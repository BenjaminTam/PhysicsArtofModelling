
\chapter{Describing motion in multiple dimensions}
\label{chap:describingmotioninnd}
In this chapter, we will learn how to extend our description of an object's motion to two and three dimensions by using vectors. We will also consider the specific case of an object moving along the circumference of a circle. 

\vspace{1cm}
\begin{learningObjectives}
\item Describe motion in a 2D plane.
\item Describe motion in 3D space.
\item Describe motion along the circumference of a circle.
\end{learningObjectives}

\section{Motion in two dimensions}

\subsection{Using vectors to describe motion in two dimensions}
We can specify the location of an object with its coordinates, and we can quantify any displacement by a vector. First consider the case of an object moving at a constant velocity in a particular direction.  We can describe the object at any time, $t$, using its position vector, $\vec r(t)$, which is a function of time:
\begin{align*}
\vec r(t=t_0)&=\vec r_1\\
\vec r(t=t_0+\Delta t)&=\vec r_2
\end{align*}
More generally, we can describe the $x$ and $y$ components of the position vector with independent functions, $x(t)$, and $y(t)$, respectively:
\begin{align*}
\vec r(t) = \begin{pmatrix}
           x(t) \\
           y(t) \\
         \end{pmatrix}= x(t) \hat x + y(t) \hat y
\end{align*}
Suppose that in a period of time $\Delta t$, the object goes from a position described by the position vector $\vec r_1$ to a position described by the position vector $\vec r_2$, as illustrated in Figure \ref{fig:DescribingMotionInND:xydrvec}. We can define a displacement vector, $\Delta \vec r=\vec r_2-\vec r_1$, and by analogy to the one dimensional case, we can define an \textbf{average} velocity vector, $\vec v$ as:
\begin{align}
\vec v = \frac{\Delta \vec r}{\Delta t}
\end{align}
\capfig{0.3\textwidth}{figures/DescribingMotionInND/xydrvec.png}{\label{fig:DescribingMotionInND:xydrvec}Illustration of a displacement vector, $\Delta \vec r = \vec r_2 -\vec r_1$, for an object that was located at position $\vec r_1$ at time $t_1$ and at position $\vec r_2$ at time $t_2=t_1+\Delta t$.}

The average velocity vector will have the same direction as $\Delta \vec r$, since it is the displacement vector divided by a scalar ($\Delta t$). The magnitude of the velocity vector, which we call ``speed'', will be proportional to the length of the displacement vector. If the object moves a large distance in a small amount of time, it will thus have a large velocity vector. This definition of the velocity vector thus has the correct intuitive properties (points in the direction of motion, is larger for faster objects).

For example, if the object went from position $(x_1,y_1)$ to position $(x_2,y_2)$ in an amount of time $\Delta t$, the average velocity vector is given by:
\begin{align*}
\vec v &= \frac{\Delta \vec r}{\Delta t}\\
&=\frac{1}{\Delta t}\begin{pmatrix}
           x_2-x_1 \\
           y_2-y_1 \\
         \end{pmatrix}\\
 &=\frac{1}{\Delta t}\begin{pmatrix}
           \Delta x \\
           \Delta y \\
         \end{pmatrix}\\     
 &=\begin{pmatrix}
           \frac{\Delta x}{\Delta t} \\
           \frac{\Delta y}{\Delta t}\\
         \end{pmatrix}\\       
 &=\begin{pmatrix}
           v_x \\
           v_y \\
         \end{pmatrix}\\    
\therefore \vec v &= v_x\hat x+v_y\hat y                     
\end{align*}
That is, the $x$ and $y$ components of the average velocity vector can be found by separately determining the average velocity in each direction. For example, $v_x=\frac{\Delta x}{\Delta t}$ corresponds to the average velocity in the $x$ direction, and can be considered independent from the velocity in the $y$ direction, $v_y$. The magnitude of the average velocity vector (i.e. the average speed), is given by:
\begin{align*}
||\vec v||&=\sqrt{v_x^2+v_y^2}=\frac{1}{\Delta t}\sqrt{\Delta x^2+\Delta y^2}=\frac{\Delta r}{\Delta t}
\end{align*}
where $\Delta r$ is the magnitude of the displacement vector. Thus, the average speed is given by the distance covered divided by the time taken to cover that distance, in analogy to the one dimensional case.

\begin{checkpointMC}{A llama runs in a field from a position $(x_1,y_1)=(\SI{2}{m},\SI{5}{m})$ to a position $(x_2,y_2)=(\SI{6}{m},\SI{8}{m})$ in a time $\Delta t=\SI{0.5}{s}$, as measured by Marcel, a llama farmer standing at the origin of the Cartesian coordinate system. What is the average speed of the llama?}
\item \SI{1}{m/s}
\item \SI{5}{m/s}
\item \SI{10}{m/s}%correct
\item \SI{15}{m/s}
\end{checkpointMC}

If the velocity of the object is not constant, then we define the \textbf{instantaneous velocity vector} by taking the limit $\Delta t\to 0$:
\begin{align}
\vec v(t) &= \lim_{\Delta t \to 0}\frac{\Delta \vec r}{\Delta t}=\frac{d\vec r}{dt}
\end{align}
which gives us the time derivative of the position vector (in one dimension, it was the time derivative of position). Writing the components of the position vector as functions $x(t)$ and $y(t)$, the instantaneous velocity becomes:
\begin{align}
\label{eqn:DescribingMotionInND:vvecdef}
\Aboxed{\vec v(t) &=\frac{d}{dt}\vec r(t) }\\
&=\frac{d}{dt} \begin{pmatrix}
           x(t) \\
           y(t) \\
         \end{pmatrix}\nonumber\\ 
&=\begin{pmatrix}
           \frac{dx}{dt}  \\
          \frac{dy}{dt}  \\
         \end{pmatrix}\nonumber\\ 
 &=\begin{pmatrix}
           v_x(t) \\
           v_y(t) \\
         \end{pmatrix}\nonumber\\   
\therefore \vec v(t) &= v_x(t)\hat x+v_y(t)\hat y  \nonumber     
\end{align}
where, again, we find that the components of the velocity vector are simply the velocities in the $x$ and $y$ direction. This means that we can treat motion in two dimensions as having two independent components: a motion along $x$ and a separate motion along $y$. This highlights the usefulness of the vector notation for allowing us to use one vector equation ($\vec v=\frac{d}{dt}\Delta \vec r$) to represent two equations (one for $x$ and one for $y$). 

Similarly the acceleration vector is given by:
\begin{align}
\label{eqn:DescribingMotionInND:avecdef}
\Aboxed{\vec a(t) &= \frac{d}{dt}\vec v(t)} \\
&=\begin{pmatrix}
           \frac{dv_x}{dt}  \\
          \frac{dv_y}{dt}  \\
         \end{pmatrix}\nonumber\\
&=\begin{pmatrix}
           a_x(t) \\
           a_y(t) \\
         \end{pmatrix}\nonumber\\
\therefore \vec a(t) &= a_x(t)\hat x+a_y(t)\hat y      \nonumber        
\end{align}

For example, if an object is at position $\vec r_0=(x_0,y_0)$ with a velocity vector $\vec v_0=v_{0x}\hat x + v_{0y}\hat y$ at time $t=0$, and has a constant acceleration vector, $\vec a = a_x\hat x+a_y\hat y$, then the velocity vector at some later time $t$, $\vec v(t)$, is given by:
\begin{align*}
\vec v(t) = \vec v_0 + \vec a t
\end{align*}
Or, if we write out the components explicitly:
\begin{align*}
\begin{pmatrix}
           v_x(t) \\
           v_y(t) \\
         \end{pmatrix} = \begin{pmatrix}
           v_{0x} \\
           v_{0y} \\
         \end{pmatrix} + \begin{pmatrix}
           a_xt \\
           a_yt \\
         \end{pmatrix}
\end{align*}
which really can be considered as two independent equations for the components of the velocity vector:
\begin{align*}
v_x(t)&=v_{0x}+a_xt \\
v_y(t)&=v_{0y}+a_yt \\
\end{align*}
which is the same equation that we had for one dimensional kinematics, but once for each coordinate. The position vector is given by:
\begin{align*}
\vec r(t) = \vec r_0 + \vec v_0 t + \frac{1}{2} \vec at^2
\end{align*}
with components:
\begin{align*}
x(t) &= x_0+v_{0x}t+\frac{1}{2}a_xt^2\\
y(t) &= y_0+v_{0y}t+\frac{1}{2}a_yt^2\\
\end{align*}
which again shows that two dimensional motion can be considered as separate and independent motions in each direction.

\begin{example}{An object starts at the origin of a coordinate system at time $t=\SI{0}{s}$, with an initial velocity vector $\vec v_0=(\SI{10}{m/s})\hat x+(\SI{15}{m/s})\hat y$. The acceleration in the $x$ direction is \SI{0}{m/s^2} and the acceleration in the $y$ direction is \SI{-10}{m/s^2}.
\begin{enumerate}[label=(\alph*)]
\item Write an equation for the position vector as a function of time.
\item Determine the position of the object at $t=\SI{10}{s}$.
\item Plot the trajectory of the object for the first \SI{5}{s} of motion.
\end{enumerate}
\ }
\label{ex:DescribingMotionInND:parabola}
\textbf{a)}We can consider the motion in the $x$ and $y$ direction separately. In the $x$ direction, the acceleration is 0, and the position is thus given by:
\begin{align*}
x(t)&=x_0+v_{0x}t\\
&=(\SI{0}{m})+(\SI{10}{m/s})t\\
&=(\SI{10}{m/s})t
\end{align*}
In the $y$ direction, we have a constant acceleration, so the position is given by:
\begin{align*}
y(t) &= y_0+v_{0y}t+\frac{1}{2}a_yt^2\\
&=(\SI{0}{m})+(\SI{15}{m/s})t+\frac{1}{2}(\SI{-10}{m/s^2})t^2\\
&=(\SI{15}{m/s})t-\frac{1}{2}(\SI{10}{m/s^2})t^2\\
\end{align*}
The position vector as a function of time can thus be written as:
\begin{align*}
\vec r(t) &= \begin{pmatrix}
           x(t) \\
           y(t) \\
          \end{pmatrix}\\
          &= \begin{pmatrix}
           (\SI{10}{m/s})t \\
           (\SI{15}{m/s})t-\frac{1}{2}(\SI{10}{m/s^2})t^2 \\
         \end{pmatrix}
\end{align*}
\textbf{b)} Using $t=\SI{10}{s}$ in the above equation gives:
\begin{align*}
\vec r(t=\SI{10}{s})&= \begin{pmatrix}
           (\SI{10}{m/s})(\SI{10}{s}) \\
           (\SI{15}{m/s})(\SI{10}{s})-\frac{1}{2}(\SI{10}{m/s^2})(\SI{10}{s})^2 \\
         \end{pmatrix}\\
         &= \begin{pmatrix}
           (\SI{100}{m}) \\
           (\SI{-350}{m})\\
         \end{pmatrix}
\end{align*}
\textbf{c)} We can plot the trajectory using python:

\begin{python}[caption=Trajectory in xy plane]
#import modules that we need
import numpy as np #for arrays of numbers
import pylab as pl #for plotting

#define functions for the x and y positions:
def x(t):
    return 10*t

def y(t):
    return 15*t-0.5*10*t**2

#define 10 values of t from 0 to 5 s:
tvals = np.linspace(0,5,10)

#calculate x and y at those 10 values of t using the functions
#we defined above:
xvals = x(tvals)
yvals = y(tvals)

#plot the result:
pl.plot(xvals,yvals, marker='o')
pl.xlabel("x [m]",fontsize=14)
pl.ylabel("y [m]",fontsize=14)
pl.title("Trajectory in the xy plane",fontsize=14)
pl.grid()
pl.show()
\end{python}
\begin{poutput}
(*\capfig{0.5\textwidth}{figures/DescribingMotionInND/parabola.png}{\label{fig:DescribingMotionInND:parabola}Parabolic trajectory of an object with no acceleration in the $x$ direction and a negative acceleration in the $y$ direction.}*)
\end{poutput}
As you can see, the trajectory is a parabola, and corresponds to what you would get when throwing an object with an initial velocity with upwards (positive $y$) and horizontal (positive $x$) components. If you look at only the $y$ axis, you will see that the object first goes up, then turns around and goes back down. This is exactly what happens when you throw a ball upwards, independently of whether the object is moving in the $x$ direction. In the $x$ direction, the object just moves with a constant velocity. The points on the graph are drawn for constant time intervals (the time between each point, $\Delta t$ is constant). If you look at the distance between points projected onto the $x$ axis, you will see that they are all equidistant and that along $x$, the motion corresponds to that of an object with constant velocity. 
\end{example}

\begin{checkpointMC}{In example \ref{ex:DescribingMotionInND:parabola}, what is the velocity vector exactly at the top of the parabola in Figure \ref{fig:DescribingMotionInND:parabola}?}
\item $\vec v=(\SI{10}{m/s})\hat x+(\SI{15}{m/s})\hat y$
\item $\vec v=(\SI{15}{m/s})\hat y$
\item $\vec v=(\SI{10}{m/s})\hat x$ %correct
\item none of the above
\end{checkpointMC}

\subsection{Accelerated motion when the velocity vector changes direction}
\label{sec:DescribingMotionInND:accvconst}
One key difference with one dimensional motion is that, in two dimensions, it is possible to have a non-zero acceleration even when the speed is constant. Recall, the acceleration \textbf{vector} is defined as the time derivative of the velocity \textbf{vector} (equation \ref{eqn:DescribingMotionInND:avecdef}). This means that if the velocity vector changes with time, then the acceleration vector is non-zero. The length of the velocity vector is called the speed. If the length of the velocity vector (speed) is constant, it is still possible that the \textbf{direction} of the velocity vector changes with time, and thus, that the acceleration vector is non-zero. In this case, the acceleration would not result in a change of speed, but rather in a change of the direction of motion. This is exactly what happens when an object goes around in a circle with a constant speed (the direction of the velocity vector changes). 
\rwcapfig[14]{0.35\textwidth}{figures/DescribingMotionInND/deltav.png}{\label{fig:DescribingMotionInND:deltav} Illustration of how the direction of the velocity vector can change when speed is constant.}

Figure \ref{fig:DescribingMotionInND:deltav} shows an illustration of a velocity vector, $\vec v(t)$, at two different times, $\vec v_1$ and $\vec v_2$, as well as the vector difference, $\Delta \vec v=\vec v_2 - \vec v_1$, between the two. In this case, the length of the velocity vector did not change with time ($||\vec v_1||=||\vec v_2||$). The acceleration vector is given by:
\begin{align*}
\vec a = \lim_{\Delta t\to 0}\frac{\Delta \vec v}{\Delta t}
\end{align*}
and will thus have a direction parallel to $\Delta \vec v$, and a magnitude that is proportional to $\Delta v$. Thus, even if the velocity vector does not change amplitude (speed is constant), the acceleration vector can be non-zero if the velocity vector changes \textit{direction}.

Let us write the velocity vector, $\vec v$, in terms of its magnitude, $v$, and a unit vector, $\hat v$, in the direction of $\vec v$:
\begin{align*}
\vec v &=v_x\hat x+v_y\hat y= v \hat v\\
v&=||\vec v||=\sqrt{v_x^2+v_y^2}\\
\hat v &= \frac{v_x}{v}\hat x+\frac{v_y}{v}\hat y\\
\end{align*}
In the most general case, both the magnitude of the velocity and its direction can change with time. That is, both the direction and the magnitude of the velocity vector are functions of time:
\begin{align*}
\vec v(t)&=v(t)\hat v(t)
\end{align*}
When we take the time derivative of $\vec v(t)$ to obtain the acceleration vector, we need to take the derivative of a product of two functions of time, $v(t)$ and $\hat v(t)$. Using the rules for taking the derivative of a product, the acceleration vector is given by:
\begin{align}
\label{eqn:DescribingMotionInND:avecdef2}
\vec a &= \frac{d}{dt}\vec v(t)= \frac{d}{dt}v(t)\hat v(t)\nonumber\\
\Aboxed{\vec a&=\frac{dv}{dt}\hat v(t)+v(t)\frac{d\hat v}{dt}}
\end{align}
and has two terms. The first term, $\frac{dv}{dt}\hat v(t)$, is zero if the speed is constant ($\frac{dv}{dt}=0$). The second term, $v(t)\frac{d\hat v}{dt}$, is zero if the direction of the velocity vector is constant ($\frac{d\hat v}{dt}=0$). In general though, the acceleration vector has two terms corresponding to the change in speed, and to the change in the direction of the velocity, respectively.

The specific functional form of the acceleration vector will depend on the path being taken by the object. If we consider the case where speed is constant, then we have:
\begin{align*}
v(t) &= v \\
\frac{dv}{dt}&=0\\
v_x^2(t)+v_y^2(t) &=v^2 \\
\therefore v_y(t)&=\sqrt{v^2-v_x(t)^2}
\end{align*}
\capfig{0.35\textwidth}{figures/DescribingMotionInND/aperpv.png}{\label{fig:DescribingMotionInND:aperpv} Illustration that the acceleration vector is perpendicular to the velocity vector if speed is constant.}
In other words, if the magnitude of the velocity is constant, then the $x$ and $y$ components are no longer independent (if the $x$ component gets larger, then the $y$ component must get smaller so that the total magnitude remains unchanged). If the speed is constant, then the acceleration vector is given by:
\begin{align}
\label{eqn:DescribingMotionInND:vecaconstv}
\vec a&=\frac{dv}{dt}\hat v(t)+v\frac{d\hat v}{dt}\nonumber\\
&=0 + v\frac{d}{dt}\hat v(t)\nonumber\\
&=v\frac{d}{dt}\left(\frac{v_x(t)}{v}\hat x+\frac{v_y(t)}{v}\hat y   )\right)\nonumber\\
&=\frac{dv_x}{dt}\hat x + \frac{d}{dt}\sqrt{v^2-v_x(t)^2}\hat y\nonumber\\
&=\frac{dv_x}{dt}\hat x + \frac{1}{2\sqrt{v^2-v_x(t)^2}}(-2v_x(t))\frac{dv_x}{dt}\hat y\nonumber\\
&=\frac{dv_x}{dt}\hat x - \frac{v_x(t)}{\sqrt{v^2-v_x(t)^2}}\frac{dv_x}{dt}\hat y\nonumber\\
&=\frac{dv_x}{dt}\hat x - \frac{v_x(t)}{v_y(t)}\frac{dv_x}{dt}\hat y\nonumber\\
\therefore\quad\Aboxed{\vec a&=\frac{dv_x}{dt} \left(\hat x - \frac{v_x(t)}{v_y(t)}\hat y\right)}
\end{align}
where most of the algebra that we did was to separate out the $x$ and $y$ components of the acceleration vector. The resulting acceleration vector is illustrated in Figure \ref{fig:DescribingMotionInND:aperpv} along with the velocity vector. Rather, a vector parallel to the acceleration vector is illustrated, as the factor of $\frac{dv_x}{dt}$ was omitted (as you recall, multiplying by a scalar only changes the length, not the direction). The velocity vector has components $v_x$ and $v_y$, which allows us to calculate the angle, $\theta$ that it makes with the $x$ axis:
\begin{align*}
\tan(\theta)=\frac{v_y}{v_x}
\end{align*}
Similarly, the vector that is parallel to the acceleration has components of $1$ and $-\frac{v_x}{v_y}$, allowing us to determine the angle, $\phi$, that it makes with the $x$ axis:
\begin{align*}
\tan(\phi)=\frac{v_x}{v_y}
\end{align*}
Note that $\tan(\theta)$ is the inverse of $\tan(\phi)$, or in other words, $\tan(\theta)=\cot(\phi)$, meaning that $\theta$ and $\phi$ are complementary and thus must sum to $\frac{\pi}{2}$ (\SI{90}{\degree}). This means that \textbf{the acceleration vector is perpendicular to the velocity vector if the speed is constant and the direction of the velocity changes}. 

In other words, when we write the acceleration vector, we can identify two components, $\vec a_{\parallel}(t)$ and $\vec a_{\perp}(t)$:
\begin{align*}
\vec a&=\frac{dv}{dt}\hat v(t)+v(t)\frac{d\hat v}{dt}\\
&=\vec a_{\parallel}(t) + \vec a_{\perp}(t)\\
\therefore \vec a_{\parallel}(t)&=\frac{dv}{dt}\hat v(t)\\
\therefore \vec a_{\perp}(t)&=v\frac{d\hat v}{dt}=\frac{dv_x}{dt} \left(\hat x - \frac{v_x(t)}{v_y(t)}\hat y\right)
\end{align*}
where $\vec a_{\parallel}(t)$ is the component of the acceleration that is parallel to the velocity vector, and is responsible for changing its magnitude, and $\vec a_{\perp}(t)$, is the component that is perpendicular to the velocity vector and is responsible for changing the direction of the motion.

\begin{checkpointMC}{A satellite moves in a circular orbit around the Earth with a constant speed. What can you say about its acceleration vector?}
\item it has a magnitude of zero.
\item it is perpendicular to the velocity vector.
\item it is parallel to the velocity vector.
\item it is in a direction other than parallel or perpendicular to the velocity vector.
\end{checkpointMC}

\subsection{Relative motion}
In the previous chapter, we examined how to convert the description of motion from one reference frame to another. Recall the one dimensional situation where we described the position of an object, $A$, using an axis $x$ as $x^A(t)$. Suppose that the reference frame, $x$, is moving with a constant speed, $v'^B$, relative to a second reference frame, $x'$. We found that the position of the object is described in the $x'$ reference frame as:
\begin{align*}
x'^A(t)=v'^Bt+x^A(t)
\end{align*}
if the origins of the two systems coincided at $t=0$. The equation above simply states that the distance of the object to the $x'$ origin is the sum of the distance from the $x'$ origin to the $x$ origin \textbf{and} the distance from the $x$ origin to the object.

In two dimensions, we proceed in exactly the same way, but use vectors instead:
\begin{align*}
\pvec r'^A(t) = \pvec v'^Bt+\vec r^A(t)
\end{align*}
where $r^A(t)$ is the position of the object as described in the $xy$ reference frame, $\pvec v'^B$, is the velocity vector describing the motion of the origin of the $xy$ coordinate system relative to an $x'y'$ coordinate system. $\pvec r'^A(t)$ is the position of the object in the $x'y'$ coordinate system. We have assumed that the origins of the two coordinate systems coincided at $t=0$ and that the axes of the coordinate systems are parallel ($x$ parallel to $x'$ and $y$ parallel to $y'$).

Note that the velocity of the object in the $x'y'$ system is found by adding the velocity of $xy$ relative to $x'y'$ and the velocity of the object in the $xy$ frame ($\vec v^A(t)$):
\begin{align*}
\frac{d}{dt}\pvec r'^A(t) &=\frac{d}{dt}(\pvec v'^Bt+\vec r^A(t))\\
&=\pvec v'^B+\vec v^A(t)
\end{align*}

As an example, consider the situation depicted in Figure \ref{fig:DescribingMotionInND:2drel}. Brice is on a boat off the shore of Nice, with a coordinate system $xy$, and is describing the position of a boat carrying Alice. He describes Alice's position as $\vec r^A(t)$ in the $xy$ coordinate system. Igor is on the shore and also wishes to describe Alice's position using the work done by Brice. Igor sees Brice's boat move with a velocity $\vec v'^B$ as measured in his $x'y'$ coordinate system. In order to find the vector pointing to Alice's position $\pvec r'^A(t)$, he adds the vector from his origin to Brice's origin ($\pvec v'^B t$) and the vector from Brice's origin to Alice $\vec r^A(t)$.

\capfig{0.7\textwidth}{figures/DescribingMotionInND/2drel.png}{\label{fig:DescribingMotionInND:2drel} Example of converting from one reference frame to another in two dimensions using vector addition.}

Writing this out by coordinate, we have:
\begin{align*}
x'^A(t)&=v'^B_xt+x^A(t)\\
y'^A(t)&=v'^B_yt+y^A(t)
\end{align*}
and for the velocities:
\begin{align*}
v_x'^A(t)&=v'^B_x+v_x^A(t)\\
v_y'^A(t)&=v'^B_y+v_y^A(t)
\end{align*}


\begin{checkpointMC}{You are on a boat and crossing a North-flowing river, from the East bank to the West bank. You point your boat in the West direction and cross the river. \chloe is watching your boat cross the river from the shore, in which direction does she measure your velocity vector to be?}
\item in the North direction
\item in the West direction
\item a combination of North and West directions
\end{checkpointMC}


\section{Motion in three dimensions}
The big challenge was to expand our description of motion from one dimension to two. Adding a third dimension ends up being trivial now that we know how to use vectors. In three dimensions, we describe the position of a point using three coordinates, so all of the vectors simply have three independent components, but are treated in exactly the same way as in the two dimensional case. The position of an object is now described by three independent functions, $x(t)$, $y(t)$, $z(t)$, that make up the three components of a position vector $\vec r(t)$:
\begin{align*}
\vec r(t) &= \begin{pmatrix}
           x(t) \\
           y(t) \\
           z(t)  \\
         \end{pmatrix}\\
\therefore \vec r(t)  &= x(t) \hat x + y(t) \hat y + z(t) \hat z
\end{align*}
The velocity vector now has three components and is defined analogously to the 2D case:
\begin{align*}
\vec v(t) &=\frac{d\vec r}{dt}
 =\begin{pmatrix}
           \frac{dx}{dt}  \\
          \frac{dy}{dt}  \\
          \frac{dz}{dt}  \\
         \end{pmatrix}
 =\begin{pmatrix}
           v_x(t) \\
           v_y(t) \\
           v_z(t) \\
         \end{pmatrix}\\   
\therefore \vec v(t) &= v_x(t)\hat x+v_y(t)\hat y+v_z(t)\hat z  \nonumber 
\end{align*}
and the acceleration is defined in a similar way:
\begin{align*}
\vec a(t)  &=\frac{d\vec v}{dt}
 =\begin{pmatrix}
           \frac{dv_x}{dt}  \\
          \frac{dv_y}{dt}  \\
          \frac{dv_z}{dt}  \\
         \end{pmatrix}
 =\begin{pmatrix}
           a_x(t) \\
           a_y(t) \\
           a_z(t) \\
         \end{pmatrix}\\   
\therefore \vec a(t) &= a_x(t)\hat x+a_y(t)\hat y+a_z(t)\hat z  \nonumber 
\end{align*}

In particular, if an object has a constant acceleration, $\vec a=a_x\hat x+a_y\hat y+a_z\hat z$, and started at $t=0$ with a position $\vec r_0$ and velocity $\vec v_0$, then its velocity vector is given by:
\begin{align*}
\vec v(t)  &= \vec v_0+\vec at=\begin{pmatrix}
           v_{0x}+ a_xt \\
           v_{0y}+ a_yt \\
           v_{0z}+ a_zt \\
         \end{pmatrix}\\
\end{align*}
and the position vector is given by:
\begin{align*}
\vec r(t)= \vec r_0+\vec v_0 t+\frac{1}{2}\vec a t^2=\begin{pmatrix}
           x_0+v_{0x}t+\frac{1}{2} a_xt^2 \\
           y_0+v_{0y}t+\frac{1}{2} a_yt^2 \\
           z_0+v_{0z}t+\frac{1}{2} a_zt^2 \\
         \end{pmatrix}\\
\end{align*}
where again, we see how writing a single vector equation (e.g. $\vec v(t) = \vec v_0+\vec at$) is really just a way to write the three independent equations that are true for each component.
\section{Circular motion}
\label{sec:describingmotioninnd:circular}
We often consider the motion of an object around a circle of fixed radius, $R$. In principle, this is motion in two dimensions, as a circle is necessarily in a two dimensional plane. However, since the object is constrained to move along the circumference of the circle, it can be thought of (and treated as) motion along a one dimensional axis that is curved. 
\capfig{0.35\textwidth}{figures/DescribingMotionInND/circle.png}{\label{fig:DescribingMotionInND:circle} Describing the motion of an object around a circle of radius $R$.}

Figure \ref{fig:DescribingMotionInND:circle} shows how we can describe motion on a circle. We could use $x(t)$ and $y(t)$ to describe the position on the circle, however, $x(t)$ and $y(t)$ are no longer independent since they have to correspond to the coordinates of points on a circle:
\begin{align*}
x^2(t)+y^2(t)=R^2
\end{align*}
Instead of using $x$ and $y$, we could think of an axis that is bent around the circle (as shown by the curved arrow in Figure \ref{fig:DescribingMotionInND:circle}, the $s$ axis). The $s$ axis is such that $s=0$ where the circle intersects the $x$ axis, and the value of $s$ increases as we move counter-clockwise along the circle. Distance along the $s$ axis thus corresponds to the distance along the circumference of the circle.

Another variable that could be used for position instead of $s$ is the angle, $\theta$, between the position vector of the object and the $x$ axis, as illustrated in Figure \ref{fig:DescribingMotionInND:circle}. If we express the angle $\theta$ in radians, then it easy to convert between $s$ and $\theta$. Recall, an angle in radians is defined as the length of an arc subtended by that angle divided by the radius of the circle. We thus have:
\begin{align}
\label{eqn:DescribingMotionInND:raddef}
\Aboxed{\theta(t)=\frac{s(t)}{R}}
\end{align}
In particular, if the object has gone around the whole circle, then $s=2\pi R$ (the circumference of a circle), and the corresponding angle is, $\theta=\frac{2\pi R}{R}=2\pi$, namely \SI{360}{\degree}. 

By using the angle, $\theta$, instead of $x$ and $y$, we are effectively using polar coordinates, with a fixed radius. As we already saw, the $x$ and $y$ positions are related to $\theta$ by:
\begin{align*}
x(t) &= R\cos(\theta(t))\\
y(t) &= R\sin(\theta(t))\\
\end{align*}
where $R$ is a constant. For an object moving along the circle, we can write its position vector, $\vec r(t)$, as:
\begin{align*}
\vec r(t)&= \begin{pmatrix}
           x(t) \\
           y(t) \\
         \end{pmatrix}
         =R \begin{pmatrix}
           \cos(\theta(t)) \\
           \sin(\theta(t)) \\
         \end{pmatrix}
\end{align*}
\capfig{0.35\textwidth}{figures/DescribingMotionInND/vcircle.png}{\label{fig:DescribingMotionInND:vcircle} The position vector, $\vec r(t)$ is always perpendicular to the velocity vector, $\vec v(t)$, for motion on a circle.}
and the velocity vector is thus given by:
\begin{align*}
\vec v(t) &=\frac{d}{dt}\vec r(t) 
=\frac{d}{dt} R \begin{pmatrix}
           \cos(\theta(t)) \\
           \sin(\theta(t)) \\
         \end{pmatrix} \\
&= R \begin{pmatrix}
           \frac{d}{dt}\cos(\theta(t)) \\
           \frac{d}{dt}\sin(\theta(t)) \\
         \end{pmatrix} \\
 &= R \begin{pmatrix}
           -\sin(\theta(t))\frac{d\theta}{dt} \\
           \cos(\theta(t))\frac{d\theta}{dt} \\
         \end{pmatrix}     
\end{align*}         
where we used the Chain Rule to calculate the time derivatives of the trigonometric functions (since $\theta(t)$ is function of time). The magnitude of the velocity vector is given by:
\begin{align*}
||\vec v|| &=\sqrt{ v_x^2+v_y^2}\\
&=\sqrt{ \left(-R\sin(\theta(t))\frac{d\theta}{dt}\right)^2+\left(R\cos(\theta(t))\frac{d\theta}{dt}\right)^2}\\
&=\sqrt{ R^2\left( \frac{d\theta}{dt}\right)^2[\sin^2(\theta(t))+\cos^2(\theta(t)]}\\
&=R\left |\frac{d\theta}{dt}\right|
\end{align*}

The position and velocity vectors are illustrated in Figure \ref{fig:DescribingMotionInND:vcircle} for an angle $\theta$ in the first quadrant ($0<\theta<\frac{\pi}{2}$). In this case, you can note that the $x$ component of the velocity is negative (in the equation above, and in the Figure). From the equation above, you can also see that $\frac{|v_x|}{|v_y|}=\tan(\theta)$, which is illustrated in Figure \ref{fig:DescribingMotionInND:vcircle}, showing that \textbf{the velocity vector is tangent to the circle} and perpendicular to the position vector. This is always the case for motion along a circle.

We can simplify our description of motion along the circle by using either $s(t)$ or $\theta(t)$ instead of the vectors for position and velocity. If we use $s(t)$ to represent position along the circumference ($s=0$ where the circle intersects the $x$ axis), then the velocity along the $s$ axis is:
\begin{align*}
v_s(t)&=\frac{d}{dt}s(t)\\
&=\frac{d}{dt}R\theta(t)\\
&=R\frac{d\theta}{dt}
\end{align*}
where we used the fact that $\theta=\frac{s}{R}$ to convert from $s$ to $\theta$. The velocity along the $s$ axis is thus precisely equal to the magnitude of the two-dimensional velocity vector (derived above), which makes sense since the velocity vector is tangent to the circle (and thus in the $s$ ``direction'').

If the object has a \textbf{constant speed}, $v_s$, along the circle and started at a position along the circumference $s=s_0$, then its position along the $s$ axis can be described as:
\begin{align*}
s(t)=s_0+v_st
\end{align*}
or, in terms of $\theta$:
\begin{align*}
\theta(t)&=\frac{s(t)}{R}=\frac{s_0}{R}+\frac{v_s}{R}t\\
&=\theta_0 + \frac{d\theta}{dt}t\\
&=\theta_0 + \omega t\\
\Aboxed{\therefore \omega &= \frac{d\theta}{dt}}
\end{align*}
where we introduced $\theta_0$ as the angle corresponding to the position $s_0$, and we introduced $\omega=\frac{d\theta}{dt}$, which is analogous to velocity, but for an angle. $\omega$ is called the \textbf{angular velocity} and is a measure of the rate of change of the angle $\theta$ (as it is the time derivative of the angle). The relation between the ``linear'' velocity $v_s$ (the magnitude of the velocity vector, which corresponds to the velocity in the direction tangent to the circle) and $\omega$ is:
\begin{align*}
\Aboxed{v_s=R\frac{d\theta}{dt}=R\omega }
\end{align*}

\begin{studentopinionOW}{A way to think about angular and linear velocity}
TO DO: Explain figure
\capfig{0.7\textwidth}{figures/DescribingMotionInND/HandPolarCoordinates.png}{\label{fig:HandPolarCoordinates} How to use your hand to better understand polar coordinates}
\end{studentopinionOW}

Similarly, if the object is accelerating, we can define an \textbf{angular acceleration}, $\alpha(t)$, as the rate of change of the angular velocity:
\begin{align*}
\alpha(t)=\frac{d\omega}{dt}
\end{align*}
which can directly be related to the acceleration in the $s$ direction, $a_s(t)$:
\begin{align*}
a_d(t) &= \frac{d}{dt}v_s\\
&=\frac{d}{dt}\omega R=R\frac{d\omega}{dt}\\
\Aboxed{a_d(t)&=R\alpha }
\end{align*}
Thus, the linear quantities (those along the $s$ axis) can be related to the angular quantities by multiplying the angular quantities by $R$:
\begin{align}
s&=R\theta\\
v_s&=R\omega\\
a_s&=R\alpha
\end{align}
If the object started at $t=0$ with a position $s=s_0$ ($\theta=\theta_0$), and an initial linear velocity $v_{0s}$ (angular velocity $\omega_0$), and has a \textbf{constant linear acceleration} around the circle, $a_s$ (angular acceleration, $\alpha$), then the position of the object can be described as:
\begin{align*}
s(t) &= s_0+v_{s0}t+\frac{1}{2}a_s t^2\\
\theta(t) &= \theta_0+\omega_0t+\frac{1}{2}\alpha t^2
\end{align*}
which corresponds to an object that is going around the circle faster and faster.

As you recall from section \ref{sec:DescribingMotionInND:accvconst}, we can compute the acceleration \textbf{vector} and identify components that are parallel and perpendicular to the velocity vector:
\begin{align*}
\vec a&=\vec a_{\parallel}(t) + \vec a_{\bot}(t)\\
&=\frac{dv}{dt}\hat v(t)+v\frac{d\hat v}{dt}\\
\end{align*}
The first term, $\vec a_{\parallel}(t)=\frac{dv}{dt}\hat v(t)$, is parallel to the velocity vector $\hat v$, and has a magnitude given by:
\begin{align*}
||\vec a_{\parallel}(t)||&=\frac{dv}{dt}=\ddt v(t)=\ddt R\omega=R\alpha
\end{align*}
That is, the component of the acceleration vector that is parallel to the velocity is precisely the acceleration in the $s$ direction (the linear acceleration). This component of the acceleration is responsible for increasing (or decreasing) the speed of the object and is zero if the object goes around the circle with a constant speed (linear or angular). 

As we saw earlier, the perpendicular component of the acceleration, $\vec a_{\bot}(t)$, is responsible for changing the direction of the velocity vector (as the object continuously changes direction when going in a circle). When the motion is around a circle, this component of the acceleration vector is called ``centripetal'' acceleration (i.e. acceleration pointing towards the centre of the circle, as we will see). We can calculate the centripetal acceleration in terms of our angular variables, noting that the unit vector in the direction of the velocity is $\hat v=-\sin(\theta)\hat x+\cos(\theta)\hat y$:
\begin{align}
\vec a_{\bot}(t)&=v\frac{d\hat v}{dt}\nonumber\\
&=(\omega R)\ddt \left[-\sin(\theta)\hat x+\cos(\theta)\hat y\right]\nonumber\\
&=\omega R \left[-\ddt\sin(\theta)\hat x+\ddt\cos(\theta)\hat y\right]\nonumber\\
&=\omega R \left[-\cos(\theta)\frac{d\theta}{dt}\hat x-\sin(\theta)\frac{d\theta}{dt}\hat y\right]\nonumber\\
&=\omega R [-\cos(\theta)\omega\hat x-\sin(\theta)\omega\hat y]\nonumber\\
\Aboxed{\vec a_{\bot}(t)&=\omega^2 R[-\cos(\theta)\hat x-\sin(\theta)\hat y]}
\end{align}
where you can easily verify that the vector $[-\cos(\theta)\hat x-\sin(\theta)\hat y]$ has unit length and points towards the centre of the circle (when the tail is placed on a point on the circle at angle $\theta$). The centripetal acceleration thus points towards the centre of the circle and has magnitude:
\begin{align}
a_c(t) = ||\vec a_{\bot}(t)||=\omega^2(t) R = \frac{v^2(t)}{R}
\end{align}
where in the last equal sign, we wrote the centripetal acceleration in terms of the speed around the circle ($v=||\vec v||=v_s$).

If an object goes around a circle, it will always have a centripetal acceleration (since its velocity vector must change direction). In addition, if the object's speed is changing, it will also have a linear acceleration, which points in the same direction as the velocity vector (it changes the velocity vector's length but not its direction).

\begin{checkpointMC}{A vicu\~na is going clockwise around a circle that is centred at the origin of an $xy$ coordinate system that is in the plane of the circle. The vicu\~na runs faster and faster around the circle. In which direction does its acceleration vector point just as the vicu\~na is at the point where the circle intersects the positive $y$ axis?}
\item In the negative $y$ direction
\item In the positive $y$ direction
\item A combination of the positive $y$ and positive $x$ directions
\item A combination of the negative $y$ and positive $x$ directions %correct
\item A combination of the negative $y$ and negative $x$ directions
\end{checkpointMC}

\subsection{Period and frequency}
When an object is moving around in a circle, it will typically complete more than one revolution. If the object is going around the circle with a constant speed, we call the motion ``uniform circular motion'', and we can define the \textbf{period and frequency} of the motion. 

The period, $T$, is defined to be the time that it takes to complete one revolution around the circle. If the object has constant angular speed $\omega$, we can find the time, $T$, that it takes to complete one full revolution, from $\theta=0$ to $\theta=2\pi$:
\begin{align}
\omega&=\frac{\Delta \theta}{T}=\frac{2\pi}{T}\nonumber\\
\Aboxed{\therefore T&=\frac{2\pi}{\omega}}
\end{align}
We would obtain the same result using the linear quantities; in one revolution, the object covers a distance of $2\pi R$ at a speed of $v$:
\begin{align*}
v&=\frac{2\pi R}{T}\\
T&=\frac{2\pi R}{v}=\frac{2\pi R}{\omega R}=\frac{2\pi}{\omega}
\end{align*}

The frequency, $f$, is defined to be the inverse of the period:
\begin{align*}
f&=\frac{1}{T}=\frac{\omega}{2\pi}
\end{align*}
and has SI units of $\si{Hz}=\si{s^{-1}}$. Think of frequency as the number of revolutions completed per second. Thus, if the frequency is $f=\SI{1}{Hz}$, the object goes around the circle once per second. 
\capfig{0.35\textwidth}{figures/DescribingMotionInND/twocircles.png}{\label{fig:DescribingMotionInND:twocircles} For a given angular velocity, the linear velocity will be larger on a larger circle ($v=\omega R$).} Given the frequency, we can of course obtain the angular velocity:
\begin{align*}
\omega = 2\pi f
\end{align*}
which is sometimes called the ``angular frequency'' instead of the angular velocity. The angular velocity can really be thought of as a frequency, as it represents the ``amount of angle'' per second that an object covers when going around a circle. The angular velocity does not tell us anything about the actual speed of the object, which depends on the radius $v=\omega R$. This is illustrated in Figure \ref{fig:DescribingMotionInND:twocircles}, where two objects can be travelling around two circles of radius $R_1$ and $R_2$ with the same angular velocity $\omega$. If they have the same angular velocity, then it will take them the same amount of time to complete a revolution. However, the outer object has to cover a much larger distance (the circumference is larger), and thus has to move with a larger linear speed.

\begin{checkpointMC}{A motor is rotating at \SI{3000}{rpm}, what is the corresponding frequency in \si{Hz}?}
\item \SI{5}{Hz}
\item \SI{50}{Hz}%correct
\item \SI{500}{Hz}
\end{checkpointMC}


\newpage
\section{Summary}
\vspace{2cm}
\begin{chapterSummary}
\item Something interesting
\end{chapterSummary}

\section{Sample problems and solutions}
\begin{problemParts}{Ethan is jumping hurdles in the Olympics. He gets a running start, moving with a velocity of $\SI{4}{m/s}$ [E], and will not slow down before jumping. The hurdle is $\SI{1}{m}$ high and the maximum speed he can have when he leaves the ground is $\SI{5}{m/s}$. (You can assume Ethan is a point particle).}
\item How close can he get to the hurdle before he has to jump?
\item What maximum height does he reach?
\item Where does he land?
\end{problemParts}

\begin{problemParts}{A cowboy swings a lasso above his head. The lasso moves in a circle of radius $\SI{1.5}{m}$ in the horizontal plane. A hawk flies toward the lasso at $\SI{50}{km/h}$. The hawk sees the end of the lasso moving at $\SI{60}{km/h}$ when the lasso is directly in front of it (see Figure \ref{fig:CowboyQuestion}). In the reference frame of the cowboy ...}
\item How long does it take for the lasso to complete one revolution?
\item What is the centripetal acceleration of the end of the lasso? 
\item What is the angular acceleration of the lasso?
\capfig{0.5\textwidth}{figures/DescribingMotionInND/CowboyQuestionGiven.png}{\label{fig:CowboyQuestion} The problem as viewed from above. This diagram depicts the moment that the end of the lasso passes in front of the hawk.}
\end{problemParts} 


\textbf{Solution:}
\begin{enumerate}[label=\alph*)]
\item Our goal is the find the period of the lasso's motion. To do this, we can use the formula, 
\begin{align*}
T&=\frac{2\pi}{\omega}
\end{align*}
for which we need the angular velocity, $\omega$. We know the radius of the lasso, so if we find the linear velocity of the end point of the lasso, we can find the angular velocity by
\begin{align*}
\omega&=\frac{v}{R}
\end{align*}
We start by using what we know about relative motion to find the linear velocity of the lasso in the cowboy's reference frame. First, we need to set up our coordinate systems. We assign the $xy$ coordinate system to the hawk's reference frame and we assign the $x'y'$ system to the cowboy's reference frame. The solution will be simplest if we align the coordinate systems so that positive $y$ and positive $y'$ are in the same direction, as in Figure \ref{fig:CowboySolution}. When we are talking about the velocity of the hawk, we will denote it with the superscript ``H", and when we are talking about the lasso, we will use ``L".\\

We want the velocity of the \textbf{lasso} in the \textbf{cowboy's reference frame}, so we want $v'^L$. To find this, we start with the velocity of the lasso in the hawk's reference frame,$v^L$ and then take into account that the hawk is moving relative to the cowboy. We do this by adding the velocity of the hawk in the cowboy's reference frame, $v'^H$ to $v^L$. So, our equation is,
\begin{align*}
v^L+v'^H&=v'^L
\end{align*}
We are adding velocities, which have both a magnitude and a direction. However, we were not given any directions in the problem, so we describe the directions with respect to our coordinate system. The way we have set up our axes, the velocity of the hawk in the cowboy's reference frame is simply $\SI{50}{km/h}$ in the positive $y'$ direction.\\

Now here's the key to solving this problem: We don't know the speed of the lasso in the cowboy's reference frame, but we do know something about its direction. Since the motion of the lasso is circular, it's velocity must be tangent to the circle. This means that when the lasso is directly in front of the hawk, its velocity must be in either the $+x'$ or $-x'$ direction. In this case, we can just choose one, so we will choose the $+x'$ direction.
\capfig{0.5\textwidth}{figures/DescribingMotionInND/CowboySolution.png}{\label{fig:CowboySolution}The two coordinate systems are aligned so that positive $y'$ and positive $y$ are in the same direction. The velocity vectors of the hawk and the lasso in the reference frame of the cowboy are shown.}
The velocity vectors $v'^H$ and $v'^L$ are shown in Figure \ref{fig:CowboySolution}. Remember that when we add two vectors they must be lined up so that the ``head" of one touches the ``tail" of the other, so there can only be one direction for $v^L$, as shown in Figure \ref{fig:CowboyVector}. 
\capfig{0.25\textwidth}{figures/DescribingMotionInND/CowboyVectorAddition.png}{\label{fig:CowboyVector} Vector addition to determine the velocity of the lasso in the cowboy's reference frame.}
This is a right angle triangle, so we use the Pythagorean theorem so solve for $v'^L$:
\begin{align*}
v'^{L^2}+v'^{H^2}&=v^{L^2}\\
&=\sqrt{v^{L^2}-v'^{H^2}}\\
&=\sqrt{(\SI{60}{km/h})^2-(\SI{50}{km/h})^2}\\
v'^L&=\SI{33}{km/h}
\end{align*}
The linear velocity of the end of the lasso at this moment is $\SI{33}{km/h}$ in the positive $x$ direction.  To find the angular velocity, first convert the linear velocity from km/h to m/s:
\begin{align*}
\frac{\SI{33}{km}}{h}\times \frac{\SI{1000}{m}}{\SI{1}{km}} \times \frac{\SI{1}{h}}{\SI{3600}{s}} &= \SI{9.2}{m/s}
\end{align*}     
Now we can substitute $\omega=\frac{v}{R}$ into $T=\frac{2\pi}{\omega}$ and solve for $T$:
\begin{align*}
T&={2\pi}\frac{R}{v}\\
&={2\pi}\frac{\SI{1.5}{m}}{\SI{9.2}{m/s}}\\
&={2\pi}\frac{\SI{1.5}{m}}{\SI{9.2}{m/s}}\\
T&=\SI{1.0}{s}
\end{align*}
$\therefore$ it takes $\SI{1.0}{s}$ for the lasso to complete one revolution.
\item The motion is circular, so it has a centripetal acceleration given by
\begin{align*}
a_c(t)&=\frac{v^2(t)}{R}
\end{align*}
To find the centripetal acceleration of the end of the lasso, we just substitute in our values for $v$ and $R$.
\begin{align*}
a_c(t)&=\frac{\SI{9.2}{m/s}^2}{\SI{1.5}{m}}\\
a_c(t)&=\SI{56}{m/s^2}
\end{align*}
$\therefore$ the centripetal acceleration of the end of the lasso is $\SI{56}{m/s^2}$ towards the centre of the circle. 
\item You may be tempted to divide the centripetal acceleration by $R$ to find the angular acceleration $\alpha$. However, the angular acceleration is the rate of change of the angular velocity. For circular motion, the angular velocity is constant, so \textbf{the angular acceleration is zero}. (Remember that in the equation $a_s=R\alpha$, $a_s$ refers to the component of acceleration that is parallel to the velocity. For circular motion, $a_s$ is zero.)
\end{enumerate}




