\chapter{The Python programming language}
\label{chap:app:python}
This appendix gives a very brief introduction to programming in python and is primarily aimed at introducing tools that are useful for the experimental side of physics. 
 \vspace{1cm}
\begin{learningObjectives}
\item Be able to perform simple algebra using python
\item Be able to plot a function in python
\item Be able to propagate uncertainties in python
\item Be able to plot and fit data to a straight line
\item Understand how to use Python to numerically calculate any integral
\end{learningObjectives}

In this textbook, we will encourage you to use computers to facilitate making calculations and displaying data. We will make use of a popular programming language called Python, as well as several ``modules'' from Python that facilitate working with numbers and data. Do not worry if you do not have any programming experience; we assume that you have none and hope that by the end of this book, you will have some capability to decrease your workload by using computer programming.

The only way to become proficient at programming is through practice. If you want to effectively learn from this chapter, it is important that you take the time to actually type the commands into a Python environment rather than simply reading through the chapter. Reading through the chapter will at least give you a sense of what is possible and some terminology, but it will not teach you programming!

\section{A quick intro to programming}
In Python, as in other programming language, the equal sign is called the \textbf{assignment operator}. Its role is to \textit{assign} the value on its right to the variable on its left. The following code does the following:
\begin{itemize}
\item \textit{assigns} the value of \code{2} to the variable \code{a}
\item \textit{assigns} the values of \code{2*a} to the variable \code{b}
\item prints out the value of the variable \code{b}
\end{itemize}

\begin{python}[caption=Declaring variables in Python] 
#This is a comment, and is ignored by Python
a = 2 
b = 2*a
print(b)
\end{python}
\begin{poutput}
4
\end{poutput}
Note that any text that follows a pound sign (\#) is intended as a comment and will be ignored by Python. Inserting comments in your code is very important for being able to understand your computer program in the future or if you are sharing your code with someone who would like to understand it. In the above example, we called the \code{print()} \textbf{function} and passed to it the variable \code{b} as an \textbf{argument}; this allowed us to print (display) the value of the variable \code{b} and verify that it was indeed equal to the number 4.


In Python, if you want to have access to ``functions'', which are more complex series of operations, then you typically need to load the \textit{module} that defines those operations. 

A large number of functions are provided in Python. Most of these functions need to be ``imported'' from ``modules''. For example, if you want to be able to take the square root of a number, then you need to load (import) the ``math module'' which contains the square root function, as in the following example:
\begin{python}[caption=Using functions from modules] 
#First, we load (import) the math module
import math as m
a = 9
b = m.sqrt(a)
print(b)
\end{python}
\begin{poutput}
3
\end{poutput}
In the above code, we loaded the math module (and renamed it \code{m}); this then allows us to use the functions that are part of that module, including the square root function (\code{m.sqrt()}).

\section{Arrays}
It is often the case that we need to represent a series of numbers. For example, imagine that you have measured the position of an object as a function of time. \textbf{Arrays} are a convenient way to hold a series of numbers that are all alike, for example, all of the values of the position and corresponding time values for the trajectory of the object. In Python, we can define variables that hold arrays instead of a single value (arrays are called ``lists'' in Python):
\begin{python}[caption=Arrays in python]
#define an array of values for the position of the object
position = [0,1,4,9,16,25]
#define an array of values for the corresponding times
time = [0,1,2,3,4,5]
\end{python}

\section{Plotting}
Several modules are available in python for plotting. We will show here how to use the \code{pylab} module (which is equivalent to the \code{matplotlib} module). For example, we can easily plot the data in the two arrays from the previous section in order to plot the position versus time for the object:
\begin{python}[caption=Plotting two arrays]
#import the pylab module
import pylab as pl

#define an array of values for the position of the object
position = [0,1,4,9,16,25]
#define an array of values for the corresponding times
time = [0,1,2,3,4,5]

#make the plot showing points and the line (.-)
pl.plot(time, position)
#add some labels:
pl.xlabel("time") #label for x-axis
pl.ylabel("position") #label for y-axis
#show the plot
pl.show()

\end{python}
\begin{poutput}
(*  \capfig{0.6\textwidth}{figures/AppendixC/positiontime.png}{Plotting two arrays.} *)
\end{poutput}

\begin{checkpointSA}{How would you modify the Python code above to show only the points, and not the line?}
\end{checkpointSA}

We can use Python to plot any mathematical function that we like. It is important to realize that computers do not have a representation of a continuous function. Thus, if we would like to plot a continuous function, we first need to evaluate that function at many points, and then plot those points. The \code{numpy} module provides many useful features for working with arrays of numbers and applying functions directly to those arrays. 

Suppose that we would like to plot the function $f(x) = cos(x^2)$ between $x=-3$ and $x=5$. In order to do this in Python, we will first generate an array of many values of $x$ between $-10$ and $25$ using the \code{numpy} package and the function \code{linspace(min,max,N)} which generates $N$ linearly spaced points between $min$ and $max$. We will then evaluate the function at all of those points to create a second array. Finally, we will plot the two arrays against each other:
\begin{python}[caption=Plotting a function of 1 variable]
#import the pylab and numpy modules
import pylab as pl
import numpy as np

#Use numpy to generate 1000 values of x between -3 and 5:
xvals =np.linspace(-3,5,1000)

#Now, evaluate the function for all of those values of x.
#We use the numpy version of cos, since it allows us to take the cos 
#of all values in the array
fvals = np.cos(xvals**2)

#make the plot showing only a line, and color it
pl.plot(xvals, fvals, color='red')
#add some labels:
pl.xlabel("time") #label for x-axis
pl.ylabel("position") #label for y-axis
#show the plot
pl.show()

\end{python}
\begin{poutput}
(*  \capfig{0.6\textwidth}{figures/AppendixC/functionplot.png}{Plotting a function using arrays.} *)
\end{poutput}

\section{The QExpy python package for experimental physics}
QExpy is a Python module that was developed with students from Queen's University to handle all aspects of undergraduate physics laboratories. In this section, we look at how to use QExpy to propagate uncertainties and to plot experimental data.

\subsection{Propagating uncertainties}
In Chapter \ref{chap:2_ModelAndExperiment}, we saw how to use the ``derivative method'' to propagate the uncertainty from measurements into the uncertainty in a value that depended on those measurements. In Example \ref{ex:Chap2:derivprop}, we propagated the uncertainties $x=\SI{3.00 \pm 0.01}{m}$ and $t=\SI{0.76\pm0.15}{s}$ to the quantity $k=\frac{t}{\sqrt x}$. We show below how easily this can be done with QExpy:

\begin{python}[caption=QExpy to propagate uncertainties] 
#First, we load the QExpy module
import qexpy as q
#Now define our measurements with uncertainties:
t = q.Measurement(0.76, 0.15) # 0.76 +/- 0.15
x = q.Measurement(3,0.1) # 3 +/- 0.1
#Now define k, which depends on t and x:
k = t/q.sqrt(x) # use the QExpy version of sqrt() since x is of type Measurement
#Print the result:
print(k)
\end{python}
\begin{poutput}
0.44 +/- 0.09
\end{poutput}
which is the result that we obtained when manually applying the derivative method. Note that we used the square root function from the QExpy module, as it ``knows'' how to take the square root of a value with uncertainty (a ``Measurement'' in the language of QExpy). 

We also saw that when we had repeated measurements of the same quantity (Section \ref{sec:c2:determiningerrors}), one could define a central value and uncertainty for that quantity by using the mean and standard deviations of the measurements. QExpy can easily take a set of measurements (an array of values) and convert them into a single quantity (a ``Measurement'') with a central value and uncertainty that correspond to the mean and standard deviation of the set of measurements:

\begin{python}[caption=QExpy to calculate mean and standard deviation] 
#First, we load the QExpy module
import qexpy as q
#We define $t$ as an array of values (note the square brackets):
t = q.Measurement([1.01,  0.76,  0.64,  0.73,  0.66])
#Choose the number of significant figures to print:
q.set_sigfigs(2)
#Print the result:
print("t = ",t)
\end{python}
\begin{poutput}
t = 0.76 +/- 0.15
\end{poutput}
By using QExpy, we do not need to tediously calculate the mean and standard deviation, as we had in Example \ref{ex:chap2:stdcalc}.


\subsection{Plotting experimental data with uncertainties}
In Chapter \ref{chap:2_ModelAndExperiment} we had presented the data in Table \ref{tab:appPython:kmes} which corresponded to our measurements of how long it took ($t$) for an object to drop a certain distance, $x$. We had also introduced  Chlo\"e's Theory of gravity that predicted that the data should be described by the following model:
\begin{align*}
t = k \sqrt{x}
\end{align*}
where $k$ was an undetermined constant of proportionality.

\begin{table}[!h]
\centering
\begin{tabular}{cccc} 
\textbf{x} [m]&\textbf{t} [s]&\textbf{$\sqrt x$}  [\si{m^{\frac{1}{2}}}]&\textbf{k}  [\si{s.m^{-\frac{1}{2}}}]\\
\hline
\hline
1.00 &0.33 &1.00 &0.33 \\ \hline
2.00 &0.74 &1.41 &0.52 \\ \hline
3.00 &0.67 &1.73 &0.39 \\ \hline
4.00 &1.07 &2.00 &0.54 \\ \hline
5.00 &1.10 &2.24 &0.49 \\ \hline
\end{tabular}
\caption{\label{tab:appPython:kmes} Measurements of the drop times, $t$, for a bowling ball to fall different distances, $x$. We have also computed $\sqrt x$ and the corresponding value of $k$. }
\end{table}

The easiest way to visualize and analyse those data is to plot them. In particular, if we plot (graph) $t$ versus $\sqrt{x}$, we  expect that the points will fall on a straight line that goes through zero, with a slope of $k$ (if the data are described by Chlo\"e's Theory). We can use QExpy to graph the data as well as determine (``fit'') for the slope of the line that best describes the data, since we expect that the slope will correspond to the value of $k$. When plotting data and fitting them to a line (or other function), it is important to make sure that the values have at least an uncertainty in the quantity that is being plotted on the $y$ axis. In this case, we have assumed that all of the measurements of time have an uncertainty of $\SI{0.15}{s}$ and that the measurements of the distance have no (or negligible) uncertainties:

\begin{python}[caption=Using QExPy to plot and fit linear data]
#First, we load the QExpy module:
import qexpy as q

#Use matplotlib as the plot engine (try using 'bokeh' instead of 'mpl')
q.plot_engine = 'mpl'

#Then we enter the data in arrays for the x and y axes.
#The values for the square root of height (x axis):
sqx = [1. , 1.41, 1.73, 2., 2.24]
#and then, the corresponding times (y-axis):
t = [ 0.33,  0.74,  0.67,  1.07,  1.1 ]

#Let us attribute an uncertainty of 0.15 to each measured values of t:
terr = 0.15

#We now make the plot. First, we create the plot object with the data.
#Note that x and y refer to the x and y axes
fig = q.MakePlot( xdata = sqx, xname = "sqrt(distance) [m^0.5]",
                  ydata = t, yerr = terr, yname = "time [s]",
                  data_name = "My data")
                  
#Ask QExpy to also determine the line of best fit                  
fig.fit("linear")
                  
#Then, we show it:
fig.show()          
\end{python}
\begin{poutput}
-----------------Fit results-------------------
Fit of  My data  to  linear
Fit parameters:
My data_linear_fit0_fitpars_intercept = -0.2 +/- 0.2,
My data_linear_fit0_fitpars_slope = 0.6 +/- 0.1

Correlation matrix: 
[[ 1.    -0.968]
 [-0.968  1.   ]]

chi2/ndof = 2.04/2
---------------End fit results----------------
(* \capfig{0.75\textwidth}{figures/AppendixC/tvssqx.png}{\label{fig:appPython:tvssqx} QExpy plot of $t$ versus $\sqrt{x}$ and line of best fit.} *)
\end{poutput}
The plot in Figure \ref{fig:appPython:tvssqx} shows that the data points are consistent with falling on a straight line, when their error bars are taken into account. We've also asked QExpy to show us the line of best fit to the data, represented by the line with the shaded area. When we asked for the line of best fit, QExpy not only drew the line, but also gave us the values and uncertainties for the slope and the intercept of the line. The shaded area around the line corresponds to other possible lines that one would obtain using different values of the slope and intercept within their corresponding uncertainties. The output also provides a line that tells us that \code{chi2/ndof = 2.04/2}; although you do not need to understand the details, this is a measure of how well the data are described by the line of best fit. Generally, the fit is assumed to be ``good'' if this ratio is close to 1 (the ratio is called ``the reduced chi-squared'').  The ``correlation matrix'' tells us how the best fit value of the slope is linked to the best fit value of the intercept, which you do not need to worry about here.


Since we expect the slope of the data to be $k$, this provides us a method to determine $k$ from the data as \SI{0.61\pm 0.13}{s.m^{-\frac{1}{2}}}. \textbf{Performing a linear fit of the data is the best way to determine a constant of proportionality between the measurements}. Finally, we expect the intercept to be equal to zero according to our model. The best fit line from QExpy has an intercept of \SI{-0.24\pm 0.22}{s}, which is slightly below, but consistent, with zero. From these data, we would conclude that our measurements are consistent with Chlo\"e's Theory. Again, remember that we can never confirm a theory, we can only exclude it; in this case, we cannot exclude Chlo\"e's Theory.

\section{Advanced topics}
This section introduces a few more advanced topics that allow you to use computer programming to simplifying many tasks. In this section, we will show you how you can write your own program to numerically estimate the value of an integral of any function.
\subsection{Defining your own functions}
Although Python provides many modules and functions, it is often useful to be able to define your own functions. For example, suppose that you would like to define a function that calculates $\frac{1}{3}x^2+\frac{1}{4}x^3+\cos(2x)$, for a given value of $x$. This is done easily using the \code{def} keyword in Python:

\begin{python}[caption=Defining a function] 
#import the math module in order to use cos
import math as m

#define our function and call it myfunction:
def myfunction(x):
  return x**2 / 3 + x**3 / 4 + m.cos(2*x)
  
#Test our function by printing out the result of evaluating it at x = 3
print( myfunction(3) )  
\end{python}
\begin{poutput}
10.710170286650365
\end{poutput}
A few things to note about the code above:
\begin{itemize}
\item Functions are defined using the \code{def} keyword followed by the name that we choose for the function (in our case, \code{myfunction})
\item If functions take arguments, those are specified in parenthesis after the name of the function (in our case, we have one argument that we chose to call \code{x})
\item After the name of the function and the arguments, we place a colon
\item The code that belongs to the function, after the colon, must be indented (this allows Python to know where the code for the function ends)
\item The function can ``return'' a value; this is done by using the \code{return} keyword. 
\item We used the ``operator'' \code{**} to take the power of a number (\code{x**2}), and the operator \code{*}, to multiply numbers. In particular, Python would not understand \code{2x} which needs to explicitly have the multiplication operator, \code{2*x} (inside of the cosine function).
\end{itemize}
In the example above, we wrote a Python function to represent a mathematical function. However, one can write a function to execute any set of tasks, not just to apply a mathematical function. Python functions are very useful in order to avoid having to repeatedly type the same code. 

Recall that the numpy module allows us to apply functions to arrays of numbers, instead of a single number. We can modify the code above slight so that, if the argument to the function, \code{x}, is an array, the function will gracefully return an array of numbers to which the function has been applied. This is done by simply replacing the call to the \code{math} version of the \code{cos} function by using the \code{numpy} version:
\begin{python}[caption=Defining a function that works on an array] 
#import the numpy module in order to use cos to an array
import numpy as np

#define our function and call it myfunction:
def myfunction(x):
  return x**2 / 3 + x**3 / 4 + np.cos(2*x)
  
#Test our function by printing out the result of evaluating it at x = 3 (same as before)
print( myfunction(3) )  

#Test it with an array
xvals = np.array([1,2,3])
print ( myfunction(xvals) )  

\end{python}
\begin{poutput}
10.710170286650365
[ 0.1671865   2.67968971 10.71017029]
\end{poutput}
where we created the array \code{xvals} using the \code{numpy} module.

\subsection{Using a loop to calculate an integral}
The ability to define our own functions in Python allows us to easily simplify complex tasks. Using ``loops'' is another way that computer programming can greatly simplify calculations that would otherwise be very tedious. In a loop, one is able to repeat the same task many times. The example below simply prints out a statement five times:
\begin{python}[caption=A simple loop] 
#A loop to print out a statement 5 times:

for i in range(5):
  print("The value of i is ",i)
\end{python}
\begin{poutput}
The value of i is  0
The value of i is  1
The value of i is  2
The value of i is  3
The value of i is  4
\end{poutput}
A few notes on the code above:
\begin{itemize}
\item The loop is defined by using the keywords \code{for ... in}
\item The value after the keyword \code{for} is the ``iterator'' variable and will have a different value each time that the code inside of the loop is run (in our case, we called the variable \code{i})
\item The value after the keyword \code{in} is an array of values that the iterator will take
\item The \code{range(N)} function returns an array of \code{N} integer value between 0 and \code{N-1} (in our case, this returns the five values 0,1,2,3,4)
\item The code to be executed at each ``iteration'' of the loop is preceded by a colon and indented (in the same way as the code for a function also follows a colon and is indented)
\end{itemize}
We now have all of the tools to evaluate an integral numerically. Recall that the integral of the function $f(x)$ between $x_a$ and $x_b$ is simply a sum:
\begin{align*}
\int_{x_a}^{x_b} f(x) dx&=\lim_{\Delta x \to 0} \sum_{i=0}^{i=N-1} f(x_{i})\Delta x\\
\Delta x &= \frac{x_b-x_a}{N}\\
x_i&=x_a+i\Delta x\\
\end{align*}
The limit of $\Delta x \to 0$ is thus equivalent to the limit $N \to \infty$. Our strategy for evaluating the integral is thus:
\begin{enumerate}
\item Define a Python function for $f(x)$
\item Create an array, \code{xvals}, of $N$ values of $x$ between $x_a$ and $x_b$
\item Evaluate the function for all those values and store those into an array, \code{fvals}
\item Loop over all of the values in the array \code{fvals}, multiply them by $\Delta x$, and sum them together.
\end{enumerate}
Let us thus use Python to evaluate the integral of the function $f(x)=4x^3+3x^2+5$ between $x=1$ and $x=5$:
\begin{python}[caption=Numerical integration of a function] 
#import numpy to work with arrays:
import numpy as np

#define our function
def f(x):
  return 4*x**3 + 3*x**2 + 5
  
#Make N and the range of integration variables:
N = 1000
xmin = 1
xmax = 5

#create the array of values of x between xmin and xmax
xvals = np.linspace(xmin, xmax, N)

#evaluate the function at all those values of x
fvals = f(xvals)

#calculate delta x
deltax = (xmax - xmin) / N

#initialize the sum to be zero:
sum = 0

#loop over the values fvals and add them to the sum
for fi in fvals:
  sum = sum + fi*deltax

#print the result:
print("The integral between {} and {} using {} steps is {:.2f} ".format(xmin, xmax, N, sum))

\end{python}
\begin{poutput}
The integral between 1 and 5 using 1000 steps is 768.42 
\end{poutput}
One can easily integrate the above function analytically and obtain the exact result of $\num{768}$. The numerical answer will approach the exact answer as we make $N$ bigger. Of course, the power of numerical integration is to use it when the function cannot be integrated analytically.

\begin{checkpointSA}{What value of $N$ should you use above in order to get within $\num{0.01}$ of the exact analytic answer?}
\end{checkpointSA}