\chapter{Guidelines for lab related activities}
\label{chapter:labs}


\begin{learningObjectives}{
 \item Learn to design an experiment to test a model with small uncertainties
 \item Practice modelling physical situations, testing the model and reviewing the model based on experimental outcomes
 \item Develop skills in general scientific writing 
 \item Learn to write scientific proposals and experimental reports
 \item Learn to review others' scientific proposals and experimental reports
 \item Practice evaluating one's own and others' experimental results
 }
\end{learningObjectives}


Experimental physics is crucial to the development and modification of theories that give us the most accurate descriptions of nature. If a physicist proposes a new description of nature, it must be tested before it is accepted to be a good description of nature, or a ``good theory''. In order for this description to be tested, oftentimes, the physicist must submit a proposal and summary of their experiment to a governing scientific body in order to obtain funding. 

Proposal and experimental report formats vary based on the type of experiment, funding source, and region of the world in which the experiment is being conducted. Despite this, proposals and experimental reports must contain certain key elements so that enough information can be conveyed to the reader so that they could test the experiment themselves. It must also be clear, concise and professional. 

The following guidelines and tips are rough outlines of the most important information to think about when writing a scientific proposal and/or experimental report. These guidelines are designed to help beginner physicists to think about scientific writing in physics. 

In addition to this, guidelines for critiquing others' proposals have also been included. In science, it is a crucial skill to be able to critically appraise scientific articles and experiments in order to advance one's own knowledge in accordance with the scientific method. Critiquing others' reports is also a tool to improve your own writing and exercise your critical thinking abilities.  

 \vspace{0.25cm}
\section{Scientific Writing}

Here are a few tips that will help you to polish your lab reports. %add more here, build on these points maybe?

\begin{itemize}
\item Subjective/imprecise terms: When writing scientifically, it is important to avoid using subjective and imprecise terms to describe any part of your experiment/results. In science, reproducibility is of utmost importance - to be as objective as possible is the goal. For example, if you stated that ``our calculated value of g was much greater than the expected value'' - here it is sufficient to say ``our calculated value of g was greater than the expected value''. Your opinion of ``how much'' it was greater shouldn't be reflected in your writing.
\item Definitive statements: it is important to avoid attributing definitive causes to your experimental outcomes in the lab. You can believe as devoutly as you want that your results were ``correct'', but your result/theory can never be ``proven'' to be correct. For example, instead of saying ``as the data exhibit, we have detected the Purple Particle'', it would be better to state that ``as the data exhibit, we have strong evidence to support our observation of the Purple Particle''. 
\item Overstatements: Overstating your findings or the importance of your experiment should generally be avoided in scientific writing, especially in proposals. For example, it might be an overstatement to describe your sliding-block experiment to measure the coefficient of kinetic friction as a ``unique and novel approach''. If many variations of your experiment already exist and the experiment is not ground-breaking in the field, it should not be stated as so.
\item Active vs. passive voice: when writing scientific papers, it is recommended to use the third person, passive voice. For example, this would mean saying ``the drop time for balls at various heights was measured'' rather than ``we measured the drop time for balls at various heights''. Generally, both passive and active voice are acceptable in scientific writing, as long as you keep it consistent throughout the text.
\end{itemize}

Grammar tips: %TODO: add more here
\begin{itemize}
\item ``Data'' is the plural of ``datum''. ``The data shows'' is thus wrong, even if it is colloquially used. The correct form is ``The data show'', or, ``These data show''.
\item Future tense vs. past tense: for all experiment proposals, you should write in the future tense. For experimental reports, you should write in the past tense. 
\end{itemize}

\begin{studentOpinion}{Emma}
\textbf{Writing and editing - how can I be more concise?}

We've all felt that our writing was lacking at some point or another. Here are some general tips to avoid overall ``wordiness'' and to increase ease of reading when writing about science: 

\begin{itemize}
\item What would you want to read?: Let's say that you wanted to know the strength of Earth's magnetic field, and how it was found, so you decide to do a literature search. Would you choose a brief, succinct article, or a wordy Magnetic Field Manifesto?
\item The kindergarten test: If you had to explain your concept to a six year old cousin, how would you break it down in a way that they could understand it? If you can't break it down enough to explain to a six year old, perhaps you need to revisit your own understanding of the concept.
\item Avoid unnecessary adjectives: while this might be ok in a creative writing class, in scientific writing, the goal is to get your point across as succinctly as possible. Using ``big'' words might be ok (as long as they properly describe what you are trying to say), but it is important to communicate your message in the simplest manner. 
\item Think about it: every time you use a comma, dash or even an ``and'', you should reconsider the brevity of your statement. In scientific writing, commas are carefully placed, and semicolons are rare. 
\item Cut it in half: For every word you read, think of another that you can cut. For every sentence that you read, think of three sentences that communicate the same idea. Pick the sentence that is the shortest and most concise. 
\item Proofread - the more, the better.
\end{itemize}
\end{studentOpinion}

\newpage
\section{Proposal and Lab Report Guidelines}

This section is intended to provide the framework for concise, professional, scientific writing. The following guidelines give the basic outline for a proposal and lab report, as well as the evaluation rubric for each. Additionally, samples of these four outlines for the experiment ``Measuring g using a pendulum'' are provided below. In the sample proposal and lab report, errors are purposefully included and addressed in the evaluations. It is important to entirely read the rest of this section to capture the common proposal/lab errors and their corresponding corrections.

\newpage
\section{Guide for writing a proposal}
 \vspace{0.25cm}
\textbf{Summary and Goal}

Write a few short sentences briefly summarizing the aim of your experiment, how you will do it and what you hope to find. Additionally, make sure to discuss feasibility and chances of success. It may be easiest to write this section last.

\textbf{Materials and Methods}

Provide a list of your materials, and clearly describe the relevant steps of the method/procedure that you will use to carry out your experiment. Also, propose a method of assessing whether or not your project was successful. 

Consider the following questions:
\begin{itemize}
\item What materials, equipment and/or tools are necessary in making your measurements?
\item What are the cost of these materials? Can they be easily obtained?
\item Where should this experiment be conducted?
\item How will you make your measurements? How many times will you make them?
\item How will you record your measurements?
\item How will you maximize the precision of your experiments?
\item How will you estimate uncertainties?
\item What issues could arise in your experiment? How do you plan to resolve these issues?
\end{itemize}

\textbf{Timeline}

Provide a timeline of your proposed plan. Describe what you will do each day/week/year. Make milestones that you hope to accomplish by set dates. If there is background theory that you need to learn, by what date will you learn it? If you need to order materials, by what date should you receive them?

\textbf{Team}

Provide the names of team members, and assign relevant duties to each member. 

\newpage
\section{Guide for reviewing a proposal}
 \vspace{0.25cm}
\textbf{Summary and Goal}

Summarize your overall evaluation of the proposal in 2-3 sentences. Focus on the experiment's methods and goals. For example, "The authors wish to drop balls from different heights to determine the value of g". You don't need to go into the specific details, just give a high level summary of the report. If the report is unclear, specify this.

Consider the following questions:
\begin{itemize}
\item Is the proposed experiment well thought out, and detailed?
\item Is the experimental procedure clear and concise?
\item Does the experimental design minimize uncertainties?
\item Is the experiment feasible? 
\item Is it possible to complete the experiment in a reasonable period of time?
\item Are the goals of the project clear and reasonable?
\end{itemize}

\textbf{Materials and Methods}

Assess the proposal's methods, and determine whether the project is well-organized and incorporates relative methods for self-evaluation.

Consider the following questions:
\begin{itemize}
\item Is the plan for carrying out the experiment well-reasoned, well-organized, and based on a sound rationale?
\item Is it possible to obtain the equipment/materials in a reasonable amount of time?
\item Are adequate resources available to the PI to conduct the experiment?
\item Does the plan incorporate a mechanism to assess success?
\item Is a troubleshooting plan in place, in case of unexpected difficulties?
\end{itemize}

\textbf{Team}

Assess whether the team members/organization are qualified to conduct the experiment.

\textbf{Overall Rating of the Experiment}

Give the experiment an overall score, based on the criteria described above.

Was the experiment...
\begin{itemize}
\item Excellent
\item Good
\item Satisfactory
\item Needs work
\item Incomplete
\end{itemize}

\newpage
\section{Guide for writing a lab report}
 \vspace{0.25cm}
\textbf{Abstract}

Write a few short sentences briefly summarizing what you did, how you did it, what you found and what went wrong in your experiment.

\textbf{Procedure}

Describe relevant theories that relate to your experiment here, and the steps to carry out your procedure. 

Consider the following questions:
\begin{itemize}
\item What are the relevant theories/principles that you used? 
\item What equations did you need? Include all necessary/relevant derivations.
\item What materials, equipment and/or tools were necessary in making your measurements?
\item Where was this experiment conducted?
\item How did you make your measurements? How many times did you make them?
\item How did you record your measurements?
\item How did you maximize the precision of your experiments?
\item How did you determine uncertainties?
\end{itemize}

\textbf{Predictions}

Because the scientific method is iterative, you want to try and predict what you will measure. After your measurement, you should examine whether you should improve on your model or if you discovered a discrepancy between theory and experiment (unlikely in introductory physics!).

Consider the following questions:
\begin{itemize}
\item Predict your measured values and uncertainties. How precise do you expect your measurements to be?
\item What assumptions did you have to make to predict your results?
\item Have these predictions influenced how you should approach your procedure? Make relevant adjustments to the procedure based on your predictions.
\end{itemize}

\textbf{Data and Analysis}

Present your data. Include relevant tables/graphs. Describe in detail how you analysed the data, including how you propagated uncertainties. Iterate on the model - is it consistent with your initial predictions?

Consider the following questions:
\begin{itemize}
\item How did you obtain the ``final'' measurement/value from your collected data?
\item What is the relative uncertainty on your value(s)?
\end{itemize}

\textbf{Discussion and Conclusion}

Summarize your findings, and address whether or not your model described the data. Discuss possible sources of error that could have influenced your results.

Consider the following questions:
\begin{itemize}
\item Were there any systematic errors that you didn't consider?
\item Did you learn anything that you didn't previously know? (eg. about the subject of your experiment, about the scientific method in general)
\item If you could redo this experiment, what would you change (if anything)?
\end{itemize}

\newpage
\subsection{Guide for reviewing a lab report}
 \vspace{0.25cm}
\textbf{Summary}

Summarize your overall evaluation of the experiment in 2-3 sentences. Focus on the experiment's method and its result. For example, "The authors dropped balls from different heights to determine the value of g". You don't need to go into the specific details, just give a high level summary of the report. If the report is unclear, specify this.

Consider the following questions:
\begin{itemize}
\item Was the experiment well thought out, and detailed?
\item Was the experimental procedure clear and concise?
\item Did the experimental design minimize uncertainties?
\end{itemize}


\textbf{Procedure}

Consider the following questions:
\begin{itemize}
\item Is the the procedure clearly and concisely described? 
\item Do you have sufficient information to repeat this experiment?
\item Is it clear which tools are needed to measure which quantities?
\item Is it specified where the experiment should take place?
\item Are clear safety precautions described?
\item Does the experiment clearly state how uncertainties were propagated and determined?
\end{itemize}

\textbf{Feasibility}

In a few sentences, describe the experiment in terms of feasibility and chances of success. 

\begin{itemize}
\item Was the experimental procedure sound?
\item Were the materials, equipment and/or tools accessible? 
\item Were the materials, equipment and/or tools appropriately chosen?
\item Was it possible to complete the experiment in a reasonable period of time?
\end{itemize}


\textbf{Overall Rating of the Experiment}

Give the experiment an overall score, based on the criteria described above.

Was the experiment...
\begin{itemize}
\item Excellent
\item Good
\item Satisfactory
\item Needs work
\item Incomplete
\end{itemize}

\newpage
\section{Sample Proposal (Measuring g using a pendulum)}
 \vspace{0.25cm}
\textbf{Summary and Goal}

One can measure the gravitational constant by measuring the period of a pendulum of a known length, requiring only a string, mass, ruler and timer. Because the experimental design can be easily adjusted and the experiment is simple, the experiment has a high chance of success.

\textbf{Materials and Methods}

The period of a pendulum of length L is easily shown to be given by:

\begin{align*}
T=2\pi \sqrt {\frac{L}{g}}
\end{align*}

Thus, by measuring the period of a pendulum as well as its length, one can determine the value of g:

\begin{align*}
g=\frac{4\pi^{2}L}{T^{2}}
\end{align*}

One can build a simple pendulum using the following materials:
\begin{itemize}
\item a mass
\item inextensible string
\item 2 metre sticks
\item stand to attach string
\item timer
\item tape
\end{itemize}

The materials listed above are all inexpensive and can be easily obtained.  It is recommended that the experiment be completed indoors at room temperature, in order to minimize any environmental effects. 

\textbf{Procedure}

1. Construction of the pendulum

One must first tie the inextensible string to the mass. Attaching the string-mass system to the stand, one must measure the distance between the end of the string on the ``mass side'' and the ``stand side'' to be exactly 1$\si{m}$ (or as close as possible to 1$\si{m}$. This is considered to be the ``length of the pendulum''. A friend should hold a ruler parallel to the string, and the string must be pulled taught when measuring 1$\si{m}$. This may take some trial and error. The mass, string and stand should be attached with knots rather than tape. The string can be secured to the stand using tape.

Next, the ruler stick used to measure the string should be attached to the top of the stand with tape. At this point, the second ruler should be used to ensure that the ruler is exactly in line with the base of the stand. After the ruler is taped to the stand, the system should be sturdy and the ruler should not wiggle on top of the stand.

2. Measurement of the period

The goal of the first part of the experiment is to measure the period of the pendulum after 20 oscillations. For each trial, the pendulum should be released from \SI{90}{\degree}. The string should be pulled taught, and the team member's eye that is measuring the angle should be situated parallel to the measuring device. The team member in charge of taking the video will start the video shortly before the pendulum is released. After releasing the pendulum, the team should measure 20 oscillations before stopping the pendulum and the video. Data from the video should be entered into the Jupiter Notebook. It is recommended that this measurement is repeated at least 5 times.

The precision of the measurement of the period will be maximized by measuring the period over 20 oscillations and dividing by 20, as this is more accurate than measuring one oscillation. The more that the measurement is repeated, the greater the precision of the experiment will be. Additionally, a slow-motion video will be taken of the pendulum to track the time of the oscillation in order to minimize error due to reaction time. 

3. Minimizing and estimating uncertainties

To propagate uncertainties, the uncertainty in the timer and ruler will be taken and half of the smallest division will be used. Foreseeable issues in this experiment may arise when trying to find a string that is optimally inextensible, as any extensibility will cause error in the results. Additionally, being able to measure exactly \SI{90}{\degree} as the drop-angle for the pendulum could be difficult. In order to correct for this, the team member who is dropping the pendulum must stand directly parallel to the measuring device, minimizing parallax error. 

The measure of success will be determined by the uncertainty and precision of the measured value of g. If the measured value of g has a relative uncertainty that is less than 10 \%, and consistent with the accepted value, then one can consider the experiment to have been carried out successfully. 

\textbf{Timeline}

One should be able to complete the experiment and analysis in approximately 1 hour and 30 minutes with the data being collected in the first 30 minutes. The remainder of the time should be spent processing the data and writing the experimental report.  

\textbf{Team}

Following the strengths of the members of the team, the following people should be responsible for leading the following tasks, while everyone participates:

\begin{itemize}
\item Group member 1: building the pendulum
\item Group member 2: taking the measurements
\item Group member 3: analysing the data
\item Group member 4: writing and formatting
\end{itemize}

 \vspace{0.25cm}
\section{Sample Reviewed Proposal (Measuring g using a pendulum)}
 \vspace{0.25cm}
\textbf{Summary and Goal}

The authors wish to measure the value of g by measuring the period of a simple pendulum, using the SHM equations and theory. Overall, this proposal is clearly explained, but is lacking certain critical details. In the theoretical section, the small angle approximation is missing. The experimental design explicitly states methods for minimizing error and uncertainties (such as the mechanism for minimizing parallax error), but could be improved. The experiment can be conducted in the specified time period, and the timeline is reasonable. The goals of the proposal are clear and realistic for the given time frame. 

\textbf{Materials and Methods}

The experimental methods are described clearly and succinctly, with most information clearly stated. For the materials list, it is stated that "a mass" must be used. Here, it should be stated that a solid, non-deformable mass should be used to minimize drag. Additionally, it is described that the pendulum string should be exactly 1$\si{m}$ long. Realistically, here it should be reasoned that the pendulum string should be a little bit smaller than 1$\si{m}$, as the measuring device (metre stick) is 1$\si{m}$, and the object being measured should never be at the limit of the measuring device. 

Most equations are described in the theory section, but it is incorrectly assumed that the period of a pendulum is independent of the drop angle for all angles. The small angle approximation is necessary for a pendulum dropped from \SI{90}{\degree}. When describing the ``construction of the pendulum'', it is described that ``the second ruler should be used to ensure that the ruler is exactly in line with the base of the stand''. Here, the accuracy of this measurement could be increased by using a square rather than a second ruler. Rather than relying on the perfect alignment of each ruler to measure the \SI{90}{\degree} angle, a square will already have the perfect \SI{90}{\degree} angle. It is stated that the pendulum will be dropped from \SI{90}{\degree} for each measurement. Varying the angle rather than keeping it constant could provide a more accurate measurement of the pendulum's period. No justification is provided for the use of 20 oscillations prior to measuring the period - it may be necessary to iterate on the reason why 20 oscillations was chosen. Additionally, in the ``measurement of the period'' section, the actual measurement of the period is described, but it is not mentioned that this data should be ``recorded''. It is important to state that data will be recorded when describing an experimental procedure. The equipment can be easily obtained and is fairly inexpensive. Adequate resources are available to the group to perform this experiment. A clear troubleshooting plan is described and a method for evaluating success is included. 

\textbf{Team}

This experiment is fairly simple and the equipment/setup is not difficult to handle. A team consisting of first year physics students should be qualified to perform this experiment.

\textbf{Overall Rating of the Proposal}

Good - this proposal was clearly explained and is scientifically sound. It was succinctly written, and most components of the experiment were clearly described. In the theory section, the small angle approximation should have been included when describing the SHM equations. The drop-angle should be varied rather than kept constant. A little more detail in the justification for using 20 oscillations is necessary. If these details were taken into account, the report would have been excellent.

\textbf{Notes on Scientific Writing}

Overall, the scientific writing in this proposal was satisfactory. The writing was clear and concise, but there were certain instances of unprofessional language. For example:
\begin{itemize}
\item In the Procedure section under ``Construction of the pendulum'', it is described that ``a friend should hold a ruler parallel to the string''. In scientific writing, team members should be referred to as ``colleagues''.
\item 
\end{itemize}

 \vspace{0.25cm}
\section{Sample Lab Report (Measuring g using a pendulum)}
 \vspace{0.25cm}
\textbf{Abstract}

In this experiment, the goal was to measure the gravitational constant by measuring the period of a pendulum of a known length. Using a string, mass, ruler and timer, the oscillations of the pendulum were measured and it was found that the gravitational constant was $\SI{7.65}+/- {0.378}{m/s^2}$. It was found that the calculated value of g had a difference of 22\% between the measured and accepted value.

\textbf{Theory}

The pendulum setup that was used exhibits simple harmonic motion (SHM), which allowed us to measure the gravitational constant by measuring the period of the pendulum. The period of a pendulum of length L can be determined with the following formula, in accordance with SHM:

\begin{align*}
T=2\pi \sqrt {\frac{L}{g}}
\end{align*}

Thus, by measuring the period of a pendulum as well as its length, one can determine the value of g:

\begin{align*}
g=\frac{4\pi^{2}L}{T^{2}}
\end{align*}

It is assumed that the frequency and period of pendulums depend on the length of the pendulum string, rather than the angle from which it was dropped. 

\textbf{Predictions}

The 1$\si{m}$ long string was measured with a ruler, which has a precision of 1$\si{mm}$. With the phone video camera and timer, it is assumed that time can be measured to within 0.5 seconds. Adding these uncertainties in quadrature, it was obtained that we can measure the gravitational constant with a relative uncertainty of 5\%. Therefore, the value we obtain for the gravitational constant should be $\SI{9.80}+/- {0.5}{m/s^2}$. 

%%TODO: I wasn’t sure exactly how to calculate these predictions, so ignore these numbers. I subbed random values in here%%

\textbf{Procedure}

1. Construction of the pendulum

To conduct the experiment, a mass, 1$\si{m}$ inextensible string, 1$\si{m}$ ruler, pendulum stand and timer was needed. First, the inextensible string was tied to the mass. After attaching the string-mass system to the stand, the distance between the end of the string on the ``mass side'' and the ``stand side'' was measured to be exactly 1$\si{m}$ (or as close as possible to 1$\si{m}$). This is considered to be the ``length of the pendulum''. A team member held the ruler parallel to the string, and the string was pulled taught when measuring 1$\si{m}$. The mass, string and stand were attached together with knots rather than tape, but tape was used to secure the string to the stand.

Next, the ruler stick that was used to measure the string was attached to the top of the stand with tape. At this point, the second ruler was used to ensure that the ruler was exactly in line with the base of the stand.

2. Measurement of the period

The experiment was conducted in a laboratory indoors. The pendulum was released from \SI{90}{\degree} and its period was measured. The system was pulled taught, and the team member's eye that was measuring the angle was parallel to the measuring device. The team member in charge of taking the video started the video shortly before the pendulum was released. After releasing the pendulum, the team measured 20 oscillations before stopping the pendulum and the video. Data from the video was entered into the Jupiter Notebook. This was repeated 5 times.

The precision of the measurement of the period was maximized by measuring the period over 20 oscillations and dividing by 20, rather than simply measuring one oscillation or 20 separate oscillations. This way, the uncertainty in the time was decreased as the timer only needs to be started and stopped once, decreasing parallax error. The slow motion video minimized the remaining reaction time measurement error. To estimate uncertainties, the uncertainty of the timer and ruler was taken and half of the smallest division was used for each.

\textbf{Data and Analysis}

Using an 100 g mass and 1.0 m ruler stick, the period of 20 oscillations was measured over 5 trials. The corresponding value of g for each of these periods was calculated. The following data for each trial and obtained value of g is provided below. 

\begin{table}[H]
\begin{tabular}{|l|l|l|l|l}
\cline{1-4}
Trial & Angle (Degrees) & Measured Period (s) & Calculated value of g ($m/s^2$) &  \\ \cline{1-4}
1     & 90              & 2.24                & 7.87                         &  \\ \cline{1-4}
2     & 90              & 2.37                & 7.03                         &  \\ \cline{1-4}
3     & 90              & 2.28                & 7.59                         &  \\ \cline{1-4}
4     & 90              & 2.26                & 7.73                         &  \\ \cline{1-4}
5     & 90              & 2.22                & 8.01                         &  \\ \cline{1-4}
\end{tabular}
\end{table}

It was found that the final calculated value of g was $\SI{7.65}+/- {0.378}{m/s^2}$. This was calculated using the mean of the calculated values of g and the standard deviation. The percent uncertainty on the calculated value of g was found to be 4.94\% and the percent error between the calculated and accepted value was found to be 22\%.

\textbf{Discussion and Conclusion}

In this experiment, it was found that the gravitational constant for Earth was $\SI{7.65}+/- {0.378}{m/s^2}$. The value of g that was calculated from the period had a difference of 22\% between the measured and accepted value, which was most likely attributed to the large number of oscillations used when measuring the period.

It is proposed that the measurement of 20 oscillations was the largest source of error. After approximately 20 oscillations, the period started to slow down due to air friction and drag. This could have altered the measurement of the actual period, making the calculation of g slightly higher than expected. Additionally, a protractor was not used to align the ruler with the pendulum before releasing it. Due of this, the angle could have been slightly more or less than \SI{90}{\degree}, making the period of the pendulum slightly larger or smaller. 

If this experiment could be redone, measuring 10 oscillations of the pendulum rather than 20 oscillations before recording the measurement could provide a more precise value of g. Additionally, a protractor would be taped to the top of the pendulum stand, with the ruler taped to the protractor. This way, the pendulum could be dropped from a near-perfect \SI{90}{\degree} rather than a rough estimate of \SI{90}{\degree}. 

 \vspace{0.25cm}
\section{Sample Reviewed Lab Report (Measuring g using a pendulum)}
 \vspace{0.25cm}
\textbf{Summary}

In this experiment, the period of oscillation of a simple pendulum was measured in order to indirectly measure the gravitational constant, g. It was found that g was $\SI{7.65}+/- {0.378}{m/s^2}$ and the experiment was not considered to be a success. The experimental procedure was well-written, and the information that was included was relevant. The theory section and the corresponding calculations needed improvement. Measures to minimize uncertainties were clearly described, but could have been improved. Some missing explanations decreased the clarity of the report.

\textbf{Procedure}

The experimental procedure was clearly written. If needed, it would be easy to reproduce this experiment with the given instructions, but improvement was needed in certain aspects. Specific tools and their uses were described. Uncertainties were minimized by performing the experiment inside, minimizing environmental disturbances. It was stated in the end of the report that a protractor would have minimized the error in the \SI{90}{\degree} angle of the pendulum. This is valid, but the experiment could have been improved if the period was measured at various angles, with a protractor. Additionally, it was described that a significant source of error could have been attributed to the slowing of the pendulum after 20 oscillations. The number 20 seems to have been chosen arbitrarily, so a more thorough initial experimental design could have improved the results. It is stated that the pendulum string should be exactly 1$\si{m}$ long. Rather,  it should be reasoned that the pendulum string should be a little bit smaller than 1$\si{m}$, as the measuring device (metre stick) is 1$\si{m}$, and the object being measured should never be at the limit of the measuring device. 

The theory section of the report needed improvement - the formula used to calculate gravity was given, but the small angle approximation was entirely missing. It was not clear that the small angle approximation was even used in the calculations. This could have accounted for much of the error in the results. Otherwise, clear safety precautions were described and methods for propagating error and uncertainty were included.

\textbf{Feasibility}

The experimental procedure was scientifically sound, other than the choice of the number of pendulum oscillations measured and recorded. Many precautions were taken to minimize error, and many were suggested afterwords as a reflection. If the small angle approximation had been used and measurements were taken from various angles, the results could have been very close to the accepted value. The materials were appropriately chosen and would have been optimal with the addition of a protractor. The time period was reasonable for completion of the experiment. It appears that the members finished the experiment and report with time to spare. 

\textbf{Overall rating of the Experiment}

Satisfactory - although the experiment was very well-written and detailed, the small angle approximation was missing from the calculations and theory section entirely. There was no apparent justification for using 20 oscillations before measuring the period and the angle from which the pendulum was dropped was not varied. A protractor could have been used to minimize the uncertainty in the drop-angle. If these factors were taken into account, this could have been an excellent experiment and the results may have been very close to the accepted value of g.

\textbf{Notes on Scientific Writing}

Overall, the scientific writing in this proposal was satisfactory. The writing was clear and concise, but there were certain instances of unprofessional language. For example:
\begin{itemize}
\item In the Theory section, the plural of ``pendulum'' is referred to as ``pendulums''. Generally in physics, we use the pluralization ``pendula''. 
\end{itemize}