
\chapter{Guidelines for lab related activities}
\label{chapter:labs}


\begin{learningObjectives}{
 \item something to learn
 }
\end{learningObjectives}

\begin{opening}
\begin{MCquestion}{A question}
\item a choice
\item another choice %correct
\end{MCquestion}
\end{opening}

\section{Scientific Writing}

Here are a few tips that will help you to polish your lab reports. %add more here, build on these points maybe?

\begin{itemize}
\item Subjective/imprecise terms: When writing scientifically, it is important to avoid using subjective and imprecise terms to describe any part of your experiment/results. In science, reproducibility is of utmost importance - to be as objective as possible is the goal. For example, if you stated that ``our calculated value of g was much greater than the expected value'' - here it is sufficient to say ``our calculated value of g was greater than the expected value''. Your opinion of ``how much'' it was greater shouldn't be reflected in your writing.
\item Definitive statements: it is important to avoid attributing definitive causes to your experimental outcomes in the lab. You can believe as devoutly as you want that your results were ``correct'', but your result/theory can never be ``proven'' to be correct. For example, instead of saying ``as the data exhibit, we have detected the Purple Particle'', it would be better to state that ``as the data exhibit, we have strong evidence to support our observation of the Purple Particle''. 
\item Overstatements: Overstating your findings or the importance of your experiment should generally be avoided in scientific writing, especially in proposals. For example, it might be an overstatement to describe your sliding-block experiment to measure the coefficient of kinetic friction as a ``unique and novel approach''. If many variations of your experiment already exist and the experiment is not ground-breaking in the field, it should not be stated as so.
\item Active vs. passive voice: when writing scientific papers, it is recommended to use the third person, passive voice. For example, this would mean saying ``the drop time for balls at various heights was measured'' rather than ``we measured the drop time for balls at various heights''. Generally, passive or active voice is acceptable in scientific writing, as long as it is kept consistent and factual.
\end{itemize}

Grammar tips: %TODO: add more here
\begin{itemize}
\item ``Data'' are actually plural. You can have multiple data!
\item Future tense vs. past tense: for all experiment proposals, you should write in the future tense. For experimental reports, you should write in the past tense. 
\end{itemize}

\begin{studentOpinion}{Emma}
\textbf{Writing and editing - how can I be more concise?}

We've all felt that our writing was lacking at some point or another. Here are some general tips to avoid overall ``wordiness'' and to increase ease of reading when writing about science: 

\begin{itemize}
\item What would you want to read?: Let's say that you wanted to know the strength of Earth's magnetic field, and how it was found, so you decide to do a literature search. Would you choose a brief, succinct article, or a wordy Magnetic Field Manifesto?
\item The kindergarten test: If you had to explain your concept to a six year old cousin, how would you break it down in a way that they could understand it? If you can't break it down enough to explain to a six year old, perhaps you need to revisit your own understanding of the concept.
\item Avoid unnecessary adjectives: while this might be ok in a creative writing class, in scientific writing, the goal is to get your point across as succinctly as possible. Using ``big'' words might be ok (as long as they properly describe what you are trying to say), but it is important to communicate your message in the simplest manner. 
\item Cut it in half: For every word you read, think of another that you can cut. For every sentence that you read, think of three sentences that communicate the same idea. Pick the sentence that is the shortest and most concise. 
\item Proofread - the more, the better.
\end{itemize}
\end{studentOpinion}

\newpage
\section{Guide for writing a lab}
 \vspace{0.25cm}
\textbf{Abstract}

Write a few short sentences briefly summarizing what you did, how you did it, what you found and what went wrong in your experiment.

\textbf{Procedure}

Describe relevant theories that relate to your experiment here, and the steps to carry out your procedure. 

Consider the following questions:
\begin{itemize}
\item What are the relevant theories/principles that you used? 
\item What equations did you need? Include all necessary/relevant derivations.
\item What materials, equipment and/or tools were necessary in making your measurements?
\item Where was this experiment conducted?
\item How did you make your measurements? How many times did you make them?
\item How did you record your measurements?
\item How did you maximize the precision of your experiments?
\item How did you estimate uncertainties?
\end{itemize}

\textbf{Predictions}

Because the scientific method is iterative, you want to try and predict what you measure. After measuring, you want to see if you should improve on your model or if you discovered a discrepancy between theory and experiment (unlikely in first year physics!)

Consider the following questions:
\begin{itemize}
\item Predict your measured values and uncertainties. How precise do you expect your measurements to be?
\item What assumptions did you have to make to predict your results?
\item Have these predictions influenced how you should approach your procedure? Make relevant adjustments to the procedure based on your predictions.
\end{itemize}

\textbf{Data and Analysis}

Present your data. Include relevant tables/graphs. Describe in detail how you analysed the data, including how you propagated uncertainties. Iterate on the model - is it consistent with your initial predictions?

Consider the following questions:
\begin{itemize}
\item How did you obtain the ``final'' measurement/value from your collected data?
\item What is the relative uncertainty on your value(s)?
\end{itemize}

\textbf{Discussion and Conclusion}

Summarize your findings, and address whether or not your model described the data. Discuss possible sources of error that could have influenced your results.

Consider the following questions:
\begin{itemize}
\item Were there any systematic errors that you didn't consider?
\item Did you learn anything that you didn't previously know? (eg. about the subject of your experiment, about the scientific method in general)
\item If you could redo this experiment, what would you change (if anything)?
\end{itemize}

\newpage
\section{Guide for writing a proposal}
 \vspace{0.25cm}
\textbf{Summary and Goal}

Write a few short sentences briefly summarizing the aim of your experiment, how you will do it and what you hope to find. Additionally, make sure to discuss feasibility and chances of success. It may be easiest to write this section last.

\textbf{Materials and Methods}

Provide a list of your materials, and clearly describe the relevant steps of the method/procedure that you will use to carry out your experiment. Also, propose a method of assessing whether or not your project was successful. 

Consider the following questions:
\begin{itemize}
\item What materials, equipment and/or tools are necessary in making your measurements?
\item What are the cost of these materials? Can they be easily obtained?
\item Where should this experiment be conducted?
\item How will you make your measurements? How many times will you make them?
\item How will you record your measurements?
\item How will you maximize the precision of your experiments?
\item How will you estimate uncertainties?
\item What issues could arise in your experiment? How do you plan to resolve these issues?
\end{itemize}

\textbf{Timeline}

Provide a timeline of your proposed plan. Describe what you will do each day/week/year. Make milestones that you hope to accomplish by set dates. If there is background theory that you need to learn, by what date will you learn it? If you need to order materials, by what date should you receive them?

\textbf{Team}

Provide the names of team members, and assign relevant duties to each member. 

\newpage
\subsection{Guide for reviewing a lab report}
 \vspace{0.25cm}
\textbf{Summary}

Summarize your overall critique of the experiment in 2-3 sentences. Focus on the experiment's method and its result. For example, "The authors dropped balls from different heights to determine the value of g". You don't need to go into the specific details, just give a high level summary of the report. If the report is unclear, specify this.

Consider the following questions:
\begin{itemize}
\item Was the experiment well thought out, and detailed?
\item Was the experimental procedure clear and concise?
\item Did the experimental design minimize uncertainties?
\end{itemize}

\textbf{Feasibility}

In a few sentences, describe the experiment in terms of feasibility and chances of success. 

\begin{itemize}
\item Was the experimental procedure sound?
\item Were the materials, equipment and/or tools accessible? 
\item Were the materials, equipment and/or tools appropriately chosen?
\item Was it possible to complete the experiment in a reasonable period of time?
\end{itemize}

\textbf{Procedure}

Consider the following questions:
\begin{itemize}
\item Is the the procedure clearly and concisely described? 
\item Do you have sufficient information to repeat this experiment?
\item Is it clear which tools are needed to measure which quantities?
\item Is it specified where the experiment should take place?
\item Are clear safety precautions described?
\item Does the experiment clearly state how uncertainties were propagated and determined?
\end{itemize}

\textbf{Overall Rating of the Experiment}

Give the experiment an overall score, based on the criterion described above.

Was the experiment...
\begin{itemize}
\item Excellent
\item Good
\item Satisfactory
\item Needs work
\item Incomplete
\end{itemize}

\newpage
\section{Guide for reviewing a proposal}
 \vspace{0.25cm}
\textbf{Summary and Goal}

Summarize your overall critique of the proposal in 2-3 sentences. Focus on the experiment's methods and goals. For example, "The authors wish to drop balls from different heights to determine the value of g". You don't need to go into the specific details, just give a high level summary of the report. If the report is unclear, specify this.

Consider the following questions:
\begin{itemize}
\item Is the proposed experiment well thought out, and detailed?
\item Is the experimental procedure clear and concise?
\item Does the experimental design minimize uncertainties?
\item Is the experiment feasible? 
\item Is it possible to complete the experiment in a reasonable period of time?
\item Are the goals of the project clear and reasonable?
\end{itemize}

\textbf{Materials and Methods}

Assess the proposal's methods, and determine whether the project is well-organized and incorporates relative methods for self-evaluation.

Consider the following questions:
\begin{itemize}
\item Is the plan for carrying out the experiment well-reasoned, well-organized, and based on a sound rationale?
\item Is it possible to obtain the equipment/materials in a reasonable amount of time?
\item Are adequate resources available to the PI to conduct the experiment?
\item Does the plan incorporate a mechanism to assess success?
\item Is a troubleshooting plan in place, in case of unexpected difficulties?
\end{itemize}

\textbf{Team}

Assess whether the team members/organization are qualified to conduct the experiment.

\textbf{Overall Rating of the Experiment}

Give the experiment an overall score, based on the criterion described above.

Was the experiment...
\begin{itemize}
\item Excellent
\item Good
\item Satisfactory
\item Needs work
\item Incomplete
\end{itemize}



