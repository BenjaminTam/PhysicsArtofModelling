
\chapter{Waves}
\label{chapter:waves}
In this chapter we introduce the tools to describe waves. Waves arise in many different physical systems (the ocean, a string, electromagnetism), but can be described by a common mathematical framework. 
\begin{learningObjectives}{
 \item Understand 
 }
\end{learningObjectives}

\begin{opening}
\begin{MCquestion}{A question}
\item a choice
\item another choice %correct
\end{MCquestion}
\end{opening}

\section{Characteristics of a wave}
\subsection{Definition and types of waves}
A travelling wave is a \textbf{disturbance that travels through} a medium. Consider the waves made by fans at a soccer game, as in Figure \ref{fig:waves:soccer}. As the wave travels horizontally through the stands, the fans only travel a short distance up and down, but they do not travel with the wave. The fans can be thought of as the medium in which the wave propagates. The elements of the medium may oscillate about an equilibrium position (fans moving up and down), but they do not travel with the wave. 
\capfig{0.6\textwidth}{figures/Waves/soccer.png}{\label{fig:waves:soccer}A transverse wave made by soccer fans moving up and down.}

When you look at the ripples made by a rock dropped in a pond (Figure \ref{fig:waves:water}), those ripples travel away from where the rock entered the water, but there is no motion of water outwards. Individual water molecules will move in little circles about an equilibrium position, but they do not move along with the wave.
\capfig{0.6\textwidth}{figures/Waves/water.png}{\label{fig:waves:water}A transverse wave travelling through water. The left panel shows the view from above as ripples move outwards. The right panel shows the motion of an individual water molecule as the wave is viewed from the side.}

If you attach a horizontal rope to a wall and move the other end up and down (Figure \ref{fig:waves:rope}), you can create a disturbance (a wave) that travels horizontally along the rope, but the parts of the rope do not move horizontally; they only move up and down, about some equilibrium position. 
\capfig{0.6\textwidth}{figures/Waves/rope.png}{\label{fig:waves:rope}A transverse wave travelling through a rope. The wave is created by moving one end of the rope up and down.}

If you clap your hands, you will create a pressure disturbance in the air that will move; this is what we call sound (a sound wave). Again, it is not air molecules that move, it is the disturbance that moves through the air. 
\capfig{0.6\textwidth}{figures/Waves/sound.png}{\label{fig:waves:sound}A longitudinal sound wave travelling through the air. The air molecules move back and forth in the same direction as the wave, but they oscillate about an equilibrium position instead of moving with the wave.}

We can distinguish between two types of waves, based on the motion of the medium through which it propagates. \textbf{Transverse waves} are those for which the elements of the medium oscillate back and forth in a direction perpendicular to the motion of the wave. \textbf{Longitudinal waves} are those for which the elements of the medium oscillate back and forth in the same direction as the motion of the wave.

Physically, a wave can only propagate through a medium if the medium can be deformed. When a particle in the medium is disturbed from its equilibrium position, it will experience a restoring force that tries to bring it back to equilibrium. Often, if the displacement of the particle from the equilibrium is small, the magnitude of that force is proportional to the displacement. Thus, the particles in the medium act as small simple harmonic oscillators, and a wave can be viewed as the superposition of many little harmonic oscillators moving back and forth in the medium. 

\subsection{Description of a wave}
We can use several quantities to describe a wave, which are illustrated in Figure \ref{fig:waves:wavelength}:
\begin{itemize}
\item The \textbf{wavelength}, $\lambda$, is the distance between two points along the wave that have the same displacement from their equilibrium position. For example, it can be the distance between two successive maxima (``peaks'') or minima (``troughs'') in the wave.
\item The \textbf{amplitude}, is the maximal distance that a particle in the medium is displaced from its equilibrium position.
\item The \textbf{velocity}, $\vec v$, is the velocity with which the disturbance propagates through the medium.
\item The \textbf{period}, $T$, is the amount of time that goes by for two successive maxima (or minima) of the wave to pass through a point in the medium.
\item The \textbf{frequency}, $f$, is the inverse of the period ($f=1/T$).
\end{itemize}
\capfig{0.6\textwidth}{figures/Waves/wavelength.png}{\label{fig:waves:wavelength}Wavelength, velocity, and amplitude for a transverse wave on a rope.}
The wavelength, speed, and period of the wave are related, since the amount of time that it takes for two successive maxima of the wave to pass through a given point will depend on the speed of the wave and the distance between maxima, $\lambda$. Since it takes a time, $T$, for two maxima a distance $\lambda$ apart to pass through a given point in the medium, the speed of the wave is given by:
\begin{align}
\label{eq:waves:speed}
\Aboxed{v = \frac{\lambda}{T}=\lambda f}
\end{align}
Thus, only two of the speed, period (frequency) and wavelength are independent. \textbf{The speed of a wave generally depends on the properties of the medium through which it is propagating and not on what is creating the wave}. For example, the speed of sound waves depends on the pressure, density, and temperature of the air through which they propagate. Thus, given the frequency of a sound wave, its wavelength is determined from the speed of sound and Equation \ref{eq:waves:speed}.

TODO: Checkpoint MC question: You shake the end of a rope up and down to send waves down the rope. Which property of the wave are you controlling? (the speed, the wavelength of the frequency (correct))?

\section{Mathematical description of a wave}
In order to describe the motion of a wave through a medium, we can describe the motion of the individual particles (or elements) of the medium as the wave passes through. Specifically, we describe the position of each particle using its displacement, $D$, from its equilibrium position. Consider a wave that is propagating through a medium in the positive $x$ direction, as depicted in Figure \ref{fig:waves:sinewave}.

\capfig{0.8\textwidth}{figures/Waves/sinewave.png}{\label{fig:waves:sinewave}Displacement as a function of position for particles in a medium as a wave passes through. The dotted line shows the dipsplacement as a function of time $\SI{1}{s}$ after the solid line, and corresponds to a wave travelling towards the right.}
The displacement of each point, $D$, is shown on the $y$ axis. The solid black line corresponds to a snapshot of the wave at time $t=0$, which has an amplitude of $A=\SI{5}{m}$, a velocity $v=\SI{1}{m/s}$ and a wavelength $\lambda=\SI{4}{m}$. The dotted line corresponds to a snapshot of the wave one second later, at $t=\SI{1}{s}$, when the wave has moved to the right by a distance $vt=\SI{1}{m}$.

At time $t=0$, we can model the displacement of each point in the medium, $D(x, t=0)$, as function of their distance from the origin, $x$:
\begin{align*}
D(x,t=0) = A\sin\left( \frac{2\pi}{\lambda}x \right)
\end{align*}
which corresponds to the displacement being 0 at the origin and at and distance $x$ which is a multiple of the wavelength, $\lambda$. If the wave moves with velocity $v$, then after a certain amount of time, $t$, all of the points that are a distance $vt$ to the right will have the same displacement as those for the wave at time $t=0$. We can state this condition as:
\begin{align*}
D(x,t=0) = D(x-vt)
\end{align*} 
That is, at some time $t$, the displacement of a point at position $x$ is found by subtracting $vt$ from $x$ and taking the displacement of the wave at $t=0$. We can thus write a function for the displacement of a point at position $x$ and a time $t$, as:
\begin{align*}
D(x,t) = A\sin\left( \frac{2\pi}{\lambda}(x-vt) \right)
\end{align*}
Noting that $v/\lambda= 1/T$, we can write this as:
\begin{align*}
D(x,t) = A\sin\left( \frac{2\pi x}{\lambda}- \frac{2\pi x}{T} \right)
\end{align*}
In the above derivation, we assumed that at time $t=0$, the displacement at $x=0$ was $D(x=0, t=0)=0$. In general, the displacement could have any value at $x=0$ and $t=0$, so we can allow the wave to shift left or right by including a phase, $\phi$, which depends on the value on displacement at $x=0$ and $t=0$:
\begin{align}
\Aboxed{D(x,t) = A\sin\left( \frac{2\pi x}{\lambda}- \frac{2\pi x}{T} + \phi \right)}
\end{align}
where $\phi=0$ corresponds to the displacement being zero at $x=0$ and $t=0$.

TODO: MC Checkpoint: what is the value of $\phi$ if the wave as an amplitude $A/2$ at $x=0$ and $t=0$?

The equation above is written in terms of the wavelength, $\lambda$, and period, $T$, of the wave. Often, one uses the ``wave number'', $k$, and the ``angular frequency'', $\omega$, to express the displacement of a particle at position $x$ and time $t$. These are defined as:
\begin{align}
k &= \frac{2\pi}{\lambda}\\
\omega &= \frac{2\pi}{T}
\end{align} 
and essentially remove the factors of $2\pi$ in the expression for $D(x,t)$, which can be written as:
\begin{align}
\Aboxed{D(x,t) = A\sin\left( kx -\omega t + \phi \right)}
\end{align}
It is important to note that the wave number, $k$, has no relation to the spring constant that we used for springs. Since the wavelength and period of the wave are related to the speed of the wave, so are the wave number and angular frequency:
\begin{align*}
v = \frac{\lambda}{T}=\frac{\frac{2\pi}{k}}{\frac{2\pi}{T}}=\frac{\omega}{k}
\end{align*}

\subsection{The wave equation}
In Chapter \ref{chapter:simpleharmonicmotion}, we saw that any physical system whose position, $x$, satisfies the following equation:
\begin{align*}
\frac{d^2x}{dt^2}=-\omega^2 x
\end{align*}
will undergo simple harmonic motion with angular frequency $\omega$, and that $x(t)$ is given by:
\begin{align*}
x(t) = A\cos(\omega t + \phi)
\end{align*}

Similarly, any system, where the displacement of a particle as a function of position and time, $D(x,t)$, satisfies the following equation:
\begin{align}
\die{^2D}{x^2}=\frac{1}{v^2}\die{^2D}{t^2}
\end{align}
is described by a wave that propagates with a speed $v$. The equation above is called the ``one-dimensional wave equation'' and would be obtained from modelling the dynamics of the system, just as the equation of motion for a simple harmonic oscillator. Just like the angular velocity of simple harmonic oscillator is a property of the system, so to is the propagation speed of a wave. The amplitude of a simple harmonic oscillator, or of a wave, have no impact on the angular frequency, or the propagation speed of the wave, respectively. We use partial derivatives in the wave equation instead of total derivatives because $D(x,t)$ is multi-variate. A possible solution to the one-dimensional wave equation is:
\begin{align*}
D(x,t) = A\sin\left( kx -\omega t + \phi \right)
\end{align*}
Furthermore, if there exists multiple solutions, $D_1(x,t)$, $D_2(x,t)$, etc, then any linear combination, $D(x,t)$, of the solutions will also be a solution to the wave equation:
\begin{align*}
D(x,t) = a_1D_1(x,t)+a_2D_2(x,t)+a_3D_3(x,t)+\dots
\end{align*}
This last property is called ``the superposition principle'', and is the result of the wave equation being linear in $D$ (it does not depend on $D^2$, for example). It easy to check that if $D_1(x,t)$ and $D_2(x,t)$ satisfy the wave equation, so does their sum. 

TODO: Question Library question, show that if $D_1(x,t)$ and $D_2(x,t)$ satisfy the 1d wave equation, so does $a_1D_1(x,t)+a_2D_2(x,t)$, where $a_i$ are constants.

In three dimensions, the displacement of a particle in the medium depends on its three spatial coordinates, $D(x,y,z,t)$, and the wave equation in Cartesian coordinates is given by:
\begin{align*}
\die{^2D}{x^2}+\die{^2D}{y^2}+\die{^2D}{z^2}&=\frac{1}{v^2}\die{^2D}{t^2}\\
\end{align*}
There are many functions that can satisfy this equation, and the best choice will depend on the physical system being modelled. 

\section{Waves on a rope}
In this section, we model the motion of transverse waves on a rope, as an example of to illustrate how one can obtain the wave equation from Newton's Second Law as well as to gain insight into what determines the speed of a wave. 
\subsection{A pulse on a rope}
We start by modelling how a single pulse propagates down a horizontal rope that is under a tension, $F_T$\footnote{We do not use $T$ for tension, so as to not confuse with the period of the wave.}. A wave is simply a series of pulses propagating down the rope. Figure \ref{fig:waves:pulse} shows how one can generate a pulse in a taught horizontal rope.
\capfig{0.8\textwidth}{figures/Waves/pulse.png}{\label{fig:waves:pulse}(Left:) Pulling upwards on a horizontal rope causes a pulse to form and propagate. After a small period of time, a pulse is seen propagating down the rope (right).}
We can model the propagation speed of a pulse by considering the speed, $v$, of point $B$ that is shown in Figure \ref{fig:waves:pulse}. Note that point $B$ is not a particle of the rope, and is instead, the location of the front of the disturbance that the pulse causes on the rope. We model the rope as being under a horizontal force of tension, $\vec F_T$, and the pulse is started by exerting a vertical force, $\vec F$, to move the end (point $A$) of the rope upwards with a speed, $v'$. Thus, by pulling upwards on the rope with a force, $\vec F$, at a speed $v'$, we can start a disturbance in the rope that will propagate with speed $v$. 

In a small amount of time, $t$, the point $A$ of the rope will have moved up by a distance $v't$, whereas point $B$ will have moved to the right by a distance $vt$. If $t$ is small enough, we can consider the points $A$, $B$, and $C$ to form the corners of a triangle. That triangle is similar to the triangle that is made by vectorially summing the applied force $\vec F$ and the tension $\vec F_T$, as shown in the top right of Figure \ref{fig:waves:pulse}. We can thus write:
\begin{align*}
\frac{F}{F_T}&=\frac{v't}{vt}=\frac{v'}{v}\\
\therefore F&= F_T \frac{v'}{v}
\end{align*}
Consider the section of rope with length $vt$ that we have raised by applying that force. If the rope has a mass per unit length $\mu$, then the mass of the rope element that was raised (between points $A$ and $B$) has a mass, $m$, given by:
\begin{align*}
m = \mu vt
\end{align*}
The momentum of that section of rope, with vertical speed given by $v'$, is thus:
\begin{align*}
p = mv' = \mu vt v'
\end{align*}
If the force, $\vec F$, was exerted for a length of time, $t$, on the mass element, it will give it an impulse, $Ft$, equal to the change in momentum of the mass element:
\begin{align*}
Ft &= \Delta p \\
Ft &= \mu vt v'\\
\therefore F &= \mu v v'
\end{align*}
We can equate this expression for $F$ with that obtained from the similar triangles to obtain an expression for the speed, $v$, of the pulse:
\begin{align*}
\mu v v' &= F_T \frac{v'}{v}\\
\therefore v&= \sqrt{\frac{F_T}{\mu}}
\end{align*}
The speed of a pulse (and wave) propagating through a rope with linear mass density $\mu$ under a tension $F_T$, is given by:
\begin{align}
\Aboxed{v&= \sqrt{\frac{F_T}{\mu}}}
\end{align}
If the tension in the rope is higher, the pulse will propagate faster. If the linear mass density of the rope is higher, then the pulse will propagate slower. 

\subsection{Reflection and transmission}
In this section, we examine what happens when a pulse travelling down a rope arrives at the end of the rope. First, consider the case illustrated in Figure \ref{fig:waves:reflectionfixed} where the end of the rope is fixed to a wall.
\capfig{0.7\textwidth}{figures/Waves/reflectionfixed.png}{\label{fig:waves:reflectionfixed}When a the end of the rope is held fixed, the reflected pulse will be inverted.}
When the pulse arrives at the wall, the rope will exert an upwards force on the wall, $\vec F_{from\; rope}$. By Newton's Third Law, the wall will then exert a downwards force on the rope, $\vec F_{from\; wall}$. The downwards force exerted on the rope will cause a downwards pulse to form, and the reflected pulse will be inverted compared to the initial pulse that arrived at the wall.

Now, consider the case when the end of the rope has a ring attached to it, so that it can slide freely up and down a post, as illustrated in Figure \ref{fig:waves:reflectionfree}.
\capfig{0.7\textwidth}{figures/Waves/reflectionfree.png}{\label{fig:waves:reflectionfree}When a the end of the rope is free, the reflected pulse will be in the same direction.}
In this case, the end of the rope will move up as the pulse arrives, which will then create a reflected pulse that is in the same direction as the incoming pulse.

Finally, consider a pulse that propagates down a rope that is tied to a second, heavier, rope, as illustrated in Figure \ref{fig:waves:transmission}. 
\capfig{0.55\textwidth}{figures/Waves/transmission.png}{\label{fig:waves:transmission}A pulse can be both reflected and transmitted as changes medium.}
When the pulse arrives at the interface between the two media (the two rope), part of the pulse will be reflected back (and stay upright), and part will be transmitted into the second medium. Note that in the second rope, the pulse will travel with a different speed.

\subsection{Deriving the wave equation}
In this section, we show how to use Newton's Second Law to derive the wave equation for transverse waves travelling down a rope with tension $F_T$ in it. Consider a small mass element of a rope, with mass $dm$, and length $dx$, as a wave pass through that point, as illustrated in Figure \ref{fig:waves:weqn}.
\capfig{0.55\textwidth}{figures/Waves/weqn.png}{\label{fig:waves:weqn}A small section of rope under tension as a wave passes through.}
We assume that the weight of the mass element is negligible compared to the force of tension that is in the rope. The only forces exerted on the mass element are from the tension in the rope, pulling on the mass element from each side. In general, the forces from tension on either side of the mass element will make different angles, $\theta$, with the horizontal, although their magnitude is the same. Let $D(x,t)$ be the vertical displacement of the mass element located at position $x$. We can write the $y$ (vertical) component of Newton's Second Law for the mass element, $dm$, as:
\begin{align*}
\sum F_y = F_{T2} - F_{T1} &= (dm)a\\
F_T\sin\theta_2 - F_T\sin\theta_1 &= dm \die{^2D}{t^2}\\
F_T(\sin\theta_2 - \sin\theta_1) &= dm \die{^2D}{t^2}
\end{align*}
where we used the fact that the force of tension has a magnitude $F_T$ on either side of the mass element, and that the acceleration of the mass in the vertical direction is the second time-derivative of $D(x,t)$. We now make the small angle approximation:
\begin{align*}
\sin\theta\approx \tan\theta = \die{D}{x}
\end{align*}
in which the sine of the angle, is approximately equal to the tangent of the angle, which is equal to the slope of the rope. Applying this approximation to Newton's Second Law:
\begin{align*}
F_T\left(\die{D}{x}\Bigr|_{right} - \die{D}{x}\Bigr|_{left}\right) &= dm \die{^2D}{t^2}
\end{align*}
where we indicated that the term in parentheses is the difference in the slope of the rope between the right side and the left side of the mass element. If the rope has linear mass density, $\mu$, then the mass of the rope element can be expressed in terms of its length, $dx$:
\begin{align*}
dm = \mu dx
\end{align*}
Replacing $dm$ in the equation of motion gives:
\begin{align*}
F_T\left(\die{D}{x}\Bigr|_{right} - \die{D}{x}\Bigr|_{left}\right) &= \mu dx \die{^2D}{t^2}\\
F_T\left(\frac{\die{D}{x}\Bigr|_{right} - \die{D}{x}\Bigr|_{left}}{dx}\right) &= \mu \die{^2D}{t^2}
\end{align*}
The term in parentheses is the difference in the first derivatives of $D(x,t)$ with respect to $x$, divided by the distance between which those derivatives are evaluated. This is precisely the definition of the second derivative with respect to $x$, so we can write:
\begin{align*}
F_T \die{^2D}{x^2} &= \mu \die{^2D}{t^2}\\
\therefore \die{^2D}{x^2} &= \frac{\mu}{F_T} \die{^2D}{t^2}\\
\end{align*}
which is precisely the wave equation:
\begin{align*}
\die{^2D}{x^2} &= \frac{1}{v^2} \die{^2D}{t^2}\\
\end{align*}
with speed:
\begin{align*}
v&= \sqrt{\frac{F_T}{\mu}}
\end{align*}
as we found earlier. Thus, we find that the speed of the propagation of the wave is related to the dynamics used when modelling the system, and is not related to the wave itself. 

\section{The speed of a wave}
In the previous section we found that the speed of a transverse wave in a rope is related to the ratio of the tension in the rope to the linear mass density of the rope:
\begin{align*}
v&= \sqrt{\frac{F_T}{\mu}}
\end{align*}
The speed of a wave in any medium is usually given by a ratio, where the numerator is a measure of how easy it is to deform the medium, and the denominator is measure of the inertia of the medium. For a rope, the tension is a measure of how stiff the rope is. A higher tension makes it more difficult to disturb the rope from equilibrium and it will ``snap back'' faster when disturbed, so the pulse will propagate faster. The heavier the rope, the harder it will be for the disturbance to propagate as the rope has more inertia, which will slow down the pulse.

The only way that a wave can propagate through a medium is if that medium can be deformed and the particles in the medium can be displaced from their equilibrium position, much like simple harmonic oscillators. The wave will propagate faster if those oscillators have a stiff spring constant and there is a strong force trying to restore them to equilibrium. However, if those oscillators have a large inertia, even with a large restoring force, they will accelerate back to their equilibrium with a smaller acceleration. 

In general, the speed of a wave is given by:
\begin{align*}
v=\sqrt{\frac{\text{Stiffness of medium}}{\text{Inertia of medium}}}
\end{align*} 
For example, the speed of longitudinal pressure waves in a solid is given by:
\begin{align*}
v=\sqrt{\frac{E}{\rho}}
\end{align*}
where $E$ is the ``elastic (or Young's) modulus'' for the material, and $\rho$ is the density of the material. The elastic modulus of a solid is a measure of how much the solid deforms when a force (or pressure) is exerted on the material.

For the propagation of longitudinal pressure waves through a fluid, the speed is given by:
\begin{align*}
v=\sqrt{\frac{B}{\rho}}
\end{align*}
where $B$ is the bulk modulus of the liquid, and $\rho$ its density. 

\section{Energy transported by a wave}
In this section, we examine how energy is transported by waves. Although no material moves along with a wave, mechanical energy can be transported by a wave, as evidenced by the damage that the waves from an earthquake can make.  
\subsection{A wave as being made of simple harmonic oscillators}
Consider a wave that is propagating through a medium, as the individual particles in the medium are displaced about their equilibrium position. We can picture the motion of one of the particles in the medium as if it were the motion of a simple harmonic oscillator\footnote{If the medium has a linear restoring force or if the amplitude of the oscillations is small.}. This is illustrated in Figure \ref{fig:waves:sinewavetimeshm}, which shows the displacement as a function of time for a point in the medium located at the origin. The displacement of that point, at $x=0$, if we choose $\phi=0$, is given by:
\begin{align*}
D(x=0,t) = A\sin(-\omega t)
\end{align*}
\capfig{0.8\textwidth}{figures/Waves/sinewavetimeshm.png}{\label{fig:waves:sinewavetimeshm}The displacement as a function of time for one particle in the medium (at $x=0$) is identical to the motion of a simple harmonic oscillator.}
The displacement of the particle in the medium is described by the same equation as the position of a simple harmonic oscillator, with angular frequency $\omega$, which is the same angular frequency as the wave. 

We can also view a snapshot of the wave in time, and model the \textbf{different} points in the medium as different oscillators that all have different displacements. This is shown in Figure \ref{fig:waves:sinewavepositionshm}.
\capfig{0.8\textwidth}{figures/Waves/sinewavepositionshm.png}{\label{fig:waves:sinewavepositionshm}The displacement as a function of position for different points in a medium. Each point in the medium can be modelled as a simple harmonic oscillator.}

\subsection{Energy transported in a one dimensional wave}
In this section, we show how to describe the energy transported by a one-dimensional wave along a rope. We can model each particle in the rope through which the wave propagates as a small simple harmonic oscillator with mass $m$, attached to a spring with spring constant $k_s$\footnote{We use $k_s$ for the spring constant, to distinguish it from $k$, the wave number.}.

Of course, there is no actual spring, but we can still determine an effective spring constant, $k_s$, from the angular velocity:
\begin{align*}
\omega &= \sqrt{\frac{k_s}{m}}\\
\therefore k_s &= \omega^2 m
\end{align*}
which corresponds to the spring constant that would give the correct angular frequency for the particle.

The total mechanical energy of one oscillator, $E_m$, can be evaluated when the oscillator is at its maximal displacement, $A$, from its equilibrium, where its kinetic energy is zero:
\begin{align*}
E_m = \frac{1}{2}k_s A^2 = \frac{1}{2}\omega^2 m A^2
\end{align*}

If the rope is infinitely long, and carries a continuous wave, it will have an infinite amount of energy, as it will corresponds to an infinite number of oscillators. Instead, let us calculate how much energy, $E_\lambda$ is stored in a wave, over one wavelength, $\lambda$. To do so, we need to evaluate how many effective oscillators are contained in the rope, over a distance $\lambda$, so that we can sum all of their energies together to obtain the energy stored in one wavelength:
\begin{align*}
E_\lambda = \sum \frac{1}{2}\omega^2 m A^2
\end{align*} 
where the sum is over the number of oscillators in one wavelength. Of course, the rope is note actually made of oscillators, but we can model each section of rope of length $dx$ has being an oscillator of mass $dm=\mu dx$, where $\mu$ is the linear mass density of the rope. The sum (integral) of the energy of the oscillators can thus be written as:
\begin{align*}
E_\lambda = \int_0^\lambda \frac{1}{2}\omega^2 \mu A^2 dx =  \frac{1}{2}\omega^2 \mu A^2 \lambda
\end{align*}
The energy stored in one wavelength is not a very useful property of a wave, since the total energy in the wave depends on the length of the wave. We can describe the rate at which energy is transmitted by the wave (its power), since we know how long, $T$, it will take the wave to travel one wavelength. The average power with which energy is transported by a wave is given by:
\begin{align*}
P = \frac{E}{T} = \frac{\frac{1}{2}\omega^2 \mu A^2 \lambda}{T}=\frac{1}{2}\omega^2 \mu A^2 v
\end{align*}
where $T$ is the period of the wave, and $v=\lambda/T$ is the speed of the wave. The power transmitted by a wave on a string (and any one-dimensional wave), is thus given by:
\begin{align}
\Aboxed{P=\frac{1}{2}\omega^2 \mu A^2 v}
\end{align}
We can see that the power transmitted by a wave goes as the amplitude, $A$, of the wave squared. It thus takes four times more energy to double the amplitude of waves that are sent down a string. 

\subsection{Energy transported in a spherical three-dimensional wave}
In this section, we show how to model the rate at which energy is transported in spherical three-dimensional wave, such as the sound wave that is generated when you clap your hands. A spherical sound wave is a pressure disturbance in the air that propagates spherically outwards from a point of emission. We can think of thin spherical shells containing air that expand and contract about their equilibrium position as the wave moves through. The motion of each shell is similar to that of a simple harmonic oscillator of mass $dm$.

TODO: This needs some sort of figure!

Consider a shell at a radial position, $r$, with thickness $dr$, and mass $dm$. If the medium has a density, $\rho$, then the mass of the shell is given by:
\begin{align*}
dm = \rho dV = \rho 4\pi r^2 dr
\end{align*}
where $dV = 4\pi r^2 dr$ is the volume of the shell. Again, if we model each shell as a simple harmonic oscillator with mass $dm$, then the energy, $dE$, stored in that oscillator is given by:
\begin{align*}
dE = \frac{1}{2}k_s A^2 =  \frac{1}{2}\omega^2 dm A^2 = \frac{1}{2}\omega^2 A^2 \rho 4\pi r^2 dr=2\pi\rho  \omega^2 A^2  r^2 dr
\end{align*}
where $\omega$ is the angular frequency of the wave, and $A$ is the amplitude of the wave. We expressed the effective spring constant, $k_s$, in terms of the angular frequency of the simple harmonic oscillator and its mass, as we did in the previous section. It makes less sense now to calculate the energy stored in one wavelength of the wave, because since the energy $dE$ stored in one shell, also depends on the location, $r$, of that shell.

The rate at which energy is transported by the wave is given by:
\begin{align*}
P = \frac{dE}{dt}
\end{align*}
We can use the Chain Rule to change this into a derivative over $r$:
\begin{align*}
P = \frac{dE}{dr}\frac{dr}{dt}=\frac{dE}{dr}v
\end{align*}
where $\frac{dr}{dt}=v$ is the speed of the wave (the rate of change of the radius of a shell). The average power transmitted by the spherical wave is thus given by:
\begin{align*}
P &=\frac{dE}{dr}v =2\pi\rho  \omega^2 A^2  r^2 v
\end{align*}
where the power depends on how far you are from the source ($r$), unlike the case for a one-dimensional wave. 

Suppose that you have a $\SI{50}{W}$ speaker emitting sound; each radial shell emanating from the speaker must transport energy at a rate of $\SI{50}{W}$. This is simply a statement that the energy radiated by the speaker has to move from one shell to the next and be conserved. Since the power transported by a shell depends on the radius of the shell, if the power transmitted by each shell is the same, then the amplitude of the wave in each shell must decrease. In particular, for a spherical wave, the amplitude will decrease as a function of distance from the source:
\begin{align*}
P& = \text{constant}\\
\therefore A&=\frac{1}{r}\sqrt{\frac{P}{2\pi\rho \omega^2 v}}
\end{align*}

The ``intensity of a wave'', $I$, is defined as the power per unit area that is transmitted by the wave. For a spherical wave at radial position $r$, with area $4\pi r^2$, the intensity of the wave is given by:
\begin{align*}
I = \frac{P}{4\pi r^2} = \frac{1}{2}\rho  \omega^2 A^2 v
\end{align*}

Usually, the intensity of a wave is something that you can measure, as it corresponds to the power delivered into some measuring device with a specific surface area. For example, we cannot directly measure the total power that is transported by the waves from an earthquake, as we would need an instrument that could encompass the entire resulting wave. Instead, we can measure the intensity of waves from the earthquake by measuring how much power is delivered into some instrument with a known surface area. By knowing our distance from the earthquake, we could then determine the total power output of the earthquake. 

The intensity is a measure of how much energy is delivered per unit area by a wave and goes down as the square of the distance from the source. If the source of the wave is an earthquake, then your house will have four times less damage than your friend's, if your house is located only twice as far from the epicentre as your friend's. You will cause four times less damage to your ears if you move only twice as far away from the stage at a rock concert.

\section{Superposition of waves and interference}
In this section, we consider what happens when two (or more) different waves propagate in a medium and interfere with each other. The superposition principle states that if $D_1(x,t)$, $D_2(x,t)$, $\dots$, are functions that satisfy the wave equation, then any linear combination of these functions, $D(x,t)$:
\begin{align*}
D(x,t) = a_1D_1(x,t)+a_2D_2(x,t)+a_3D_3(x,t)+\dots
\end{align*}
will also satisfy the wave equation.

The wave equation comes from modelling the physics of propagating a disturbance through a medium, but does not describe the type of disturbance. For example, you can imagine shaking the end of a rope at a given frequency, which will result in a wave of a specific wavelength travelling down the rope, at a speed determined by the wave equation. If you then shake the rope to create a wave with a different frequency, that wave will have a different wavelength and be described by a different function, but will still propagate with the same speed. You can create two different waves in the same rope, and both will satisfy the wave equation, but the waves can have different amplitudes and frequencies.

Suppose that you hold one end of the rope and shake it with a specific frequency, creating waves that are described by:
\begin{align*}
D_1(x,t) = A_1\sin(k_1x-\omega_1t+\phi_1)
\end{align*}
Your friend, at the other end of the rope shakes the rope with a different frequency, creating waves that propagate in the opposite direction and that are described by:
\begin{align*}
D_2(x,t) = A_2\sin(k_2x+\omega_2t+\phi_2)
\end{align*}
The superposition principle states that the net displacement at any position $x$ at some time $t$ is found by summing these two waves together:
\begin{align*}
D(x,t) = A_1\sin(k_1x-\omega_1t+\phi_1) + A_2\sin(k_2x+\omega_2t+\phi_2)
\end{align*}

The superposition of waves is illustrated in Figure \ref{fig:waves:superposition}, which shows three waves, and their resulting sum in the bottom most panel.
\capfig{0.7\textwidth}{figures/Waves/superposition.png}{\label{fig:waves:superposition}The superposition of three waves to create a resulting wave shown in the bottom panel. The waves are shown at an instant in time.}

The resulting wave is created by the ``interference'' of the three waves, and mathematically is simply a sum of the three individual waves at each position (and instant in time). The resulting wave in this example has a rather complicated shape, that is no longer described by a sine function. However, by the superposition principle, it is a valid solution to the wave equation.

The individual waves in the top three panels of Figure \ref{fig:waves:superposition} all have an amplitude of $\SI{5}{m}$. The resulting wave, at some points (e.g. at $x=\SI{2}{m}$), has an amplitude that is larger than any of the individual waves; we say that, at those positions, the individual waves have ``constructively interfered''. In other locations (e.g. at $x=\SI{6}{m}$), the resulting wave has a smaller amplitude than the individual waves, and we say that the individual waves have ``destructively interfered''. The interference between waves can be observed easily on a water surface, for example by observing the constructive and destructive interference pattern of waves that originate from two pebbles being dropped at the same time a certain distance apart. Constructive interference between waves is also thought to be behind some reports of gigantic waves observed out at sea.

If two waves have the same wavelength and amplitude, it is possible for them to completely interfere destructively, resulting in no net wave. Similarly, they can also completely constructively interfere, resulting in a wave with a much larger amplitude. Complete destructive and constructive interference is illustrated in the left and right panel of Figure \ref{fig:waves:interference}, respectively.
\capfig{0.7\textwidth}{figures/Waves/interference.png}{\label{fig:waves:interference}Destructive (left) and constructive (right) interference of waves.}

\section{Standing waves}
As we saw in the last section, when waves have the same frequency, it is possible for them to interfere completely, either destructively or constructively. In particular, if you have a string that is fixed at one end, and you vibrate the other end, you will create waves that travel to the fixed end. Those will then be inverted after reflection, and will interfere with the original wave and have the same frequency. In general, the resulting wave will be quite complicated, but if you choose the frequency precisely, then the resulting wave can have a large amplitude. This special type of resulting wave is called a ``standing wave'' as it does not appear to travel along the string. Instead, each point on the string will oscillate with an amplitude that depends on where the point is located along on the string. In contrast, for a travelling wave, all of the points oscillate with the same amplitude.

Three standing waves of different frequencies (wavelengths) are illustrated in Figure \ref{fig:waves:standing}.
\capfig{0.7\textwidth}{figures/Waves/standing.png}{\label{fig:waves:standing}The first three standing waves on a string.}

The solid line in each of the three panels corresponds to one particular snapshot of the standing wave at a particular instant in time. The dashed lines correspond to snapshots at different times. In particular, there is a time where the displacement of all points on the string is zero. Each point on the string vibrates with a different amplitude, which corresponds to the solid line (and the opposite dashed line). Certain points do not oscillate at all; these are called ``nodes''. The points at the end of the string are always nodes. Certain points vibrate with a maximal amplitude; these are called ``anti-nodes''.

In general, if you pluck a taught string (such as a guitar string), you will create many waves, with many different frequencies, that propagate outwards from the point where the string was plucked. Those waves will be reflected by the ends of the string and interfere with each other. Most of the waves will interfere in a complicated way and decay away. Those waves that have the correct frequency to create standing waves will persist on the string for a longer period of time. The string will eventually vibrate as a superposition of the fundamental frequency (the standing wave with one anti-node, also called the first harmonic), and the higher ``harmonics'' (those standing waves with more anti-nodes).

The wavelength of the fundamental standing wave for a string of length, $L$, is given by the condition:
\begin{align*}
\lambda = 2L
\end{align*}
In general, the $n$th harmonic will have a wavelength of:
\begin{align}
\Aboxed{\lambda_n = \frac{2L}{n}\quad\quad n=1,2,3,\dots}
\end{align}
The corresponding frequency is give by:
\begin{align}
\Aboxed{f_n = \frac{nv}{2L}}
\end{align}
where $v=f\lambda$ is the speed of the waves on the string.

A standing wave is the result of two waves of the same frequency and amplitude travelling in opposite directions. Thus, there is no energy that is transmitted by a standing wave (e.g. through the nodes at the end of the string). Although we described standing waves for a string, these are not restricted to one dimensional waves. For example, the membrane of a drum can also support standing waves.

In general, most object can be characterized by a harmonic (or ``resonant'') frequency that corresponds to the standing waves that can exist in the object. If that object is, say, shaken, many waves will propagate through the object and cancel out, except those that have the resonant frequency. Relatively small vibrations, if at the correct frequency, can lead to large standing waves that can result in damage to the object. 

\subsection{Mathematical description of a standing wave}
A standing wave is the result of two identical waves, travelling in opposite directions, interfering. Consider the waves described by $D_1(x,t)$ and $D_2(x,t)$ that are modelled as follows:
\begin{align*}
D_1(x,t) &= A\sin(kx-\omega t)\\
D_2(x,t) &= A\sin(kx+\omega t)\\
\end{align*}
These two waves are identical, but travel in opposite directions (due to the sign in front of the $\omega t$). The superposition of these waves is given by:
\begin{align*}
D(x,t) &= D_1(x,t) + D_2(x,t)\\
&=A\Bigr(\sin(kx-\omega t)+\sin(kx+\omega t)\Bigl)
\end{align*}
We can use the following trigonometric identity to combine these into single term:
\begin{align*}
\sin\theta_1+\sin\theta_2 = 2\sin\left(\frac{\theta_1+\theta_2}{2} \right) \cos\left( \frac{\theta_1-\theta_2}{2}\right)
\end{align*}
The resulting wave is thus given by:
\begin{align*}
D(x,t) &= 2A\sin\left(\frac{kx-\omega t + kx+\omega t}{2} \right) \cos\left( \frac{kx-\omega t - kx-\omega t}{2}\right)\\
&=2A\sin(kx)\cos(\omega t)
\end{align*}
If this wave describes the wave on a string of length $L$ with both ends held fixed, and we set the origin of our coordinate system at one end of the string, then we require that the displacement at $x=0$ and $x=L$ is always zero. The first condition is always true, and the second requires that:
\begin{align*}
D(x=L,t) &= 0\\
\sin(kL) &= 0\\
\therefore kL &= n\pi \quad\quad n=1,2,3,\dots
\end{align*}
and $kL$ must be a multiple of $2\pi$. In terms of the wavelength, $\lambda$, this gives:
\begin{align*}
\frac{2\pi}{\lambda}L &= n\pi\\
\therefore \lambda&= \frac{2L}{n}
\end{align*}
as we argued before. The standing wave for the $n$-th harmonic is thus described by
\begin{align}
\Aboxed{D(x,t)=2A\sin\left(\frac{n\pi}{L}x\right)\cos(\omega t) }
\end{align}
The point at position $x$, will behave like a simple harmonic oscillator and oscillate with an amplitude given by:
\begin{align*}
A(x) = 2A\sin\left(\frac{n\pi}{L}x\right)
\end{align*}
Each point on the string will vibrate with the same angular frequency, $\omega$, but with a different amplitude, depending on their position. For the $n$-th harmonic, the nodes of the standing wave are located at:
\begin{align*}
\sin\left(\frac{n\pi}{L}x\right) &=0\\
\frac{n\pi}{L}x &= m\pi \quad\quad m=0,1,2,\dots\\
\therefore x &= m\frac{L}{n} 
\end{align*}
Thus, for example, the second node ($m=2$) of the third harmonic ($n=3$), is located at $x=2L/3$, as can be seen in the bottom panel of Figure \ref{fig:waves:standing}. The anti-nodes are located at:
\begin{align*}
\frac{n\pi}{L}x &= m\frac{\pi}{2} \quad\quad m=1,3,5,7,\dots\\
\therefore x&=m\frac{L}{2n}
\end{align*}
where, for example, the first anti-node of the first harmonic is located at $x=L/2$, as can be seen in the top panel of Figure \ref{fig:waves:standing}.
\newpage
\section{Summary}

\begin{chapterSummary}
A travelling wave is the propagation of a disturbance with a speed, $v$, through a medium. Particles in the medium oscillate back and forth, about an equilibrium position, as a wave passes through the medium, but no matter moves in the direction of the wave. Only mechanical energy is transmitted by a wave.

In a transverse wave, the particles in the medium oscillate in a direction that is perpendicular to the velocity of the wave. In a longitudinal wave, the particles of the medium oscillated in a direction that is co-linear with the velocity of the wave.

A wave is described by it frequency, $f$, its wavelength, $\lambda$, its amplitude, $A$, and its speed, $v$. We can also use the period of the wave, $T$, in lieu of the frequency. The frequency and wavelength of a wave are related to each other by the speed of the wave:
\begin{align*}
v = \lambda f
\end{align*}

Mathematically, a one-dimensional travelling wave moving in the positive $x$ direction can be described by:
\begin{align*}
D(x,t) = A \sin(kx-\omega t + \phi)
\end{align*}
where $D(x,t)$ is the displacement of the particle in the medium at position $x$ at time $t$. $\phi$ is the phase of the wave and depends on our choice of when $t=0$. $k$ is the wave number of the wave, and $\omega$ its angular frequency. These are related to the wavelength and frequency, respectively:
\begin{align*}
k &= \frac{2\pi}{\lambda}\\
\omega &= 2\pi f = \frac{2\pi}{T}
\end{align*}

If the a dynamical model (e.g. Newton's Second Law) of a system leads to an equation with the following form:
\begin{align*}
\die{^2D}{x^2}=\frac{1}{v^2}\die{^2D}{t^2}
\end{align*}
then waves with a speed of $v$ can propagate in the medium, if there is a disturbance.

The speed of a wave on a rope of linear mass density $\mu$ under a tension $F_T$, is given by:
\begin{align*}
v=\sqrt{\frac{F_T}{\mu}}
\end{align*}

Generally, the speed of a wave in medium is related to the elasticity of the medium when it is deformed and the inertia of the particles in the medium. In order for a wave to propagate through a medium, the particles in the medium must be able to be displaced from their equilibrium position.

A pulse travelling through a rope will get reflected at the end of the rope and travel back in the opposite direction. If the end of the rope is fixed, the reflected pulse will be inverted. If the end of the rope can move, the reflected pulse will be in the same direction as the incoming pulse.

A one-dimensional wave will transfer energy at an average rate:
\begin{align*}
P = \frac{1}{2}\omega^2\mu A^2 v 
\end{align*}

A three dimensional spherical wave through a medium with density $\rho$ will transfer energy at an average rate:
\begin{align*}
P = 2\pi\rho\omega^2r^2 v
\end{align*}
at a distance $r$ from the source of the wave. The amplitude of a spherical wave will decrease as the distance away from the source increases:
\begin{align*}
A =  A&=\frac{1}{r}\sqrt{\frac{P}{2\pi\rho \omega^2 v}}
\end{align*}
The intensity of a spherical wave is defined as the power per unit area transferred by the wave, and is given by:
\begin{align*}
P=\frac{P}{4\pi r^2}=\frac{1}{2}\rho\omega^2A^2v
\end{align*}

The superposition principle states that if $D_1(x,t)$, $D_2(x,t)$, $\dots$, are functions that satisfy the wave equation, then any linear combination of these functions, $D(x,t)$:
\begin{align*}
D(x,t) = a_1D_1(x,t)+a_2D_2(x,t)+a_3D_3(x,t)+\dots
\end{align*}
will also satisfy the wave equation. 

Different waves can interfere constructively or destructively in a medium, and the resulting wave is given by the sum of the functions describing the interfering waves. 

Standing waves are formed when waves of the same frequency and amplitude travelling in opposite directions interfere. For standing waves on a string, the displacement of a particle on the string is given by:
\begin{align*}
D(x,t)=2A\sin\left(\frac{n\pi}{L}x\right)\cos(\omega t)
\end{align*}
where $n$ is the number of the harmonic and $L$ is the length of the string. In particular, a particle at position $x$ will move up and down as a simple harmonic oscillator with amplitude:
\begin{align*}
A(x) = 2A\sin\left(\frac{n\pi}{L}x\right)
\end{align*}
The condition for a standing wave on a string is that the length of the string must be equal to a multiple of half the wavelength:
\begin{align*}
L &= n\frac{\lambda}{2}\quad\quad n=1,2,3,\dots\\
\lambda &= \frac{2L}{n}\\
f &= \frac{nv}{2L}
\end{align*}


\end{chapterSummary}

\newpage
\begin{importantEquations}
This is an important equation
\begin{align*}
E = mc^2
\end{align*}

\end{importantEquations}


\newpage
\section{Thinking about the material}
\subsection{Reflect and research}

\begin{enumerate}
\item Look up a video of the Tacoma Narrows bridge failing, and explain what happened.
\end{enumerate}
\subsection{To try at home}
Confirm that the reflected pulse from a rope on a string is inverted when the end of the rope is fixed.
\subsection{To try in the lab}

\newpage
\section{Sample problems and solutions}
\subsection{Problems}


\newpage
\subsection{Solutions}


