
\chapter{Rotational motion}
\label{chapter:rotationalmotion}
In this Chapter, we introduce the concepts of torque and moment of inertia. These will allow us to apply Newton's Second Law to rotating systems. 

\begin{learningObjectives}{
 \item Understand how to use vector quantities for describing the kinematics of rotations.
 \item Understand how to use torque to determine angular acceleration.
 \item Understand the moment of inertia and to calculate it.
 \item Understand conditions for static and dynamic equilibrium.
 }
\end{learningObjectives}

\begin{opening}
\begin{MCquestion}{A question}
\item a choice
\item another choice %correct
\end{MCquestion}
\end{opening}

\section{Rotational kinematic vectors}

\section{Torques}

\section{Moment of inertia}


\newpage
\section{Summary}

\begin{chapterSummary}{
\item Something that was learned
}
\end{chapterSummary}

\newpage
\begin{importantEquations}
This is an important equation
\begin{align*}
E = mc^2
\end{align*}

\end{importantEquations}


\newpage
\section{Thinking about the material}
\subsection{Reflect and research}

\begin{enumerate}
\item Something to research more.
\end{enumerate}
\subsection{To try at home}

\begin{tQuestion}Try doing this \end{tQuestion}

\subsection{To try in the lab}

\newpage
\section{Sample problems and solutions}
\subsection{Problems}
\begin{problemParts}{A question\label{Q:chaptertitle:q1}}
\item How close can he get to the hurdle before he has to jump?
\item What maximum height does he reach?
\end{problemParts}

\newpage
\subsection{Solutions}
\begin{solution}{\ref{Q:chaptertitle:q1}}
{
the solution
}
\end{solution}

