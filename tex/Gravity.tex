
\chapter{Gravity}
\label{chapter:gravity}
In previous chapters, we have so far learned about Newton's Theory of Classical Mechanics, which allowed us to model the motion of an object based on the forces acting on the object. In this chapter, we present the theories that describe the force of gravity itself. We will see several theories of gravity and focus primarily on Newton's Universal Theory of Gravity. 

\begin{learningObjectives}{
 \item something to learn
 }
\end{learningObjectives}

\begin{opening}
\begin{MCquestion}{A question}
\item a choice
\item another choice %correct
\end{MCquestion}
\end{opening}

\section{Kepler's Laws}
Although humans have long been fascinated by the motion of objects in the sky, it was Johannes Kepler, in the early seventeenth century, that was the first to write down quantitative rules that described the motions of planets around the Sun. His theory was based on the extensive and detailed observations recorded by Tycho Brahe in the late sixteenth century. 

Kepler proposed three laws that described all of the data that Tycho Brahe had collected about planetary motion:
\begin{enumerate}
\item The path of a planet around the Sun is described by an ellipse with the Sun at once of its foci.
\item Planets move in such a way that the area swept by a line connecting the planet in the Sun in a given period of time is constant, independent of the location of the planet.
\item The ratio between the orbital periods squared of two planets is equal to the ratio of the semi-major axes of their orbits cubed:
\begin{align*}
\left(\frac{T_1}{T_2}\right)^2=\left(\frac{s_1}{s_2}\right)^3
\end{align*}
\end{enumerate}

\subsection{Kepler's First Law}
Kepler noticed that the motion of all planets followed the path of an ellipse with the Sun located at one of its foci. Figure \ref{fig:gravity:ellipse} shows a diagram of an ellipse, along with its two foci, its semi-major axis, $s$, its semi-minor axis, $b$, and its eccentricity, $e$. The eccentricity is a measure of how ``flat'' the ellipse is; an eccentricity of zero corresponds to the special case of a circle. 
\capfig{0.5\textwidth}{figures/Gravity/ellipse.png}{\label{fig:gravity:ellipse}A ellipse, showing its two foci, its semi-major axis, $s$, its semi-minor axis, $b$, and its eccentricity, $e$. A circle is a special case of an ellipse with both foci overlapping and an eccentricity of $e=0$.}

\subsection{Kepler's Second Law}
Kepler's Second Law relates to the speed of a planet in an elliptical orbit and states that the area swept by a line connecting the planet and the Sun in a given period of time is fixed. This is illustrated in Figure \ref{fig:gravity:ellipse2}, which shows the elliptical orbit of a planet around the Sun located at one of the foci, and the area swept out when the planet goes from $A$ to $B$ and from $C$ to $D$. 
\capfig{0.5\textwidth}{figures/Gravity/ellipse2.png}{\label{fig:gravity:ellipse2}Illustration of Kepler's Second Law, showing the area that is ``swept'' by a planet in a fixed period of time. }

In particular, Kepler's Second Law states that the two areas that are shown in the figure will have the same area if the planet took the same amount of time to travel between points $A$ and $B$ as it did to travel between points $C$ and $D$. Because the points $C$ and $D$ are further away from the Sun than points $A$ and $B$, the distance between points $C$ and $D$ must be smaller than the distance between points $A$ and $B$ for the two areas to be the same. This, in turn, implies that the planet must be moving slower between $C$ and $D$ than between points $A$ and $B$. As we will see in a later chapter, Kepler's Second Law is equivalent to the statement that the angular momentum of the planet is conserved. 

TODO: Checkpoint: At which point in the orbit is the speed of the planet the greatest?

\section{Newton's Universal Theory of Gravity}

\subsection{The gravitational field}

\subsection{Gauss' Law}

\section{Gravitational potential energy}

\subsection{Mechanical energy with gravity}

\subsection{Satellite orbits}
discuss apparent weight.

\section{Einstein's Theory of General Relativity}

\newpage
\section{Summary}

\begin{chapterSummary}
We learned about gravity.
\end{chapterSummary}

\newpage
\begin{importantEquations}
This is an important equation
\begin{align*}
E = mc^2
\end{align*}

\end{importantEquations}


\newpage
\section{Thinking about the material}
\subsection{Reflect and research}

\begin{enumerate}
\item When you look at the night sky, how can you tell the difference between a planet and a star?
\item What was the relationship between Tycho Brahe and Johannes Kepler?
\item How did Tycho Brahe collect all the data that Kepler used?
\item How much time elapsed between Kepler publishing his laws and Newton publishing his Universal Theory of Gravity?
\item What was Kepler's original intention when he was synthesized Tycho Brahe's observations? What was he hoping to show?
\item What was Ptolemy's theory of gravity based upon?
\item Who was the first to suggest that planets revolved around the Sun instead of the Earth?
\item Explain how the force of gravity from the moon results in tides on both sides of the Earth.
\end{enumerate}
\subsection{To try at home}

\begin{tQuestion}Try doing this \end{tQuestion}

\subsection{To try in the lab}
Theory project: Prove, based on Newton's Universal Theory of Gravity, that the motion of orbiting bodies is given by a conical section. You may find the Lagrangian formulation to be easier for this task!

\newpage
\section{Sample problems and solutions}
\subsection{Problems}
\begin{problemParts}{A question\label{Q:chaptertitle:q1}}
\item How close can he get to the hurdle before he has to jump?
\item What maximum height does he reach?
\end{problemParts}

\newpage
\subsection{Solutions}
\begin{solution}{\ref{Q:chaptertitle:q1}}
{
the solution
}
\end{solution}

