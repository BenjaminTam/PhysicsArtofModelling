
\chapter{Gravity}
\label{chapter:gravity}
In previous chapters, we have so far learned about Newton's Theory of Classical Mechanics, which allowed us to model the motion of an object based on the forces acting on the object. In this chapter, we present the theories that describe the force of gravity itself. We will see several theories of gravity and focus primarily on Newton's Universal Theory of Gravity. 

\begin{learningObjectives}{
 \item Understand Kepler's Laws.
 \item Understand Newton's Universal Theory of Gravity. 
 \item Understand Gauss' Law and the gravitational field.
 \item Understand how to use energy to describe orbits.
 \item Understand how Einstein's General Theory of Relativity differs from Newton's theory of gravity.
 }
\end{learningObjectives}

\begin{opening}
\begin{MCquestion}{A question}
\item a choice
\item another choice %correct
\end{MCquestion}
\end{opening}

\section{Kepler's Laws}
Although humans have long been fascinated by the motion of objects in the sky, it was Johannes Kepler, in the early seventeenth century, that was the first to write down quantitative rules that described the motions of planets around the Sun. His theory was based on the extensive and detailed observations recorded by Tycho Brahe in the late sixteenth century. 

Kepler proposed three laws that described all of the data that Tycho Brahe had collected about planetary motion:
\begin{enumerate}
\item The path of a planet around the Sun is described by an ellipse with the Sun at once of its foci.
\item Planets move in such a way that the area swept by a line connecting the planet in the Sun in a given period of time is constant, independent of the location of the planet.
\item The ratio between the orbital periods squared of two planets is equal to the ratio of the semi-major axes of their orbits cubed:
\begin{align*}
\left(\frac{T_1}{T_2}\right)^2=\left(\frac{s_1}{s_2}\right)^3
\end{align*}
\end{enumerate}

\subsection{Kepler's First Law}
Kepler noticed that the motion of all planets followed the path of an ellipse with the Sun located at one of its foci. Figure \ref{fig:gravity:ellipse} shows a diagram of an ellipse, along with its two foci, its semi-major axis, $s$, its semi-minor axis, $b$, and its eccentricity, $e$. The eccentricity is a measure of how far a focus is from the centre of the ellipse. A larger eccentricity thus corresponds to a ``flatter'' ellipse. Note that a circle is just a special case of an ellipse, with both foci located at the centre of the circle.

\capfig{0.5\textwidth}{figures/Gravity/ellipse.png}{\label{fig:gravity:ellipse}A ellipse, showing its two foci, its semi-major axis, $s$, its semi-minor axis, $b$, and its eccentricity, $e$.}

TODO: Checkpoint: Show two ellipses, determine which one has the largest eccentricity.

\subsection{Kepler's Second Law}
Kepler's Second Law relates to the speed of a planet in an elliptical orbit and states that the area swept by a line connecting the planet and the Sun in a given period of time is fixed. This is illustrated in Figure \ref{fig:gravity:ellipse2}, which shows the elliptical orbit of a planet around the Sun located at one of the foci, and the area swept out when the planet goes from $A$ to $B$ and from $C$ to $D$. 
\capfig{0.5\textwidth}{figures/Gravity/ellipse2.png}{\label{fig:gravity:ellipse2}Illustration of Kepler's Second Law, showing the area that is ``swept'' by a planet in a fixed period of time. }

In particular, Kepler's Second Law states that the two areas that are shown in the figure will have the same area if the planet took the same amount of time to travel between points $A$ and $B$ as it did to travel between points $C$ and $D$. Because the points $C$ and $D$ are further away from the Sun than points $A$ and $B$, the distance between points $C$ and $D$ must be smaller than the distance between points $A$ and $B$ for the two areas to be the same. This, in turn, implies that the planet must be moving slower between $C$ and $D$ than between points $A$ and $B$. As we will see in a later chapter, Kepler's Second Law is equivalent to the statement that the angular momentum of the planet is conserved. 

TODO: Checkpoint: At which point in the orbit is the speed of the planet the greatest?

\subsection{Kepler's Third Law}
Kepler's Third Law is quantitative and relates the orbital periods ($T$) and the semi-major axes ($s$) between any two planets in orbit around the Sun:
\begin{align*}
\left(\frac{T_1}{T_2}\right)^2=\left(\frac{s_1}{s_2}\right)^3
\end{align*}
We can re-arrange this relation so that all of the quantities related to one planet are on the same side of the equal sign:
\begin{align*}
\frac{T_1^2}{s_1^3}=\frac{T_2^2}{s_2^3}=\text{constant}
\end{align*}
In other words, the ratio between the orbital period squared and the semi-major axis cubed is a constant for all planets which cannot depend on the planet itself. Later in this chapter, we will use Newton's Universal Theory of Gravity to evaluate the constant, and we will find that it depends on properties of the Sun. 

TODO: Checkpoint MC if you double the radius of a circular orbit, what happens to the orbital speed?

\section{Newton's Universal Theory of Gravity}
Newton supposedly gained insight into the gravitational force by observing an apple falling from a tree and concluding that if it is the same force that makes apples fall at sea level and at the top of the mountain, perhaps that force can be exerted all the way up to the moon. It is rather remarkable that Newton was able to make the connection between falling apples and the motion of the moon around the Earth to find a single theory to describe both situations.

We should be clear that the theory of gravity is a different theory than Newton's ``Laws of Motion'' (Newton's Three Laws). The Laws of Motion introduce the concepts of force and inertial mass, and prescribe how to use those concepts in order to model motion using kinematics. Newton's Universal Theory of Gravity is a theory that describes the force of gravity that two bodies with (gravitational) mass exert on each other.

Newton's Universal Theory of Gravity states that if two bodies with masses $M_1$ and $M_2$, located at positions $\vec r_1$ and $\vec r_2$, respectively, are separated by a distance, $r$, then $M_2$ will exert an attractive force on $M_1$, $\vec F_{12}$, given by:
\begin{align}
\vec F_{12}=-G\frac{M_1M_2}{r^2}\hat r_{21}
\end{align}
where $\hat r_{21}$ is the unit vector from $M_2$ to $M_1$:
\begin{align*}
\vec r_{21} &= \vec r_2 - \vec r_1\\
\hat r_{21} &= \frac{1}{r} \vec r_{21}
\end{align*}
as shown in Figure \ref{fig:gravity:gvectors}. $\vec F_{12}$ should be read as ``the force on body 1 from body 2''. $G=\SI{6.67e-11}{Nm^2/kg^2}$ is Newton's Universal Constant of Gravity. It should be noted that Newton's theory for the force of gravity written in this form only applies to either point masses (separated by a distance $r$) or spherical bodies whose centres are separated by a distance $r$
\capfig{0.4\textwidth}{figures/Gravity/gvectors.png}{\label{fig:gravity:gvectors}Illustration of the vectors involved in Newton's Universal Theory of Gravity.}

Originally, Newton argued that the force of gravity would be proportional to the masses of the bodies, and inversely proportional to the square of the distance between them:
\begin{align*}
F_{12}\propto \frac{M_1M_2}{r^2}
\end{align*}
and $G$ is simply the constant of proportionality.

Because of Newton's Third Law, body 1 exerts a force on body 2 that is equal in magnitude but opposite in direction:
\begin{align*}
\vec F_{12} = -\vec F_{21}
\end{align*} 

\begin{example}{Calculate the magnitude of the force of gravity between yourself and a person standing $\SI{50}{cm}$ from you and compare that to your weight at the surface of the Earth (the force of gravity exerted by the Earth on you).}
If we assume that the two people have a mass of $\SI{50}{kg}$, the force of gravity exerted by one on the other, if they are separated by $\SI{50}{cm}$, is given by:
\begin{align*}
F=G\frac{M_1M_2}{r^2}=(\SI{6.67e-11}{Nm^2/kg^2})\frac{(\SI{50}{kg})(\SI{50}{kg})}{(\SI{0.5}{m})^2}=\SI{6.67e-7}{N}
\end{align*}
This is a very small force, compared to their weight, $F_g$:
\begin{align*}
F_g=M_1g=(\SI{50}{kg})(\SI{9.8}{kg/N})=\SI{490}{N}
\end{align*}
which is approximately 700 million times bigger. 

\textbf{Discussion:} The force of gravity is a very weak force when compared to other forces in Nature, such as the electric force between two charges. Newton's Universal Constant of Gravity is very small, so the force of gravity between two objects is very small unless either of the masses involved are very large, or the distance between them is very small. In general, when modelling the motion of objects on the Earth, it is generally safe to ignore the forces of gravity between objects and only include their weight (the force of gravity from the Earth). 
\end{example}

TODO: Checkpoint question MC: Choose the order of magnitude that corresponds to the mass of the Earth.

\begin{example}{Geosynchronous satellites are satellites that are placed in a circular orbit around the Earth in such a way that their orbital period is synchronized with the $\SI{24}{h}$ rotation of the Earth about the axis that goes through the poles. The advantage of geosynchronous satellites is that they are always above the same point on Earth, which makes them useful for establishing communication networks. At what altitude must geosynchronous satellites be placed?}
When a satellite orbits the Earth, the only force on the satellite is the force of gravity from the Earth. Since the satellite is in a circular orbit, that force of gravity must point towards the centre of the Earth in such a way that the satellite has the correct radial acceleration, $a_R$, to stay in uniform circular motion:
\begin{align*}
a_r=\frac{v^2}{R}
\end{align*}
where $v$ is the speed of the satellite, and $R$ is the distance between the satellite and the centre of the Earth (i.e. the centre of the circle). The magnitude of the force of gravity on the satellite of mass $m$ is given by:
\begin{align*}
F = G\frac{Mm}{R^2}
\end{align*}
where $M$ is the mass of the Earth. The radial component\footnote{The force of gravity and the acceleration are both co-linear and along the line that joins the centre of the Earth to the satellite.} of Newton's Second Law written for the satellite is given by:
\begin{align*}
\sum F_r = F &= ma_r\\
\therefore G\frac{Mm}{R^2}&=m\frac{v^2}{R}
\end{align*}
The speed of the satellite can be found from the fact that it must travel a distance of $2\pi r$ in a period $T=\SI{24}{h}$:
\begin{align*}
v=\frac{2\pi R}{T}
\end{align*}
which we can substitute into the equation from Newton's Second Law to find the distance $R$:
\begin{align*}
G\frac{Mm}{R^2}&=m\frac{v^2}{R}\\
G\frac{M}{R^2}&=\frac{(2\pi R)^2}{T^2R}\\ 
G\frac{M}{R^2}&=\frac{4\pi^2 R}{T^2}\\ 
\therefore R&=\sqrt[3]{G\frac{MT^2}{4\pi^2}}\\
&=\sqrt[3]{(\SI{6.67e-11}{Nm^2/kg^2})\frac{(\SI{5.97e24}{kg})(\SI{86400}{s})^2}{4\pi^2}}\\
&=\SI{42.2e6}{m}
\end{align*}
which corresponds to the distance between the satellite and the centre of the Earth. To obtain the ``altitude'', $h$, namely the distance from the surface of the Earth to the satellite, we must subtract the radius of the Earth, $R_\oplus=\SI{6.371e6}{m}$ from this distance:
\begin{align*}
h = R-R_\oplus = \SI{35.9e6}{m}
\end{align*}
Thus, geosynchronous satellites are located at an altitude of approximately $\SI{36000}{km}$.
\end{example}

\begin{example}{\label{ex:gravity:gofr}The acceleration due to Earth's gravity depends on the force that the Earth exerts on an object. Using the Earth's mass and radius, determine the acceleration due to Earth gravity at the surface of the Earth. Also, determine the altitude where the acceleration due to Earth's gravity is half of that at the Earth's surface.}
We can find the acceleration due to Earth gravity by determining the acceleration of a mass $m$ upon which gravity is the only acting force. In other words, we model an object that is in free-fall a distance $R$ away from the centre of the Earth. Newton's Second Law can be used in one dimension (corresponding to the direction that connects the falling mass to the centre of the Earth):
\begin{align*}
\sum F &= G\frac{Mm}{R^2}=ma\\
\therefore a&=G\frac{M}{R^2}
\end{align*}
where $M=\SI{5.97e24}{kg}$ is the mass of the Earth. At the surface of the Earth, $R=R_\oplus=\SI{6.371e6}{m}$:
\begin{align*}
a&=G\frac{M}{R_\oplus^2}=(\SI{6.67e-11}{Nm^2/kg^2})\frac{(\SI{5.97e24}{kg})}{(\SI{6.371e6}{m})^2}\\
&=\SI{9.81}{m/s^2}
\end{align*}
which, of course, is the value of $g$ that we have been using so far for the acceleration due to gravity near the surface of the Earth. To find the altitude at which this is reduced by half, we first find the value of $R$ that corresponds to this acceleration:
\begin{align*}
\frac{1}{2}G\frac{M}{R_\oplus^2}&=G\frac{M}{R^2}\\
\therefore R &=\sqrt{2}R_\oplus = \SI{9.0e6}{m}
\end{align*}
which corresponds to an altitude of $h=R-R_\oplus=\SI{2640}{km}$, which is well above the Earth's atmosphere.

\textbf{Discussion:} Very precise measurements of the acceleration due to Earth's gravity can be used to determine the shape of the Earth (which is not a perfect sphere). 
\end{example}

\subsection{Weight and apparent weight}
You have probably seen images of astronauts floating around the International Space Station (ISS) or other orbiting vessels, and heard of the term ``weightlessness''  to describe their motion. The ISS is in orbit at an altitude of approximately $\SI{400}{km}$ where the force of Earth's gravity is far from negligible (in Example \ref{ex:gravity:gofr} we showed that one needs to go to an altitude of $\SI{2640}{km}$ for the force to be reduced by half of that at the surface of the Earth). The contradiction is resolved by understanding that the popular term ``weightless'' is imprecise from a physics perspective, since the astronauts clearly still have a non-negligible weight from Earth's gravity.

The correct term to use from a physics perspective is to say that the \textit{apparent weight} of the astronauts is zero when they are floating around. Weight is the magnitude of the force of gravity exerted by the Earth. Apparent weight is the force that one measures when standing on a spring scale, which is equal to the normal force exerted by the spring scale on the person. 

Consider a person standing on a spring scale at the North pole of the Earth (top free-body diagram in Figure \ref{fig:gravity:apparentweight}). The only two forces exerted on the person are their weight, $\vec F_g$, and the normal force, $\vec N$, exerted by the spring scale. Since the person is not accelerating, the normal force and the weight have the same magnitude and opposite directions. The scale will thus read the actual weight of the person.

\capfig{0.4\textwidth}{figures/Gravity/apparentweight.png}{\label{fig:gravity:apparentweight}The apparent weight, given by the normal force, is different at the Earth's equator because a person's acceleration is non-zero as they rotate with the Earth.}

Consider, instead, a person standing on a spring scale at the equator of the Earth (Figure \ref{fig:gravity:apparentweight}). That person is accelerating, because they are rotating with the Earth in uniform circular motion. Again, the only forces exerted on the person are their weight and the normal force exerted by the scale. The sum of the forces must now be equal to the person's mass, $m$, times the radial acceleration, $a_r$, that is necessary for them to follow the surface of the Earth as the Earth rotates about its axis. Newton's Second Law for the person allows us to find the magnitude of the normal force:
\begin{align*}
\sum F &= F_g-N=ma_r=m\frac{v^2}{R}\\
\therefore N &= F_g - m\frac{v^2}{R}\\
&=G\frac{Mm}{R^2} -  m\frac{v^2}{R}\\
&=m\left(G\frac{M}{R^2} - \frac{v^2}{R}  \right)\\
&=m\left(g - \frac{v^2}{R}  \right)
\end{align*}
where $M$ is the mass of the Earth, $R$ is the radius of the Earth, $v$ is the speed at the surface of the Earth due to the Earth's rotation, and, in the last line, we used the result from Example \ref{ex:gravity:gofr} where we had determined the value of $g$ in terms of the mass and radius of the Earth. We see that the normal force is reduced compared to what it would be if the Earth were not rotating ($v=0$) or if one is standing at one of the poles. Your apparent weight, which you can measure by standing on a spring scale, is thus smaller at the equator than it is at the poles. The quantity in parenthesis can be thought of as a modified or ``effective'' value of $g$ at the equator. To an observer at the equator, unaware of the Earth's rotation, the acceleration of objects as they fall towards the ground is smaller than at the poles. 

TODO: Checkpoint MC, what is the effective value of $g$ at the equator? (Need to figure out v...). 

If you are circling the Earth a distance $R$ from the centre of the Earth at a constant speed $v$, it is possible for you apparent weight to be zero. Imagine standing on a scale in an aircraft that is circling the Earth and measuring your apparent weight with the spring scale. As the speed of the aircraft increases, your apparent weight decreases:
\begin{align*}
N=m\left(G\frac{M}{R^2} - \frac{v^2}{R}  \right)
\end{align*}
At a certain speed, $v$, you apparent weight is zero and you feel weightless:
\begin{align*}
G\frac{M}{R^2} &= \frac{v^2}{R}\\
\therefore v&= \sqrt{G\frac{M}{R} }
\end{align*}
which is what we call being ``in circular orbit'' around the Earth. That is, your acceleration is exactly that which is due to gravity, since gravity is the only force that is acting on you. Another way to feel weightless is when you are in free-fall (e.g. the first few seconds of a parachute jump from an airplane). One can think of being in orbit as continuously falling towards the centre of the Earth, but with an initial velocity in a direction such that you never actually collide with the Earth. The feeling of weightlessness will occur any time that the only force exerted on you is the force of gravity, which does not require you to be in a circular orbit. If you are in a spacecraft and the only force on the spacecraft is from gravity (i.e. no rockets or wings are exerting any forces), then everything in the spacecraft will have the same acceleration, since gravity is the only force acting on anything in the spacecraft, and it will appear that everything is just floating in the spacecraft. To an outside observer, it would obviously be clear that the spacecraft and its contents are all accelerating.

At any position that is not on the equator or the poles, the direction of the effective acceleration due to Earth's gravity is not directed towards the centre of the Earth. If you hang a mass from a string (creating a plumb line), the string will point slightly away from the centre of the Earth. This is illustrated in Figure \ref{fig:gravity:apparentweight2} which shows the forces of tension and gravity that are exerted on a mass hanging from a string located away from the equator or poles. The tension in the string cannot point towards the centre of the Earth, because the net force must point towards to centre of the circle about which that location on Earth is rotating. 
\capfig{0.4\textwidth}{figures/Gravity/apparentweight2.png}{\label{fig:gravity:apparentweight2}Away from the equator and poles, a plumb line does not point towards the centre of the Earth.}

TODO: Question library question about finding the angle between the plumbline and the true vertical at some location (give latitude so that they have to figure out how $\theta$ relates to latitude (it is latitude)

\subsection{The gravitational field}

\subsection{Gauss' Law}

\section{Gravitational potential energy}

\subsection{Mechanical energy with gravity}

\subsection{Satellite orbits}
discuss apparent weight.

\subsection{Escapge velocity}

\section{Einstein's Theory of General Relativity}

\newpage
\section{Summary}

\begin{chapterSummary}
We learned about gravity.
\end{chapterSummary}

\newpage
\begin{importantEquations}
This is an important equation
\begin{align*}
E = mc^2
\end{align*}

\end{importantEquations}


\newpage
\section{Thinking about the material}
\subsection{Reflect and research}

\begin{enumerate}
\item When you look at the night sky, how can you tell the difference between a planet and a star?
\item What was the relationship between Tycho Brahe and Johannes Kepler?
\item How did Tycho Brahe collect all the data that Kepler used?
\item How much time elapsed between Kepler publishing his laws and Newton publishing his Universal Theory of Gravity?
\item What was Kepler's original intention when he synthesized Tycho Brahe's observations? What was he hoping to show?
\item What was Ptolemy's theory of gravity based upon?
\item Who was the first to suggest that planets revolved around the Sun instead of the Earth?
\item Explain how the force of gravity from the moon results in tides on both sides of the Earth.
\end{enumerate}
\subsection{To try at home}

\begin{tQuestion}Try doing this \end{tQuestion}

\subsection{To try in the lab}
Theory project: Prove, based on Newton's Universal Theory of Gravity, that the motion of orbiting bodies is given by a conical section. You may find the Lagrangian formulation to be easier for this task!

\newpage
\section{Sample problems and solutions}
\subsection{Problems}
\begin{problemParts}{A question\label{Q:chaptertitle:q1}}
\item How close can he get to the hurdle before he has to jump?
\item What maximum height does he reach?
\end{problemParts}

\newpage
\subsection{Solutions}
\begin{solution}{\ref{Q:chaptertitle:q1}}
{
the solution
}
\end{solution}

