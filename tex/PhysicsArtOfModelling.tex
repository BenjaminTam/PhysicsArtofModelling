\documentclass[12pt]{book}
\usepackage{comment}
\specialcomment{answer}{\textbf{Answer: }}{}


%Decide whether to show answers:
%Hides the answers to short answer questions
%\excludecomment{answer}
%Shows multiple choice answers
\newcommand{\correct}{\textbf{(Correct)}}
%Hides multiple choice answers
%\newcommand{\correct}{}


\usepackage{mathtools} % for \Aboxed
\usepackage{paralist}
\usepackage{calc}
\usepackage{subfig}
\usepackage{setspace}
\usepackage{amssymb}
\usepackage{amsmath}
\usepackage{amstext}
\usepackage[font={small,it}]{caption}
\usepackage[pdftex]{graphicx} %Does not work in pressbooks!!!
\usepackage{fancyhdr,lastpage}
\usepackage{url}
\usepackage{longtable}
\usepackage{comment}
\usepackage{ifthen}
\usepackage{color}
\usepackage[colorlinks=true,linkcolor=blue]{hyperref}
\usepackage[explicit]{titlesec}
\usepackage{lmodern}
\usepackage{listings}
\usepackage{parskip}
\usepackage[table]{xcolor}
\usepackage{enumitem}
\usepackage{wrapfig}
\usepackage[framemethod=TikZ]{mdframed}
\usepackage{titlesec} %for spacing around titles
\usepackage{caption}
\usepackage[separate-uncertainty = true]{siunitx}
\usepackage{float}
\restylefloat{table}
\usepackage{multicol}
%\lstset{language=Python,showstringspaces=false,commentstyle=} 

%%Some math and other shortcuts
\newcommand{\chloe}{Chlo\"e~}
\newcommand{\die}[2]{\frac{\partial #1}{\partial #2}}
\newcommand{\ddt}{\frac{d}{dt}}
\newcommand{\lagd}{\mathcal{L}}
\newcommand{\code}[1]{\texttt{#1}}

\newcommand{\pvec}[1]{\vec{#1}\mkern2mu\vphantom{#1}} % for a vector of a primed quantity, e.g. \vec p ', should be \pvec p'

%Colours
\definecolor{mygreen}{rgb}{0.2,0.6,0}
\definecolor{TBgreen}{rgb}{0.0,0.5,0.0}
\definecolor{TBblue}{rgb}{0.0, 0.58, 0.71}
\definecolor{TBred}{rgb}{0.8, 0.25, 0.33}
\definecolor{TBorange}{rgb}{0.89, 0.35, 0.13}

%Stuff for writing code:

\lstset{ %
  belowskip=0pt,
  aboveskip=0pt,
  caption=\relax,
  backgroundcolor=\color{white},   % choose the background color; you must add \usepackage{color} or \usepackage{xcolor}
  basicstyle=\footnotesize,        % the size of the fonts that are used for the code
  breakatwhitespace=false,         % sets if automatic breaks should only happen at whitespace
  breaklines=true,                 % sets automatic line breaking
  captionpos=t,                    % sets the caption-position to bottom
  commentstyle=\color{mygreen},    % comment style
  deletekeywords={...},            % if you want to delete keywords from the given language
  escapeinside={(*}{*)},          % if you want to add LaTeX within your code
  extendedchars=true,              % lets you use non-ASCII characters; for 8-bits encodings only, does not work with UTF-8
  frame=none,	                   % adds a frame around the code
  keepspaces=true,                 % keeps spaces in text, useful for keeping indentation of code (possibly needs columns=flexible)
  keywordstyle=\color{blue},       % keyword style
  language=Python,                 % the language of the code
  otherkeywords={*,...},           % if you want to add more keywords to the set
  numbers=none,                    % where to put the line-numbers; possible values are (none, left, right)
  numbersep=5pt,                   % how far the line-numbers are from the code
  numberstyle=\tiny\color{black}, % the style that is used for the line-numbers
  rulecolor=\color{black},         % if not set, the frame-color may be changed on line-breaks within not-black text (e.g. comments (green here))
  showspaces=false,                % show spaces everywhere adding particular underscores; it overrides 'showstringspaces'
  showstringspaces=false,          % underline spaces within strings only
  showtabs=false,                  % show tabs within strings adding particular underscores
  stepnumber=1,                    % the step between two line-numbers. If it's 1, each line will be numbered
  stringstyle=\color{red},     % string literal style
  tabsize=2,	                   % sets default tabsize to 2 spaces
  title=\lstname                   % show the filename of files included with \lstinputlisting; also try caption instead of title
}

%Environments for writing code
\DeclareCaptionFont{white}{\color{white}}
\DeclareCaptionFormat{listing}{\colorbox{gray}{\parbox{\textwidth}{#1#2#3}}}

\captionsetup[lstlisting]{format=listing,labelfont=white,textfont=white}
\renewcommand{\lstlistingname}{Python Example}
\renewcommand{\lstlistlistingname}{List of \lstlistingname s}

\lstnewenvironment{python}[1][]{
  \lstset{#1, language=Python}%
  \renewcommand\lstlistingname{Python Code}
}{}

\lstnewenvironment{poutput}{
 \lstset{caption=\mbox{}, language=,aboveskip=-3pt}
 \addtocounter{lstlisting}{-1}
 \renewcommand\lstlistingname{Output}
}{}


%%Pretty chapter headings:
\newlength\chapnumb
\setlength\chapnumb{4cm}

\titleformat{\chapter}[block]
{\normalfont\sffamily}{}{0pt}
{\parbox[b]{\chapnumb}{%
   \fontsize{120}{110}\selectfont\thechapter}%
  \parbox[b]{\dimexpr\textwidth-\chapnumb\relax}{%
    \raggedleft%
    \hfill{\LARGE#1}\\
    \rule{\dimexpr\textwidth-\chapnumb\relax}{0.4pt}}}
\titleformat{name=\chapter,numberless}[block]
{\normalfont\sffamily}{}{0pt}
{\parbox[b]{\chapnumb}{%
   \mbox{}}%
  \parbox[b]{\dimexpr\textwidth-\chapnumb\relax}{%
    \raggedleft%
    \hfill{\LARGE#1}\\
    \rule{\dimexpr\textwidth-\chapnumb\relax}{0.4pt}}}


%%%spacing around titles
\setlength{\parindent}{0pt}
\parskip = \baselineskip

%spacing around captions (e.g. caption after a table)
\captionsetup{belowskip=6pt,aboveskip=4pt}

%\titlespacing*{\chapter}
%{0pt}{0ex}{0ex}
\titlespacing*{\section}
{0pt}{4pt-\parskip}{2pt-\parskip}
\titlespacing*{\subsection}
{0pt}{4pt-\parskip}{1pt-\parskip}
\titlespacing*{\subsubsection}
{0pt}{4pt-\parskip}{1pt-\parskip}

%%% Spacing in lists:
\setlist{nosep}

%%Verticall spacing between table rows
\renewcommand{\arraystretch}{1.5}

\setlength{\intextsep}{12pt}

%space before itemized list:
%\setlength{\topsep}{-10pt} %does nothing?

%%Simplifed figure environment:

\newenvironment{capfig}[3]{\begin{center}\includegraphics[width=#1]{#2}\captionof{figure}{#3}\end{center}}{}


%Wrap figure environments (right or left). Argument #1 (default value 12, specified as optional), is the number of 
%lines that the figure should take.
%space around wrap figures:
%\setlength{\intextsep}{20pt}%
%\setlength{\columnsep}{5pt}%
\newenvironment{Rwcapfig}[4][0]{
\begingroup
%\setlength{\intextsep}{0pt}%
\setlength{\columnsep}{10pt}%
\begin{wrapfigure}[#1]{R}{#2}\centering\includegraphics[width=#2]{#3}\caption{#4}\end{wrapfigure}}{\endgroup}

\newenvironment{rwcapfig}[4][0]{
\begingroup
%\setlength{\intextsep}{0pt}%
\setlength{\columnsep}{10pt}%
\begin{wrapfigure}[#1]{r}{#2}\centering\includegraphics[width=#2]{#3}\caption{#4}\end{wrapfigure}}{\endgroup}

\newenvironment{Lwcapfig}[4][0]{
\begingroup
%\setlength{\intextsep}{0pt}%
\setlength{\columnsep}{10pt}%
\begin{wrapfigure}[#1]{L}{#2}\centering\includegraphics[width=#2]{#3}\caption{#4}\end{wrapfigure}}{\endgroup }

\newenvironment{lwcapfig}[4][0]{
\begingroup
%\setlength{\intextsep}{0pt}%
\setlength{\columnsep}{10pt}%
\begin{wrapfigure}[#1]{l}{#2}\centering\includegraphics[width=#2]{#3}\caption{#4}\end{wrapfigure}}{\endgroup }

%Environments (in the works) %%%%%%%%%%%%%%%%%%%%%%%%%%%%%%%%%%%%%%%%%%%%%%%%%%%%%%%%%%%%%%%%%%%%%%%%%%%%%%%%%%%%%%%%%%%%%%%%%%%%%%%%%%%%%%%%%%%%%%%%%%%%%%%%%%%%%%%%%%%%%%%%%%%%%%%
%% Environments to Keep
%%Learning Objectives:
\newenvironment{learningObjectives}[1]{%
\mdfsetup{%
    frametitle={%
        \tikz[baseline=(current bounding box.east),outer sep=0pt]
        \node[anchor=west,rectangle,fill=TBblue, minimum width=5cm, rounded corners=0.2cm]
        {\strut Learning Objectives};},
    frametitlefont=\color{white}\sffamily\bfseries,
    innertopmargin=0pt,linecolor=TBblue,%
    linewidth=1pt,topline=true,%
    frametitleaboveskip=\dimexpr-\ht\strutbox\relax%
}
\vspace{10pt}
\begin{mdframed}\begin{itemize}[label=\textcolor{TBblue}{\textbullet}] #1 \end{itemize}}{\end{mdframed}}


%%Opening Question:
\newenvironment{opening}[1]{%
\mdfsetup{%
    frametitle={%
        \tikz[baseline=(current bounding box.east),outer sep=0pt]
        \node[anchor=west,rectangle,fill=TBblue, minimum width=5cm,rounded corners=0.2cm]
        {\strut Think About It};},
    frametitlefont=\color{white}\sffamily\bfseries,
    innertopmargin=0pt,linecolor=TBblue,backgroundcolor=TBblue!5,%
    linewidth=1pt,topline=true,%
    frametitleaboveskip=\dimexpr-\ht\strutbox\relax%
}
\vspace{10pt}
\begin{mdframed}[nobreak=true]\relax #1}{%
\end{mdframed}}

%Problems
\newcounter{problem}[chapter]
\def\theproblem{\thechapter-\arabic{problem}}

\newenvironment{problem}[2]
  {\refstepcounter{problem}\textbf{Problem \theproblem: }#2 (\hyperref[#1]{Solution})} %
  {\vspace{2ex}}
  
\newenvironment{problemParts}[2]
  {\refstepcounter{problem}\textbf{Problem \theproblem: }#2 (\hyperref[#1]{Solution}) %
  \begin{enumerate}[label=\alph*),topsep=-10pt]}%
   {\end{enumerate}}
  {\vspace{2ex}}
  
\newenvironment{MCquestion}[1]{#1%
   \begin{enumerate}[label=\Alph*),topsep=-10pt]}{%
   \end{enumerate}}

%Example Box with counter  
\newcounter{example}[chapter]
\def\theexample{\thechapter-\arabic{example}}
\newenvironment{example}[2]{% 
\refstepcounter{example}%
\mdfsetup{%
    frametitle={% 
        \tikz[baseline=(current bounding box.east),outer sep=0pt]
        \node[anchor=west,rectangle,fill=TBred, minimum width=5cm,rounded corners=0.2cm]
        {\strut Example~\theexample};},
    frametitlefont=\color{white}\sffamily\bfseries,
    skipabove=\strutbox,
    innertopmargin={0.5cm},
    linecolor=TBred,%
    linewidth=1pt,topline=true,%
    frametitleaboveskip=\dimexpr-\ht\strutbox\relax%
    %frametitleaboveskip=-\strutbox
}
\begin{mdframed}[]\relax #1 \leavevmode \\ \newline { \color{TBred}\textsf{\large Solution}} \\ {\color{TBred}\rule[10pt]{\textwidth}{1pt}} \\ #2}{%
\end{mdframed}}

%Review Box
\newenvironment{review}[1]{%
\mdfsetup{%
    frametitle={%
        \tikz[baseline=(current bounding box.east),outer sep=0pt]
        \node[anchor=west,rectangle,fill=TBorange, minimum width=5cm,rounded corners=0.2cm]
        {\strut Review Topics};},
    frametitlefont=\color{white}\sffamily\bfseries,
    innertopmargin=0pt,linecolor=TBorange,backgroundcolor=TBorange!5,%
    linewidth=1pt,topline=true,%
    frametitleaboveskip=\dimexpr-\ht\strutbox\relax%
}
\vspace{10pt}
\begin{mdframed}[nobreak=true]\relax #1}{%
\end{mdframed}}

%Reflect and Research
\newenvironment{reflectresearch}[1]{%
\mdfsetup{%
    frametitle={%
        \tikz[baseline=(current bounding box.east),outer sep=0pt]
        \node[anchor=west,rectangle,fill=TBgreen!80, minimum width=5cm,rounded corners=0.2cm]
        {\strut Reflect and Research};},
    frametitlefont=\color{white}\sffamily\bfseries,
    innertopmargin=0pt,linecolor=TBgreen!80,backgroundcolor=TBgreen!5,%
    linewidth=1pt,topline=true,%
    frametitleaboveskip=\dimexpr-\ht\strutbox\relax%
}
\vspace{10pt}
\begin{mdframed}\begin{enumerate}[itemsep=1ex] #1
   \end{enumerate}}
{\end{mdframed}}



%%Checkpoint question in a box, with counter:
\newcounter{checkpoint}[chapter]
\def\thecheckpoint{\thechapter-\arabic{checkpoint}}

\newenvironment{checkpoint}[1]{%
\refstepcounter{checkpoint}%
\mdfsetup{%
    frametitle={%
        \tikz[baseline=(current bounding box.east),outer sep=0pt]
        \node[anchor=west,rectangle,fill=TBgreen!80, minimum width=5cm,rounded corners=0.2cm]
        {\strut Checkpoint~\thecheckpoint};},
    frametitlefont=\color{white}\sffamily\bfseries,
    innertopmargin=0pt,linecolor=TBgreen!80,backgroundcolor=TBgreen!5,%
    linewidth=1pt,topline=true,%
    frametitleaboveskip=\dimexpr-\ht\strutbox\relax%
}
\vspace{10pt}
\begin{mdframed}[nobreak=true]\relax #1}{%
\end{mdframed}}


%Student Opinion with option for student name
\newenvironment{studentOpinion}[2]{%
\mdfsetup{%
    frametitle={%
        \tikz[baseline=(current bounding box.east),outer sep=0pt]
        \node[anchor=west,rectangle,fill=TBorange, minimum width=5cm,rounded corners=0.2cm]
        {\strut #1's Thoughts};},
    frametitlefont=\color{white}\sffamily\bfseries,
    linecolor=TBorange,%
    linewidth=1pt,topline=true,
    skipabove=\strutbox,
    innertopmargin={0.5cm},%
    frametitleaboveskip=\dimexpr-\ht\strutbox\relax%
}
\vspace{10pt}
\begin{mdframed}[]\relax #2}{%
\end{mdframed}}

%%Important Equations:
\newenvironment{importantEquations}[1]{%
\mdfsetup{%
    frametitle={%
        \tikz[baseline=(current bounding box.east),outer sep=0pt]
        \node[anchor=west,rectangle,fill=TBblue, minimum width=5cm,rounded corners=0.2cm]
        {\strut Important Equations};},
    frametitlefont=\color{white}\sffamily\bfseries,
    innertopmargin=0pt,linecolor=TBblue,%
    linewidth=1pt,topline=true,%
    frametitleaboveskip=\dimexpr-\ht\strutbox\relax%
}
\vspace{10pt}
\begin{mdframed}[nobreak=false]\relax #1}{%
\end{mdframed}}

%%Key Takeaways:
\newenvironment{chapterSummary}[1]{%
\mdfsetup{%
    frametitle={%
        \tikz[baseline=(current bounding box.east),outer sep=0pt]
        \node[anchor=west,rectangle,fill=TBblue, minimum width=5cm,rounded corners=0.2cm]
        {\strut Key Takeaways};},
    frametitlefont=\color{white}\sffamily\bfseries,
    innertopmargin=0pt,linecolor=TBblue,backgroundcolor=TBblue!5,%
    linewidth=1pt,topline=true,%
    frametitleaboveskip=\dimexpr-\ht\strutbox\relax%
}
\vspace{10pt}
\begin{mdframed}[]\relax #1}{%
\end{mdframed}}

\newcounter{solution}[chapter]
\def\thesolution{\thechapter-\arabic{solution}}

\newenvironment{solution}[2]{\refstepcounter{solution}\textbf{Solution to problem \ref{#1}:} #2}
{\vspace{2ex}}

%%Reflect and Research Questions with Counter
\newcounter{tQuestion}[chapter]
\def\therQuestion{\thechapter-\arabic{tQuestion}}

\newenvironment{tQuestion}[1]{\refstepcounter{tQuestion}%
    \textbf{Activity~\therQuestion: }#1}
          
\usepackage[paper=letterpaper,
            %includefoot, % Uncomment to put page number above margin
            marginparwidth=.0in,     % Length of section titles
            marginparsep=.05in,       % Space between titles and text
            margin=1in,               % 1 inch margins
            includemp]{geometry}

\setcounter{secnumdepth}{2}
\setcounter{tocdepth}{3}

\begin{document}
\title{Introductory physics: The art of building models}
\author{Ryan D. Martin}
\pagenumbering{roman}
\maketitle
\tableofcontents
\pagenumbering{arabic}

%%Copyright 2017 R.D. Martin
%This book is free software: you can redistribute it and/or modify it under the terms of the GNU General Public License as published by the Free Software Foundation, either version 3 of the License, or (at your option) any later version.
%
%This book is distributed in the hope that it will be useful, but WITHOUT ANY WARRANTY; without even the implied warranty of MERCHANTABILITY or FITNESS FOR A PARTICULAR PURPOSE.  See the GNU General Public License for more details, http://www.gnu.org/licenses/.
\chapter{The Scientific Method and Physics}
\label{Introduction}

\begin{checkpointMC}{A good scientific theory...}
\item Must explain the physical world; may or may not be experimentally verifiable.
\item Prove our models to be correct; must be experimentally verifiable.
\item Describe the physical world; must be experimentally verifiable.
\item Must disprove other theories; may or may not be experimentally verifiable.
\end{checkpointMC}


\begin{learningObjectives}
\item Understand the Scientific Method
\item Define the scope of Physics
\item Understand the difference between theory and model
\item Have a sense of how a physicist thinks
\end{learningObjectives}

\section{Science and the Scientific Method}
Science is an attempt to \textit{describe} the world around us. It is important to note that describing the world around us is not the same as \textit{explaining} the world around us. Science aims to answer the question ``How?'' and not the question ``Why?''. As we develop our description of the physical world, you should remember this important distinction and resist the urge to ask ``Why?''.

The Scientific Method is a prescription for coming up with a description of the physical world that anyone can challenge and improve through performing experiments. If we come up with a description that can describe many observations, or the outcome of many different experiments, then we usually call that description a ``Scientific Theory''. We can get some insight into the Scientific Method through a simple example. 

Imagine that we wish to describe how long it takes for a tennis ball to reach the ground after being released from a certain height. One way to proceed is to describe how long it takes for a tennis ball to drop \SI{1}{\meter}, and then to describe how long it takes for a tennis ball to drop \SI{2}{\meter}, etc. We could generate a giant table showing how long it takes a tennis ball to drop from any given height. Someone would then be able to perform an experiment to measure how long a tennis ball takes to drop \SI{1}{\meter} or \SI{2}{\meter} and see if their measurement is consistent with the tabulated values. If we collected the descriptions for all possible heights, then we would effectively have a valid and testable scientific theory that describes how long it takes tennis balls to drop from any height.

Suppose that a budding scientist, let's call her Chlo\"e, then came along and noticed that there is a pattern in the theory that can be described much more succinctly and generally than by using a giant table. In particular, suppose that she notices that, mathematically, the time, $t$, that it takes for a tennis ball to drop a height, $h$, is proportional to the square root of the height:
\begin{equation*}
t \propto \sqrt{h}
\end{equation*}

\begin{example}{Use Chlo\"e's Theory ($t \propto \sqrt{h}$) to determine how much longer it will take for an object to drop by \SI{2}{\meter} than it would to drop by \SI{1}{\meter}:}
When we have a proportionality law (with a $\propto$) sign, we can always change this to an equal sign by introducing a constant, which we will call $k$:
\begin{align*}
t &\propto \sqrt{h} \\
\rightarrow t&=k\sqrt{h}
\end{align*}
Let $t_1$ be the time to fall a distance $h_1=\SI{1}{\meter}$, and $t_2$ be the time to fall a distance $h_2=\SI{2}{\meter}$. In terms of our unknown constant, $k$, we have:
\begin{align*}
t_1 &=k\sqrt{h_1}=k \sqrt{(\SI{1}{\meter})}\\
t_2 &=k\sqrt{h_2}=k \sqrt{(\SI{2}{\meter})}\\
\end{align*}
By taking the ratio, $\frac{t_1}{t_2}$, our unknown constant $k$ will cancel:
\begin{align*}
\frac{t_1}{t_2}&=\frac{\sqrt{(\SI{1}{\meter})}}{\sqrt{(\SI{2}{\meter})}}=\frac{1}{\sqrt 2}\\
\therefore t_2 &= \sqrt{2} t_1
\end{align*}
and we find that it will take $\sqrt{2}\sim 1.41$ times longer to drop by \SI{2}{\meter} than it will by \SI{1}{\meter}.
\end{example}

Chlo\"e's ``Theory of Tennis Ball Drop Times'' is appealing because it is succinct, and it also allows us to make \textbf{verifiable predictions}. That is, using this theory, we can predict that it will take a tennis ball $\sqrt 2$ times longer to drop from \SI{2}{\meter} than it will from \SI{1}{\meter}, and then perform an experiment to verify that prediction. If the experiment agrees with the prediction, then we conclude that Chlo\"e's theory adequately describes the result of that particular experiment. If the experiment does not agree with the prediction, then we conclude that the theory is not an adequate description of that experiment, and we try to find a new theory.

Chlo\"e's theory is also appealing because it can describe not only tennis balls, but the time it takes for other objects to fall as well. Scientists can then set out to continue testing her theory with a wide range of objects and drop heights to see if it describes those experiments as well. Inevitably, they will discover situations where Chlo\"e's theory fails to adequately describe the time that it takes for objects to fall (can you think of an example?).

We would then develop a new ``Theory of Falling Objects'' that would include Chlo\"e's theory that describes most objects falling, and additionally, a set of descriptions for the fall times for cases that are not described by Chlo\"e's theory. Ideally, we would seek a new theory that would also describe the new phenomena not described by Chlo\"e's theory in a succinct manner. There is of course no guarantee, ever, that such a theory would exist; it is just an optimistic hope of scientists to find the most general and succinct description of the physical world. 

This example highlights that applying the Scientific Method is an iterative process. Loosely, the prescription for applying the Scientific Method is:
\begin{enumerate}
\item Identify and describe a process that is not currently described by a theory.
\item Look at similar processes to see if they can be described in a similar way.
\item Improve the description to arrive at a ``Theory'' that can be generalized to make predictions.
\item Test predictions of the theory on new processes until a prediction fails.
\item Improve the theory.
\end{enumerate}

\begin{checkpointMC}{Fill in the blanks:

Physics is a branch of science that here the behaviour of the universe. When doing physics, we attempt to answer the question here things work the way they do.}
\item explains, ``Why?''
\item describes, ``How?''
\end{checkpointMC}

\section{Theories and models}
For the purpose of this textbook, we wish to introduce a distinction in what we mean by ``theory'' and by ``model''. We will consider a ``theory'' to be a set of statements that gives us a broad description, applicable to several phenomena and that allow us to make verifiable prediction. We will consider a ``model'' to be a situation-specific description of a phenomenon \textit{based on a theory}. Using the example from the previous section, our theory would be that the fall time of an object is proportional to the square root of the drop height, and a model would be applying that theory to describe a tennis ball falling by \SI{4.2}{\meter}.

This textbook will introduce the theories from Classical Physics, which were mostly established and tested between the seventeenth and nineteenth centuries. We will take it as given that readers of this textbook are not likely to perform experiments that challenge those well-established theories. The main challenge will be, given a theory, to define a model that describes a particular situation, and then to test that model. This introductory physics course is thus focused on thinking of ``doing physics'' as the task of correctly modelling a situation.

\begin{studentopinion}{Difference between a model and a theory}

``Model'' and ``Theory'' are sometimes used interchangeably among scientists. In physics, it is particularly important to distinguish between these two terms.

A model provides an immediate understanding of something based on a theory. For example, if you would like to model the launch of your toy rocket into space, you might run a computer simulation of the launch based on various theories of repulsion that you have learned. In this case, the model is the computer simulation, which describes what will happen to the rocket. This model depends on various theories that have been extensively tested such as Newton's Laws of motion, Fluid dynamics, etc. 
\begin{itemize}
\item``Model'': Your homemade rocket computer simulation
\item``Theory'': Newton's Laws of motion, Fluid dynamics
\end{itemize}

With this analogy, we can quickly see that the ``model'' and ``theory'' are not interchangeable. If they were, we would be saying that all of Newton's Laws of Motion depend on the success of your piddly toy rocket computer simulation!
\end{studentopinion}

\begin{checkpointMC}{Models cannot be scientifically tested, only theories can be tested.}
\item True
\item False
\end{checkpointMC}

\section{Fighting intuition}
It is important to remember to fight one's intuition when applying the scientific method. Certain theories, such as Quantum Mechanics, are very counter-intuitive. For example, in Quantum Mechanics, it is possible for an object to be in two locations at the same time. In the Theory of Special Relativity, it is possible for two people to disagree on whether two events occurred at the same time. Both of these theories have however not been invalidated by any experiment.

There is no requirement in science that a theory be ``pretty'' or intuitive. The only requirement is that a theory describe experimental data. One should then take care in not forcing one's preconceived notions into interpreting a theory. For example, Quantum Mechanics does not actually predict that objects can be in two locations at once, only that objects behave \textit{as if} they were in two locations at once. A famous example is Schr\"odinger's cat, which can be modelled as being both alive and dead at the same time. However, just because we model it that way does not mean that it really is alive and dead at the same time. 

\textbf{``Explanations aren't satisfying because they're beautiful, they're beautiful because they're satisfying. - Rebecca Goldstein''}

\section{The scope of Physics}
Physics describes a wide range of phenomena within the physical sciences, ranging from the behaviour of microscopic particles that make up matter to the evolution of the entire Universe. We often distinguish between ``classical'' and ``modern'' physics depending on when the theories were developed, and we can further subdivide these areas of physics depending on the scale or the type of the phenomena that are described.

The word physics comes from Ancient Greek and translates to ``nature'' or ``knowledge of nature''. The goal of physics is to develop theories from which mathematical models can be derived to describe a particular observation. One of the ambitious goals of physicists is to develop a single theory that describes all of nature, instead of having multiple theories to describe different categories of phenomena. This is in stark contrast to other fields of science, as Rutherford famously quipped: ``All science is either physics or stamp collecting''. That is, physicists hope that there exists one single mathematical theory (like Chlo\"e's theory of falling objects) that describes the entire physical world. In Biology, for example, this would not be a reasonable goal, as one needs to describe every single living being, and there is no overarching ``theory of what all living things look like''. Currently, physicists have been able to narrow down the number of theories required to describe all of the physical world to only three, which is impressive (the theory of gravity, the theory of the strong nuclear force, and physicists have now further unified the weak nuclear force with electromagnetism to make the ``electroweak force'').


\subsection{Classical Physics}
This textbook is focused on classical physics, which corresponds to the theories that were developed before 1905.
\subsubsection{Mechanics}
Mechanics describes most of our everyday experiences, such as how objects move, including how planets move under the influence of gravity. Isaac Newton was the first to formally develop a theory of mechanics, using his ``Three Laws'' to describe the behaviour of objects in our everyday experience. His famous work published in 1687, ``Philosophiae Naturalis Principia Mathematica'' (``The Principia'') also included a theory of gravity that describes the motion of celestial objects. 

Following the 1781 discovery of the planet Uranus by William Herschel, astronomers noticed that the orbit of the planet was not well described by Newton's theory. This led Urbain Le Verrier (in Paris) and John Couch Adams (in Cambridge) to predict the location of a new planet that was disturbing the orbit of Uranus rather than to claim that Newton's theory was incorrect. The planet Neptune was subsequently discovered by Le Verrier in 1846, one year after the prediction, and seen as a resounding confirmation of Newton's theory. 

In 1859, Urbain Le Verrier also noted that Mercury's orbit around the Sun is different than that predicted by Newton's theory. Again, a new planet was proposed, ``Vulcan'', but that planet was never discovered and the deviation of Mercury's orbit from Newton's prediction remained unexplained until 1915, when Albert Einstein introduced a new, more complete, theory of gravity, called ``General Relativity''. This is a good example of the scientific method; although the discovery of Neptune was consistent with Newton's theory, it did not prove that the theory is correct, only that it correctly described the motion of Uranus. The discrepancy that arose when looking at Mercury ultimately showed that Newtons' theory of gravity fails to provide a proper description of planetary orbits in the proximity of very massive objects (Mercury is the closest planet to the Sun). 

\begin{checkpointMC}{Albert Einstein's postulation of General Relativity was useful in explaining the precession of Mercury because:}
\item It showed that Newton's model of Mercury was correct. (wrong)
\item It showed that Newton's theory correctly described the motion of Uranus, but did not correctly describe the motion of more massive objects. (right)
\item It proved that Verrier's theory was correct, but did not correctly describe the motion of more massive objects. (wrong)
\item It proved Einstein's theory of General Relativity to be correct. (wrong)
\end{checkpointMC}
 

\subsubsection{Electromagnetism}
Electromagnetism describes electric charges and magnetism. At first, it was not realized that electricity and magnetism were connected. Charles Augustin de Coulomb published in 1784 the first description of how electric charges attract and repel each other. Magnetism was discovered in the ancient world, when people noticed that lodestone (rocks made from magnetized magnetite mineral) could attract iron tools. In 1819, Oersted discovered that moving electric charges could influence a compass needle, and several subsequent experiments were carried out to discover how magnets and moving electric charges interact.

In 1865, James Clerk Maxwell published ``A Dynamical Theory of the Electromagnetic Field'', wherein he first proposed a theory that unified electricity and magnetism as two facets of the same phenomenon. One important concept from Maxwell's theory is that light is an electromagnetic wave with a well-defined speed. This uncovered some potential issues with the theory as it required an absolute frame of reference in which to describe the propagation of light. Experiments in the late 1800s failed to detect the existence of this frame of reference.

\subsection{Modern Physics}
In 1905, Albert Einstein published three major papers that set the foundation for what we now call ``Modern Physics''. These papers covered the following areas that were not well-described by classical physics:
\begin{itemize}
\item A description of Brownian motion that implied that all matter is made of atoms
\item A description of the photoelectric effect that implied that light is made of particles
\item A description of the motion of very fast objects that implied that mass is equivalent to energy, and that time and distance are relative concepts
\end{itemize}
In order to accommodate Einstein's descriptions, physicists had to dramatically re-formulate new theories. 

\subsubsection{Quantum mechanics and particle physics}
Quantum mechanics is a theory that was developed in the 1920s to incorporate Einstein's conclusion that light is made of particles (or rather, quantized lumps of energy called quanta) and describe Nature at the smallest scales. This could only be done at the expense of determinism, leading to a theory that could not predict how particular situations evolve in time, but only the probabilities that certain outcomes will be realized. Quantum mechanics was further refined during the twentieth century into Quantum Field Theory, which led to the Standard Model of particle physics that describes our current understanding of matter through the theories of the electroweak and strong forces.

\subsubsection{The Special and General Theories of Relativity}
In 1905, Einstein published his ``Special Theory of Relativity'', which describes how light propagates without the need for an absolute frame of reference, thus solving the problem introduced by Maxwell. This required physicists to consider space and time on an equal footing (``Space-time''), rather than two independent aspects of the natural world, and led to a flurry of odd, but verified, experimental predictions. One such prediction is that time flows slower for objects moving fast, which has been experimentally verified by flying, precise, atomic clocks on air planes and satellites. In 1915, Einstein further refined his theory into General Relativity, which is our best current description of gravity and includes a description of Mercury's orbit which was not described by Newton's theory.

\subsubsection{Cosmology and astrophysics} 
\rwcapfig[12]{0.45\textwidth}{figures/Introduction/galaxies_in_Coma_cluster.jpeg}{\label{fig:galaxies_in_Coma_cluster}A galaxy in the Coma cluster of galaxies (credit:NASA).}
Cosmology describes processes at the largest scales and is mostly based on applying General Relativity to the scale of the Universe. For example, cosmology describes how our Universe started from the Big Bang and how large scale structures, such as galaxies and clusters of galaxies, have formed and evolved into our present day Universe. 

Astrophysics is focused on describing the formation and the evolution of stars, galaxies, and other ``astrophysical objects'' such as neutron stars and black holes. 

\subsubsection{Particle astrophysics}
Particle astrophysics is a relatively new field that makes use of subatomic particles produced by astrophysical objects to learn both about the objects \textit{and} about the particles. For example, the 2015 Nobel Prize in Physics was awarded to Art McDonald (a Canadian physicist from Queen's University) for using neutrinos\footnote{Neutrinos are the lightest subatomic particles that we know of} produced by the Sun to both learn about the nature of neutrinos and about how the Sun works. 

\section{Thinking like a physicist}
In a sense, physics can be thought of as the most fundamental of the sciences, as it describes the interactions of the smallest constituents of matter. In principle, if one can precisely describe how protons, neutrons, and electrons interact, then one can completely describe how a human brain thinks. In practice, the theories of particle physics lead to equations that are too difficult to solve for systems that include as many particles as a human brain. In fact, they are too difficult to solve exactly for even rather small systems of particles such as atoms bigger than helium (containing several protons, neutrons and electrons). 

We have a number of other fields of science to cover complex systems of particles interacting. Chemistry can be used to describe what happens to systems consisting of many atoms and molecules. In a living being, it is too difficult to keep track of systems of atoms and molecules, so we use Biology to describe living systems. 

One of the key qualities required to be an effective physicist is an ability to understand how to apply a theory and develop a model to describe a phenomenon. Just like any other skill, it takes practice to become good at developing models. Students that graduate with a physics degree are thus often sought for jobs that require critical thinking and the ability to develop quantitative models, which covers many fields from outside of physics such as finance or Big Data. This textbook thus tries to emphasize practice with developing models, while also providing a strong background in the theories of classical physics. 

\newpage
\section{Summary}
\vspace{2cm}
\begin{chapterSummary}
\item Science attempts to \textit{describe} the physical world (answers the question ``How?'', not ``Why?'').
\item The Scientific Method provides a prescription for arriving at theories that describe the physical world, that can be 
experimentally verified.
\item The Scientific Method is necessarily an iterative process where theories are continuously updated as new experimental data are acquired.
\item An experiment can only disprove a theory, not confirm it in any general sense.
\item Theories are typically valid only in well-defined situations.
\item Physics covers a wide scale of phenomena ranging from the Universe down to subatomic particles.
\item Classical physics encompasses the theories developed before 1905, when Einstein introduced the need for Quantum Mechanics and the Theorie(s) of Relativity.
\end{chapterSummary}

\begin{reflectresearch}{Research the following topics and try to answer these questions:}

\begin{checkpointMC}{Which of the following is a branch of modern physics?}
\item Newtonian mechanics
\item Classical electrodynamics
\item Quantum chromodynamics
\end{checkpointMC}

\begin{checkpointMC}{What particle helps to give mass to all of the massive elementary particles?}
\item Up quark
\item Neutrino
\item Photon
\item Higgs Boson
\end{checkpointMC}

\begin{checkpointMC}{In the Double Slit Experiment, the behaviour of an electron travelling through a double slit depends on whether or not ...}
\item The electron was in an excitable state.
\item An observer was measuring the system.
\item he electron had a certain critical speed.
\end{checkpointMC}

\begin{checkpointMC}{Name that physicist! Who was the first to propose that the universe is expanding?}
\item Richard Feynman
\item Stephen Hawking
\item Edwin Hubble
\item Erwin Schr\"odinger
\end{checkpointMC}

\end{reflectresearch}
%Copyright 2017 R.D. Martin
%This book is free software: you can redistribute it and/or modify it under the terms of the GNU General Public License as published by the Free Software Foundation, either version 3 of the License, or (at your option) any later version.
%
%This book is distributed in the hope that it will be useful, but WITHOUT ANY WARRANTY; without even the implied warranty of MERCHANTABILITY or FITNESS FOR A PARTICULAR PURPOSE.  See the GNU General Public License for more details, http://www.gnu.org/licenses/.
\chapter{Comparing Model and Experiment}
\label{chap:2_ModelAndExperiment}
In this chapter, we will learn about the process of doing science and lay the foundations for developing skills that will be of use throughout your scientific careers. In particular, we will start to learn how to test a model with an experiment, as well as learn to estimate whether a given result or model makes sense.
\vspace{1cm}
\begin{learningObjectives}
\item Be able to estimate orders of magnitude
\item Understand units
\item Understand the process of building a model and performing an experiment
\item Understand uncertainties in experiments
\item Be able to use a computer for simple data analysis
\end{learningObjectives}

\section{Orders of magnitude}
Although one should try to fight intuition when building a model to describe a particular phenomenon, one should not abandon critical thinking and should always ask if a model (or a prediction of the model) makes sense. One of the most straightforward ways to verify if a model makes sense is to ask whether it predicts the correct order of magnitude for a quantity. Usually, the order of magnitude for a quantity can be determined by making a very simple model, ideally one that you can work through in your head. When we say that a prediction gives the right ``order of magnitude'', we usually mean that the prediction is within a factor of ``a few'' (up to a factor of 10) of the correct answer.

\begin{example}{How many ping pong balls can you fit into a school bus? Is it of order 10,000, or 100,000, or more?}
Our strategy is to estimate the volumes of a school bus and of a ping pong ball, and then calculate how many times the volume of the ping pong ball fits into the volume of the school bus.

We can model a school bus as a box, say $\SI{20}{\meter}\times \SI{2}{\meter}\times\SI{2}{\meter}$, with a volume of \SI{80}{\meter\cubed}$\sim$\SI{100}{\meter\cubed}. We can model a ping pong ball as a sphere with a diameter of \SI{0.03}{\meter} (\SI{3}{\centi\meter}). When stacking the ping pong balls, we can model them as little cubes with a side given by their diameter, so the volume of a ping pong ball, for stacking, is $\sim$ \SI{0.00003}{\meter\cubed}=\SI{3e-5}{\meter\cubed}. If we divide \SI{100}{\meter\cubed} by \SI{3e-5}{\meter\cubed}, using scientific notation:
\begin{align*}
\frac{\SI{100}{\meter\cubed}}{\SI{3e-5}{\meter\cubed}}=\frac{\num{1e2}}{\num{3e-5}}=\frac{1}{3}\times 10^7\sim 3\times 10^6
\end{align*}
Thus, we expect to be able to fit about three million ping pong balls in a school bus. 
\end{example}

\begin{checkpointSA}{Fill in the following table, giving the order of magnitude in meters of the sizes of different physical objects. Feel free to Google these!}
\begin{center}
\begin{tabular}{|c|c| }
\hline  
\textbf{Object}&\textbf{Order of magnitude}\\
\hline
Proton&\\ \hline
Nucleus of atom&\\ \hline
Hydrogen atom&\\ \hline
Virus&\\ \hline
Human skin cell&\\ \hline
Width of human hair&\\ \hline
Human &\SI{1}{\meter}\\ \hline
Height of Mt. Everest&\\ \hline
Radius of Earth&\\ \hline
Radius of the Sun&\\ \hline
Distance to the Moon&\\ \hline
Radius of the Milky Way&\\ \hline
\end{tabular}
\end{center}
\end{checkpointSA}


\section{Units and dimensions}
In 1999, the NASA Mars Climate Orbiter disintegrated in the Martian atmosphere because of a mixup in the units used to calculate the thrust needed to slow the probe and place it in orbit about Mars. A computer program provided by a private manufacturer used units of pounds seconds to calculate the change in momentum of the probe instead of the Newton seconds expected by NASA. As a result, the probe was slowed down too much and disintegrated in the Martian atmosphere. This example illustrates the need for us to \textbf{use and specify units} when we talk about the properties of a physical quantity, and it also demonstrates the difference between a dimension and a unit.

``Dimensions'' can be thought of as types of measurements. For example, length and time are both dimensions. A unit is the standard that we choose to quantify a dimension. For example, meters and feet are both units for the dimension of length, whereas seconds and jiffys\footnote{A jiffy is a unit used in electronics and generally corresponds to either 1/50 or 1/60 seconds.} are units for the dimension of time.

When we compare two numbers, for example a prediction from a model and a measurement, it is important that both quantities have the same dimension \textit{and} be expressed in the same unit.
\begin{checkpointMC}{The speed limit on a highway:}
\item has dimension of length over time and can be expressed in units of kilometers per hour %correct
\item has dimension of length can be expressed in units of kilometers
\item has dimension of time over length and can be expressed in units of meters per second
\item has dimension of time and can be expressed in units of meters
\end{checkpointMC}

\subsection{Base dimensions and their SI units}
In order to facilitate communication of scientific information, the International System of units (SI for the french, Syst\`eme International d'unit\'es) was developed. This allows us to use a well-defined convention for which units to use when expressing quantities. For example, the SI unit for the dimension of length is the meter and the SI unit for the dimension of time is the second.

In order to simplify the SI unit system, a fundamental (base) set of dimensions was chosen and the SI units were defined for those dimensions. Any other dimension can always be re-expressed in terms of the base dimensions shown in table \ref{tab:ModelAndExperiment:SIunits} and thus in terms of the corresponding base SI units.

\begin{table}[!h]
\centering
\begin{tabular}{ll }
\textbf{Dimension}&\textbf{SI unit}\\
\hline
\hline
Length [L]& meter [m]\\ \hline
Time [T]& seconds[s] \\ \hline
Mass [M]& kilogram [kg]\\ \hline
Temperature [$\Theta$]& kelvin [K] \\ \hline
Electric current [I]& amp\`ere [A]\\ \hline
Amount of substance [N]& mole [mol] \\ \hline
Luminous intensity [J]& candela [cd] \\ \hline
Dimensionless [0]& unitless [] \\ \hline
\end{tabular}
\caption{\label{tab:chap2:SIunits} Base dimensions and their SI units with abbreviations.}
\end{table}

From the base dimensions, one can obtain ``derived'' dimensions such as ``speed'' which is a measure of how fast an object is moving. The dimension of speed is $\frac{L}{T}$ (length over time) and the corresponding SI unit is m/s (meters per second)\footnote{Note that we can also write meters per second as m$\cdot$s$^{-1}$, but we often use a divide by sign if the power of the unit in the denominator is 1.} and corresponds to a measure of how much distance an object can cover per unit time (the higher the speed, the larger the distance covered per unit time). Table \ref{tab:ModelAndExperiment:DerivedSIunits} shows a few derived dimensions and their corresponding SI units.

\begin{table}[!h]
\centering
\begin{tabular}{lll }  
\textbf{Dimension}&\textbf{SI unit}&\textbf{SI base units}\\
\hline
\hline
Speed [L/T]& meter per second [m/s] & [m/s]\\ \hline
Frequency [1/T]& hertz [Hz] & [1/s]\\ \hline
Force [M$\cdot$L$\cdot$T$^{-2}$]& newton [N]&[kg$\cdot$m$\cdot$s$^{-2}$]\\ \hline
Energy [M$\cdot$L$^2\cdot$T$^{-2}$]& joule [J]&[N$\cdot$m=kg$\cdot$m$^2\cdot$s$^{-2}$] \\ \hline
Power [M$\cdot$L$^2\cdot$T$^{-3}$]& watt [W]&[J/s=kg$\cdot$m$^2\cdot$s$^{-3}$]\\ \hline
Electric Charge [I$\cdot$ T]& coulomb [C]&[A$\cdot$ s] \\ \hline
Voltage [M$\cdot$L$^2\cdot$T$^{-3}\cdot$I$^{-1}$]& volt [V]&[J/C=kg$\cdot$m$^2\cdot$s$^{-3}\cdot$A$^{-1}$] \\ \hline
\end{tabular}
\caption{\label{tab:chap2:DerivedSIunits} Example of derived dimensions and their SI units with abbreviations.}
\end{table}

By convention, we can indicate the dimension of a quantity, $X$, by writing it in square brackets, $[X]$. For example, $[X]=I$, would mean that the quantity $X$ has dimensions of electric current. Similarly, we can indicate the SI units of $X$ with $SI[X]$; since $X$ has dimensions of current, $SI[X]=A$.

\subsection{Dimensional analysis}
We call ``dimensional analysis'' the process of working out the dimensions of a quantity in terms of the base dimensions. A few simple rules allow us to easily work out the dimensions of a derived quantity. Suppose that we have two quantities, $X$ and $Y$, both with dimensions. We then have the following rules to find the dimension of a quantity that depends on $X$ and $Y$:
\begin{enumerate}
\item You can only add or subtract two quantities if they have the same dimension: $[X+Y]=[X]=[Y]$
\item The dimension of the product is the product of the dimensions: $[XY]=[X]\cdot[Y]$
\item The dimension of the ratio is the ratio of the dimensions:$[X/Y]=[X]/[Y]$
\end{enumerate}

The next two examples show how to apply dimensional analysis to obtain the unit or dimension of a derived quantity. 

\begin{example}{Acceleration has SI units of ms$^{-2}$ and force has dimensions of mass multiplied by acceleration. What are the dimensions and SI units of force, expressed in terms of the base dimensions and units?}
We can start by expressing the dimension of acceleration, since we know from its SI units that it must have dimension of length over time squared.
\begin{align*}
[acceleration] = \frac{L}{T^2}
\end{align*}
Since force has dimension of mass times acceleration, we have:
\begin{align*}
[force] = \frac{M\cdot L}{T^2}
\end{align*}
and the SI units of force are thus:
\begin{align*}
SI[force] = \frac{kg \cdot m}{s^2}
\end{align*}
Force is such a common dimension that it, like many other derived dimensions, has its own derived SI unit, the Newton [N].
\end{example}

\begin{example}{Use Table \ref{tab:ModelAndExperiment:DerivedSIunits} to show that voltage has the same dimension as force multiplied by speed and divided by electric current.}
According to Table \ref{tab:ModelAndExperiment:DerivedSIunits}, voltage has dimensions:
\begin{align*}
[voltage]=M\cdot L^2 \cdot T^{-3}\cdot I^{-1}
\end{align*}
while force, speed and current have dimensions:
\begin{align*}
[force]&=M\cdot L\cdot T^{-2} \\
[speed]&=L\cdot T^{-1}\\
[current]&=I
\end{align*}
The dimension of force multiplied by speed divided by electric charge
\begin{align*}
[\frac{force\cdot speed}{current}]&=\frac{[force]\cdot [speed]}{[current]}=\frac{M\cdot L\cdot T^{-2} \cdot L\cdot T^{-1} }{I}\\
&=M\cdot L^2 \cdot T^{-3}\cdot I^{-1}
\end{align*}
where, in the last line, we combined the powers of the same dimensions. By inspection, this is the same dimension as voltage.
\end{example}

When you build a model to describe a situation, your model will typically provide a value for a quantity that you are interested in modelling. You should always use dimensional analysis to ensure that the dimension of the quantity your model predicts has the correct dimension. For example, suppose that you model the speed, $v$, that an object has after falling from a height of \SI{100}{\meter} on the surface of the planet Mars. Presumably, $v$ will depend on the mass and radius of the planet. You can be guaranteed that your model for $v$ is incorrect if the dimension of $v$ is not speed. Dimensional analysis should always be used to check that your model is not incorrect (note that getting the correct dimension is not a guarantee of the model being correct, only that it is ``not definitely wrong''). Similarly, you should also use order of magnitude estimates to evaluate whether your model gives a reasonable prediction.

\begin{checkpointMC}{In Chlo\"e's theory of falling objects from Chapter \ref{Introduction}, the time, $t$, for an object to fall a distance, $x$, was given by $t=k\sqrt{x}$. What must the SI units of Chlo\"e's constant, $k$, be?}
\item \si{T.L^{\frac{1}{2}}}
\item \si{T.L^{-\frac{1}{2}}}
\item \si{s.m^{\frac{1}{2}}}
\item \si{s.m^{-\frac{1}{2}}} %correct
\end{checkpointMC}

\section{Making measurements}
Having introduced some tools for the modelling aspect of physics, we now address the other side of physics, namely performing experiments. Since the goal of developing theories and models is to describe the real world, we need to understand how to make meaningful measurements that test our theories and models.

Suppose that we wish to test Chlo\"e's theory of falling objects from Chapter \ref{Introduction}:
\begin{align*}
t=k\sqrt{x}
\end{align*}
which states that the time, $t$, for any object to fall a distance, $x$, from the surface of the Earth is given by the above relation. The theory assumes that Chlo\"e's constant, $k$, is the same for any object falling any distance on the surface of the Earth.

One possible way to test Chlo\"e's theory of falling objects is to measure $k$ for different drop heights to see if we always obtain the same value. Results of such an experiment are presented in Table \ref{tab:ModelAndExperiment:kmes}, where the time, $t$, was measured for a bowling ball to fall distances of $x$ between \SI{1}{\meter} and \SI{5}{\meter}. The table also shows the values computed for $\sqrt x$ and the corresponding value of $k=\frac{t}{\sqrt x}$:

\begin{table}[!h]
\centering
\begin{tabular}{cccc} 
\textbf{x} [m]&\textbf{t} [s]&\textbf{$\sqrt x$}  [\si{m^{\frac{1}{2}}}]&\textbf{k}  [\si{s.m^{-\frac{1}{2}}}]\\
\hline
\hline
1.00 &0.33 &1.00 &0.33 \\ \hline
2.00 &0.74 &1.41 &0.52 \\ \hline
3.00 &0.67 &1.73 &0.39 \\ \hline
4.00 &1.07 &2.00 &0.54 \\ \hline
5.00 &1.10 &2.24 &0.49 \\ \hline
\end{tabular}
\caption{\label{tab:chap2:kmes} Measurements of the drop times, $t$, for a bowling ball to fall different distances, $x$. We have also computed $\sqrt x$ and the corresponding value of $k$. }
\end{table}

When looking at Table \ref{tab:ModelAndExperiment:kmes}, it is clear that each drop height gave a different value of $k$, so at face value, we would claim that Chlo\"e's theory is incorrect, as there does not seem to be a value of $k$ that applies to all situations. However, we would be incorrect in doing so unless we understood \textit{the precision of the measurements} that we made. Suppose that we repeated the measurement multiple times at a fixed drop height of $x=\SI{3}{m}$, and obtained the values in Table \ref{tab:ModelAndExperiment:kmes_3m}.

\begin{table}[!h]
\centering
\begin{tabular}{cccc} 
\textbf{x} [m]&\textbf{t} [s]&\textbf{$\sqrt x$}  [\si{m^{\frac{1}{2}}}]&\textbf{k}  [\si{s.m^{-\frac{1}{2}}}]\\
\hline
\hline
3.00 &1.01 &1.73 &0.58 \\ \hline
3.00 &0.76 &1.73 &0.44 \\ \hline
3.00 &0.64 &1.73 &0.37 \\ \hline
3.00 &0.73 &1.73 &0.42 \\ \hline
3.00 &0.66 &1.73 &0.38 \\ \hline
\end{tabular}
\caption{\label{tab:chap2:kmes_3m} Repeated measurements of the drop time, $t$, for a bowling ball to fall a distance $x=\SI{3}{m}$. We have also computed $\sqrt x$ and the corresponding value of $k$. }
\end{table}

This simple example highlights the critical aspect of making any measurement: it is impossible to make a measurement with infinite precision. The values in Table \ref{tab:ModelAndExperiment:kmes_3m} show that if we repeat the exact same experiment, we are likely to measure different values for a single quantity. In this case, for a fixed drop height, $x=\SI{3}{m}$, we obtained a spread in values of the drop time, $t$, between roughly \SI{0.6}{s} and \SI{1.0}{s}. Does this mean that it is hopeless to do science, since we can never repeat measurements? Thankfully, no! It does however require that we deal with the inherent imprecision of measurements in a formal manner.

\subsection{Measurement uncertainties}
The values in Table \ref{tab:ModelAndExperiment:kmes_3m} show that for a fixed experimental setup (a drop height of \SI{3}{m}), we are likely to measure a spread in the values of a quantity (the time to drop). We can quantify this ``uncertainty'' in the value of the measured time by quoting the measured value of $t$ by providing a ``central value'' and an ``uncertainty'':
\begin{align*}
t = \SI{0.76 \pm 0.15}{s}
\end{align*}
where \SI{0.76}{s} is called the ``central value'' and \SI{0.15}{s} the ``uncertainty'' or the ``error''\footnote{We use the word error as a synonym for uncertainty, not ``mistake''.}. When we present a number with an uncertainty, we mean that we are ``pretty certain'' that the true value is in the range that we quote. In this case, the range that we quote is that $t$ is between \SI{0.61}{s} and \SI{0.91}{s} (given by \SI{0.76}{s} - \SI{0.15}{s} and \SI{0.76}{s} + \SI{0.15}{s}). When we say that we are ``pretty sure'' that the value is within the quoted range, we usually mean that there is a 68\% chance of this being true and allow for the possibility that the true value is actually outside the range that we quoted. The value of 68\% comes from statistics and the normal distribution which you can learn about on the internet or in a more advanced course. 

\subsubsection{Determining the central value and uncertainty}
The tricky part when performing a measurement is to decide how to assign a central value and an uncertainty. For example, how did we come up with $t=\SI{0.76 \pm 0.15}{s}$ from the values in Table \ref{tab:ModelAndExperiment:kmes_3m}? 

Determining the uncertainty and central value on a measurement is greatly simplified when one can repeat the same measurement multiple times, as we did in Table \ref{tab:ModelAndExperiment:kmes_3m}. With repeatable measurements, a reasonable choice for the central value and uncertainty is to use the mean and standard deviation of the measurements, respectively.

If we have $N$ measurements of some quantity $t$, $\{t_1, t_2, t_3, \dots t_N\}$, then the mean, $\bar t$, and standard deviation, $\sigma_t$, are defined as:
\begin{align}
\bar t &= \frac{1}{N}\sum_{i=1}^{i=N} t_i=\frac{t_1 +t_2 +t_3 +\dots+ t_N}{N} \\
\sigma_t^2 &=\frac{1}{N-1}\sum_{i=1}^{i=N}(t_i-\bar t)^2 = \frac{(t_1-\bar t)^2+(t_2-\bar t)^2+(t_3-\bar t)^2+\dots+(t_N-\bar t)^2}{N-1} \\
\sigma_t &=\sqrt{\sigma_t^2}
\end{align}
The mean is just the arithmetic average of the values, and the standard deviation, $\sigma_t$, requires one to first calculate the mean, then the variance ($\sigma^2_t$, the square of the standard deviation). You should also note that for the variance, we divide by $N-1$ instead of $N$. The standard deviation and variance are quantities that come from statistics and are a good measure of how spread out the values of $t$ are about their mean.

\begin{example}{Calculate the mean and standard deviation of the values for $k$ from Table \ref{tab:ModelAndExperiment:kmes_3m}.}
\label{ex:chap2:stdcalc}
In order to calculate the standard deviation, we first need to calculate the mean of the $N=5$ values of $k$: $\{0.58, 0.44, 0.37, 0.42, 0.38 \}$. The mean is given by:
\begin{align*}
\bar k = \frac{0.58 + 0.44 + 0.37 + 0.42 + 0.38}{5}=\SI{0.44}{s.m^{-\frac{1}{2}}}
\end{align*}
We can now calculate the variance:
\begin{align*}
\sigma^2_k &= \frac{1}{4}[(0.58-0.44)^2+(0.44-0.44)^2\\
         &+(0.37-0.44)^2+(0.42-0.44)^2+(0.38-0.44)^2]=\SI{7.3e-3}{s^2.m}
\end{align*}
and the standard deviation is then given by the square root of the variance:
\begin{align*}
\sigma_k=\sqrt{0.0073}=\SI{0.09}{s.m^{-\frac{1}{2}}}
\end{align*}
Using the mean and standard deviation, we would quote our value of $k$ as $k=\SI{0.44 \pm 0.09}{s.m^{-\frac{1}{2}}}$.
\end{example}
Any value that we measure should always have an uncertainty. In the case where we can easily repeat the measurement, we should do so to evaluate how reproducible it is, and the standard deviation of those values is usually a good first estimate of the uncertainty in a value\footnote{In practice, the standard deviation is an overly conservative estimate of the error and we would use the error on the mean, which is the standard deviation divided by the square root of the number of measurements.}. Sometimes, the measurements cannot easily be reproduced; in that case, it is still important to determine a reasonable uncertainty, but in this case, it usually has to be estimated. Table \ref{tab:ModelAndExperiment:uncertainties} shows a few common types of measurements and how to determine the uncertainties in those measurements. 

\begin{table}[!h]
\centering
\begin{tabular}{p{3in}p{3in}} 
\textbf{Type of measurement} &\textbf{How to determine central value and uncertainty} \\
\hline
\hline
Repeated measurements & Mean and standard deviation \\ \hline
Single measurement with a graduated scale (e.g. ruler, digital scale, analogue meter) & Closest value and half of the smallest division\\ \hline
Counted quantity & Counted value and square root of the value \\ \hline
\end{tabular}
\caption{\label{tab:chap2:uncertainties} Different types of measurements and how to assign central values uncertainties.}
\end{table}
\Lwcapfig[11]{0.4\textwidth}{figures/ModelAndExperiment/ruler.png}{\label{fig:ModelAndExperiment:ruler}The length of the grey rectangle would be quoted as $L=\SI{2.8\pm0.5}{cm}$ using the rule of ``half the smallest division''.}
For example, we would quote the length of the grey object in Figure \ref{fig:ModelAndExperiment:ruler} to be $L=\SI{2.8\pm0.5}{cm}$ based on the rules in Table \ref{tab:ModelAndExperiment:uncertainties}, since \SI{2.8}{cm} is the closest value on the ruler that matches the length of the object and \SI{0.5}{mm} is half of the smallest division on the ruler. Using half of the smallest division of the ruler means that our uncertainty range covers one full division. Note that it is usually better to reproduce a measurement to evaluate the uncertainty instead of using half of the smallest division, although half of the smallest division should be the lower limit on the uncertainty. That is, by repeating the measurements and obtaining the standard deviation, you should see if the uncertainty is \textit{larger} than half of the of the smallest division, not smaller.


The \textbf{relative uncertainty} in a measured value is given by dividing the uncertainty by the central value, and expressing the result in percent. For example, the relative uncertainty in $t=\SI{0.76\pm 0.15}{s}$ is given by $\frac{0.15}{0.76}=20\%$. The relative uncertainty gives an idea of how precisely a value was determined. Typically, a value above 10\% means that it was not a very precise measurement, and we would generally consider a value smaller than 1\% to correspond to quite a precise measurement. 

\subsubsection{Random and systematic sources of error/uncertainty}
It is important to note that there are two possible sources of uncertainty in a measurement. The first is called ``statistical'' or ``random'' and occurs because it is impossible to exactly reproduce a measurement. For example, every time you lay down a ruler to measure something, you might shift it slightly one way or the other which will affect your measurement. The important property of random sources of uncertainty is that if you reproduce the measurement many times, these will tend to cancel out and the mean can usually be determined to high precision with enough measurements. 

The other source of uncertainty is called ``systematic''. Systematic uncertainties are much more difficult to detect and to estimate. One example would be trying to measure something with a scale that was not properly tarred (where the 0 weight was not set). You may end up with very small random errors when measuring the weights of object (very repeatable measurements), but you would have a hard time noticing that all of your weights were offset by a certain amount unless you had access to a second scale. Some common examples of systematic uncertainties are: incorrectly calibrated equipment, parallax error when measuring distance, reaction times when measuring time, effects of temperature on materials, etc.

In this section, we want to further emphasize the difference between ''error'' and ''mistake''. ''Uncertainty'' or ''error'' in a measurement comes from an unavoidable source that affects experimental results. A ''mistake'' also affects experimental results, but is preventable. If a ''mistake'' occurs in physics, the experiment is generally re-done and the previous data is discarded. The broad term ''human error'' may refer to a variety of errors or mistakes, so it is best to avoid this term in experimental physics (especially in your lab reports)!

The following table refers to examples of error that may be confused with ''human error'' that could be more accurately explained:

\begin{table}[!h]
\centering
\begin{tabular}{p{3in}p{3in}} 
\textbf{Situation} &\textbf{Source of Error} \\
\hline
\hline
While taking measurements, your line of sight was not completely parallel to the measuring device. & This is parallax error - a type of random error.\\ \hline
You incorrectly performed calculations. & Mistake! Redo the calculations.\\ \hline
A draft of wind in the lab slightly altered the direction of your ball rolling down an incline. & This is an environmental error - a type of random error (in this situation).\\ \hline
Your hand slipped while holding the ruler - the object was measured to be twice its original size! & Mistake! Redo this experiment and discard the data.\\ \hline
When timing an experiment, you don't hit the ''STOP'' button exactly when the experiment stops. & Reaction time error - a type of random error.\\ \hline
\end{tabular}
\caption{\label{tab:ModelAndExperiment:uncertainties} Different situations in the lab and type of error.}
\end{table}

\subsubsection{Propagating uncertainties}
Going back to the data in Table \ref{tab:ModelAndExperiment:kmes_3m}, we found that for a known drop height of $x=\SI{3}{m}$, we measured different values of the drop time, which we found to be $t=\SI{0.76 \pm 0.15}{s}$ (using the mean and standard deviation). We also calculated a value of $k$ corresponding to each value of $t$, and found $k=\SI{0.44 \pm 0.09}{s.m^{-\frac{1}{2}}}$. Suppose that we did not have access to the individual values of $t$, but only to the value of $t=\SI{0.76 \pm 0.15}{s}$ with uncertainty. How do we calculate a value for $k$ with uncertainty? In order to answer this question, we need to know how to ``propagate'' the uncertainties in a measured value to the uncertainty in a valued derived from the measurements. We briefly present different methods for propagating uncertainties, before advocating for the use of computers to do the calculations for you.

\textbf{1. Estimate using relative uncertainties}
The relative uncertainty in a measurement gives us an idea of how precisely a value was determined. Any quantity that depends on that measurement should have a precision that is similar; that is we expect the relative uncertainty on $k$ to be similar to that in $t$. For $t$, we saw that the relative uncertainty was approximately 20\%. If we take the central value of $k$ to be the central value of $t$ divided by $\sqrt x$, we find:
\begin{align*}
k=\frac{(\SI{0.76}{s})}{\sqrt{(\SI{3}{m})}}=\SI{0.44}{s.m^{-\frac{1}{2}}}
\end{align*} 
Since we expect the relative uncertainty in $k$ to be approximately 20\%, then the absolute uncertainty is given by:
\begin{align*}
\sigma_k = (0.2) k=\SI{0.09}{s.m^{-\frac{1}{2}}}
\end{align*}
which is close to the value obtained by averaging the five values of $k$ in Table \ref{tab:ModelAndExperiment:kmes_3m}.

\textbf{2. The Min-Max method}\\
A pedagogical way to determine $k$ and its uncertainty is to use the ``Min-Max method''. Since $k=\frac{t}{\sqrt x}$, $k$ will be the biggest when $t$ is the biggest, and the smallest when $t$ is the smallest. We can thus determine ``minimum'' and ``maximum'' values of $k$ corresponding to the minimum value of $t$, $t^{min}=\SI{0.61}{s}$ and the maximum value of $t$, $t^{max}=\SI{0.91}{s}$:
\begin{align*}
k^{min} &= \frac{t^{min}}{\sqrt x}=\frac{0.61\,s}{\sqrt{(3\,m)}} = \SI{0.35}{s.m^{-\frac{1}{2}}}\\
k^{max} &= \frac{t^{max}}{\sqrt x}=\frac{0.91\,s}{\sqrt{(3\,m)}} = \SI{0.53}{s.m^{-\frac{1}{2}}}\\
\end{align*}
This gives us the range of values of $k$ that correspond to the range of values of $t$. We can choose the middle of the range as the central value of $k$ and half of the range as the uncertainty:
\begin{align*}
\bar k &= \frac{1}{2}(k^{min}+k^{max})= \SI{0.44}{s.m^{-\frac{1}{2}}}\\
\sigma_k &= \frac{1}{2}(k^{max}-k^{min})= \SI{0.09}{s.m^{-\frac{1}{2}}}\\
\therefore k&= \SI{0.44 \pm 0.09}{s.m^{-\frac{1}{2}}}
\end{align*}
which, in this case, gives the same value as that obtained by averaging the individual values of $k$. While the Min-Max method is useful for illustrating the concept of propagating uncertainties, we usually do not use it in practice as it tends to overestimate the uncertainty. 

\textbf{3. The derivative method}
In the example above, we assumed that the value of $x$ was known precisely (and we chose 3\,m) which of course is not realistic. Let us suppose that we have measured $x$ to within \SI{1}{cm} so that $x=\SI{3.00 \pm 0.01}{m}$. The task is now to calculate $k=\frac{t}{\sqrt{x}}$ when both $x$ and $t$ have uncertainties.

The derivative method lets us propagate the uncertainty in a general way, so long as the relative uncertainties on all quantities are ``small'' (less than 10-20\%). If we have a function, $F(x,y)$ that depends on multiple variables with uncertainties (e.g. $x\pm\sigma_x$, $y\pm\sigma_y$), then the central value and uncertainty in $F(x,y)$ are given by:
\begin{align}
\bar F &= F(\bar x, \bar y) \nonumber \\
\sigma_F &= \sqrt{\left(\die{F}{x}\sigma_x \right)^2 + \left(\die{F}{y}\sigma_y \right)^2 }
\end{align}
That is, the central value of the function $F$ is found by evaluating the function at the central values of $x$ and $y$. The uncertainty in $F$, $\sigma_F$ is found by taking the quadrature sum of the partial derivatives of $F$ evaluated at the central values of $x$ and $y$ multiplied by the uncertainties in the corresponding variables that $F$ depends on. The uncertainty will contain one term in the sum per variable that $F$ depends on. At the end of the chapter, we will show you how to calculate this easily with a computer, so do not worry about getting comfortable with partial derivatives (yet!). Note that the partial derivative, $\die{F}{x}$ is simply the derivative of $F(x,y)$ relative to $x$ evaluated as if $y$ were a constant. Also, when we say ``add in quadrature'', we mean square the quantities, add them, and then take the square root (same as you would do to calculate the hypotenuse of a right-angle triangle).

\begin{example}{Use the derivative method to evaluate $k=\frac{t}{\sqrt{x}}$ for $x=\SI{3.00 \pm 0.01}{m}$ and $t=\SI{0.76\pm0.15}{s}$.}
\label{ex:ModelAndExperiment:derivprop}
Here, $k=k(x,t)$ is a function of both $x$ and $t$. The central value is easily found:
\begin{align*}
\bar k = \frac{t}{\sqrt{x}} = \frac{(\SI{0.76}{s})}{\sqrt{(\SI{3}{m})}}=\SI{0.44}{s.m^{-\frac{1}{2}}}\end{align*}
Next, we need to determine and evaluate the partial derivative of $k$ with respect to $t$ and $x$:
\begin{align*}
\die{k}{t}&=\frac{1}{\sqrt{x}}\frac{d}{dt}t=\frac{1}{\sqrt{x}}=\frac{1}{\sqrt{(\SI{3}{m})}}=\SI{0.58}{m^{-\frac{1}{2}}}\\
\die{k}{x}&=t\frac{d}{dx}x^{-\frac{1}{2}}=-\frac{1}{2}tx^{-\frac{3}{2}}= -\frac{1}{2}(\SI{0.76}{s})(\SI{3.00}{m})^{-\frac{3}{2}}=-\SI{0.073}{s.m^{-\frac{3}{2}}}
\end{align*}
And finally, we plug this into the quadrature sum to get the uncertainty in $k$:
\begin{align*}
\sigma_k&=\sqrt{\left(\die{k}{x}\sigma_x \right)^2 + \left(\die{k}{t}\sigma_t \right)^2 } = \sqrt{\left((\SI{0.073}{s.m^{-\frac{3}{2}}}) (\SI{0.01}{m}) \right)^2 + \left((\SI{0.58}{m^{-\frac{1}{2}}})(\SI{0.15}{s}) \right)^2 } \\
&=\SI{0.09}{s.m^{-\frac{1}{2}}}
\end{align*}
So we find that:
\begin{align*}
k&= \SI{0.44 \pm 0.09}{s.m^{-\frac{1}{2}}}
\end{align*}
which is consistent with what we found with the other two methods.

We should ask ourselves if the value we found is reasonable, since we also included an uncertainty in $x$ and would expect a bigger uncertainty than in the previous calculations where we only had an uncertainty in $t$. The reason that the uncertainty in $k$ has remained the same is that the relative uncertainty in $x$ is very small, $\frac{0.01}{3.00}\sim 0.3\%$, so it contributes very little compared to the 20\% uncertainty from $t$. 
\end{example}

The derivative method leads to a few simple short cuts when propagating the uncertainties for simple operations, as shown in Table \ref{tab:ModelAndExperiment:prop_uncertainties}. A few rules to note:
\begin{enumerate}
\item Uncertainties should be combined in quadrature
\item For addition and subtraction, add the absolute uncertainties in quadrature
\item For multiplication and division, add the relative uncertainties in quadrature
\end{enumerate}

\begin{table}[!h]
\centering
\begin{tabular}{p{2.5in}p{2in}} 
\textbf{Operation to get $z$} &\textbf{Uncertainty in $z$} \\
\hline
\hline
$z=x+y$ (addition) &  $\sigma_z=\sqrt{\sigma_x^2+\sigma_y^2}$ \\ \hline
$z=x-y$ (subtraction) & $\sigma_z=\sqrt{\sigma_x^2+\sigma_y^2}$ \\ \hline
$z=xy$ (multiplication) & $\sigma_z=xy\sqrt{\left(\frac{\sigma_x}{x}\right)^2+\left(\frac{\sigma_y}{y}\right)^2}$ \\ \hline
$z=\frac{x}{y}$ (division) & $\sigma_z=\frac{x}{y}\sqrt{\left(\frac{\sigma_x}{x}\right)^2+\left(\frac{\sigma_y}{y}\right)^2}$ \\ \hline
$z=f(x)$ (a function of 1 variable) &$\sigma_z=\left|\frac{df}{dx}\sigma_x \right|$ \\ \hline
\end{tabular}
\caption{\label{tab:ModelAndExperiment:prop_uncertainties} How to propagate uncertainties from measured values $x\pm\sigma_x$ and $y\pm\sigma_y$ to a quantity $z(x,y)$ for common operations.}
\end{table}

\begin{checkpointSA}{We have measured that a llama can cover a distance of \SI{20.0 \pm 0.5}{m} in \SI{4.0\pm 0.5}{s}. What is the speed (with uncertainty) of the llama?}
%5.0 +/- 0.6 m/s
\end{checkpointSA}


\subsection{Reporting measured values}
Now that you know how to attribute an uncertainty to a measured quantity and then propagate that uncertainty to a derived quantity, you are ready to present your measurement to the world. In order to conduct ``good science'', your measurements should be reproducible, clearly presented, and precisely described. Here are general rules to follow when reporting a measured number:
\begin{enumerate}
\item Indicate the units, preferably SI units (use derived SI units, such as newtons, when appropriate)
\item Include a sentence describing how the uncertainty was determined (if it is a direct measurement, how did you choose the uncertainty? If it is a derived quantity, how did you propagate the uncertainty?)
\item Show no more than 2 ``significant digits''\footnote{Significant digits are those excluding leading and trailing zeroes.} in the uncertainty and format the central value to the same decimal as the uncertainty. 
\item Use scientific notation when appropriate (usually numbers bigger than 1000 or smaller than 0.01).
\item Factor out the power 10 from the central value and uncertainty (e.g. \SI{10123\pm 310}{m} would be \SI{10.12\pm 0.31e3}{m} or \SI{101.2\pm 3.1e2}{m} 
\end{enumerate}

\begin{checkpointMC}{Someone has measured the average height of tables in the laboratory to be \SI{1.0535}{m} with a standard deviation of \SI{0.0525}{m}. What is the best way to present this measurement?}
\item \SI{1.0535\pm 0.0525}{m}
\item \SI{1.054\pm 0.053}{m}
\item \SI{105.4\pm 5.3e-2}{m}
\item \SI{105.35\pm 5.25}{cm}
\end{checkpointMC}

\subsection{Comparing model and measurement - discussing a result}
In order to advance science, we make measurements and compare them to a theory or model prediction. We thus need a precise and consistent way to compare measurements with each other and with predictions. Suppose that we have measured a value for Chlo\"e's constant $k= \SI{0.44 \pm 0.09}{s.m^{-\frac{1}{2}}}$. Of course, Chlo\"e's theory does not predict a value for $k$, only that fall time is proportional to the square root of the distance fallen. Isaac Newton's Universal Theory of Gravity does predict a value for $k$ of \SI{0.45}{s.m^{-\frac{1}{2}}} with negligible uncertainty. In this case, since the model (theoretical) value easily falls within the range given by our uncertainty, we would say that our measurement is consistent (or compatible) with the theoretical prediction. 

Suppose that instead, we had measured $k=\SI{0.55 \pm 0.08}{s.m^{-\frac{1}{2}}}$ so that the lowest value compatible with our measurement, $k=\SI{0.55}{s.m^{-\frac{1}{2}}}-\SI{0.08}{s.m^{-\frac{1}{2}}}=\SI{0.47}{s.m^{-\frac{1}{2}}}$ is not compatible with Newton's prediction. Would we conclude that our measurement invalidates Newton's theory? The answer is: it depends... And what ``it depends on'' should always be discussed any time that you present a measurement (even it it happened that your measurement is compatible with a prediction - maybe that was a fluke). Below, we list a few common points that should be addressed when presenting a measurement that will guide you into deciding whether your measurement is consistent with a prediction:
\begin{itemize}
\item How was the uncertainty determined and/or propagated?
\item Are there systematic effects that were not taken into account when determining the uncertainty? (e.g. reaction time, parallax, something difficult to reproduce).
\item Are the relative uncertainties reasonable?
\item What assumptions were made in calculating your measured value?
\item What assumptions were made in determining the model prediction? 
\end{itemize}
In the above, our value of $k= \SI{0.55 \pm 0.08}{s.m^{-\frac{1}{2}}}$ is the result of propagating the uncertainty in $t$ which was found by using the standard deviation of the values of $t$. It is thus conceivable that the true value of $t$, and therefore of $k$, is outside the range that we quote. Since our value of $k$ is still quite close to the theoretical value, we would not claim to have invalidated Newton's theory with this measurement. Our uncertainty in $k$ is $\sigma_k=\SI{0.08}{s.m^{-\frac{1}{2}}}$, and the difference between our measured and the theoretical value is only $1.25\sigma_k$, so very close to the value of the uncertainty. 

In a similar way, we would discuss whether two different measurements, each with an uncertainty, are compatible. If the ranges given by uncertainties in two values overlap, then they are clearly consistent and compatible. If on the other hand, the ranges do not overlap, they could be inconsistent, or the discrepancy might instead be the result of how the uncertainties were determined and the measurements could still be considered consistent. 




\newpage
\section{Summary}
\vspace{2cm}
\begin{chapterSummary}
\item Measurable quantities have dimensions and units.
\item A physical quantity should always be reported with units, preferably SI units.
\item When you build a model to predict a physical quantity, you should always ask if the prediction makes sense (Does it have a reasonable order of magnitude? Does it have the right dimensions?).
\item Any quantity that you measure will have an uncertainty.
\item Almost any quantity that you determine from a model or theory will also have an uncertainty.
\item The best way to determine an uncertainty is to repeat the measurement and use the mean and standard deviation of the measurements as the central value and uncertainty.
\item You have to pay special attention to systematic uncertainties, which are difficult to determine. You should always think of ways that your measured values could be wrong, even after repeated measurements.
\item Relative uncertainties tell you whether your measurement is precise.
\item If you expect two measured quantities to be linearly related (one is proportional to the other), plot them to find out! Use a computer to do so!
\end{chapterSummary}
%%Copyright 2017 R.D. Martin
%This book is free software: you can redistribute it and/or modify it under the terms of the GNU General Public License as published by the Free Software Foundation, either version 3 of the License, or (at your option) any later version.
%
%This book is distributed in the hope that it will be useful, but WITHOUT ANY WARRANTY; without even the implied warranty of MERCHANTABILITY or FITNESS FOR A PARTICULAR PURPOSE.  See the GNU General Public License for more details, http://www.gnu.org/licenses/.
\chapter{Describing motion in one dimension}
\label{chapter:describingmotionin1d}
In this chapter, we will introduce the tools required to describe motion in one dimension. In later chapters, we will use the theories of physics to model the motion of objects, but first, we need to make sure that we have the tools to describe the motion. We generally use the word ``kinematics'' to label the tools for describing motion (e.g. speed, acceleration, position, etc), whereas we refer to ``dynamics'' when we use the laws of physics to predict motion (e.g. what motion will occur if a force is applied to an object). 

\begin{learningObjectives}
{\item Describe motion in 1D using functions and defining an axis.
\item Define position, velocity, speed, and acceleration.
\item Use calculus to describe motion.
\item Define the meaning of an inertial frame of reference.
\item Use Galilean and Lorentz transformations to convert the description of an object's position from one inertial frame to another.}
\end{learningObjectives}

\begin{opening}
\begin{MCquestion} {You throw a ball upwards with an initial speed $v$. Assume there is no air resistance. When you catch the ball, its speed will be...}
\item greater than $v$.
\item equal to $v$. \correct
\item less than $v$.
\end{MCquestion}
\end{opening}


The most simple type of motion to describe is that of a particle that is constrained to move along a straight line (one-dimensional motion); much like a train along a straight piece of track. When we say that we want to describe the motion of the particle (or train), what we mean is that we want to be able to say where it is at what time. Formally, we want to know the particle's \textbf{position as a function of time}, which we will label as $x(t)$. The function will only be meaningful if:
\begin{itemize}
\item we specify an $x$-axis and the direction that corresponds to increasing values of $x$
\item we specify an origin where $x=0$
\item we specify the units for the quantity, $x$.
\end{itemize}
That is, unless all of these are specified, you would have a hard time describing the motion of an object to one of your friends over the phone. 

\capfig{0.4\textwidth}{figures/DescribingMotionIn1D/1daxis.png}{\label{fig:DescribingMotionIn1D:1daxis.png}In order to describe the motion of the grey ball along a straight line, we introduce the x-axis, represented by an arrow to indicate the direction of increasing $x$, and the location of the origin, where $x=\SI{0}{m}$. Given our choice of origin, the ball is currently at a position of $x=\SI{0.5}{m}$.
}
Consider Figure \ref{fig:DescribingMotionIn1D:1daxis.png} where we would like to describe the motion of the grey ball as it moves along a straight line. In order to quantify where the ball is, we introduce the ``$x$-axis'', illustrated by the black arrow. The direction of the arrow corresponds to the direction where $x$ increases (i.e. becomes more positive). We have also chosen a point where $x=0$, and by convention, we choose to express $x$ in units of meters (the S.I. unit for the dimension of length).

Note that we are completely free to choose both the direction of the $x$-axis and the location of the origin. The $x$-axis is a mathematical construct that we introduce in order to describe the physical world; we could just as easily have chosen for it to point in the opposite direction with a different origin. Since we are completely free to choose where we define the $x$-axis, we should choose the option that is most convenient to us. 

\section{Motion with constant speed}
Now suppose that the ball in Figure \ref{fig:DescribingMotionIn1D:1daxis.png} is rolling, and that we recorded its x position every second in a table and obtained the values in Table \ref{tab:DescribingMotionIn1D:1dmotion} (we will ignore measurement uncertainties and pretend that the values are exact).
\begin{table}[!h]
\centering
\begingroup
\renewcommand{\arraystretch}{1.0}
\begin{tabular}{cc}
\textbf{Time [s]}&\textbf{X position [m]}\\
\hline
\hline
\SI{0.0}{s}& \SI{0.5}{m}\\ \hline
\SI{1.0}{s}& \SI{1.0}{m}\\ \hline
\SI{2.0}{s}& \SI{1.5}{m}\\ \hline
\SI{3.0}{s}& \SI{2.0}{m}\\ \hline
\SI{4.0}{s}& \SI{2.5}{m}\\ \hline
\SI{5.0}{s}& \SI{3.0}{m}\\ \hline
\SI{6.0}{s}& \SI{3.5}{m}\\ \hline
\SI{7.0}{s}& \SI{4.0}{m}\\ \hline
\SI{8.0}{s}& \SI{4.5}{m}\\ \hline
\SI{9.0}{s}& \SI{5.0}{m}\\ \hline
\end{tabular}
\caption{\label{tab:DescribingMotionIn1D:1dmotion} Position of a ball along the x-axis recorded every second.}
\endgroup
\end{table}
The easiest way to visualize the values in the table is to plot them on a graph. Plotting position as a function of time is one of the most common graphs to make in physics, since it is often a complete description of the motion of an object. We can easily plot these values in Python:
\begin{python}[caption=Plotting position versus time] 
#First, we load pylab module for plotting
import pylab as pl
#We define t as a list of values (note the square brackets):
t = [0.0, 1.0, 2.0, 3.0, 4.0, 5.0, 6.0, 7.0, 8.0, 9.0]
#Similarly, we define the corresponding positions:
x = [0.5, 1.0, 1.5, 2.0, 2.5, 3.0, 3.5, 4.0, 4.5, 5.0]
#Define the plot:
pl.plot(t,x,'.')# the '.' means that it will show the actual points instead of a line
#Set the range of the axes, add some labels and a grid
pl.ylim(0,6)
pl.xlim(0,10)
pl.xlabel('time [s]')
pl.ylabel('position [m]')
pl.grid()
#Show the plot
pl.show()
\end{python}
\begin{poutput}
(* \capfig{0.7\textwidth}{figures/DescribingMotionIn1D/1dxvst.png}{\label{fig:DescribingMotionIn1D:1dxvst}Plot of position as a function of time using the values from Table \ref{tab:DescribingMotionIn1D:1dmotion}.} *)
\end{poutput}

The data plotted in Figure \ref{fig:DescribingMotionIn1D:1dxvst} show that the $x$ position of the ball increases linearly with time (i.e. it is a straight line). This means that in equal time increments, the ball will cover equal distances. Note that we also had the liberty to choose when we define $t=0$; in this case, we chose that time is zero when the ball is at $x=\SI{0.5}{m}$. 

\begin{checkpoint}{Using the data from Table \ref{tab:DescribingMotionIn1D:1dmotion}, at what position along the x-axis will the ball be when time is $t=\SI{9.5}{s}$, if it continues its motion undisturbed?} %5.25m
\end{checkpoint} 

Since the position as a function of time for the ball plotted in Figure \ref{fig:DescribingMotionIn1D:1dxvst} is linear, we can summarize our description of the motion using a function, $x(t)$, instead of having to tabulate the values as we did in Table \ref{tab:DescribingMotionIn1D:1dmotion}. The function will have the functional form:
\begin{align*}
x(t) = a + b t
\end{align*}
The constant $a$ is the ``offset'' of the function, the value that the function has at $t=\SI{0}{s}$. The constant $b$ is the slope and gives the rate of change of the position as a function of time. We can determine the values for the constants $a$ and $b$ by choosing any two rows from Table \ref{tab:DescribingMotionIn1D:1dmotion} (to determine 2 unknown quantities, you need 2 equations), and obtain 2 equations and 2 unknowns. For example, choosing the points where $t=\SI{0}{s}$ and $t=\SI{2.0}{s}$:
\begin{align*}
x(t=\SI{0}{s})&=\SI{0.5}{m}=a + b(\SI{0}{s}) \\
x(t=\SI{2.0}{s})&=\SI{1.5}{m}=a + b(\SI{2.0}{s}) \\
\end{align*}
The first equation immediately gives $a = \SI{0.5}{m}$, which we can substitute into the second equation to get $b$:
\begin{align*}
\SI{1.5}{m}&=a + b(\SI{2.0}{s}) = \SI{0.5}{m} + b(\SI{2.0}{s})\\
\therefore b &=\frac{(\SI{1.5}{m})-(\SI{0.5}{m})}{(\SI{2.0}{s})}=\SI{0.5}{m/s}
\end{align*}
which gives us the functional form for $x(t)$:
\begin{align*}
x(t) = (\SI{0.5}{m}) + (\SI{0.5}{m/s}) t
\end{align*}
where you should note that $a$ and $b$ have different dimensions. Since $a$ is added to something that must then give dimensions of length (for position, $x$), $a$ has dimensions of length. $b$ is multiplied by time, and that product must have dimensions of length as well; $b$ thus has dimensions of length over time, or ``speed'' (with S.I. units of \si{m/s}).

We can generalize the description of an object whose position increases linearly with time as:
\begin{align}
\label{eqn:DescribingMotionIn1D:1dxvst_noa}
\Aboxed{x(t) = x_0 + v_xt}
\end{align}
where $x_0$ is the position of the object at time $t=\SI{0}{s}$ ($a$ from above), and $v_x$ corresponds to the distance that the object covers per unit time ($b$ from above) along the x-axis. We call $v_x$ the ``velocity'' of the object. If $v_x$ is large, then the object covers more distance in a given time, i.e. it moves faster. If $v_x$ is a negative number, then the object moves in the negative $x$ direction.

\capfig{0.7\textwidth}{figures/DescribingMotionIn1D/1dturn.png}{\label{fig:DescribingMotionIn1D:1dturn}Position as a function of time for an object.}
\begin{checkpoint}
\begin{MCquestion}{Referring to Figure \ref{fig:DescribingMotionIn1D:1dturn}, what can you say about the motion of the object? }
\item The object moved faster and faster between $t=\SI{0}{s}$ and $t=\SI{30}{s}$, then slowed down to a stop at $t=\SI{60}{s}$.
\item The object moved in the positive x-direction between $t=\SI{0}{s}$ and $t=\SI{30}{s}$, and then turned around and moved in the negative x-direction between $t=\SI{30}{s}$ and $t=\SI{60}{s}$. %correct
\item The object moved faster between $t=\SI{0}{s}$ and $t=\SI{30}{s}$ than it did between $t=\SI{30}{s}$ and $t=\SI{60}{s}$.
\end{MCquestion}
\end{checkpoint}

\capfig{0.7\textwidth}{figures/DescribingMotionIn1D/1d2objects.png}{\label{fig:DescribingMotionIn1D:1d2objects}Positions as a function of time for two objects.}
\begin{checkpoint}
\begin{MCquestion}{Referring to Figure \ref{fig:DescribingMotionIn1D:1d2objects}, what can you say about the motion of the two objects? }
\item Object 1 is slower than Object 2
\item Object 1 is more than twice as fast as Object 2 %correct
\item Object 1 is less than twice as fast as Object 2
\end{MCquestion}
\end{checkpoint}

\section{Motion with constant acceleration}
Until now, we have considered motion where the velocity is a constant (i.e. where velocity does not change with time). Suppose that we wish to describe the position of a falling object that we released from rest at time $t=\SI{0}{s}$. The object will start with a velocity of \SI{0}{m/s} and it will \textbf{accelerate} as it falls. We say that an object is ``accelerating'' if its velocity is not constant. As we will see in later chapters, objects that fall near the surface of the Earth experience a constant acceleration (their velocity changes at a constant rate).

Formally, we define acceleration as the rate of change of velocity. Recall that velocity is the rate of change of position, so acceleration is to velocity what velocity is to position. In particular, we saw that if the velocity, $v_x$, is constant, then position as a function of time is given by:
\begin{align}
x(t) = x_0 + v_xt \tag{\ref{eqn:DescribingMotionIn1D:1dxvst_noa}}
\end{align} 
In analogy, if the acceleration is constant, then the velocity as a function of time is given by:
\begin{align}
\label{eqn:DescribingMotionIn1DL1dvvst}
\Aboxed{v_x(t) = v_{0x} + a_xt }
\end{align}
where $a_x$ is the ``acceleration'' and $v_{0x}$ is the velocity of the object at $t=0$. We can work out the dimensions of acceleration for this equation to make sense. Since we are adding $v_{0x}$ and $a_xt$, we need the dimensions of $a_xt$ to be velocity:
\begin{align*}
[a_xt] &= \frac{L}{T} \\
[a_x][t] &= \frac{L}{T} \\
[a_x]T&= \frac{L}{T} \\
[a_x]&= \frac{L}{T^2} \\
\end{align*}
Acceleration thus has dimensions of length over time squared, with corresponding S.I. units of m/s$^2$ (meters per second squared or meters per second per second). 

Now that we have an understanding of acceleration, how do we describe the position of an object that is accelerating? We cannot use equation \ref{eqn:DescribingMotionIn1D:1dxvst_noa}, since it is only correct if the velocity is constant. 

\capfig{0.1\textwidth}{figures/DescribingMotionIn1D/1daxis_vertical.png}{\label{fig:DescribingMotionIn1D:1daxis_vertical} X-axis for an object that starts at rest at $x=\SI{0}{m}$ when $t=\SI{0}{s}$ and falls downwards (in the direction of increasing $x$).}

Let us work out the corresponding equation for position as a function of time for accelerated motion using the x-axis depicted in Figure \ref{fig:DescribingMotionIn1D:1daxis_vertical}. We will determine $x(t)$ for the grey ball that starts at rest ($v_{0x}=\SI{0}{m/s}$) at the position $x=\SI{0}{m}$ at time $t=\SI{0}{s}$ with a constant positive acceleration $a_x=\SI{10}{m/s\squared}$. We would like to use equation \ref{eqn:DescribingMotionIn1D:1dxvst_noa}, but we cannot because it only applies if the velocity is constant. To remedy this, we pretend (we ``approximate'') that for a very small amount of time, the velocity is almost constant. Let us take a very small interval in time, say $\Delta t=\SI{0.001}{s}$, and approximate that the velocity is constant during that interval. 

At $t=\SI{0}{s}$, we have $x=\SI{0}{m}$, $v_{0x}=\SI{0}{m/s}$ and $a_x=\SI{10}{m/s\squared}$. We can use equation \ref{eqn:DescribingMotionIn1DL1dvvst} to find the velocity at $t=\Delta t$ (at the end of the first interval):
\begin{align*}
v_x(t=\Delta t) &= v_{0x} + a_x\Delta t\\
&=(\SI{0}{m/s})+ a_x\Delta t\\&=a_x\Delta t
\end{align*}

The average velocity during the first interval, $v_1^{avg}$ is then given by averaging the velocity at the beginning and at the end of the interval:
\begin{align*}
v_1^{avg}(t=\Delta t)&=\frac{1}{2}\left[ v(t=0) + v(t=\Delta t)\right]\\
&=\frac{1}{2}\left(v_{0x}+a_x\Delta t\right)\\
&=\frac{1}{2}\left((\SI{0}{m/s})+a_x\Delta t\right)\\
&=\frac{1}{2}(\SI{10}{m/s^2})(\SI{0.001}{s})\\
&=\SI{0.005}{m/s}
\end{align*}
Using the average velocity during the interval, we can use equation \ref{eqn:DescribingMotionIn1D:1dxvst_noa} to find the position at $t=\Delta t$: 
\begin{align*}
x(t=\Delta t) &= x_0 +v_1^{avg}\Delta t\\
&=(\SI{0}{m}) + \frac{1}{2}a_x(\Delta t)^2\\
&= \frac{1}{2}(\SI{10}{m/s^2})(\SI{0.001}{s})^2\\
&=\SI{0.000005}{m}
\end{align*}
Thus, at time $t=\SI{0.001}{s}$, the object will have a velocity of $v=\SI{0.005}{m/s}$ and be at a position $x=\SI{0.000005}{m}$. We can now use these values as the starting velocity and position for the next interval in time. Using variables, at the beginning of the second interval, the velocity is $v(t=\Delta t)=a_x\Delta t$ and at the end of the second interval, it will be $v(t=2\Delta t)=2a_x\Delta t$. The average velocity during the second interval is thus given by:
\begin{align*}
v_2^{avg}(t=2\Delta t)&= \frac{1}{2}\left[v(t=\Delta t)+v(t=2\Delta t) \right]\\
&=\frac{1}{2}(a_x\Delta t+2a_x\Delta t)\\
&=\frac{3}{2}a_x\Delta t\\
&=\frac{3}{2}(\SI{10}{m/s^2})(\SI{0.001}{s})\\
&=\SI{0.015}{m/s}
\end{align*}
To find the position at the end of the second time interval, when $t=2\Delta t$, we use equation \ref{eqn:DescribingMotionIn1D:1dxvst_noa} again, but with a different starting position and the average velocity that we just found:
\begin{align*}
x(t=2\Delta t) &= x(t=\Delta t) +v_2^{avg}\Delta t\\
&= \frac{1}{2}a_x(\Delta t)^2+\frac{3}{2}a(\Delta t)^2\\
&= \frac{1}{2}a_x(2\Delta t)^2\\
&=\frac{1}{2}(\SI{10}{m/s^2})(2\times\SI{0.001}{s})^2=\SI{0.00002}{m}
\end{align*}
You can carry out this exercise to ultimately find the position at any time. However, if you carry it out over a few more intervals, you may notice the following pattern: For the Nth interval when $t=N\Delta t$ at the end of the interval, we have:
\begin{align*}
v(t=(N-1)\Delta t) &= a_x (N-1) \Delta t &\text{(at beginning of interval N)}\\
v(t=N\Delta t) &= a_x N \Delta t &\text{(at end of interval N)}\\
v_N^{avg}&=\frac{1}{2}a_x(2N-1)\Delta t&\text{(average during interval)}\\
x(t=N\Delta t)&=\frac{1}{2}a_x(N\Delta t)^2&\text{(position at end of interval)}
\end{align*}

The last line gives us exactly what we were after, namely the position as a function of time for a constant acceleration, $a_x$, when the object started at rest at a position of $x=\SI{0}{m}$:
\begin{align}
\label{eqn:DescribingMotionIn1D:1dxoft_novonoxo}
 x(t) = \frac{1}{2} a_x t^2
\end{align}

If at $t=0$, the object had an initial position along the x-axis of $x_0$, then the position $x(t)$ would be shifted by an amount $x_0$:

\begin{align}
\label{eqn:DescribingMotionIn1D:1dxoft_novo}
 x(t) = x_0+\frac{1}{2} a_x t^2
\end{align}

Finally, if the object had an initial speed $v_{0x}$ at $t=0$, one can easily reproduce the iterations above to find that we need to add an additional term to account for this. We arrive at the general description of the position of an object moving in a straight line with acceleration, $a_x$:
\begin{align}
\label{eqn:DescribingMotionIn1D:1dxvst}
\Aboxed{ x(t) = x_0+v_{0x}t+ \frac{1}{2}a_xt^2}
\end{align}
Note that equation \ref{eqn:DescribingMotionIn1D:1dxvst_noa} is just a special case of the above when $a=0$. 

\begin{example}{A ball is thrown upwards with a velocity of \SI{10}{m/s}. After what distance will the ball stop before falling back down? Assume that gravity causes a constant downwards acceleration of \SI{9.8}{m/s^2}.}
\label{ex:DescribingMotionIn1D:ballupandown}
We will solve this problem in the following steps:
\begin{enumerate}[topsep=-10pt]
\item Setup a coordinate system (define the x-axis).
\item Identify the condition that corresponds to the ball stopping its upwards motion and falling back down.
\item Determine the distance at which the ball stopped.
\end{enumerate}
Since we throw the ball upwards with an initial velocity upwards, it makes sense to choose an x-axis that points up and has the origin at the point where we release the ball. With this choice, referring to the variables in equation \ref{eqn:DescribingMotionIn1D:1dxvst}, we have:
\begin{align*}
x_0&=0\\
v_{0x}&=+\SI{10}{m/s}\\
a_x&=\SI{-9.8}{m/s^2}
\end{align*}
where the initial velocity is in the positive x-direction, and the acceleration, $a_x$, is in the negative direction (the velocity will be getting smaller and smaller, so its rate of change is negative).

The condition for the ball to stop at the top of the trajectory is that its velocity will be zero (that is what it means to stop). We can use equation \ref{eqn:DescribingMotionIn1DL1dvvst} to find what time that corresponds to:
\begin{align*}
v(t) &= v_{0x}+a_xt\\
0 &= (\SI{10}{m/s}) + (\SI{-9.8}{m/s^2})t\\
\therefore t&=\frac{(\SI{10}{m/s})}{(\SI{9.8}{m/s^2})}=\SI{1.02}{s}
\end{align*}
Now that we know that it took \SI{1.02}{s} to reach the top of the trajectory, we can find how much distance was covered:
\begin{align*}
x(t) &= x_0+v_{0x}t+ \frac{1}{2}a_xt^2\\
x &= (\SI{0}{m})+(\SI{10}{m/s})(\SI{1.02}{s})+\frac{1}{2}(\SI{-9.8}{m/s^2})(\SI{1.02}{s})^2 = \SI{5.10}{m}
\end{align*}
and we find that the ball will rise by \SI{5.10}{m} before falling back down. 
\end{example}

\subsection{Visualizing motion with constant acceleration}

When an object has a constant acceleration, its velocity and position as a function of time are described by the two following equations:
\begin{align*}
v(t) &= v_{0x} + a_xt\\
x(t) &= x_0+v_{0x}t+ \frac{1}{2}a_xt^2
\end{align*}
where the velocity changes linearly with time, and the position changes quadratically with time (it goes as $t^2$). Figure \ref{fig:DescribingMotionIn1D:1dxvvst_aconst} shows the position and the speed as a function of time for the ball from example \ref{ex:DescribingMotionIn1D:ballupandown} for the first three seconds of the motion.

\capfig{0.7\textwidth}{figures/DescribingMotionIn1D/1dxvvst_aconst.png}{\label{fig:DescribingMotionIn1D:1dxvvst_aconst} Position and speed as a function of time for the ball in example \ref{ex:DescribingMotionIn1D:ballupandown}.}

We can divide the motion into three parts (shown by the vertical dashed lines in Figure \ref{fig:DescribingMotionIn1D:1dxvvst_aconst}):

\textbf{1) Between $t=\SI{0}{s}$ and $t=\SI{1.02}{s}$}

At time $t=\SI{0}{s}$, the ball starts at a position of $x=\SI{0}{m}$ (left) and a speed of $v_{0x}=\SI{10}{m/s}$ (right). During the first second of motion, the position, $(t)$, increases (the ball is moving up), until the position stops increasing at $t=\SI{1.02}{s}$, as found in example \ref{ex:DescribingMotionIn1D:ballupandown}. During that time, the velocity decreases linearly from \SI{10}{m/s} to \SI{0}{m/s} due to the constant negative acceleration from gravity. At $t=\SI{1,02}{s}$, the velocity is instantaneously \SI{0}{m/s} and the ball is momentarily at rest (as it reaches the top of the trajectory before falling back down).

\textbf{2) Between $t=\SI{1.02}{s}$ and $t=\SI{2.04}{s}$}

At $t=\SI{1.02}{s}$, the velocity continues to decrease linearly (it becomes more and more negative) as the ball start to fall back down faster and faster. The position also starts decreasing just after $t=\SI{1,02}{s}$, as the ball returns back down to the point of release. At $t=\SI{2.04}{s}$, the ball returns to the point from which it was thrown, and the ball is going with the same velocity (\SI{10}{m/s}) as when it was released, but in the opposite direction (downwards).

\textbf{3) After $t=\SI{2.04}{s}$}

If nothing is there to stop the ball, it continues to move downwards with ever increasing velocity. The position continues to become more negative and the velocity continues to become larger in magnitude and more negative.

\begin{checkpoint}{Make a sketch of the acceleration as a function of time corresponding to the position and velocity shown in Figure \ref{fig:DescribingMotionIn1D:1dxvvst_aconst}.}
\end{checkpoint}

\subsection{Speed versus velocity}
In the previous example, our language was not quite as precise as it should be when conducting science. Specifically, we need a way to distinguish the situation when the velocity is decreasing (becoming more negative), while the object is actually going faster and faster (after $t=\SI{1.02}{s}$ in Figure \ref{fig:DescribingMotionIn1D:1dxvvst_aconst}). We will use the term \textbf{speed} to refer to how fast an object is moving (how much distance it covers per unit time), and we will use the term \textbf{velocity} to also indicate the direction of the motion. In other words, the speed is the absolute value of the velocity\footnote{This is true for one-dimensional motion, whereas in two or more dimensions, velocity is a vector and speed is the magnitude of that vector.}. The speed is thus always positive, whereas the velocity can also be negative.

With this vocabulary, the speed of the ball in Figure \ref{fig:DescribingMotionIn1D:1dxvvst_aconst} decreases between $t=\SI{0}{s}$ and $t=\SI{1.02}{s}$, and increases thereafter. On the other hand, the velocity continuously decreases (it is always becoming more and more negative). Velocity is thus the more general term since it tells us both the speed and the direction of the motion. 

\section{Using calculus to describe motion}
Objects do not necessarily have a constant velocity or acceleration. We thus need to extend our description of the position and velocity of an object to a more general case. This can be done in much the same way as we introduced accelerated motion; namely by pretending that during a very small interval in time, $\Delta t$, the velocity and acceleration are constant, and then considering the motion as the sum over many small intervals in time. In the limit that $\Delta t$ tends to zero, this will be an accurate description. 

\subsection{Instantaneous and average velocity}

Suppose that an object is moving with a non constant velocity, and covers a distance $\Delta x$ in an amount of time $\Delta t$. We can define an \textbf{average velocity}, $v^{avg}$:
\begin{align*}
v^{avg}= \frac{\Delta x}{\Delta t}
\end{align*}
That is, regardless of our choice of time interval, $\Delta t$, we can always calculate the average velocity, $v^{avg}$, over the time interval. That average velocity will be an average over the interval, between some time $t$ and $t+\Delta t$. If we shrink the time interval, and take the limit $\Delta t\to 0$, we can define the \textbf{instantaneous velocity}:
\begin{align*}
v = \lim_{\Delta t\to 0} \frac{\Delta x}{\Delta t}
\end{align*}
The instantaneous velocity is the velocity only in that small instant in time where we choose $\Delta x$ and $\Delta t$. Another way to read this equation is that the velocity, $v$, is the slope of the graph of $x(t)$. Recall that the slope is the ``rise over run'', in other words, the change in $x$ divided by the corresponding change in $t$. Indeed, when we had no acceleration, the position as a function of time, equation \ref{eqn:DescribingMotionIn1D:1dxvst_noa}, explicitly had the velocity as the slope of a linear function:
 \begin{align*}
 x(t) = v_{0x}+v_xt
 \end{align*}
 If we go back to Figure \ref{fig:DescribingMotionIn1D:1dxvvst_aconst}, where velocity was no longer constant, we can indeed see that the graph of the velocity versus time, $v(t)$, corresponds to the instantaneous slope of the graph of position versus time, $x(t)$. For $t<\SI{1.02}{s}$, the slope of the $x(t)$ graph is positive but decreasing (as is $v(t)$). At $t=\SI{1.02}{s}$, the slope of $x(t)$ is instantaneously \SI{0}{m/s} (as is the velocity). Finally, for $t>\SI{1.02}{s}$, the slope of $x(t)$ is negative and increasing in magnitude, as is $v(t)$.

Leibniz and Newton were the first to develop mathematical tools to deal with calculations that involve quantities that tend to zero, as we have here for our time interval $\Delta t$. Nowadays, we call that field of mathematics ``calculus'', and we will make use of it here. Using the vocabulary of calculus, rather than saying that ``instantaneous velocity is the slope of the graph of position versus time at some point in time'', we say that ``instantaneous velocity is the time derivative of position as a function of time''. We also use a slightly different notation so that we do not have to write the limit $\lim_{\Delta t\to 0}$:
\begin{align}
\label{eqn:DescribingMotionIn1D:vdef}
\Aboxed{v(t)=\lim_{\Delta t\to 0} \frac{\Delta x}{\Delta t}=\frac{dx}{dt}=\frac{d}{dt} x(t)}
\end{align}
where we can really think of $dt$ as $\lim_{\Delta t\to 0}\Delta t$, and $dx$ as the corresponding change in position over an \textit{infinitesimally} small time interval $dt$.

Similarly, we introduce the \textbf{instantaneous acceleration}, as the time derivative of $v(t)$:
\begin{align}
\Aboxed{a_x(t)=\frac{dv}{dt}=\frac{d}{dt}v(t)}
\end{align}

\begin{studentOpinion}{Olivia}
When looking at a graph of position versus time, it is sometimes hard to tell at first glance whether the speed of the object is increasing or decreasing. This section gives us an easy way to figure it out. The velocity is the instantaneous slope of the graph $x(t)$, so the speed is the ``steepness" of that slope. Simply draw a few lines that are tangent to (meaning just touching) the curve, and see what happens as time increases. If the lines get steeper, the object is speeding up. If they are getting flatter, the object is slowing down.
\capfig{0.7\textwidth}{figures/DescribingMotionIn1D/SpeedingSlowing.png}{\label{fig:DescribingMotionIn1D:speedingslowing}Two graphs of $x(t)$ showing tangent lines. Left: the object is speeding up (positive velocity, positive acceleration). Right: the object is slowing down (positive velocity, negative acceleration).} 
From here, you can also figure out what the direction of the acceleration is. If an object is speeding up, the acceleration and velocity must be in the same direction (i.e. both positive or both negative). If the object is slowing down, they must be in opposite directions. Imagine the graphs in Figure \ref{fig:DescribingMotionIn1D:speedingslowing} are describing the motion of a person running in heavy wind. In the graph on the left, the person is running with the wind ($v(t)$ and $a(t)$ positive), and in the second graph the person is running against the wind ($v(t)$ positive and $a(t)$ negative). 
\end{studentOpinion} 



\subsection{Using calculus to obtain acceleration from position}
Suppose that we know the function for position as a function of time, and that it is given by our previous result (for the case when the acceleration $a_x$ is constant):
\begin{align*}
x(t)=x_0+v_{0x}t+\frac{1}{2}a_xt^2
\end{align*}
The velocity is given by taking the derivative of $x(t)$ with respect to time:
\begin{align*}
v(t)&=\frac{dx}{dt}=\frac{d}{dt}\left(x_0+v_{0x}t+\frac{1}{2}a_xt^2\right)\\
&=v_{0x}t+a_xt
\end{align*}
as we found before, in equation \ref{eqn:DescribingMotionIn1DL1dvvst}. The acceleration is then given by the time-derivative of the velocity:
\begin{align*}
a_x &= \frac{dv}{dt}=\frac{d}{dt}\left(v_{0x}t+a_xt\right)\\
&=a_x
\end{align*}
as expected.


\begin{checkpoint}
\begin{MCquestion}{Chlo\"e has been working on a detailed study of how vicu\~nas\footnote{Never heard of vicu\~nas? Internet!} run, and found that their position as a function of time when they start running is well modelled by the function $x(t)=(\SI{40}{m/s^2})t^2+(\SI{20}{m/s^3})t^3$. What is the acceleration of the vicu\~nas?}
\item $a_x(t)=\SI{40}{m/s^2}$
\item $a_x(t)=\SI{80}{m/s^2}$
\item $a_x(t)=\SI{40}{m/s^2}+(\SI{20}{m/s^3})t$
\item $a_x(t)=\SI{80}{m/s^2}+(\SI{120}{m/s^3})t$ % correct
\end{MCquestion}
\end{checkpoint}

\subsection{Using calculus to obtain position from acceleration}
Now that we saw that we can use derivatives to determine acceleration from position, we will see how to do the reverse and use acceleration to determine position. Let us suppose that we have a constant acceleration, $a_x(t)=a_x$, and that we know that at time $t=\SI{0}{s}$, the object had a speed of $v_{0x}$ and was located at a position $x_0$. 

Since we only know the acceleration as a function of time, we first need to find the velocity as a function of time. We start with:
\begin{align*}
a_x(t)=a_x=\frac{d}{dt} v(t)
\end{align*}
which tells us that we know the slope (derivative) of the function $v(t)$, but not the actual function. In this case, we must do the opposite of taking the derivative, which in calculus is called taking the ``anti-derivative'' with respect to $t$ and has the symbol $\int dt$. In other words, if:
\begin{align*}
\frac{d}{dt} v(t) =a_x(t)
\end{align*}
then:
\begin{align*}
v(t) =\int a_x(t) dt
\end{align*}
Since in this case, $a_x(t)$ is a constant, $a_x$, the anti-derivative is easily found:
\begin{align*}
\int a_xdt = a_xt + C
\end{align*}
The velocity is thus given by:
\begin{align*}
v(t) &=\int a_x dt =a_xt+C
\end{align*}
The constant $C$ is determined by what we call our ``initial conditions''. In this case, we stated that at time $t=0$, the velocity should be $v_{0x}$. The constant $C$ is thus $v_{0x}$:
\begin{align*}
v(t) &=C+a_x t =v_{0x}+a_xt
\end{align*}
and we recover the formula for velocity when the acceleration is constant. Now that we know the velocity as a function of time, we can take one more anti-derivative with respect to time to obtain the position:
\begin{align*}
v(t) &= \frac{dx}{dt}\\
\therefore x(t) &= \int v(t)dt 
\end{align*}
In the case where acceleration is constant, this gives:
\begin{align*}
 x(t) &= \int v(t)dt\\
 &=\int (v_{0x}+a_xt )dt\\
 &=v_{0x}t+\frac{1}{2}a_xt^2+C'
\end{align*} 
where $C'$ is a different constant than the one we had when determining velocity. The constant is given by our initial conditions. If the object was located at position $x=x_0$ at time $t=0$, then $C'=x_0$ and we recover the equation for position as a function of time for constant acceleration:
\begin{align*}
x(t)=x_0+v_{0x}t+\frac{1}{2}a_xt^2
\end{align*}

\begin{checkpoint}
\begin{MCquestion}{The acceleration of a cricket jumping sideways is observed to increase linearly with time, that is, $a_x(t)=a_0+jt$, where $a_0$ and $j$ are constants. What can you say about the velocity of the cricket as a function of time?}
\item it is constant
\item it increases linearly with time ($v(t)\propto t$)
\item it increases quadratically with time ($v(t)\propto t^2$) %correct
\item it increases with the cube of time ($v(t)\propto t^3$)
\end{MCquestion}
\end{checkpoint}

\begin{checkpoint}
\begin{MCquestion}{Choose the graph of $x(t)$ for the case when acceleration is given by $\cos(\omega t)$, where $\omega$ is a constant. The velocity and position are zero at $t=0$
\capfig{0.7\textwidth}{figures/DescribingMotionIn1D/xfromacheckpoint.png}{\label{fig:DescribingMotionIn1D:xfromacheckpoint} Choose the correct position versus time graph.}}
\item Figure A
\item Figure B
\item Figure C %correct
\end{MCquestion}
\end{checkpoint}



\section{Relative motion}
In order to describe the motion of an object confined to a straight line, we introduced an axis ($x$) with a specified direction (in which $x$ increases) and an origin (where $x=0$). Sometimes, it can be more convenient to use an axis that is \textit{moving}. For example, consider a person, Alice, moving inside of a train headed for the French town of Nice. The train is moving with a constant speed, $v'^B$ as measured from the ground. Suppose that another person, Brice, describes Alice's position using the function $x^A(t)$ using an x-axis defined inside of the train car ($x=0$ where Brice is sitting, and positive $x$ is in the direction of the train's motion), as depicted in Figure \ref{fig:DescribingMotionIn1D:TrainABC} below. As long as any person is in the train with Brice, they will easily be able to describe Alice's motion using the x-axis that is moving with the train. Suppose that the train goes through the French town of Hossegor, where a surfer, Igor, watches the train go by. If Igor wishes to describe Alice's motion, it is easier for him to use a different axis, say $x'$, that is fixed to the ground and not moving with the train. 
\capfig{0.7\textwidth}{figures/DescribingMotionIn1D/TrainABC.png}{\label{fig:DescribingMotionIn1D:TrainABC}Alice is walking in the train and her position is described by both Brice, who is sitting in the train (using the $x$ axis), and Igor, who is at rest on the ground (using the $x'$ axis).} 

Since Brice already went through the work of determining the function $x^A(t)$ in the \textbf{reference frame} of the train, we wish to determine how to \textit{transform} $x^A(t)$ into the reference frame of the train station, $x'^A(t)$, so that Igor can also describe Alice's motion. In other words, we wish to describe Alice's motion in two different \textit{reference frames}.


A reference frame is simply a choice of coordinates, in this case, a choice of x-axis. Ideally, in physics, we prefer to use \textit{inertial} reference frames, which are reference frames that are either ``at rest'' or that are moving at a constant speed relative to a frame that we consider at rest.
 
 
In principle, if you blocked out all of the windows in the train, it would not be possible for Alice and Brice to determine if the train is moving at constant speed or if it is stopped. Thus, the concept of a ``rest frame'' is itself arbitrary. It is not possible to define a frame of reference that is truly at rest. Even Igor's frame of reference, the train station, is on the planet Earth, which is moving around the Sun with a speed of \SI{108000}{km/h}.


Not only is it impossible to define a frame of reference that is truly at rest, the rules from transforming from one frame to the other depend on the speed between the reference frames. Our common experience is described by what we call ``Galilean Relativity'', but if the speed between trains is very large, close to the speed of light, then we need to use Einstein's Special Theory of Relativity.

Referring to Figure \ref{fig:DescribingMotionIn1D:TrainABC}, we wish to use Brice's description of Alice's motion, $x^A(t)$, and convert it into a description, $x'^A(t)$ that Igor can use in the train station. Since Brice is at rest in the train, the speed of Brice \textit{relative} to Igor is $v'^B(t)$. The first step is for Igor to describe Brice's position, $x'^B(t)$, (that is, the position of Brice's origin). Assume that we choose $t=0$ to be the point in time where the two origins are aligned. Since the train is moving at a constant speed, $v_B$ (as measured by Brice), then the position of Brice's origin as measured from Igor's origin is given by:
\begin{align*}
x'^B(t)=v'^Bt
\end{align*}
Now that Igor can describe the position of the origin of Brice's coordinate system, he can use Brice's description of Alice's motion. Recall that $x^A(t)$ is Brice's measure of Alice's distance from his origin. Similarly, $x'^B(t)$, is Igor's measure of the distance from his origin to Brice's origin. Thus, to obtain Alice's distance from Igor's origin, we simply add the distance, $x'^B(t)$, from Igor's origin to Brice's origin, and then add, $x^A(t)$, the distance from Brice's origin to Alice. Thus:
\begin{align}
\Aboxed{x'^A(t)=x'^B(t)+x^A(t)=v'^Bt+x^A(t)}
\end{align}
which tells us how to obtain the position of object A in the $x'$ reference frame, when $x^A(t)$ is the description the object's position in the $x$ reference frame which is moving with a velocity $v'^B$ relative to the $x'$ reference frame.

Since we know the position of Alice as measured in Igor's frame of reference, we can now easily find her velocity and her acceleration, as measured by Igor. Her velocity as measured by Igor, $v'^A$, is given by the time-derivative of her position measured in Igor's frame of reference:
\begin{align}
v'^A(t)&=\frac{d}{dt}x'^A(t)\\
&=\frac{d}{dt}(v'^Bt+x^A(t))\\
&=v'^B+\frac{d}{dt}x^A(t)\\
&=v'^B+v^A(t)
\end{align}
where $v^A(t)=\frac{d}{dt}x^A(t)$ is Alice's speed as measured by Brice, in the train. That is, the velocity of Alice as measured by Igor is the sum of the velocity of the train relative to the ground and the velocity of Alice relative to the train, which makes sense. If we now determine Alice's acceleration, $a'^A(t)$, as measured by Igor, we find:
\begin{align}
a'^A(t)&=\frac{d}{dt}v'^A(t)\\
&=\frac{d}{dt}(v'^B+v^A)\\
&=0+\frac{d}{dt}v^A(t)\\
&=a^A
\end{align}
where we have explicitly used the fact that the train is moving at constant velocity ($\frac{d}{dt}v'^B=0$). Here we find that both Brice and Igor will measure the same number when referring to Alice's acceleration (if the train is moving at a constant velocity). This is a particularity of ``inertial'' frame of references: accelerations do not depend on the reference frame, as long as the reference frames are moving with a constant velocity relative to each other. As we will see later, forces exerted on an object are directly related to the acceleration experienced by that object. Thus, the forces on an object do not depend on the choice of inertial reference frame. 

\begin{example}{A large boat is sailing North at a speed of $v'^B=\SI{15}{m/s}$ and a restless passenger is walking about on the deck. Chlo\"e, another passenger on the boat, finds that the passenger is walking at a constant speed of $v^A=\SI{3}{m/s}$ towards the South (opposite the direction of the boat's motion). Marcel is watching the boat pass by from the shore. What velocity (magnitude and direction) does Marcel measure for the restless passenger?}
First, we must choose coordinate systems in the boat and on the shore. On the boat, let us define an $x$ axis that is positive in the North direction and has an origin such that the position of the restless passenger was $x^A(t=0)=0$ at time $t=0$. In Chlo\"e's reference frame, the passenger is thus described by:\\
\begin{align*}
x^A(t)=v^At=(\SI{-3}{m/s})t
\end{align*}
where we note that $v^A$ is negative since the passenger is moving the in negative $x$ direction (the passenger is walking towards the South, but we chose positive $x$ to be in the North direction). On shore, we choose an $x'$ axis that also is positive in the North direction. We can choose the origin such that the origin of the boat's coordinate system was $x'=0$. The origin of the boat's coordinate system as measured by Marcel (on shore) is thus:\\
\begin{align*}
x'^B(t)=v'^Bt=(\SI{15}{m/s})t
\end{align*}
The position of the passenger, $x'^A(t)$, as measured by Marcel, is then given by adding the position of the boat's origin and the position of the passenger as measured from the boat's origin:\\
\begin{align*}
x'^A(t) &= x'^B(t)+x^A(t)\\
&= v'^Bt + v^At \\
&= (v'^B+v^A)t\\
&= ((\SI{15}{m/s})+(\SI{-3}{m/s}))t\\
&= (\SI{12}{m/s})t
\end{align*}
To find the velocity of the passenger as measured by Marcel, we take the time derivative:\\
\begin{align*}
v'^A &= \frac{d}{dt}x'^A(t)\\
&= \frac{d}{dt} \left((v'^B+v^A)t\right)\\
&=(v'^B+v^A)\\
&=((\SI{15}{m/s})+(\SI{-3}{m/s}))\\
&=\SI{12}{m/s}
\end{align*}
Since this is a positive number, Marcel still sees the passenger moving in the North direction (the direction of his positive $x'$ axis), but with a speed of \SI{12}{m/s}, which is less than that of the boat. On the boat, the passenger appears to be walking towards the South, but the net motion of the passenger relative to the ground is still in the North direction, as their speed is less than that of the boat.
\end{example}


\newpage
\section{Summary}
\begin{chapterSummary}
To describe motion in one dimension, we must define an axis with:
\begin{enumerate}
\item An origin (where $x=0$)
\item A direction (the direction in which $x$ increases).
\end{enumerate}

We describe the position of an object with a function $x(t)$ that \textit{depends} on time. The rate of change of position is called ``velocity'', $v_x(t)$, and the rate of change of velocity is called ``acceleration'', $a_x(t)$:
\begin{align*}
v_x(t)&=\lim_{\Delta t\to 0}\frac{\Delta x}{\Delta t}=\frac{dx}{dt}\\
a_x(t)&=\lim_{\Delta t\to 0}\frac{\Delta v}{\Delta t}=\frac{dv_x}{dt}
\end{align*}
Given the acceleration, one can find the velocity and position:
\begin{align*}
v_x(t)&=\int a_x(t)dt\\
x(t)&=\int v_x(t)dt
\end{align*}
With a constant acceleration, $a_x(t)=a_x$, if the object had velocity $v_{0x}$ and position $x_0$ at $t=0$:\footnote{We did not derive the third of these kinematic equations in this chapter, but it is derived in problem \ref{prob:kinematicDerivation}.} 
\begin{align*}
v_x(t)&=v_{0x}t+a_xt\\
x(t)&=x_0+v_{0x}t+\frac{1}{2}a_xt^2\\
v^2-v_0^2&=2a\Delta x 
\end{align*}
An inertial frame of reference is one that is moving with a constant velocity. It is impossible to define a frame of reference that is truly ``at rest'', so we consider inertial frames of reference only relative to other frames of reference that we also consider to be inertial. If an object has position $x^A$ as measured in a frame of reference $x$ that is moving at constant speed $v'^B$ as measured in a second frame of reference $x'$, then in the $x'$ reference frame, the kinematic quantities for the object are obtained by the Galilean transformation:
\begin{align*}
x'^A(t) &= v'^Bt + x^A(t)\\
v'^A(t) &=v'^B+v^A(t)\\
a'^A(t) &= a(t)
\end{align*}
\end{chapterSummary}

\newpage
\begin{importantEquations}
\begin{multicols}{2}
\begin{center}
\textbf{Position, Velocity, and\\ Acceleration:}
\begin{align*}
v_x(t)&=\lim_{\Delta t\to 0}\frac{\Delta x}{\Delta t}=\frac{dx}{dt}\\
a_x(t)&=\lim_{\Delta t\to 0}\frac{\Delta v}{\Delta t}=\frac{dv_x}{dt}\\
v_x(t)&=\int a_x(t)dt\\
x(t)&=\int v_x(t)dt
\end{align*}
\textbf{Kinematic Equations:}
\begin{align*}
v_x(t)&=v_{0x}t+a_xt\\
x(t)&=x_0+v_{0x}t+\frac{1}{2}a_xt^2\\
v^2-v_0^2&=2a\Delta x 
\end{align*}
\end{center}
\columnbreak
\begin{center}
\textbf{Relative Motion:}\\
\begin{align*}
x'^A(t) &= v'^Bt + x^A(t)\\
v'^A(t) &=v'^B+v^A(t)\\
a'^A(t) &= a(t)
\end{align*}
\end{center}
\end{multicols}
\end{importantEquations}


\newpage
\section{Thinking about the material}
\subsection{Reflect and research}
\begin{enumerate}
\item Look up the depth of a competition diving pool. What is the relationship between the height of the diving platform and the minimum pool depth? Why? If the designers of the pool assumed that every diver drops straight down off the diving board, would the pool still be safe for divers that jump up first?
\item When did Galileo Galilei first describe his principles of Galilean Relativity?
\item In Galileo's ``Dialogue Concerning the Two Chief World Systems'', what example did he use to describe relative motion?
\item Imagine that you are a judge, trying to charge an irresponsible driver for speeding on the highway. In the courtroom, he argues that in his own frame of reference, he was sitting still with respect to his car. In fact, he says that it was the officer, parked on the side of the highway that was speeding. You realize that in his reference frame, he is indeed correct - but that's not what matters! How do you explain the relative motion of driving laws to this sneaky offender, in order to serve him justice?
\end{enumerate}

\subsection{To try at home}
\begin{tquestion}Design an experiment that you could perform at home to find the value of $g$, the acceleration due to gravity near the surface of the Earth. You will find the following equation to be useful:
\begin{align*} 
x(t)=x_0+v_{0x}+\frac{1}{2}gt^2$
\end{align*} 
Hint: Try to find a linear relationship in which $g$ is the slope. 
\end{tquestion}
\subsection{To try in the lab}

\section{Sample Problems and Solutions}
\subsection{Problems}
\begin{problem}{soln:describingmotionin1d:derivtimeindependent}{
\label{prob:describingmotionin1d:derivtimeindependent} Derive a kinematic equation that is independent of time. Specifically, derive: $v^2-v_0^2=2ax$, starting with equations \ref{eqn:DescribingMotionIn1DL1dvvst} and \ref{eqn:DescribingMotionIn1D:1dxvst}.}
\end{problem}

\begin{problemParts}{soln:describingmotionin1d:velociraptor}{\label{prob:describingmotionin1d:velociraptor}Rob is riding his bike at a speed of $\SI{8}{m/s}$. He passes by a velociraptor, as one often does, who is eating by the side of the road. The velociraptor begins chasing him. The velociraptor accelerates from rest at a rate of $\SI{4}{m/s^2}$.} 
\item Assuming it takes 3 seconds for the velociraptor to react, how long does it take from the moment Rob passes by for the velicoraptor to catch up to him? 
\item If there is a safe place 70 metres from where Rob passes the velociraptor, will Rob make it there in time to escape being eaten?  
\end{problemParts}

\begin{problem}{soln:describingmotionin1d:accelerationtime} {\label{prob:describingmotionin1d:accelerationtime}Figure \ref{fig:describingmotionin1d:accelerationtime} shows a graph of the acceleration, $a(t)$, of a particle moving in one dimension. Draw the corresponding velocity and position graphs. Assume that $v(0)=0$ and $x(0)=0$.}
\end{problem}
\capfig{0.7\textwidth}{figures/DescribingMotionIn1D/accelerationproblem.png}{\label{fig:describingmotionin1d:accelerationtime}A graph of acceleration as a function of time. The scale and units are arbitrary.}

\newpage
\subsection{Solutions}
\begin{solution}{prob:describingmotionin1d:derivtimeindependent}\label{soln:describingmotionin1d:derivtimeindependent}We start with the equations for position and velocity that we derived in this chapter:
\begin{align*}
x&=x_0+v_0t+\frac{1}{2}at^2\\
v&=v_0+at
\end{align*}
The first equation can be written as:
\begin{align*}
\Delta x&=v_0t+\frac{1}{2}at^2
\end{align*}
Our goal is to find an equation that is independent of time $t$. We start by isolating $t$ in our equation for velocity:
\begin{align*}
v&=v_0+at\\
t&=\frac{v-v_0}{a}
\end{align*}
We then substitute this value of $t$ into our equation for $\Delta x$:
\begin{align*}
\Delta x&=v_0t+\frac{1}{2}at^2\\
\Delta x&=v_0\left(\frac{v-v_0}{a}\right)+\frac{1}{2}a\left(\frac{v-v_0}{a}\right)^2
\end{align*}
We want the left hand side to be $2a\Delta x$, so we multiply each term by $2a$:
\begin{align*}
2a\Delta x&=(2a)v_0\left( \frac{v-v_0}{a}\right) +(2a)\frac{1}{2}a\left( \frac{v-v_0}{a}\right) ^2\\
2a\Delta x&=(2v_0)a\left(\frac{v-v_0}{a}\right)+a^2\left( \frac{v-v_0}{a}\right) ^2\\
2a\Delta x&=2v_0(v-v_0)+(v-v_0)^2
\end{align*}
We distribute $2v_0$ into the brackets. Then we expand the third term and get:
\begin{align*}
2a\Delta x&=(2v_0v-2v_0^2)+(v_0-v^2)(v_0-v^2)\\
2a\Delta x&=(2v_0v-2v_0^2)+(v_0^2-2v_0v+v^2)
\end{align*}
All that's left to do is collect like terms, and we get the formula we are looking for:
\begin{align*}
2a\Delta x&=2v_0v-2v_0^2+v_0^2-2v_0v+v^2\\
2a\Delta x&=(v^2)+(2v_0v-2v_0v)+(v_0^2-2v_0^2)\\
2a\Delta x&=v^2-v_0^2\\
v^2-v_0^2&=2a\Delta x\\
\therefore \text{QED}
\end{align*}
If you choose a coordinate system such that $x_0$, this equation becomes $v^2-v_0^2=2ax$.
\end{solution}

\newpage
\begin{solution}{prob:describingmotionin1d:velociraptor}\label{soln:describingmotionin1d:velociraptor}
We start by choosing our coordinate system. The solution is simplest if the $x$ axis is positive in the direction of motion and has an origin at the point where Rob passes the velociraptor. We set $t=0$ to be the moment the velociraptor starts running.\\

\capfig{\textwidth}{figures/DescribingMotionIn1D/velociraptorquestion.png}{\label{fig:DescribingMotionIn1D:velociraptorproblem1D} Rob is being chased by a velociraptor. At $t=0$, Rob is a distance $x_{0R}$ from the velociraptor. Safety is $\SI{70}{m}$ away from the origin.}

\begin{enumerate}[label=(\alph*)]
\item What do we mean by ``catch up"? It means that Rob and the velociraptor are in the same place at the same time. So, we are interested in the value of $t$ when $x_R=x_V$. 

We need two equations, one describing Rob's position and one describing the position of the velociraptor. Rob is moving at a constant velocity, so his position is described by:
\begin{align*}
x_R&=x_{0R}+v_{R}t
\end{align*}
The velociraptor has a constant acceleration, so its position is described by:
\begin{align*}
x_V&=x_{0V}+v_{0V}t+\frac{1}{2}a_Vt^2
\end{align*}
We can use a table to take stock of our known values:
\begin{table}[H]
\centering
\label{KnownsUnknownsSampleProb1D}
\begin{tabular}{|c|c|}
\hline
\textbf{Rob}          & \textbf{Velociraptor}  \\ \hline
$x_{0R} = ?$          & $x_{0V} = \SI{0}{m}$   \\
$v_R = \SI{8}{m/s}$   & $v_{0V} = \SI{0}{m/s}$ \\
                   & $a_V = \SI{4}{m/s^2}$  \\                                     
\end{tabular}
\end{table}

$x_{0R}$ is Rob's position at the instant the velociraptor starts running. The value of $x_{0R}$ is unknown but can be easily solved for. It takes 3 seconds for the velociraptor to react, so at $t=0$, Rob has moved $(\SI{8}{m/s})\times (\SI{3}{s}) = \SI{24}{m} = x_{0R}$ (where we used the formula $x=vt$).\\

Since $v_{0V}=0$ (the velociraptor starts running from rest) and $x_{0V}=0$, we can write our equations as:
\begin{align*}
x_R&=x_{0R}+v_{R}t\\
x_V&=\frac{1}{2}a_Vt^2
\end{align*}
 

Remember that we want to find $t$ when $x_R$=$x_V$. Setting the above equations equal to one another gives:
\begin{align*}
x_{0R}+v_{R}t&=\frac{1}{2}a_Vt^2 
\end{align*}
that we can rearrange to get the quadratic:
\begin{align*}
0&=\frac{1}{2}a_Vt^2-v_{R}t-x_{0R} 
\end{align*}

Solving the quadratic gives $t=\SI{6}{s}$. This doesn't quite give us the answer we want. We want to know how long it takes the velociraptor to catch up \textit{from the moment Rob passes by}, so we have to add on the $\SI{3}{s}$ reaction time, giving a total time of $\SI{9}{s}$.\\

\item We can use this solution to figure out whether Rob makes it to safety. The velociraptor catches up after 9 seconds. In 9 seconds, Rob has travelled a distance of $(\SI{8}{m/s})\times (\SI{9}{s}) = \SI{72}{m}$. The shelter is only $\SI{70}{m}$ away, so Rob gets to safety in time!
\end{enumerate}
\end{solution}

\newpage
\begin{solution}{prob:describingmotionin1d:accelerationtime}\label{soln:describingmotionin1d:accelerationtime}
\capfig{0.7\textwidth}{figures/DescribingMotionIn1D/velocitypositionsolution.png}{\label{fig:DescribingMotionIn1D:velocitypositionproblem1D}Graphs of $v(t)$ and $x(t)$ corresponding to the accleration versus time graph given in the question.}
We start by drawing the graph of $v(t)$ from the graph of $a(t)$. Solutions may vary, but a few key features must be present:
\begin{itemize}
\item Velocity is zero at $t=0$.
\item When acceleration is negative, the velocity is decreasing. When the acceleration is positive, the velocity is increasing. 
\item When the acceleration is zero, the graph of $v(t)$ is a horizontal line
\end{itemize}
We can get the graph of $x(t)$ from the graph of $v(t)$. The graph of $x(t)$ should have these features:
\begin{itemize}
\item Position is zero at $x=0$
\item When the velocity is negative, $x(t)$ is decreasing. When the velocity is positive, $x(t)$ is increasing
\item The particle turns around (the position goes from decreasing to increasing) when the velocity changes sign. 
\item When the velocity is negative and decreasing, or if it is positive and increasing, the magnitude of the slope of $x(t)$ increases. When velocity is positive and decreasing or negative and increasing, the magnitude of the slope decreases. When velocity is constant, the slope of $x(t)$ does not change. 
\begin{itemize}
\item Note: This is the same as saying that when the velocity and acceleration are both negative or both positive (when they are in the same direction), the slope of $x(t)$ increases in magnitude; when the acceleration and velocity are in opposite directions, the slope of $x(t)$ decreases in magnitude. 
\end{itemize}
\end{itemize}
\end{solution}



%
\chapter{Describing motion in multiple dimensions}
\label{chapter:describingmotioninnd}
In this chapter, we will learn how to extend our description of an object's motion to two and three dimensions by using vectors. We will also consider the specific case of an object moving along the circumference of a circle. 

\vspace{1cm}
\begin{learningObjectives}
\item Describe motion in a 2D plane.
\item Describe motion in 3D space.
\item Describe motion along the circumference of a circle.
\end{learningObjectives}

\section{Motion in two dimensions}

\subsection{Using vectors to describe motion in two dimensions}
We can specify the location of an object with its coordinates, and we can quantify any displacement by a vector. First consider the case of an object moving at a constant velocity in a particular direction.  We can describe the object at any time, $t$, using its position vector, $\vec r(t)$, which is a function of time:
\begin{align*}
\vec r(t=t_0)&=\vec r_1\\
\vec r(t=t_0+\Delta t)&=\vec r_2
\end{align*}
More generally, we can describe the $x$ and $y$ components of the position vector with independent functions, $x(t)$, and $y(t)$, respectively:
\begin{align*}
\vec r(t) = \begin{pmatrix}
           x(t) \\
           y(t) \\
         \end{pmatrix}= x(t) \hat x + y(t) \hat y
\end{align*}
Suppose that in a period of time $\Delta t$, the object goes from a position described by the position vector $\vec r_1$ to a position described by the position vector $\vec r_2$, as illustrated in Figure \ref{fig:DescribingMotionInND:xydrvec}. We can define a displacement vector, $\Delta \vec r=\vec r_2-\vec r_1$, and by analogy to the one dimensional case, we can define an \textbf{average} velocity vector, $\vec v$ as:
\begin{align}
\vec v = \frac{\Delta \vec r}{\Delta t}
\end{align}
\capfig{0.3\textwidth}{figures/DescribingMotionInND/xydrvec.png}{\label{fig:DescribingMotionInND:xydrvec}Illustration of a displacement vector, $\Delta \vec r = \vec r_2 -\vec r_1$, for an object that was located at position $\vec r_1$ at time $t_1$ and at position $\vec r_2$ at time $t_2=t_1+\Delta t$.}

The average velocity vector will have the same direction as $\Delta \vec r$, since it is the displacement vector divided by a scalar ($\Delta t$). The magnitude of the velocity vector, which we call ``speed'', will be proportional to the length of the displacement vector. If the object moves a large distance in a small amount of time, it will thus have a large velocity vector. This definition of the velocity vector thus has the correct intuitive properties (points in the direction of motion, is larger for faster objects).

For example, if the object went from position $(x_1,y_1)$ to position $(x_2,y_2)$ in an amount of time $\Delta t$, the average velocity vector is given by:
\begin{align*}
\vec v &= \frac{\Delta \vec r}{\Delta t}\\
&=\frac{1}{\Delta t}\begin{pmatrix}
           x_2-x_1 \\
           y_2-y_1 \\
         \end{pmatrix}\\
 &=\frac{1}{\Delta t}\begin{pmatrix}
           \Delta x \\
           \Delta y \\
         \end{pmatrix}\\     
 &=\begin{pmatrix}
           \frac{\Delta x}{\Delta t} \\
           \frac{\Delta y}{\Delta t}\\
         \end{pmatrix}\\       
 &=\begin{pmatrix}
           v_x \\
           v_y \\
         \end{pmatrix}\\    
\therefore \vec v &= v_x\hat x+v_y\hat y                     
\end{align*}
That is, the $x$ and $y$ components of the average velocity vector can be found by separately determining the average velocity in each direction. For example, $v_x=\frac{\Delta x}{\Delta t}$ corresponds to the average velocity in the $x$ direction, and can be considered independent from the velocity in the $y$ direction, $v_y$. The magnitude of the average velocity vector (i.e. the average speed), is given by:
\begin{align*}
||\vec v||&=\sqrt{v_x^2+v_y^2}=\frac{1}{\Delta t}\sqrt{\Delta x^2+\Delta y^2}=\frac{\Delta r}{\Delta t}
\end{align*}
where $\Delta r$ is the magnitude of the displacement vector. Thus, the average speed is given by the distance covered divided by the time taken to cover that distance, in analogy to the one dimensional case.

\begin{checkpointMC}{A llama runs in a field from a position $(x_1,y_1)=(\SI{2}{m},\SI{5}{m})$ to a position $(x_2,y_2)=(\SI{6}{m},\SI{8}{m})$ in a time $\Delta t=\SI{0.5}{s}$, as measured by Marcel, a llama farmer standing at the origin of the Cartesian coordinate system. What is the average speed of the llama?}
\item \SI{1}{m/s}
\item \SI{5}{m/s}
\item \SI{10}{m/s}%correct
\item \SI{15}{m/s}
\end{checkpointMC}

If the velocity of the object is not constant, then we define the \textbf{instantaneous velocity vector} by taking the limit $\Delta t\to 0$:
\begin{align}
\vec v(t) &= \lim_{\Delta t \to 0}\frac{\Delta \vec r}{\Delta t}=\frac{d\vec r}{dt}
\end{align}
which gives us the time derivative of the position vector (in one dimension, it was the time derivative of position). Writing the components of the position vector as functions $x(t)$ and $y(t)$, the instantaneous velocity becomes:
\begin{align}
\label{eqn:DescribingMotionInND:vvecdef}
\Aboxed{\vec v(t) &=\frac{d}{dt}\vec r(t) }\\
&=\frac{d}{dt} \begin{pmatrix}
           x(t) \\
           y(t) \\
         \end{pmatrix}\nonumber\\ 
&=\begin{pmatrix}
           \frac{dx}{dt}  \\
          \frac{dy}{dt}  \\
         \end{pmatrix}\nonumber\\ 
 &=\begin{pmatrix}
           v_x(t) \\
           v_y(t) \\
         \end{pmatrix}\nonumber\\   
\therefore \vec v(t) &= v_x(t)\hat x+v_y(t)\hat y  \nonumber     
\end{align}
where, again, we find that the components of the velocity vector are simply the velocities in the $x$ and $y$ direction. This means that we can treat motion in two dimensions as having two independent components: a motion along $x$ and a separate motion along $y$. This highlights the usefulness of the vector notation for allowing us to use one vector equation ($\vec v=\frac{d}{dt}\Delta \vec r$) to represent two equations (one for $x$ and one for $y$). 

Similarly the acceleration vector is given by:
\begin{align}
\label{eqn:DescribingMotionInND:avecdef}
\Aboxed{\vec a(t) &= \frac{d}{dt}\vec v(t)} \\
&=\begin{pmatrix}
           \frac{dv_x}{dt}  \\
          \frac{dv_y}{dt}  \\
         \end{pmatrix}\nonumber\\
&=\begin{pmatrix}
           a_x(t) \\
           a_y(t) \\
         \end{pmatrix}\nonumber\\
\therefore \vec a(t) &= a_x(t)\hat x+a_y(t)\hat y      \nonumber        
\end{align}

For example, if an object is at position $\vec r_0=(x_0,y_0)$ with a velocity vector $\vec v_0=v_{0x}\hat x + v_{0y}\hat y$ at time $t=0$, and has a constant acceleration vector, $\vec a = a_x\hat x+a_y\hat y$, then the velocity vector at some later time $t$, $\vec v(t)$, is given by:
\begin{align*}
\vec v(t) = \vec v_0 + \vec a t
\end{align*}
Or, if we write out the components explicitly:
\begin{align*}
\begin{pmatrix}
           v_x(t) \\
           v_y(t) \\
         \end{pmatrix} = \begin{pmatrix}
           v_{0x} \\
           v_{0y} \\
         \end{pmatrix} + \begin{pmatrix}
           a_xt \\
           a_yt \\
         \end{pmatrix}
\end{align*}
which really can be considered as two independent equations for the components of the velocity vector:
\begin{align*}
v_x(t)&=v_{0x}+a_xt \\
v_y(t)&=v_{0y}+a_yt \\
\end{align*}
which is the same equation that we had for one dimensional kinematics, but once for each coordinate. The position vector is given by:
\begin{align*}
\vec r(t) = \vec r_0 + \vec v_0 t + \frac{1}{2} \vec at^2
\end{align*}
with components:
\begin{align*}
x(t) &= x_0+v_{0x}t+\frac{1}{2}a_xt^2\\
y(t) &= y_0+v_{0y}t+\frac{1}{2}a_yt^2\\
\end{align*}
which again shows that two dimensional motion can be considered as separate and independent motions in each direction.

\begin{example}{An object starts at the origin of a coordinate system at time $t=\SI{0}{s}$, with an initial velocity vector $\vec v_0=(\SI{10}{m/s})\hat x+(\SI{15}{m/s})\hat y$. The acceleration in the $x$ direction is \SI{0}{m/s^2} and the acceleration in the $y$ direction is \SI{-10}{m/s^2}.
\begin{enumerate}[label=(\alph*)]
\item Write an equation for the position vector as a function of time.
\item Determine the position of the object at $t=\SI{10}{s}$.
\item Plot the trajectory of the object for the first \SI{5}{s} of motion.
\end{enumerate}
\ }
\label{ex:DescribingMotionInND:parabola}
\textbf{a)}We can consider the motion in the $x$ and $y$ direction separately. In the $x$ direction, the acceleration is 0, and the position is thus given by:
\begin{align*}
x(t)&=x_0+v_{0x}t\\
&=(\SI{0}{m})+(\SI{10}{m/s})t\\
&=(\SI{10}{m/s})t
\end{align*}
In the $y$ direction, we have a constant acceleration, so the position is given by:
\begin{align*}
y(t) &= y_0+v_{0y}t+\frac{1}{2}a_yt^2\\
&=(\SI{0}{m})+(\SI{15}{m/s})t+\frac{1}{2}(\SI{-10}{m/s^2})t^2\\
&=(\SI{15}{m/s})t-\frac{1}{2}(\SI{10}{m/s^2})t^2\\
\end{align*}
The position vector as a function of time can thus be written as:
\begin{align*}
\vec r(t) &= \begin{pmatrix}
           x(t) \\
           y(t) \\
          \end{pmatrix}\\
          &= \begin{pmatrix}
           (\SI{10}{m/s})t \\
           (\SI{15}{m/s})t-\frac{1}{2}(\SI{10}{m/s^2})t^2 \\
         \end{pmatrix}
\end{align*}
\textbf{b)} Using $t=\SI{10}{s}$ in the above equation gives:
\begin{align*}
\vec r(t=\SI{10}{s})&= \begin{pmatrix}
           (\SI{10}{m/s})(\SI{10}{s}) \\
           (\SI{15}{m/s})(\SI{10}{s})-\frac{1}{2}(\SI{10}{m/s^2})(\SI{10}{s})^2 \\
         \end{pmatrix}\\
         &= \begin{pmatrix}
           (\SI{100}{m}) \\
           (\SI{-350}{m})\\
         \end{pmatrix}
\end{align*}
\textbf{c)} We can plot the trajectory using python:

\begin{python}[caption=Trajectory in xy plane]
#import modules that we need
import numpy as np #for arrays of numbers
import pylab as pl #for plotting

#define functions for the x and y positions:
def x(t):
    return 10*t

def y(t):
    return 15*t-0.5*10*t**2

#define 10 values of t from 0 to 5 s:
tvals = np.linspace(0,5,10)

#calculate x and y at those 10 values of t using the functions
#we defined above:
xvals = x(tvals)
yvals = y(tvals)

#plot the result:
pl.plot(xvals,yvals, marker='o')
pl.xlabel("x [m]",fontsize=14)
pl.ylabel("y [m]",fontsize=14)
pl.title("Trajectory in the xy plane",fontsize=14)
pl.grid()
pl.show()
\end{python}
\begin{poutput}
(*\capfig{0.5\textwidth}{figures/DescribingMotionInND/parabola.png}{\label{fig:DescribingMotionInND:parabola}Parabolic trajectory of an object with no acceleration in the $x$ direction and a negative acceleration in the $y$ direction.}*)
\end{poutput}
As you can see, the trajectory is a parabola, and corresponds to what you would get when throwing an object with an initial velocity with upwards (positive $y$) and horizontal (positive $x$) components. If you look at only the $y$ axis, you will see that the object first goes up, then turns around and goes back down. This is exactly what happens when you throw a ball upwards, independently of whether the object is moving in the $x$ direction. In the $x$ direction, the object just moves with a constant velocity. The points on the graph are drawn for constant time intervals (the time between each point, $\Delta t$ is constant). If you look at the distance between points projected onto the $x$ axis, you will see that they are all equidistant and that along $x$, the motion corresponds to that of an object with constant velocity. 
\end{example}

\begin{checkpointMC}{In example \ref{ex:DescribingMotionInND:parabola}, what is the velocity vector exactly at the top of the parabola in Figure \ref{fig:DescribingMotionInND:parabola}?}
\item $\vec v=(\SI{10}{m/s})\hat x+(\SI{15}{m/s})\hat y$
\item $\vec v=(\SI{15}{m/s})\hat y$
\item $\vec v=(\SI{10}{m/s})\hat x$ %correct
\item none of the above
\end{checkpointMC}

\subsection{Accelerated motion when the velocity vector changes direction}
\label{sec:DescribingMotionInND:accvconst}
One key difference with one dimensional motion is that, in two dimensions, it is possible to have a non-zero acceleration even when the speed is constant. Recall, the acceleration \textbf{vector} is defined as the time derivative of the velocity \textbf{vector} (equation \ref{eqn:DescribingMotionInND:avecdef}). This means that if the velocity vector changes with time, then the acceleration vector is non-zero. The length of the velocity vector is called the speed. If the length of the velocity vector (speed) is constant, it is still possible that the \textbf{direction} of the velocity vector changes with time, and thus, that the acceleration vector is non-zero. In this case, the acceleration would not result in a change of speed, but rather in a change of the direction of motion. This is exactly what happens when an object goes around in a circle with a constant speed (the direction of the velocity vector changes). 
\rwcapfig[14]{0.35\textwidth}{figures/DescribingMotionInND/deltav.png}{\label{fig:DescribingMotionInND:deltav} Illustration of how the direction of the velocity vector can change when speed is constant.}

Figure \ref{fig:DescribingMotionInND:deltav} shows an illustration of a velocity vector, $\vec v(t)$, at two different times, $\vec v_1$ and $\vec v_2$, as well as the vector difference, $\Delta \vec v=\vec v_2 - \vec v_1$, between the two. In this case, the length of the velocity vector did not change with time ($||\vec v_1||=||\vec v_2||$). The acceleration vector is given by:
\begin{align*}
\vec a = \lim_{\Delta t\to 0}\frac{\Delta \vec v}{\Delta t}
\end{align*}
and will thus have a direction parallel to $\Delta \vec v$, and a magnitude that is proportional to $\Delta v$. Thus, even if the velocity vector does not change amplitude (speed is constant), the acceleration vector can be non-zero if the velocity vector changes \textit{direction}.

Let us write the velocity vector, $\vec v$, in terms of its magnitude, $v$, and a unit vector, $\hat v$, in the direction of $\vec v$:
\begin{align*}
\vec v &=v_x\hat x+v_y\hat y= v \hat v\\
v&=||\vec v||=\sqrt{v_x^2+v_y^2}\\
\hat v &= \frac{v_x}{v}\hat x+\frac{v_y}{v}\hat y\\
\end{align*}
In the most general case, both the magnitude of the velocity and its direction can change with time. That is, both the direction and the magnitude of the velocity vector are functions of time:
\begin{align*}
\vec v(t)&=v(t)\hat v(t)
\end{align*}
When we take the time derivative of $\vec v(t)$ to obtain the acceleration vector, we need to take the derivative of a product of two functions of time, $v(t)$ and $\hat v(t)$. Using the rules for taking the derivative of a product, the acceleration vector is given by:
\begin{align}
\label{eqn:DescribingMotionInND:avecdef2}
\vec a &= \frac{d}{dt}\vec v(t)= \frac{d}{dt}v(t)\hat v(t)\nonumber\\
\Aboxed{\vec a&=\frac{dv}{dt}\hat v(t)+v(t)\frac{d\hat v}{dt}}
\end{align}
and has two terms. The first term, $\frac{dv}{dt}\hat v(t)$, is zero if the speed is constant ($\frac{dv}{dt}=0$). The second term, $v(t)\frac{d\hat v}{dt}$, is zero if the direction of the velocity vector is constant ($\frac{d\hat v}{dt}=0$). In general though, the acceleration vector has two terms corresponding to the change in speed, and to the change in the direction of the velocity, respectively.

The specific functional form of the acceleration vector will depend on the path being taken by the object. If we consider the case where speed is constant, then we have:
\begin{align*}
v(t) &= v \\
\frac{dv}{dt}&=0\\
v_x^2(t)+v_y^2(t) &=v^2 \\
\therefore v_y(t)&=\sqrt{v^2-v_x(t)^2}
\end{align*}
\capfig{0.35\textwidth}{figures/DescribingMotionInND/aperpv.png}{\label{fig:DescribingMotionInND:aperpv} Illustration that the acceleration vector is perpendicular to the velocity vector if speed is constant.}
In other words, if the magnitude of the velocity is constant, then the $x$ and $y$ components are no longer independent (if the $x$ component gets larger, then the $y$ component must get smaller so that the total magnitude remains unchanged). If the speed is constant, then the acceleration vector is given by:
\begin{align}
\label{eqn:DescribingMotionInND:vecaconstv}
\vec a&=\frac{dv}{dt}\hat v(t)+v\frac{d\hat v}{dt}\nonumber\\
&=0 + v\frac{d}{dt}\hat v(t)\nonumber\\
&=v\frac{d}{dt}\left(\frac{v_x(t)}{v}\hat x+\frac{v_y(t)}{v}\hat y   )\right)\nonumber\\
&=\frac{dv_x}{dt}\hat x + \frac{d}{dt}\sqrt{v^2-v_x(t)^2}\hat y\nonumber\\
&=\frac{dv_x}{dt}\hat x + \frac{1}{2\sqrt{v^2-v_x(t)^2}}(-2v_x(t))\frac{dv_x}{dt}\hat y\nonumber\\
&=\frac{dv_x}{dt}\hat x - \frac{v_x(t)}{\sqrt{v^2-v_x(t)^2}}\frac{dv_x}{dt}\hat y\nonumber\\
&=\frac{dv_x}{dt}\hat x - \frac{v_x(t)}{v_y(t)}\frac{dv_x}{dt}\hat y\nonumber\\
\therefore\quad\Aboxed{\vec a&=\frac{dv_x}{dt} \left(\hat x - \frac{v_x(t)}{v_y(t)}\hat y\right)}
\end{align}
where most of the algebra that we did was to separate out the $x$ and $y$ components of the acceleration vector. The resulting acceleration vector is illustrated in Figure \ref{fig:DescribingMotionInND:aperpv} along with the velocity vector. Rather, a vector parallel to the acceleration vector is illustrated, as the factor of $\frac{dv_x}{dt}$ was omitted (as you recall, multiplying by a scalar only changes the length, not the direction). The velocity vector has components $v_x$ and $v_y$, which allows us to calculate the angle, $\theta$ that it makes with the $x$ axis:
\begin{align*}
\tan(\theta)=\frac{v_y}{v_x}
\end{align*}
Similarly, the vector that is parallel to the acceleration has components of $1$ and $-\frac{v_x}{v_y}$, allowing us to determine the angle, $\phi$, that it makes with the $x$ axis:
\begin{align*}
\tan(\phi)=\frac{v_x}{v_y}
\end{align*}
Note that $\tan(\theta)$ is the inverse of $\tan(\phi)$, or in other words, $\tan(\theta)=\cot(\phi)$, meaning that $\theta$ and $\phi$ are complementary and thus must sum to $\frac{\pi}{2}$ (\SI{90}{\degree}). This means that \textbf{the acceleration vector is perpendicular to the velocity vector if the speed is constant and the direction of the velocity changes}. 

In other words, when we write the acceleration vector, we can identify two components, $\vec a_{\parallel}(t)$ and $\vec a_{\perp}(t)$:
\begin{align*}
\vec a&=\frac{dv}{dt}\hat v(t)+v(t)\frac{d\hat v}{dt}\\
&=\vec a_{\parallel}(t) + \vec a_{\perp}(t)\\
\therefore \vec a_{\parallel}(t)&=\frac{dv}{dt}\hat v(t)\\
\therefore \vec a_{\perp}(t)&=v\frac{d\hat v}{dt}=\frac{dv_x}{dt} \left(\hat x - \frac{v_x(t)}{v_y(t)}\hat y\right)
\end{align*}
where $\vec a_{\parallel}(t)$ is the component of the acceleration that is parallel to the velocity vector, and is responsible for changing its magnitude, and $\vec a_{\perp}(t)$, is the component that is perpendicular to the velocity vector and is responsible for changing the direction of the motion.

\begin{checkpointMC}{A satellite moves in a circular orbit around the Earth with a constant speed. What can you say about its acceleration vector?}
\item it has a magnitude of zero.
\item it is perpendicular to the velocity vector.
\item it is parallel to the velocity vector.
\item it is in a direction other than parallel or perpendicular to the velocity vector.
\end{checkpointMC}

\subsection{Relative motion}
In the previous chapter, we examined how to convert the description of motion from one reference frame to another. Recall the one dimensional situation where we described the position of an object, $A$, using an axis $x$ as $x^A(t)$. Suppose that the reference frame, $x$, is moving with a constant speed, $v'^B$, relative to a second reference frame, $x'$. We found that the position of the object is described in the $x'$ reference frame as:
\begin{align*}
x'^A(t)=v'^Bt+x^A(t)
\end{align*}
if the origins of the two systems coincided at $t=0$. The equation above simply states that the distance of the object to the $x'$ origin is the sum of the distance from the $x'$ origin to the $x$ origin \textbf{and} the distance from the $x$ origin to the object.

In two dimensions, we proceed in exactly the same way, but use vectors instead:
\begin{align*}
\pvec r'^A(t) = \pvec v'^Bt+\vec r^A(t)
\end{align*}
where $r^A(t)$ is the position of the object as described in the $xy$ reference frame, $\pvec v'^B$, is the velocity vector describing the motion of the origin of the $xy$ coordinate system relative to an $x'y'$ coordinate system. $\pvec r'^A(t)$ is the position of the object in the $x'y'$ coordinate system. We have assumed that the origins of the two coordinate systems coincided at $t=0$ and that the axes of the coordinate systems are parallel ($x$ parallel to $x'$ and $y$ parallel to $y'$).

Note that the velocity of the object in the $x'y'$ system is found by adding the velocity of $xy$ relative to $x'y'$ and the velocity of the object in the $xy$ frame ($\vec v^A(t)$):
\begin{align*}
\frac{d}{dt}\pvec r'^A(t) &=\frac{d}{dt}(\pvec v'^Bt+\vec r^A(t))\\
&=\pvec v'^B+\vec v^A(t)
\end{align*}

As an example, consider the situation depicted in Figure \ref{fig:DescribingMotionInND:2drel}. Brice is on a boat off the shore of Nice, with a coordinate system $xy$, and is describing the position of a boat carrying Alice. He describes Alice's position as $\vec r^A(t)$ in the $xy$ coordinate system. Igor is on the shore and also wishes to describe Alice's position using the work done by Brice. Igor sees Brice's boat move with a velocity $\vec v'^B$ as measured in his $x'y'$ coordinate system. In order to find the vector pointing to Alice's position $\pvec r'^A(t)$, he adds the vector from his origin to Brice's origin ($\pvec v'^B t$) and the vector from Brice's origin to Alice $\vec r^A(t)$.

\capfig{0.7\textwidth}{figures/DescribingMotionInND/2drel.png}{\label{fig:DescribingMotionInND:2drel} Example of converting from one reference frame to another in two dimensions using vector addition.}

Writing this out by coordinate, we have:
\begin{align*}
x'^A(t)&=v'^B_xt+x^A(t)\\
y'^A(t)&=v'^B_yt+y^A(t)
\end{align*}
and for the velocities:
\begin{align*}
v_x'^A(t)&=v'^B_x+v_x^A(t)\\
v_y'^A(t)&=v'^B_y+v_y^A(t)
\end{align*}


\begin{checkpointMC}{You are on a boat and crossing a North-flowing river, from the East bank to the West bank. You point your boat in the West direction and cross the river. \chloe is watching your boat cross the river from the shore, in which direction does she measure your velocity vector to be?}
\item in the North direction
\item in the West direction
\item a combination of North and West directions
\end{checkpointMC}


\section{Motion in three dimensions}
The big challenge was to expand our description of motion from one dimension to two. Adding a third dimension ends up being trivial now that we know how to use vectors. In three dimensions, we describe the position of a point using three coordinates, so all of the vectors simply have three independent components, but are treated in exactly the same way as in the two dimensional case. The position of an object is now described by three independent functions, $x(t)$, $y(t)$, $z(t)$, that make up the three components of a position vector $\vec r(t)$:
\begin{align*}
\vec r(t) &= \begin{pmatrix}
           x(t) \\
           y(t) \\
           z(t)  \\
         \end{pmatrix}\\
\therefore \vec r(t)  &= x(t) \hat x + y(t) \hat y + z(t) \hat z
\end{align*}
The velocity vector now has three components and is defined analogously to the 2D case:
\begin{align*}
\vec v(t) &=\frac{d\vec r}{dt}
 =\begin{pmatrix}
           \frac{dx}{dt}  \\
          \frac{dy}{dt}  \\
          \frac{dz}{dt}  \\
         \end{pmatrix}
 =\begin{pmatrix}
           v_x(t) \\
           v_y(t) \\
           v_z(t) \\
         \end{pmatrix}\\   
\therefore \vec v(t) &= v_x(t)\hat x+v_y(t)\hat y+v_z(t)\hat z  \nonumber 
\end{align*}
and the acceleration is defined in a similar way:
\begin{align*}
\vec a(t)  &=\frac{d\vec v}{dt}
 =\begin{pmatrix}
           \frac{dv_x}{dt}  \\
          \frac{dv_y}{dt}  \\
          \frac{dv_z}{dt}  \\
         \end{pmatrix}
 =\begin{pmatrix}
           a_x(t) \\
           a_y(t) \\
           a_z(t) \\
         \end{pmatrix}\\   
\therefore \vec a(t) &= a_x(t)\hat x+a_y(t)\hat y+a_z(t)\hat z  \nonumber 
\end{align*}

In particular, if an object has a constant acceleration, $\vec a=a_x\hat x+a_y\hat y+a_z\hat z$, and started at $t=0$ with a position $\vec r_0$ and velocity $\vec v_0$, then its velocity vector is given by:
\begin{align*}
\vec v(t)  &= \vec v_0+\vec at=\begin{pmatrix}
           v_{0x}+ a_xt \\
           v_{0y}+ a_yt \\
           v_{0z}+ a_zt \\
         \end{pmatrix}\\
\end{align*}
and the position vector is given by:
\begin{align*}
\vec r(t)= \vec r_0+\vec v_0 t+\frac{1}{2}\vec a t^2=\begin{pmatrix}
           x_0+v_{0x}t+\frac{1}{2} a_xt^2 \\
           y_0+v_{0y}t+\frac{1}{2} a_yt^2 \\
           z_0+v_{0z}t+\frac{1}{2} a_zt^2 \\
         \end{pmatrix}\\
\end{align*}
where again, we see how writing a single vector equation (e.g. $\vec v(t) = \vec v_0+\vec at$) is really just a way to write the three independent equations that are true for each component.
\section{Circular motion}
We often consider the motion of an object around a circle of fixed radius, $R$. In principle, this is motion in two dimensions, as a circle is necessarily in a two dimensional plane. However, since the object is constrained to move along the circumference of the circle, it can be thought of (and treated as) motion along a one dimensional axis that is curved. 
\capfig{0.35\textwidth}{figures/DescribingMotionInND/circle.png}{\label{fig:DescribingMotionInND:circle} Describing the motion of an object around a circle of radius $R$.}

Figure \ref{fig:DescribingMotionInND:circle} shows how we can describe motion on a circle. We could use $x(t)$ and $y(t)$ to describe the position on the circle, however, $x(t)$ and $y(t)$ are no longer independent since they have to correspond to the coordinates of points on a circle:
\begin{align*}
x^2(t)+y^2(t)=R^2
\end{align*}
Instead of using $x$ and $y$, we could think of an axis that is bent around the circle (as shown by the curved arrow in Figure \ref{fig:DescribingMotionInND:circle}, the $s$ axis). The $s$ axis is such that $s=0$ where the circle intersects the $x$ axis, and the value of $s$ increases as we move counter-clockwise along the circle. Distance along the $s$ axis thus corresponds to the distance along the circumference of the circle.

Another variable that could be used for position instead of $s$ is the angle, $\theta$, between the position vector of the object and the $x$ axis, as illustrated in Figure \ref{fig:DescribingMotionInND:circle}. If we express the angle $\theta$ in radians, then it easy to convert between $s$ and $\theta$. Recall, an angle in radians is defined as the length of an arc subtended by that angle divided by the radius of the circle. We thus have:
\begin{align}
\label{eqn:DescribingMotionInND:raddef}
\Aboxed{\theta(t)=\frac{s(t)}{R}}
\end{align}
In particular, if the object has gone around the whole circle, then $s=2\pi R$ (the circumference of a circle), and the corresponding angle is, $\theta=\frac{2\pi R}{R}=2\pi$, namely \SI{360}{\degree}. 

By using the angle, $\theta$, instead of $x$ and $y$, we are effectively using polar coordinates, with a fixed radius. As we already saw, the $x$ and $y$ positions are related to $\theta$ by:
\begin{align*}
x(t) &= R\cos(\theta(t))\\
y(t) &= R\sin(\theta(t))\\
\end{align*}
where $R$ is a constant. For an object moving along the circle, we can write its position vector, $\vec r(t)$, as:
\begin{align*}
\vec r(t)&= \begin{pmatrix}
           x(t) \\
           y(t) \\
         \end{pmatrix}
         =R \begin{pmatrix}
           \cos(\theta(t)) \\
           \sin(\theta(t)) \\
         \end{pmatrix}
\end{align*}
\capfig{0.35\textwidth}{figures/DescribingMotionInND/vcircle.png}{\label{fig:DescribingMotionInND:vcircle} The position vector, $\vec r(t)$ is always perpendicular to the velocity vector, $\vec v(t)$, for motion on a circle.}
and the velocity vector is thus given by:
\begin{align*}
\vec v(t) &=\frac{d}{dt}\vec r(t) 
=\frac{d}{dt} R \begin{pmatrix}
           \cos(\theta(t)) \\
           \sin(\theta(t)) \\
         \end{pmatrix} \\
&= R \begin{pmatrix}
           \frac{d}{dt}\cos(\theta(t)) \\
           \frac{d}{dt}\sin(\theta(t)) \\
         \end{pmatrix} \\
 &= R \begin{pmatrix}
           -\sin(\theta(t))\frac{d\theta}{dt} \\
           \cos(\theta(t))\frac{d\theta}{dt} \\
         \end{pmatrix}     
\end{align*}         
where we used the Chain Rule to calculate the time derivatives of the trigonometric functions (since $\theta(t)$ is function of time). The magnitude of the velocity vector is given by:
\begin{align*}
||\vec v|| &=\sqrt{ v_x^2+v_y^2}\\
&=\sqrt{ \left(-R\sin(\theta(t))\frac{d\theta}{dt}\right)^2+\left(R\cos(\theta(t))\frac{d\theta}{dt}\right)^2}\\
&=\sqrt{ R^2\left( \frac{d\theta}{dt}\right)^2[\sin^2(\theta(t))+\cos^2(\theta(t)]}\\
&=R\left |\frac{d\theta}{dt}\right|
\end{align*}

The position and velocity vectors are illustrated in Figure \ref{fig:DescribingMotionInND:vcircle} for an angle $\theta$ in the first quadrant ($0<\theta<\frac{\pi}{2}$). In this case, you can note that the $x$ component of the velocity is negative (in the equation above, and in the Figure). From the equation above, you can also see that $\frac{|v_x|}{|v_y|}=\tan(\theta)$, which is illustrated in Figure \ref{fig:DescribingMotionInND:vcircle}, showing that \textbf{the velocity vector is tangent to the circle} and perpendicular to the position vector. This is always the case for motion along a circle.

We can simplify our description of motion along the circle by using either $s(t)$ or $\theta(t)$ instead of the vectors for position and velocity. If we use $s(t)$ to represent position along the circumference ($s=0$ where the circle intersects the $x$ axis), then the velocity along the $s$ axis is:
\begin{align*}
v_s(t)&=\frac{d}{dt}s(t)\\
&=\frac{d}{dt}R\theta(t)\\
&=R\frac{d\theta}{dt}
\end{align*}
where we used the fact that $\theta=\frac{s}{R}$ to convert from $s$ to $\theta$. The velocity along the $s$ axis is thus precisely equal to the magnitude of the two-dimensional velocity vector (derived above), which makes sense since the velocity vector is tangent to the circle (and thus in the $s$ ``direction'').

If the object has a \textbf{constant speed}, $v_s$, along the circle and started at a position along the circumference $s=s_0$, then its position along the $s$ axis can be described as:
\begin{align*}
s(t)=s_0+v_st
\end{align*}
or, in terms of $\theta$:
\begin{align*}
\theta(t)&=\frac{s(t)}{R}=\frac{s_0}{R}+\frac{v_s}{R}t\\
&=\theta_0 + \frac{d\theta}{dt}t\\
&=\theta_0 + \omega t\\
\Aboxed{\therefore \omega &= \frac{d\theta}{dt}}
\end{align*}
where we introduced $\theta_0$ as the angle corresponding to the position $s_0$, and we introduced $\omega=\frac{d\theta}{dt}$, which is analogous to velocity, but for an angle. $\omega$ is called the \textbf{angular velocity} and is a measure of the rate of change of the angle $\theta$ (as it is the time derivative of the angle). The relation between the ``linear'' velocity $v_s$ (the magnitude of the velocity vector, which corresponds to the velocity in the direction tangent to the circle) and $\omega$ is:
\begin{align*}
\Aboxed{v_s=R\frac{d\theta}{dt}=R\omega }
\end{align*}

\begin{studentopinionOW}{A way to think about angular and linear velocity}
TO DO: Explain figure
\capfig{0.7\textwidth}{figures/DescribingMotionInND/HandPolarCoordinates.png}{\label{fig:HandPolarCoordinates} How to use your hand to better understand polar coordinates}
\end{studentopinionOW}

Similarly, if the object is accelerating, we can define an \textbf{angular acceleration}, $\alpha(t)$, as the rate of change of the angular velocity:
\begin{align*}
\alpha(t)=\frac{d\omega}{dt}
\end{align*}
which can directly be related to the acceleration in the $s$ direction, $a_s(t)$:
\begin{align*}
a_d(t) &= \frac{d}{dt}v_s\\
&=\frac{d}{dt}\omega R=R\frac{d\omega}{dt}\\
\Aboxed{a_d(t)&=R\alpha }
\end{align*}
Thus, the linear quantities (those along the $s$ axis) can be related to the angular quantities by multiplying the angular quantities by $R$:
\begin{align}
s&=R\theta\\
v_s&=R\omega\\
a_s&=R\alpha
\end{align}
If the object started at $t=0$ with a position $s=s_0$ ($\theta=\theta_0$), and an initial linear velocity $v_{0s}$ (angular velocity $\omega_0$), and has a \textbf{constant linear acceleration} around the circle, $a_s$ (angular acceleration, $\alpha$), then the position of the object can be described as:
\begin{align*}
s(t) &= s_0+v_{s0}t+\frac{1}{2}a_s t^2\\
\theta(t) &= \theta_0+\omega_0t+\frac{1}{2}\alpha t^2
\end{align*}
which corresponds to an object that is going around the circle faster and faster.

As you recall from section \ref{sec:DescribingMotionInND:accvconst}, we can compute the acceleration \textbf{vector} and identify components that are parallel and perpendicular to the velocity vector:
\begin{align*}
\vec a&=\vec a_{\parallel}(t) + \vec a_{\bot}(t)\\
&=\frac{dv}{dt}\hat v(t)+v\frac{d\hat v}{dt}\\
\end{align*}
The first term, $\vec a_{\parallel}(t)=\frac{dv}{dt}\hat v(t)$, is parallel to the velocity vector $\hat v$, and has a magnitude given by:
\begin{align*}
||\vec a_{\parallel}(t)||&=\frac{dv}{dt}=\ddt v(t)=\ddt R\omega=R\alpha
\end{align*}
That is, the component of the acceleration vector that is parallel to the velocity is precisely the acceleration in the $s$ direction (the linear acceleration). This component of the acceleration is responsible for increasing (or decreasing) the speed of the object and is zero if the object goes around the circle with a constant speed (linear or angular). 

As we saw earlier, the perpendicular component of the acceleration, $\vec a_{\bot}(t)$, is responsible for changing the direction of the velocity vector (as the object continuously changes direction when going in a circle). When the motion is around a circle, this component of the acceleration vector is called ``centripetal'' acceleration (i.e. acceleration pointing towards the centre of the circle, as we will see). We can calculate the centripetal acceleration in terms of our angular variables, noting that the unit vector in the direction of the velocity is $\hat v=-\sin(\theta)\hat x+\cos(\theta)\hat y$:
\begin{align}
\vec a_{\bot}(t)&=v\frac{d\hat v}{dt}\nonumber\\
&=(\omega R)\ddt \left[-\sin(\theta)\hat x+\cos(\theta)\hat y\right]\nonumber\\
&=\omega R \left[-\ddt\sin(\theta)\hat x+\ddt\cos(\theta)\hat y\right]\nonumber\\
&=\omega R \left[-\cos(\theta)\frac{d\theta}{dt}\hat x-\sin(\theta)\frac{d\theta}{dt}\hat y\right]\nonumber\\
&=\omega R [-\cos(\theta)\omega\hat x-\sin(\theta)\omega\hat y]\nonumber\\
\Aboxed{\vec a_{\bot}(t)&=\omega^2 R[-\cos(\theta)\hat x-\sin(\theta)\hat y]}
\end{align}
where you can easily verify that the vector $[-\cos(\theta)\hat x-\sin(\theta)\hat y]$ has unit length and points towards the centre of the circle (when the tail is placed on a point on the circle at angle $\theta$). The centripetal acceleration thus points towards the centre of the circle and has magnitude:
\begin{align}
a_c(t) = ||\vec a_{\bot}(t)||=\omega^2(t) R = \frac{v^2(t)}{R}
\end{align}
where in the last equal sign, we wrote the centripetal acceleration in terms of the speed around the circle ($v=||\vec v||=v_s$).

If an object goes around a circle, it will always have a centripetal acceleration (since its velocity vector must change direction). In addition, if the object's speed is changing, it will also have a linear acceleration, which points in the same direction as the velocity vector (it changes the velocity vector's length but not its direction).

\begin{checkpointMC}{A vicu\~na is going clockwise around a circle that is centred at the origin of an $xy$ coordinate system that is in the plane of the circle. The vicu\~na runs faster and faster around the circle. In which direction does its acceleration vector point just as the vicu\~na is at the point where the circle intersects the positive $y$ axis?}
\item In the negative $y$ direction
\item In the positive $y$ direction
\item A combination of the positive $y$ and positive $x$ directions
\item A combination of the negative $y$ and positive $x$ directions %correct
\item A combination of the negative $y$ and negative $x$ directions
\end{checkpointMC}

\subsection{Period and frequency}
When an object is moving around in a circle, it will typically complete more than one revolution. If the object is going around the circle with a constant speed, we call the motion ``uniform circular motion'', and we can define the \textbf{period and frequency} of the motion. 

The period, $T$, is defined to be the time that it takes to complete one revolution around the circle. If the object has constant angular speed $\omega$, we can find the time, $T$, that it takes to complete one full revolution, from $\theta=0$ to $\theta=2\pi$:
\begin{align}
\omega&=\frac{\Delta \theta}{T}=\frac{2\pi}{T}\nonumber\\
\Aboxed{\therefore T&=\frac{2\pi}{\omega}}
\end{align}
We would obtain the same result using the linear quantities; in one revolution, the object covers a distance of $2\pi R$ at a speed of $v$:
\begin{align*}
v&=\frac{2\pi R}{T}\\
T&=\frac{2\pi R}{v}=\frac{2\pi R}{\omega R}=\frac{2\pi}{\omega}
\end{align*}

The frequency, $f$, is defined to be the inverse of the period:
\begin{align*}
f&=\frac{1}{T}=\frac{\omega}{2\pi}
\end{align*}
and has SI units of $\si{Hz}=\si{s^{-1}}$. Think of frequency as the number of revolutions completed per second. Thus, if the frequency is $f=\SI{1}{Hz}$, the object goes around the circle once per second. 
\capfig{0.35\textwidth}{figures/DescribingMotionInND/twocircles.png}{\label{fig:DescribingMotionInND:twocircles} For a given angular velocity, the linear velocity will be larger on a larger circle ($v=\omega R$).} Given the frequency, we can of course obtain the angular velocity:
\begin{align*}
\omega = 2\pi f
\end{align*}
which is sometimes called the ``angular frequency'' instead of the angular velocity. The angular velocity can really be thought of as a frequency, as it represents the ``amount of angle'' per second that an object covers when going around a circle. The angular velocity does not tell us anything about the actual speed of the object, which depends on the radius $v=\omega R$. This is illustrated in Figure \ref{fig:DescribingMotionInND:twocircles}, where two objects can be travelling around two circles of radius $R_1$ and $R_2$ with the same angular velocity $\omega$. If they have the same angular velocity, then it will take them the same amount of time to complete a revolution. However, the outer object has to cover a much larger distance (the circumference is larger), and thus has to move with a larger linear speed.

\begin{checkpointMC}{A motor is rotating at \SI{3000}{rpm}, what is the corresponding frequency in \si{Hz}?}
\item \SI{5}{Hz}
\item \SI{50}{Hz}%correct
\item \SI{500}{Hz}
\end{checkpointMC}


\newpage
\section{Summary}
\vspace{2cm}
\begin{chapterSummary}
\item Something interesting
\end{chapterSummary}

\section{Sample problems and solutions}
\begin{problemParts}{Ethan is jumping hurdles in the Olympics. He gets a running start, moving with a velocity of $\SI{4}{m/s}$ [E], and will not slow down before jumping. The hurdle is $\SI{1}{m}$ high and the maximum speed he can have when he leaves the ground is $\SI{5}{m/s}$. (You can assume Ethan is a point particle).}
\item How close can he get to the hurdle before he has to jump?
\item What maximum height does he reach?
\item Where does he land?
\end{problemParts}

\begin{problemParts}{A cowboy swings a lasso above his head. The lasso moves in a circle of radius $\SI{1.5}{m}$ in the horizontal plane. A hawk flies toward the lasso at $\SI{50}{km/h}$. The hawk sees the end of the lasso moving at $\SI{60}{km/h}$ when the lasso is directly in front of it (see Figure \ref{fig:CowboyQuestion}). In the reference frame of the cowboy ...}
\item How long does it take for the lasso to complete one revolution?
\item What is the centripetal acceleration of the end of the lasso? 
\item What is the angular acceleration of the lasso?
\capfig{0.5\textwidth}{figures/DescribingMotionInND/CowboyQuestionGiven.png}{\label{fig:CowboyQuestion} The problem as viewed from above. This diagram depicts the moment that the end of the lasso passes in front of the hawk.}
\end{problemParts} 


\textbf{Solution:}
\begin{enumerate}[label=\alph*)]
\item Our goal is the find the period of the lasso's motion. To do this, we can use the formula, 
\begin{align*}
T&=\frac{2\pi}{\omega}
\end{align*}
for which we need the angular velocity, $\omega$. We know the radius of the lasso, so if we find the linear velocity of the end point of the lasso, we can find the angular velocity by
\begin{align*}
\omega&=\frac{v}{R}
\end{align*}
We start by using what we know about relative motion to find the linear velocity of the lasso in the cowboy's reference frame. First, we need to set up our coordinate systems. We assign the $xy$ coordinate system to the hawk's reference frame and we assign the $x'y'$ system to the cowboy's reference frame. The solution will be simplest if we align the coordinate systems so that positive $y$ and positive $y'$ are in the same direction, as in Figure \ref{fig:CowboySolution}. When we are talking about the velocity of the hawk, we will denote it with the superscript ``H", and when we are talking about the lasso, we will use ``L".\\

We want the velocity of the \textbf{lasso} in the \textbf{cowboy's reference frame}, so we want $v'^L$. To find this, we start with the velocity of the lasso in the hawk's reference frame,$v^L$ and then take into account that the hawk is moving relative to the cowboy. We do this by adding the velocity of the hawk in the cowboy's reference frame, $v'^H$ to $v^L$. So, our equation is,
\begin{align*}
v^L+v'^H&=v'^L
\end{align*}
We are adding velocities, which have both a magnitude and a direction. However, we were not given any directions in the problem, so we describe the directions with respect to our coordinate system. The way we have set up our axes, the velocity of the hawk in the cowboy's reference frame is simply $\SI{50}{km/h}$ in the positive $y'$ direction.\\

Now here's the key to solving this problem: We don't know the speed of the lasso in the cowboy's reference frame, but we do know something about its direction. Since the motion of the lasso is circular, it's velocity must be tangent to the circle. This means that when the lasso is directly in front of the hawk, its velocity must be in either the $+x'$ or $-x'$ direction. In this case, we can just choose one, so we will choose the $+x'$ direction.
\capfig{0.5\textwidth}{figures/DescribingMotionInND/CowboySolution.png}{\label{fig:CowboySolution}The two coordinate systems are aligned so that positive $y'$ and positive $y$ are in the same direction. The velocity vectors of the hawk and the lasso in the reference frame of the cowboy are shown.}
The velocity vectors $v'^H$ and $v'^L$ are shown in Figure \ref{fig:CowboySolution}. Remember that when we add two vectors they must be lined up so that the ``head" of one touches the ``tail" of the other, so there can only be one direction for $v^L$, as shown in Figure \ref{fig:CowboyVector}. 
\capfig{0.25\textwidth}{figures/DescribingMotionInND/CowboyVectorAddition.png}{\label{fig:CowboyVector} Vector addition to determine the velocity of the lasso in the cowboy's reference frame.}
This is a right angle triangle, so we use the Pythagorean theorem so solve for $v'^L$:
\begin{align*}
v'^{L^2}+v'^{H^2}&=v^{L^2}\\
&=\sqrt{v^{L^2}-v'^{H^2}}\\
&=\sqrt{(\SI{60}{km/h})^2-(\SI{50}{km/h})^2}\\
v'^L&=\SI{33}{km/h}
\end{align*}
The linear velocity of the end of the lasso at this moment is $\SI{33}{km/h}$ in the positive $x$ direction.  To find the angular velocity, first convert the linear velocity from km/h to m/s:
\begin{align*}
\frac{\SI{33}{km}}{h}\times \frac{\SI{1000}{m}}{\SI{1}{km}} \times \frac{\SI{1}{h}}{\SI{3600}{s}} &= \SI{9.2}{m/s}
\end{align*}     
Now we can substitute $\omega=\frac{v}{R}$ into $T=\frac{2\pi}{\omega}$ and solve for $T$:
\begin{align*}
T&={2\pi}\frac{R}{v}\\
&={2\pi}\frac{\SI{1.5}{m}}{\SI{9.2}{m/s}}\\
&={2\pi}\frac{\SI{1.5}{m}}{\SI{9.2}{m/s}}\\
T&=\SI{1.0}{s}
\end{align*}
$\therefore$ it takes $\SI{1.0}{s}$ for the lasso to complete one revolution.
\item The motion is circular, so it has a centripetal acceleration given by
\begin{align*}
a_c(t)&=\frac{v^2(t)}{R}
\end{align*}
To find the centripetal acceleration of the end of the lasso, we just substitute in our values for $v$ and $R$.
\begin{align*}
a_c(t)&=\frac{\SI{9.2}{m/s}^2}{\SI{1.5}{m}}\\
a_c(t)&=\SI{56}{m/s^2}
\end{align*}
$\therefore$ the centripetal acceleration of the end of the lasso is $\SI{56}{m/s^2}$ towards the centre of the circle. 
\item You may be tempted to divide the centripetal acceleration by $R$ to find the angular acceleration $\alpha$. However, the angular acceleration is the rate of change of the angular velocity. For circular motion, the angular velocity is constant, so \textbf{the angular acceleration is zero}. (Remember that in the equation $a_s=R\alpha$, $a_s$ refers to the component of acceleration that is parallel to the velocity. For circular motion, $a_s$ is zero.)
\end{enumerate}





%
\chapter{Newton's Laws}
In this chapter, we introduce Newton's Laws, which is a succinct theory of physics that describes an incredibly large number of phenomena in the natural world. Newton's Laws are one possible formulation of what we call ``Classical Physics'' (as opposed to ``Modern Physics'' which include Quantum Mechanics and Special Relativity). Newton's Laws make the connection between dynamics (the causes of motion) and the kinematics of motion (the description of that motion). 
\label{chap:NewtonsLaws}
 \vspace{1cm}
\begin{learningObjectives}
\item Understand Newton's Three Laws
\item Understand the concept of force and how to identify a force
\item Understand the concepts of mass and inertia
\item Understand free body diagrams
\end{learningObjectives}

\section{Newton's Three Laws}
Newton's classical theory of physics is based on the three following laws:
\begin{itemize}
\item \textbf{Law 1}: An object will remain in its state of motion, be it at rest or moving with constant velocity, unless a net external force is exerted on the object.
\item \textbf{Law 2}: An object's acceleration is proportional to the net force exerted \textbf{on the object}, inversely proportional to the mass of the object, and in the same direction as the net force exerted on the object.
\item \textbf{Law 3}: If one object exerts a force on another object, the second object exerts a force on the first object that is equal in magnitude and opposite in direction.
\end{itemize}
The three statements above are sufficient to describe almost all of the natural phenomena that we experience in our lives. Concepts such as energy, centre of mass, torque, etc, which you may have already encountered, are derived naturally from Newton's Laws. In order to build models to describe specific experiments or observations using Newton's Laws, one needs to understand the two main mathematical concepts that are introduced by the theory: force and mass. A few comments on each of the three laws are first provided before the concepts of force and mass are developed further.

\subsection{Newton's First Law}
Newton's First Law is often referred to as the law of inertia which was originally stated by Galileo. The first law is counter-intuitive, as our experience is that if you push a block on a table and let it go, it will eventually stop. Indeed, Aristotle proposed that the natural state of objects is to be at rest. As a result of Newton's theory, we now understand that if you model a block sliding on a table, one must include a force of friction between the table and the block that acts to slow it down; the block is thus not in a situation where no net external force is exerted on the object.

Newton's First Law is useful in defining what we call an ``inertial frame of reference'', which is a frame of reference in which Newton's First Law holds true. A frame of reference can be thought of as a coordinate system which can be moving. For example, if a train is moving with constant velocity, we can consider the train as an inertial frame of reference since objects in the train would follow Newton's First Law for observers that are in the train. If a train passenger placed an object on a table, they would observe that the object does not spontaneously start moving; if they slide an object on a frictionless table, they would observe that it keeps on sliding at constant velocity. However, if the train is accelerating forwards, then an object placed on a frictionless table would appear, for observers in the train's frame of reference, to be accelerating in the direction opposite to that of the train, and violate Newton's First Law. To an observer on the ground, looking into the train through a window the object would appear to move with the same constant velocity as when it was placed on the table. In a similar way, when you are in a car, Newton's First Law holds if the car is going at constant velocity, but if the car goes around a curve (and thus accelerates even is speed is constant), you will find that all objects in the car suddenly appear to be pushed towards the outside of the curve.

Newton's First Law thus allows us to define an inertial frame of reference; Newton's Three Laws only hold in inertial frames of reference.

\begin{checkpointMC}{You are in an elevator accelerating upwards.}
\item The elevator is an inertial frame of reference.
\item The elevator is not an inertial frame of reference.%correct
\end{checkpointMC}

\subsection{Newton's Second Law}
Newton's Second Law is often written as a vector equation:
\begin{align*}
\sum \vec F = m\vec a
\end{align*}
where $\sum \vec F$ is the vector sum of the forces exerted on an object, $\vec a$ is the acceleration vector of the object, and $m$ is the ``inertial mass'' of the object. As we will see, a force is represented by a vector, and the sum of the force vectors on an object is often called the ``net force''. Recall that using vectors to write an equation is just a shorthand for writing the equation out for each component. In three dimensions, this would thus correspond to three independent scalar equations (one for each component):
\begin{align*}
\sum F_x &= ma_x \\
\sum F_y &= ma_y \\
\sum F_z &= ma_z
\end{align*}

Newton's Second Law is the foundation for Classical Physics, in which we seek to be able to describe the motion of any object. The motion of an object is fully specified by its acceleration as long as we know the position and velocity at a specific point in time. That is, by knowing the position and velocity of the object at a point in time and its acceleration, we can describe its motion both in the future and in the in past; we call Classical Physics a deterministic theory (as opposed to, say, Quantum Mechanics, which would only tell us the probability that a particle would be at some particular position in the future). The right side of the equation is thus the kinematic description of the object; if we know the acceleration, we know everything that the object will do.

The left side of the equation contains all of the ``dynamics''; force is the tool that Newton introduced in order to be able to determine the acceleration of an object. Newton's Second Law thus tells how to determine the kinematics of an object by using the concept of forces; it relates the dynamics to the kinematics. Having already covered kinematics, we will now focus on understanding dynamics and how to develop models that allow us to calculate the net force on an object. The inertial mass, $m$, is a specific property of an object that tells us how large an acceleration it will experience based on a given net force. Thus, objects with different masses will experience different accelerations if subject to the same net force.

\begin{checkpointMC}{Object 1 has twice the inertial mass of object 2. If both objects have the same acceleration vector.}
\item The net force on both objects is the same.
\item The net force on object 1 is twice that on object 2. %correct
\item The net force on object 1 is half of that on object 2.
\end{checkpointMC}


\subsection{Newton's Third Law}
Newton's Third Law relates the forces that two objects exert on each other. It is important to understand that the forces that are mentioned in the Newton's Third Law are exerted on \textit{different} objects. If object A exerts a force on object B, then object B will also exert a force on object A. The two forces have the same magnitude but opposite directions. Sometimes, the forces are called ``action'' and ``reaction'' forces, although this is misleading, because it makes it sound that the reaction force in in response to some voluntary action force. However, inanimate object can exert forces, and so this can lead to needless confusion as to which force is the reaction force.

It does not matter which force you choose to call the action (reaction) force. If a block is pushing down on a table (action force), then the table is pushing up on the block (reaction force). However, one could just as well say that the table is pushing up on the block (action force) so the block is pushing down on the table (reaction force). It does not matter which force you call the action force. This can be confusing, because if you choose to push on a wall (exerting an action force), then the wall exerts a force on you (the reaction force). If you choose not to push on the wall (exerting no force), then the wall does not exert the reaction force. This leads to people thinking that the reaction force is in response to an action force exerted by a sentient being, which is not the case. You can call the force that you choose to exert on the wall the reaction force and Newton's Laws will still work just as well!

Newton's Third law often leads to confusion when Newton's Second Law is applied. Recall that Newton's Second Law involves the sum of the forces on a particular object. The \textbf{two forces that are mentioned in Newton's Third Law are not exerted on the same object}, so they would never appear together in the sum of the forces from Newton's Second Law, and they never cancel each other. 

\begin{checkpointMC}{You push a heavy block in the North direction. The block is twice as heavy as you are. Which statement is true?}
\item The block exerts half of the force on you, in the North direction.
\item The block exerts the same force on you, but in the South direction. %correct
\item The block exerts double of the force on you, in the South direction.
\item The block is inanimate and thus does not exert a force on you. 
\end{checkpointMC}

\section{Force}
A force is a mathematical tool that is introduced in Newton's theory of physics. A force is not a real ``thing''; there are no forces in the real world, you cannot give someone a force, or buy a force at the supermarket. A force is a purely mathematical tool, so it is important to fight your intuition about what a force is and to stick to well-defined rules for identifying forces to build models.

Mathematically, a \textbf{force is represented by a vector}, and thus has a magnitude and a direction. The SI unit for the magnitude of a force is the Newton, abbreviated $\si{N}$. A force is used to describe how the motion of an object is affected by external agents. It is important to note that a force can be exerted by an inanimate being; that is, there is no intent - no conscious decision to push or pull - associated with a force.

When you push a block along a horizontal surface, we would model the motion of the block as being related to a force that you exert on the block in the direction that you are pushing and with a magnitude that is proportional to how hard you are pushing. Newton's third law states that the block will exert a force on you that is of equal magnitude but in the opposite direction; if we want to model \textit{your motion}, we will need to include that force exerted by the block \textit{on you}. 

If you are pulling on a cart, we would model the motion of the cart by including a force that is exerted on the cart by you. The force would be represented by a vector in the direction that you are pulling with a magnitude based on how hard you are pulling. Similarly, to model your motion, we would include a force vector that is equal in magnitude and opposite in direction to represent the force exerted by the cart on you. When modelling the motion of an object, it is important to consider only the forces exerted on that object.

One way to quantify a force is to use a spring scale. Springs have a natural ``rest length'' if not acted upon by external forces. If you try to stretch a spring, it will ``want'' to come back to its normal rest length; it exerts a force on your hand in the opposite direction from the one you are pulling on the spring. You may have noticed that the more you stretch a spring, the harder you have to pull on it. We can quantify the magnitude of a force by the distance that the forces causes a spring to stretch, since that distance increases with what we conceptualize as a force. For example, one could designate a ``standard spring'' to be one that extends (or compresses) by $\SI{1}{cm}$ when a force of $\SI{1}{N}$ is exerted on the spring. 

\subsection{Types of forces}
\label{sec:newtonslaws:typesofforces}
When modelling the dynamics of an object, we need to identify all of the items that can influence the motion of the object; we do this using the concept of force and identifying all of the forces exerted on that object. Some of the forces can be classified as ``contact forces'' as they arise from something making contact with the object (such as you pushing on the object). Other forces can be exerted ``at a distance''; for example, the force of gravity from the Earth can be exerted on a bird in flight, even if the bird is not in contact with the Earth. In reality, contact forces arise because the electrons from two objects repel each other. When you push against a wall, the reason that you feel a resistance is because the electrons on your hand are repelled by the electrons on the wall; you never actually ``touch'' the wall\footnote{As a matter of fact, it is impossible to ever touch anything, you can just get really close!}!

In this section, we list and describe the most common types of forces that arise. When determining the forces that are acting on an object, it is usually a good idea to run down this list to see if any of these forces should be included. Again, try to fight your intuition about what a force ``feels'' like and instead be objective in determining whether any of the forces below should be included based on their characteristics.

\subsubsection{Weight}
Weight is the force exerted by gravity. While all objects with mass exert an attractive force of gravity on all other objects with mass, that force is usually negligible unless the mass of one of the objects is very large. For an object near the surface of the Earth, we can, to a very good degree of approximation, assume that the only force of gravity on the object is from the Earth. We usually label the force of gravity on an object as $\vec F_g$. All objects near the surface of the Earth will experience a weight, as long as they have a mass. If an object has a mass, $m$, and is located near the surface of the Earth, it will experience a force (its weight) that is given by:
\begin{align*}
\vec F_g = m\vec g
\end{align*}
where $\vec g$ is the Earth's ``gravitational field'' vector and points towards the centre of the Earth. Near the surface of the Earth, the magnitude of the gravitational field is approximately $g=\SI{9.8}{N/kg}$. The gravitational field is a measure of the strength of the force of gravity from the Earth (it is the gravitational force per unit mass). The magnitude of the gravitational field is weaker as you move further from the centre of the Earth (e.g. at the top of a mountain, or in Earth's orbit). The gravitational field is also different on different planets; for example, at the surface of the moon, it is approximately $g_m=\SI{1.62}{N/kg}$ (six times less) - thus the weight of an object is six times less at the surface of the moon (but its mass is still the same). As we will see, the magnitude of the gravitational field from any spherical body of mass $M$ (e.g a planet) is given by:
\begin{align*}
g(r) = G\frac{M}{r^2}
\end{align*}
where $G=\SI{6.67e-11}{}$ is Newton's constant of gravity, and $r$ is the distance from the centre of the object. 

\capfig{0.1\textwidth}{figures/NewtonsLaws/weight.png}{\label{fig:newtonslaws:weight}The weight force on an object near the surface of the Earth points towards the centre of the Earth (downwards).}
Although we have not yet introduced the concept of mass, it is worth emphasizing that mass and weight are different (they have different dimensions). Mass is an intrinsic property of an object, whereas weight is a force of gravity that is exerted on that object because it has mass. On Earth, when we measure our weight, we usually do so by standing on a spring scale, which is designed to measure a force by compressing a spring. We are thus measuring $mg$, which can easily be related to our mass since, on Earth, weight and mass are related by a factor of $g=\SI{9.8}{N/kg}$; this is usually what leads to the confusion between mass and weight. As you gain mass, your weight increases, and vice versa.

\begin{checkpointMC}{A person standing on a scale finds that they weigh $\SI{80}{kg}$.}
\item They exert an upwards force on the Earth with a magnitude of $\SI{80}{N}$.
\item They exert an upwards force on the Earth with a magnitude of $\SI{784}{N}$.%correct
\item They exert an downwards force on the Earth with a magnitude of $\SI{80}{N}$.
\item They exert an downwards force on the Earth with a magnitude of $\SI{784}{N}$.
\item They exert no force on the Earth.
\end{checkpointMC}


\subsubsection{Normal forces}
Normal forces are exerted when two surfaces are in contact and ``pushing'' against each other. For example, if a block is resting on a horizontal table, the table will exert a normal force on the block that is upwards. The force is called ``normal'' because it is normal (i.e. perpendicular) to the interface between the two objects. The normal force exerted by a surface onto an object points in the direction from the surface to the object in such as way that it is perpendicular to the interface between the surface and the object. Because of Newton's Third Law, whenever an object experiences a normal force from a surface, the object also exerts a force of the same magnitude (in the opposite direction) on the surface. The magnitude of the normal force exerted by a surface onto an object, in general, depends on the other forces that are exerted on the object. For example, if a block is on a table, it will experience a stronger normal force if you exert a downwards force on the block.

Figure \ref{fig:newtonslaws:normal} shows two examples of the normal force on a block that is exerted by a surface (it is explicitly assumed that the block also experiences a downwards force from gravity that is not shown). In both cases, the normal force, $\vec N$, is perpendicular to the interface and in the direction that goes from the interface towards the object.

\capfig{0.5\textwidth}{figures/NewtonsLaws/normal.png}{\label{fig:newtonslaws:normal}The normal force, $\vec N$, exerted by a horizontal surface on a block (left side) and by an inclined surface (right side). In both cases, the normal force on the object is perpendicular to the interface between the object and the surface and points in the direction from the interface towards the object.}


\subsubsection{Frictional forces}
A frictional force can exist at the interface between two surfaces and is always perpendicular to the normal force that corresponds to that interface. A frictional force is used to model the resistance that is felt when one tries to slide an object along a surface. The frictional force is used to model the details of how two surfaces interact at a microscopic level; since surfaces are never perfectly flat, two surfaces will never slide without resistance as the various bumps and valleys of the two surfaces will interact (Figure \ref{fig:newtonslaws:fsurfaces}). Furthermore, even if the two surfaces were perfectly smooth, the electrons on the two surfaces would still interact and lead to an effective force when one surface moves with respect to the other. 

\capfig{0.3\textwidth}{figures/NewtonsLaws/fsurfaces.png}{\label{fig:newtonslaws:fsurfaces}Illustration that the frictional force between surfaces can be thought of as arising from microscopic imperfection in the surfaces, although even two perfectly smooth surfaces would still interact. }

One distinguishes between two types of frictional forces: kinetic and static, depending on whether the surfaces are sliding with respect to each other (kinetic) or not (static). Because of Newton's Third Law, the objects associated with each surface will both experience a frictional force (same magnitude, opposite direction).

The frictional force exerted on an object is always parallel to the surface of the object. For the kinetic force of friction, the force is exerted in the direction that is opposite to the motion of the object relative to the surface. For the static force of friction, the force is exerted in the direction that is opposite to the \textit{impeding motion}. If a block is sliding towards the right on a table (Figure \ref{fig:newtonslaws:friction}, left), it will experience a kinetic force of friction that is to the left. The table will then experience a force of friction that is to the right (Newton's Third Law). If there is a heavy crate on the ground which you try to push but does not move (Figure \ref{fig:newtonslaws:friction}, right), there is a force of static friction exerted by the ground on the object that is in the opposite direction that you are pushing. 

One key difference between the static and kinetic friction forces is that the static force can vary in magnitude; the static force of friction on the crate increases as you push harder, until you push hard enough to overcome the maximal force of static friction that can exist between the ground and the crate. Often, the force of kinetic friction is smaller than the static force of friction; you may have noticed that you have to push very hard to get an object sliding, but once it is sliding, you do not need to push as hard to keep it moving.

The magnitude of the kinetic friction force between two surfaces, $f_k$, is modelled as being proportional to the normal force between the two surfaces:
\begin{align*}
f_k=\mu_kN
\end{align*}
where $\mu_k$ is called the ``coefficient of kinetic friction'' and depends on the two surfaces. If you push down on an object, it is more difficult to slide it along a surface, because the normal force, and thus the kinetic friction force increases.

Similarly, the maximum value of the static friction force between to surfaces, $f_s$, is modelled as:
\begin{align*}
f_s\leq\mu_sN
\end{align*}
where $\mu_s$ is called the ``coefficient of static friction'' and the inequality sign is used to indicate that the force of static friction has a maximum value, but that its magnitude depends on the other forces being exerted on the object. For example, if you do not push against a crate on a horizontal surface, there is no force of static friction on the crate (as long as no other forces are exerted that are parallel to the surface).

\capfig{0.5\textwidth}{figures/NewtonsLaws/friction.png}{\label{fig:newtonslaws:friction} (Left:) A block sliding to the right on a horizontal surface (not shown). The force of kinetic friction is always perpendicular to the normal force and opposite of the direction of motion. (Right:) A block that is being acted upon by an external force $\vec F$ to the right. A force of static friction is perpendicular to the normal force and opposite the direction of ``impeding motion'' - without the force of static friction, the block would start to accelerate towards the right, so the force of static friction is to the left.}

\subsubsection{Tension forces}
Tension forces are ``pulling'' forces that are applied by a rope or other non rigid media (e.g. a chain) which cannot usually be used to push\footnote{If you attached a rigid rod to an object and pulled on the rigid rod, you could call the force exerted by the rod on the object a force of tension, even if the rod is rigid.}. If you attach a rope to a crate and use the rope to pull the crate, we call the force exerted by the rope onto the crate a force of tension.

When you pull on a rope that is attached to a wall at the other end, we say that the rope is under tension, or that the tension force is present throughout the rope. If you pull really hard on the rope, it is harder to displace the centre of the rope (or any other point) than if you did not pull on the rope at all. It thus makes sense to view the tension as being present throughout the rope. The force of tension that a rope can apply onto an object depends on what is pulling on the rope at the other end. A rope can be used to change the direction of a force, as illustrated in Figure \ref{fig:newtonslaws:tension}, which shows a pulley and rope being used to lift a block vertically by applying a horizontal force to the rope.
 
\capfig{0.25\textwidth}{figures/NewtonsLaws/tension.png}{\label{fig:newtonslaws:tension} A force $\vec F$ is applied to a rope, which goes around a pulley and is attached to a crate. The rope exerts a force of tension $\vec T$ to the crate. If the pulley and rope are massless, then the magnitude of the applied force is equal to that of the tension force, and the rope and pulley effectively allow one to change the direction of the applied force vector.}

The same tension is present throughout sections of the rope that can move freely. Imagine a rope lying on the ground and someone pressing down with their foot on the rope at its midpoint. If you pull on one end of the rope with your hand, there will be a tension in the section of the rope between your hand and the foot that is pressing on the rope, but the other side of the rope will be slack; the tension is thus different in different sections of the rope. As we will see in later chapters, if a rope goes around a pulley that is accelerating and has mass, then the tension in the rope on either side of the pulley is different; this is similar to the tension being different on either side of the foot pressing down on the rope. 

\subsubsection{Drag forces}
Drag forces are exerted on an object that is moving through a fluid (a gas or a liquid). As an object moves through a fluid, the fluid must be displaced which results in a net force opposing the motion of the object. Drag forces are thus always in the opposite direction of the motion of the object relative to the fluid, similar to friction. Often, one hears the term ``air friction'' which refers to the drag force experienced by an object that is moving through the air. 

There is no good general model for calculating the magnitude of the drag force on any object moving through any fluid. This usually has to be measured; while good software exist for simulating drag, you will still ultimately need to test your new airplane design in a wind tunnel to measure the drage force.

 The magnitude of the drag force generally depends on the cross-section of the object (the area of the object as seen when looking at the object in the direction of motion), the speed of the object, and the visocity of the fluid (how difficult it is to displace the fluid). For small objects moving relatively slowly through a fluid (e.g. pollen falling through the air), the drag force is usually proportional to speed, whereas for larger objects moving faster through a fluid (e.g car or airplane in air) the drag force is usually proportional to speed squared.

\subsubsection{Spring forces}
Spring forces are those forces that are exerted by those materials and object that can be compressed or extended. A common example is a simple coil spring, which has a natural rest length. If the spring is extended, the spring will exert ``restoring forces'' on both ends of the spring that are directed towards the centre of the spring. If the spring is compressed, the spring will exert restoring forces that point away from the centre of the spring. In either case, the spring will exert forces that would allow it to come back to its rest length.

Most springs, if they are not stretched or compressed too much, will exert a restoring force that is given by Hooke's Law:
\begin{align*}
\vec F = -kx \hat x
\end{align*}
where $\vec F$ is the force exerted by the spring, $k$ is called the ``spring constant'' of the spring, and $x$ is the amount that the spring has been stretched or compressed. The negative sign indicates that the restoring force from the spring will be in the opposite direction that the spring length was changed, and it is assumed that the $x$ axis is parallel to the length of the spring with the origin located where the spring is at rest. This is illustrated in Figure \ref{fig:newtonslaws:spring}.

\capfig{0.7\textwidth}{figures/NewtonsLaws/spring.png}{\label{fig:newtonslaws:spring} A spring is attached to a fixed wall on one side and a movable block on the other. The $x$ axis is chosen to describe the position of the block and the origin corresponds to the point where the spring is not extended or compressed (the top row). The $x$ axis is chosen so that positive values of $x$ correspond to the spring being extended. On the bottom left, the spring is extended by a distance $x$ (the position of the block has positive $x$), and the force from the spring on the block is in the negative $x$ direction. On the bottom right, the spring is compressed (the position of the block has negative $x$), and the force from the spring is in the positive $x$ direction.}

\begin{checkpointMC}{In Figure \ref{fig:newtonslaws:spring}, we chose the positive $x$ axis to correspond positions when the spring is extended and verified that Hooke's Law ($\vec F=-kx\hat x$) holds. If we had chosen the positive direction to correspond to compression (positive $x$ to the left), would Hooke's Law still correctly describe the direction of the force exerted by the spring on the block?}
\item Yes. % correct
\item No.
\end{checkpointMC}


\subsubsection{``Applied'' forces}
``Applied'' forces is just a general ``catch-all'' term for specifying forces that are not described above. For example, the force applied by a person onto an object is often referred to as an applied force. 

\section{Mass and inertia}
Mass is a property of an object that quantifies how much matter the object contains. In SI units, mass is measured in kilograms. One kilogram is defined to be the mass of a cylinder that is made of a platinum-iridium alloy that is kept at the international Bureau of Weights and Measures, in France. All other masses are obtained by comparison to this standard. 

Newton's Second Law introduces the concept of mass as that property of the object that determines how large of an acceleration it will experience given a net force exerted on that object. In principle, one can compare the accelerations of different bodies to that of the international standard to determine their mass in kilograms. For example, under a given net force, if an object's acceleration is half of that of the standard kilogram, the object has a mass of $\SI{2}{kg}$. 

In the context of Newton's Second Law, mass is a measure of the inertia of an object; that is, it is a measure of how that particular object resists a change in motion due to a force (we can think of a large acceleration as a large change in motion, as the velocity vector of the object will change more). For this reason, the mass that appears in Newton's Second Law is referred to as ``inertial mass''.

As you recall, the weight of an object is given by the mass of the object multiplied by the strength of the gravitational field, $\vec g$. There is no reason that the mass that is used to calculate weight, $F_g=mg$, has to be the same quantity as the mass that is used to calculate inertia $F=ma$. Thus, people will sometimes make the distinction between ``gravitational mass'' (the mass that you use to calculate weight and the force of gravity) and ``inertial mass'' as described above. Very precise experiments have been carried out to determine if the gravitational and inertial masses are equal. So far, experiments have been unable to detect any difference between the two quantities. As we will see, both Newton's Universal Theory of Gravity and Einstein Theory of General Relativity assume that the two are indeed equal. In fact, it is a key requirement for Einstein's Theory that the two be equal (the assumption that they are equal is called the ``Equivalence Principle''). You should however keep in mind that there is no physical reason that the two are the same, and that as far as we know, it is a coincidence!

Unless stated otherwise, we will not make any distinction between gravitational and inertial mass and assume that they are equal. We will simply use the term ``mass'' and only clarify the type of mass when relevant (e.g. when we cover gravity).

\section{Applying Newton's Laws}
Now that we have introduced all of the concepts from Newton's Theory of Classical Physics, we present some general strategies for building models that use the theory. Recall that if we can describe the motion of all objects of interest to us, we have described everything that we can. Newton's Second Law allows us to determine the acceleration of an object based on the net force acting on the object. Once we have determined the accelerations of all objects of interest we have built a complete model. 

The most important step in applying Newton's Theory is to identify the forces that are exerted on an object. The most important step in applying Newton's Theory is to identify the forces that are exerted on an object. The most important step in applying Newton's Theory is to identify the forces that are exerted on an object. Now that you have read it three times, you realize this step is important, right?!

The strategy for building a model for the motion of an object using Newton's Theory is straightforward:
\begin{enumerate}
\item Identify an inertial frame of reference in which to build the model.
\item Identify the forces acting on the object (did we mention that this step is important?).
\item Draw a free-body diagram.
\item Apply Newton's Second Law.
\end{enumerate}

\subsection{Identifying the forces}
The first step in applying Newton's theory is to identify all of the forces that are acting on an object. This can be done by asking yourself: ``what could possibly be pushing or pulling on the object'', as well as running through the list of forces that we enumerated in section \ref{sec:newtonslaws:typesofforces} to identify if any of them are relevant here. For easy reference, we reproduce the types of forces here:
\begin{itemize}
\item Weight (is the object near the surface of a planet?)
\item Normal forces ( is the object in contact with any surface? There could be more than one!)
\item Frictional forces (are there static or kinetic friction forces?)
\item Tension forces (is something like a rope pulling on the object?)
\item Drag forces (is the object moving through a fluid?)
\item Spring forces (is there a spring pushing or pulling on the object?)
\item Applied forces (is anything else pushing or pulling on the object?)
\end{itemize}

\begin{example}{A block of mass $m$ is at rest on a horizontal table, as shown in Figure \ref{fig:newtonslaws:blockH}. What forces are exerted on the block? }
\capfig{0.2\textwidth}{figures/NewtonsLaws/blockH.png}{\label{fig:newtonslaws:blockH} A block on a horizontal table.}
The forces on the block are illustrated in Figure \ref{fig:newtonslaws:blockH_forces} and are:
\begin{enumerate}
\item $\vec F_g$, its weight.
\item $\vec N$, a normal force exerted by the plane. The normal force is perpendicular to the interface between the table and the block. It points upwards in ``reaction'' to the downwards force that the block exerts onto the table. The downwards force from the block onto the table is not shown, since that force is not exerted on the block but on the table.
\end{enumerate}
\capfig{0.2\textwidth}{figures/NewtonsLaws/blockH_forces.png}{\label{fig:newtonslaws:blockH_forces} Forces on a block on a horizontal table.}
\end{example}


\begin{example}{A block of mass $m$ is at rest on a inclined surface, as shown in Figure \ref{fig:newtonslaws:blockI}. What forces are exerted on the block? }
\label{ex:newtonslaws:blockI}
\capfig{0.2\textwidth}{figures/NewtonsLaws/blockI.png}{\label{fig:newtonslaws:blockI} A block on an inclined surface.}
The forces on the block are illustrated in Figure \ref{fig:newtonslaws:blockI_forces} and are:
\begin{enumerate}
\item $\vec F_g$, its weight.
\item $\vec N$, a normal force exerted by the inclined plane.
\item $\vec f_s$, a force of static friction exerted by the inclined plane. Without this force, the block would slide down. The force is in the direction opposite of impeding motion and is parallel to the interface (and perpendicular to the normal force).
\end{enumerate}

\capfig{0.2\textwidth}{figures/NewtonsLaws/blockI_forces.png}{\label{fig:newtonslaws:blockI_forces} Forces on block on an inclined surface.}
\end{example}

\begin{example}{A block of mass $m$ is at rest on a wedge-shaped block of mass $M$ itself at rest on a horizontal table, as shown in Figure \ref{fig:newtonslaws:2blockswedge}. What forces are exerted on each of the two blocks? }
\label{ex:newtonslaws:2blockswedge}
\capfig{0.2\textwidth}{figures/NewtonsLaws/2blockswedge.png}{\label{fig:newtonslaws:2blockswedge} A block resting on a wedge-shaped block.}
Since it will be too messy to draw all of the forces on the same diagram, we have drawn each block separately in Figure \ref{fig:newtonslaws:2blockswedge_forces}. 

Usually, when multiple blocks are stacked on each other, it is easiest to start with the forces on the top block. In this case, the top block is in the same condition as the block from Example \ref{ex:newtonslaws:blockI}. The forces on the top block are:
\begin{enumerate}
\item $\vec F_g^m$, its weight.
\item $\vec N^m$, a normal force from the wedge-shaped block.
\item $\vec f_s^m$, a force of static friction exerted by the wedge-shaped block.
\end{enumerate}

The wedge-shaped block has the following forces exerted on it:
\begin{enumerate}
\item $\vec F_g^M$, its weight.
\item $\vec N^M$, a normal force exerted by the small block. Note that this force is equal in magnitude and opposite in direction to $\vec N^m$ (the two forces, $\vec N^m$ and $\vec N^M$, which are on different objects, are an action/reaction pair of forces).
\item $\vec f_s^M$, a force of friction exerted by the small block (again, this forms an action/reaction pair of forces with  $\vec f_s^m$). 
\item $N_2^M$, a normal force exerted by the table.
\end{enumerate}


The forces for both blocks are shown in Figure \ref{fig:newtonslaws:2blockswedge_forces}.
\capfig{0.5\textwidth}{figures/NewtonsLaws/2blockswedge_forces.png}{\label{fig:newtonslaws:2blockswedge_forces} Forces on the block and the wedge-shaped block.}
\end{example}


\subsection{Free body diagrams}
In order to analyse the forces on an object more clearly, it is a very good idea to draw a ``Free-Body Diagram'' (FBD). A free-body diagram is simply a diagram where we draw the forces on a single object and represent the object as a point. Because the object is a point, we do not worry where on the object the forces are exerted\footnote{In later chapters, we will see that for extended bodies, it does matter where the forces are applied. However, Newton's Laws as presented so far are only valid for objects that can be represented by a small point (a ``point mass'').}.

For Example \ref{ex:newtonslaws:2blockswedge}, we would draw one free-body diagram for each object (each mass), as shown in Figure \ref{fig:newtonslaws:2blockswedge_fbd}.
\capfig{0.5\textwidth}{figures/NewtonsLaws/2blockswedge_fbd.png}{\label{fig:newtonslaws:2blockswedge_fbd} Free-body diagram for the block and the wedge-shaped block from Example \ref{ex:newtonslaws:2blockswedge}.}

\begin{example}{Two blocks, of masses $m_1$ and $m_2$, are placed on an inclined plane that makes an angle $\theta$ with the horizontal. The blocks are connected by a massless string, as shown in Figure \ref{fig:newtonslaws:2blocksI}. The two blocks are sliding and accelerating downwards with an acceleration, $\vec a$, as shown. The coefficient of kinetic friction between the plane and either block is $\mu_k$. Draw a free-body diagram for each block.}
\label{ex:newtonslaws:2blocksI}
\capfig{0.2\textwidth}{figures/NewtonsLaws/2blocksI.png}{\label{fig:newtonslaws:2blocksI} Two connected blocks sliding down an inclined plane.}
First, we identify the forces on each mass (each block), which we then use to make the free-body diagram shown in Figure \ref{fig:newtonslaws:2blocksI_fbd}. On mass $m_1$, the forces are:

\begin{enumerate}
\item $\vec F_{g1}$, its weight.
\item $\vec N_1$, a normal force exerted by the inclined plane.
\item $\vec f_{k1}$, a force of kinetic friction exerted by the inclined plane. The force is opposite of the direction of motion, and has a magnitude given by $f_{k1}=\mu_kN_1$.
\item $\vec T$, a force of tension from the string. 
\end{enumerate}

On mass $m_2$, the forces are:

\begin{enumerate}
\item $\vec F_{g2}$, its weight.
\item $\vec N_2$, a normal force from the inclined plane.
\item $\vec f_{k2}$, a force of kinetic friction exerted by the inclined plane. The force is opposite of the direction of motion, and has a magnitude given by $f_{k2}=\mu_kN_2$.
\item $-\vec T$, a force of tension from the string. This is the same force as on $m_1$, but in the opposite direction. We chose to label the force as $-\vec T$, instead of using a different variable, since it is just the negative of the vector that represents the tension force on $m_1$. 
\end{enumerate}

In Figure \ref{fig:newtonslaws:2blocksI_fbd}, we have shown the forces on each block using a free-body diagram. We also reproduced the vector for the acceleration (we drew the vector for the acceleration using a thicker arrow to indicate that it has a different dimension). We also reproduced the angle $\theta$ in the free-body diagram, as this is helpful once the free-body diagram is used with Newton's Second Law.

\capfig{0.5\textwidth}{figures/NewtonsLaws/2blocksI_fbd.png}{\label{fig:newtonslaws:2blocksI_fbd} Free-body diagram for the blocks $m_1$ and $m_2$ from Figure \ref{fig:newtonslaws:2blocksI}.}
\end{example}

\subsection{Using Newton's Second Law}
Applying Newton's Second Law is straightforward once all of the forces exerted on a object have been identified. You should thus make sure that you spend most of your time drawing a good and complete free-body diagram before proceeding.

Newton's Second Law is a vector equation that relates the vector sum of all forces exerted on a object and the acceleration vector of the object. This corresponds to one scalar equation per component of the vector.

\begin{align*}
\sum \vec F &=m\vec a\\
\sum F_x &= ma_x \\
\sum F_y &= ma_y \\
\sum F_z &= ma_z
\end{align*}

In order to use Newton's Second Law, we thus need to introduce a coordinate system so that we can work with the components of the vectors (forces and acceleration) in that coordinate system. Usually, a good choice of coordinate system is one where the $x$ axis is parallel to the acceleration vector. Figure \ref{fig:newtonslaws:2blocksI} shows the free-body diagram from the $m_1$ block from the previous example (Example \ref{ex:newtonslaws:2blocksI}) along with a good choice of coordinate system. 

\capfig{0.2\textwidth}{figures/NewtonsLaws/2blocksI_fbd_m1.png}{\label{fig:newtonslaws:2blocksI_fbd_m1} Free-body diagram and choice of coordinate system for the $m_1$ blocks from Figure \ref{fig:newtonslaws:2blocksI}, Example \ref{ex:newtonslaws:2blocksI}.}

To apply Newton's Second Law using the free-body diagram from Figure \ref{fig:newtonslaws:2blocksI_fbd_m1}, we first start by looking at the $x$ components of the vectors:
\begin{align*}
\sum F_x = T-f_{k1}-F_{g1}\sin\theta &= m_1 a\\
\therefore T-f_{k1}-F_{g1}\sin\theta &= m_1 a
\end{align*}
where the tension $\vec T = T\hat x+0\hat y$ and the friction force ($\vec f_{k1}=-f_{k1}\hat x=0\hat y$ are in the $x$ direction. The force of gravity, $\vec F_{g1}=\sin\theta hat x-\cos\theta \hat y$ has a component in the $x$ direction, whereas the normal force, $\vec N=0\hat x+N\hat y$ only has a component in the $y$ direction. The acceleration vector, $\vec a=a\hat x+0\hat y$ is, by construction, only in the $x$ direction. The $y$ component of Newton's Second Law for mass $m_1$ is thus:
\begin{align*}
\sum F_y = N-F_{g1}&=0\\
\therefore N-F_{g1}&=0
\end{align*}
The two equations that we obtained above for $x$ and $y$ fully specify the motion of the $m_1$ block if all quantities are known\footnote{Since we have two equations, we technically only need to specify all but two quantities to be able to fully mode the motion of the block.}. A few notes:
\begin{itemize}
\item When applying Newton's Second Law, analyze each mass in the problem separately. It does not matter that block $m_1$ is connected by a rope to block $m_2$. Once you have determined all of the forces exerted on $m_1$, you can write Newton's Second Law for $m_1$.
\item Newton's Second Law is a vector equation; this means that it is true for each (scalar) component of the vectors involved.
\item You can choose the coordinate system, so choose one that makes it easy to write out the vector components. A good choice is to choose $x$ to be parallel to the acceleration vector. The choice of coordinate system is only made in order to allow you to write out the components of Newton's Second Law based on the free-body diagram.
\item Treat each mass separately (since Newton's Second Law is only true for an individual mass). This means that each mass will have its own free-body diagram and that you can choose the coordinate system that is most convenient for a given free-body diagram. In particular, this means that you do not need to choose the same coordinate system for different masses in a problem.
\end{itemize}

\begin{example}{A block of mass $m_1$ is placed on an incline that makes an angle of $\theta$ with the horizontal. The block of mass $m_1$ is connected by a massless string through a massless pulley to a second block of mass $m_2$ which rests on a horizontal surface. The blocks are accelerating in such a way that the block of mass $m_1$ is accelerating down the incline, as shown in Figure \ref{ex:newtonslaws:2blocksIH}. The coefficient of kinetic friction between either block and the surface it is resting on is $\mu_k$. Write Newton's Second Law for both blocks.}
\label{ex:newtonslaws:2blocksHI}
\capfig{0.5\textwidth}{figures/NewtonsLaws/2blocksHI.png}{\label{fig:newtonslaws:2blocksHI} Two blocks connected by a massless string and massless pulley. Both blocks are accelerating.}

First, we identify the forces on each mass (each block). On mass $m_1$, the forces are:

\begin{enumerate}
\item $\vec F_{g1}$, its weight.
\item $\vec N_1$, a normal force exerted by the inclined plane.
\item $\vec f_{k1}$, a force of kinetic friction exerted by the inclined plane. The force is opposite of the direction of motion, and has a magnitude given by $f_{k1}=\mu_kN_1$.
\item $\vec T_1$, a force of tension from the string. 
\end{enumerate}

On mass $m_2$, the forces are:

\begin{enumerate}
\item $\vec F_{g2}$, its weight.
\item $\vec N_2$, a normal force from the horizontal surface.
\item $\vec f_{k2}$, a force of kinetic friction exerted by the horizontal surface. The force is opposite of the direction of motion, and has a magnitude given by $f_{k2}=\mu_kN_2$.
\item $\vec T_1$, a force of tension from the string. This force has the same magnitude as the tension force $\vec T_1$ exerted on mass $m_1$, because the pulley is massless. 
\end{enumerate}

We can then proceed to draw the free-body diagram for each mass, and use that to write out Newton's Second Law. For mass $m_1$, the free-body diagram is shown in Figure \ref{fig:newtonslaws:2blocksHI_fbd_m1}. We have chosen a coordinate system that has the $x$ axis parallel to the acceleration of the block, and the $y$ axis upwards and perpendicular to the $x$ axis, as shown. 

\capfig{0.2\textwidth}{figures/NewtonsLaws/2blocksHI_fbd_m1.png}{\label{fig:newtonslaws:2blocksHI_fbd_m1} Free-body diagram for $m_1$.}

For $m_1$, we can write Newton's Second Law, starting with the $x$ components:
\begin{align*}
\sum F_x = F_{g1}\sin\theta-f_{k1}-T_1&=m_1a_1\\
\therefore m_1 g\sin\theta -\mu_k N_1 - T_1 &= m_1 a_1
\end{align*}
where, in the second line, we used the magnitude of the weight ($m_1g$) and of the force of kinetic friction $\mu_kN_1$. For the $y$ component of Newton's Second Law, in which the acceleration has no component, we have:
\begin{align*}
\sum F_y = N_1 - F_{g1}\cos\theta &= 0\\
\therefore N_1=m_1g\cos\theta
\end{align*}
which shows us the magnitude of the normal force can easily be expressed in terms of the weight ($F_{g1}=m_1g$) and the angle of the incline.

For $m_2$, we can proceed in much the same way, choosing a different coordinate system, since the acceleration vector for $m_2$ points in a different direction (we don't have to choose a different coordinate system, but we can if we find it makes things easier). The free-body diagram for $m_2$ is shown in Figure \ref{fig:newtonslaws:2blocksHI_fbd_m2} along with our choice of coordinate system.

\capfig{0.2\textwidth}{figures/NewtonsLaws/2blocksHI_fbd_m2.png}{\label{fig:newtonslaws:2blocksHI_fbd_m2} Free-body diagram for $m_2$.}

We start by writing out the $x$ component of Newton's Second Law for $m_2$:
\begin{align*}
\sum F_x = T_2 - f_{k2} &= m_2 a_2\\
\therefore T_2 - \mu_k N_2 = m_2 a_2
\end{align*}
where again, we expressed the kinetic force of friction using the normal force and the coefficient of kinetic friction. The $y$ component of Newton's Second Law gives:
\begin{align*}
\sum F_y = F_{g2}-N_2 &=0\\
\therefore N_2 = m_2g
\end{align*}
where again, we expressed the weight in terms of the mass and $g$, and find that the normal force can be expressed in terms of the weight. 

Now that we have written Newton's Second Law \textbf{for each mass}, we can write all four equations that we obtained to describe \textbf{the system of two masses}. We should also note that the magnitude of the tension forces are the same for the two masses ($T_1=T_2=T$), and that since the masses are connected by a string, the magnitude of their acceleration vectors are the same ($a_1=a_2=a$). Using this, we can describe the full system with the following 4 equations:
\begin{align*}
m_1 g\sin\theta -\mu_k N_1 - T &= m_1 a\\
N_1=m_1g\cos\theta\\
T - \mu_k N_2 = m_2 a\\
N_2 = m_2g
\end{align*}
Of the variables above ($m_1$, $m_2$, $\mu_k$, $T$, $N_1$, $N_2$, $a$), one would only need to specify all but four of them to fully describe the motion of the system. For example, if one specifies the two masses and the coefficient of kinetic friction, all of the other variables can be determined.

\end{example}

\newpage
\section{Summary}
\vspace{1cm}
\begin{chapterSummary}
\item Newton's Three Laws are a theory of classical physics that allow the motion of an object to be fully described by introducing the concepts of force and mass.
\item Newton's First Law allows inertial frames to be defined and introduces the concept of inertia. 
\item Newton's Second Law connects dynamics and kinematics by relating the forces exerted on an object to its acceleration.
\item Newton's Third Law prescribes how the forces exerted by two objects on each other are related.
\item A force is a mathematical tool introduced in Newton's theory.
\item The concept of mass is introduced as a quantity of matter. Inertial mass refers to how that quantity of matter resists acceleration, whereas gravitational mass refers to how that quantity of mass experiences the force of gravity. As far as we can tell, inertial and gravitational mass are the same, but we don't know why. 
\item When applying Newton's theory, the most important part is to identify the forces that act on one object. This can be represented graphically by using a free-body diagram.
\item When applying Newton's Second Law, one needs to choose a coordinate system so that Newton's Second Law can be written out for each component. It is usually good to choose the coordinate system such that the $x$ axis is parallel to the acceleration vector of the object.
\end{chapterSummary}


\section{Thinking about the material}

\subsection{Finding more context}
\begin{enumerate}
\item What was the name of the publication in which Newton's published his three laws, and when was it published?
\item When did Galileo come up with his principle of inertia?
\end{enumerate}

\subsection{Experiments to try at home}

\subsection{Experiment to try in the lab}
\begin{enumerate}
\item How would you make an experiment to determine whether gravitational and inertial mass are equal?
\end{enumerate}



%
\chapter{Applying Newton's Laws}
\label{chap:ApplyingNewtonsLaws}
In this chapter, we take a closer look at how to use Newton's Laws to build models to describe motion. Whereas the previous chapter was focused on identifying the forces that are acting on an object, this chapter focuses on using those forces to describe the motion of the object.

Newton's Laws are meant to describe ``point particles'', that is, objects that can be thought of as a point and thus have no orientation. A block sliding down a hill, a person on a merry-go-round, a bird flying through the air can all be modelled as point particles, as long as we do not need to model their orientation. In all of these cases, we can model the forces on the object using a free-body diagram as the location of where the forces are applied on the object do not matter. In later chapters, we will introduce the tools required to apply Newton's Second Law to objects that can rotate, where we will see that the location of where a force is exerted matters.

\begin{learningObjectives}
{
\item Understand when an object's motion can be modelled as one dimensional (linear).
\item Be able to develop models for objects undergoing linear motion.
\item Be able to develop models for objects undergoing circular motion.
\item Be able to develop models for objects undergoing arbitrary three dimensional motion.
\item Understand the forces involved in circular motion, and understand that ``centripetal'' and ``centrifugal'' forces are not really forces.
}
\end{learningObjectives}

\begin{opening}
\begin{MCquestion}{If a person swings on a swing where the ropes are damaged, where are the ropes most likely to break? }
\item at the bottom of the trajectory, when the speed is the greatest. \correct
\item at the top of the trajectory, when the speed is zero.
\item at the point in the trajectory where the speed is one half of its maximal value.
\end{MCquestion}
\end{opening}



%%%%%%%%%%%%%%%%%%%%%%%%%%%%%%%%%%%%
%% Beginning of chapter content
%%%%%%%%%%%%%%%%%%%%%%%%%%%%%%%%%%%%
\section{Statics}
When using Newton's Laws to model an object, one can identify two broad categories of situations: static and dynamic. In static situations, the acceleration of the object is zero. By Newton's Second Law, this means that the vector sum of the forces (and torques, as we will see in a later chapter) exerted on an object must be zero. In dynamic situations, the acceleration of the object is non-zero. 

For static problems, since the acceleration vector is zero, we can choose a coordinate system in a way that results in as many forces as possible being aligned with the axes (so that we minimize the number of forces that we need to break up into components).

\begin{example}{You push horizontally with a force $\vec F$ on a box of mass $m$ that is resting against a vertical wall, as shown in Figure \ref{fig:applyingnewtonslaws:blockwall}. The coefficient of static friction between the wall and the box is $\mu_s$. What is the minimum magnitude of the force that you must exert for the box to remain stationary?}
\capfig{0.2\textwidth}{figures/ApplyingNewtonsLaws/blockwall.png}{\label{fig:applyingnewtonslaws:blockwall} A horizontal force exerted on box that is resting against a wall.}

Since the acceleration of the box is zero, the vector sum of the forces exerted on the box is zero. We start by identifying the forces exerted on the box; these are:
\begin{enumerate}
\item $\vec F$, the horizontal force that you exert on the box.
\item $\vec F_g$, the weight of the box, with magnitude $mg$.
\item $\vec N$, a normal force exerted by the wall on the box. The force is in the horizontal direction, in the opposite direction to $\vec F$.
\item $\vec f_s$, a vertical force of static friction between the wall and the box. The force points upwards as the ``impeding motion'' of the block is downwards. The force will have at most a magnitude of $f_s\leq\mu_s N$, since the force of static friction depends on the other forces exerted on the object.
\end{enumerate}
The forces are shown in the free-body diagram in Figure \ref{fig:applyingnewtonslaws:blockwall_fbd}, along with our choice of coordinate system which was chosen so that all forces are either in the $x$ or $y$ direction. 
\capfig{0.25\textwidth}{figures/ApplyingNewtonsLaws/blockwall_fbd.png}{\label{fig:applyingnewtonslaws:blockwall_fbd} Free-body diagram of the forces exerted on the box.}
The $x$ component of Newton's Second Law is:
\begin{align*}
\sum F_x = F - N &=0\\
\therefore N = F
\end{align*}
which tells us that the normal force exerted by the wall has the same magnitude as the applied force, $\vec F$. The $y$ component of Newton's Second Law is:
\begin{align*}
\sum F_y = f_s - F_g &=0\\
\therefore f_s -mg &=0\\
\therefore f_s = mg\\
\end{align*}
which tells us that the force of friction must have the same magnitude as the weight. This makes sense, since they are the only forces with components in the $y$ direction, and thus, they must cancel each other out. 

The force of friction will be less than or equal to $\mu_sN$, and thus less than or equal to $\mu_s F$, since $\vec F$ and $\vec N$ have the same magnitude (from the $x$ component of Newton's Second Law). Furthermore, since $f_s=mg$, we can write:
\begin{align*}
f_s &\leq \mu_s F\\
\therefore mg &\leq \mu_s F\\
\therefore \frac{mg}{\mu_s} &\leq F
\end{align*}
which gives us the condition that $F\geq mg/\mu_s$, and thus the minimum magnitude of $F$ in order to keep the box from sliding down.

Although we used the lesser than or equal to sign in the above equations, we could have used an equal sign if we were confident that the force of friction has its maximal magnitude, $f_s=\mu_sN$. The maximal magnitude of the force of friction is proportional to the force that we exert (since $N=F$); if we want to exert the least amount of force $F$, then we need the force of friction to be equal to its maximal magnitude which needs to be equal to the weight of the box. 

\textbf{Discussion:} This model for the minimal required force makes sense because:
\begin{itemize}
\item The dimension of $mg/\mu_s$ is force.
\item If the mass of the box is increased, then one needs to push harder against the box to keep it up.
\item If the coefficient of static friction, $\mu_s$, is increased, one does not need to push as hard. 
\end{itemize}
\end{example}

\section{Linear motion}
We can describe the motion of an object whose \textit{velocity vector does continuously change direction} as ``linear'' motion. For example, an object that moves along a straight line in a particular direction, then abruptly changes direction and continues to move in a straight line can be modelled as undergoing linear motion over two different segments (which we would model individually). An object moving around a circle, with its velocity vector continuously changing direction, would not be considered to be undergoing linear motion. For example, paths of objects undergoing linear and non-linear motion are illustrated in Figure \ref{fig:applyingnewtonslaws:linearmotion}.
\capfig{0.3\textwidth}{figures/ApplyingNewtonsLaws/linearmotion.png}{\label{fig:applyingnewtonslaws:linearmotion} (Left:) Displacement vectors for an object undergoing three segments that can each be modelled as linear motion. (Right:) Path of an object whose velocity vector changes continuously and cannot be considered as linear motion.}

When an object undergoes linear motion, we always model the motion of the object over straight segments separately. Over one such segment, the acceleration vector will be co-linear with the displacement vector of the object (parallel or anti-parallel - note that the acceleration can change direction as it would from a spring force, but will always be co-linear with the displacement).

\begin{example}{\label{ex:applyingnewtonslaws:block}A block of mass $m$ is placed at rest on an incline that makes an angle $\theta$ with respect to the horizontal, as shown in Figure \ref{fig:applyingnewtonslaws:blockI}. The block is nudged slightly so that the force of static friction is overcome and the block starts to accelerate down the incline. At the bottom of the incline, the block slides on a horizontal surface. The coefficient of kinetic friction between the block and the incline is $\mu_{k1}$, and the coefficient of kinetic friction between the block and horizontal surface is $\mu_{k2}$. If one assumes that the block started at rest a distance $L$ from the bottom of the incline, how far along the horizontal surface will the block slide before stopping?}

\capfig{0.5\textwidth}{figures/ApplyingNewtonsLaws/blockI.png}{\label{fig:applyingnewtonslaws:blockI} A block slides down an incline before sliding on a flat surface and stopping. }

We can identify that this is linear motion that we can break up into two segments: (1) the motion down the incline, and (2), the motion along the horizontal surface. We will thus identify the forces, draw the free-body diagram for the block, and use Newton's Second Law twice, once for each segment.

It is often useful to describe the motion in words to help us identify the steps required in building a model for the block. In this case we could say that:
\begin{enumerate}
\item The block slides down the incline and accelerates in the direction of motion. By identifying the forces and applying Newton's Second Law, we can determine its acceleration which will be parallel to the incline.
\item The block will reach a certain speed at the bottom of the incline, which we can determine from kinematics by knowing that the block travelled a distance $L$, with a known acceleration and that it started at rest.
\item The block will decelerate along the horizontal surface. Again, by identifying the forces and using Newton's Second Law, we will be able to determine the acceleration of the block.
\item The block will stop after having travelled an unknown distance, which we can find by using kinematics and knowing the acceleration of the block as well as its initial velocity at the bottom of the incline.
\end{enumerate}

Our first step is thus to identify the forces on the block while it is on the incline. These are:
\begin{enumerate}
\item $\vec F_{g}$, its weight.
\item $\vec N_1$, a normal force exerted by the incline.
\item $\vec f_{k1}$, a force of kinetic friction exerted by the incline. The force is opposite of the direction of motion, and has a magnitude given by $f_{k1}=\mu_{k1}N_1$.
\end{enumerate}
These are shown on the free-body diagram in Figure \ref{fig:applyingnewtonslaws:blockI_fbd1}. As usual, we drew the acceleration, $\vec a_1$, on the free-body diagram, and chose the direction of the $x$ axis to be parallel to the acceleration. 

\capfig{0.25\textwidth}{figures/ApplyingNewtonsLaws/blockI_fbd1.png}{\label{fig:applyingnewtonslaws:blockI_fbd1} Free-body diagram for the block when it is on the incline.}

Writing out the $x$ component of Newton's Second Law, and using the fact that the acceleration is in the $x$ direction ($\vec a=a_1\hat x$):
\begin{align*}
\sum F_x = F_g\sin\theta - f_{k1} &= ma_1\\
\therefore mg\sin\theta - \mu_{k1} N_1 &= ma_1
\end{align*}
where we expressed the magnitude of the kinetic force of friction in terms of the normal force exerted by the plane, and the weight in terms of the mass and gravitational field, $g$. The $y$ component of Newton's Second Law can be written:
\begin{align*}
\sum F_y = N_1-F_g\cos\theta &= 0\\
\therefore N_1 = mg\cos\theta
\end{align*}
which we used to express the normal force in terms of the weight. We can use this expression for the normal force by substituting it into the equation we obtained from the $x$ component to find the acceleration along the incline:
\begin{align*}
mg\sin\theta - \mu_{k1} N_1 &= ma_1\\
mg\sin\theta - \mu_{k1} mg\cos\theta&= ma_1\\
\therefore a_1 &= g(\sin\theta-\mu_{k1}\cos\theta)
\end{align*}
Now that we know the acceleration down the incline, we can easily find the velocity at the bottom of the incline using kinematics. We choose the origin of the $x$ axis to be zero where the block started ($x_0=0$), so that the block is at position $x=L$ at the bottom of the incline. Using kinematics, we can find the speed, $v$, given that the initial speed, $v_0=0$:
\begin{align*}
v^2-v_0^2&=2a_1(x-x_0)\\
v^2&=2a_1L\\
\therefore v &= \sqrt{2a_1L}\\
&=\sqrt{2Lg(\sin\theta-\mu_{k1}\cos\theta)}
\end{align*}
We can now proceed to build a model for the second segment. We first identify the forces on the block when it is on the horizontal surface; these are:
\begin{enumerate}
\item $\vec F_{g1}$, its weight.
\item $\vec N_2$, a normal force exerted by the horizontal surface. This is in general different than the normal force exerted when the block was on the inclined plane. 
\item $\vec f_{k2}$, a force of kinetic friction exerted by the horizontal surface. The force is opposite of the direction of motion, and has a magnitude given by $f_{k2}=\mu_{k2}N_2$.
\end{enumerate}
The forces are illustrated by the free-body diagram in Figure \ref{fig:applyingnewtonslaws:blockI_fbd2}, where we showed the acceleration vector, $\vec a_2$, which we determined to be to the left since the block is decelerating. We also chose an $xy$ coordinate system such that the $x$ axis is anti-parallel to the acceleration, so that the motion is in the positive $x$ direction (and the acceleration in the negative $x$ direction).

\capfig{0.2\textwidth}{figures/ApplyingNewtonsLaws/blockI_fbd2.png}{\label{fig:applyingnewtonslaws:blockI_fbd2} Free-body diagram for the block when it is sliding along the horizontal surface. We (arbitrarily) chose the positive $x$ direction to be in the direction of motion and anti-parallel to the acceleration. We could easily have chosen the opposite direction.}

Writing out the $x$ component of Newton's Second Law:
\begin{align*}
\sum F_x = -f_{k2} &= -ma_2\\
\therefore \mu_{k2}N_2 &= ma_2
\end{align*}
where we expressed the force of kinetic friction using the normal force. We  have to be careful here with the sign of the acceleration; the equation that we wrote implies that $a_2$ is a positive number, since $\mu_{k2}$ is positive and $N_2$ is also positive (it is the magnitude of the normal force). $a_2$ is the magnitude of the acceleration, and we included the fact that the acceleration points in the negative $x$ direction when we put a negative sign in the first line. The $x$ component of the acceleration is $-a_2$, and the vector is given by $\vec a_2=-a_2\hat x$.

The $y$ component of Newton's Second Law will allow us to find the normal force:
\begin{align*}
\sum F_y = N_2 -F_g &=0\\
\therefore N_2 = mg
\end{align*}
which we can substitute back into the $x$ equation to find the magnitude of the acceleration along the horizontal surface:
\begin{align*}
ma_2 &=\mu_{k2}N_2 \\
\therefore a_2&=\mu_{k2}g
\end{align*}
Now that we have found the acceleration along the horizontal surface, we can use kinematics to find the distance that the block travelled before stopping. We choose the origin of the $x$ axis to be the bottom of the incline ($x_0=0$), the acceleration is negative $a_x = -a_2 = -mu_{k2}g$, the final speed is zero, $v=0$, and the initial speed, $v_0$ is given by our model for the first segment. Using one of the kinematic equations:
\begin{align*}
v^2-v_0^2&=2(-a_2)(x-x_0)\\
v_0^2&=2a_2x\\
\therefore x &=\frac{1}{2a_2}v_0^2\\
&=\frac{1}{2\mu_{k2}g}2Lg(\sin\theta-\mu_{k1}\cos\theta)\\
\therefore x&=\frac{(\sin\theta-\mu_{k1}\cos\theta)}{\mu_{k2}}L
\end{align*}
\textbf{Discussion:} The model for the distance $x$ that it takes the block to stop makes sense because:
\begin{itemize}
\item All of the terms in the fraction are dimensionless, so the value of $x$ will have the same dimension as $L$. 
\item If we make $L$ bigger, then $x$ will be bigger (if we release the block from higher up on the incline, it will have more time to accelerate and will slide further before stopping).
\item If we make $\mu_{k1}$ bigger, then $x$ will be smaller: if we increase friction on the incline, the block will have a smaller acceleration and smaller speed at the bottom.
\item If we increase the friction with the horizontal plane (increase $\mu_{k2}$), then $x$ will be reduced (it won't slide as far if there is more friction on the horizontal plane).
\item If we increase $\theta$, the numerator will be larger, so $x$ will increase (the block will accelerate more down a steeper incline and end up further).
\end{itemize} 

\end{example}

\begin{checkpoint}
\begin{MCquestion}{A present is placed at rest on a plane that is inclined, at a distance $L$ from the bottom of the incline, much like the box in Example \ref{ex:applyingnewtonslaws:block} above. At the bottom of the incline, the box is determined to have a speed $v$. If the box is instead released from a distance of $4L$ from the bottom of the incline, what will its speed at the bottom of the incline be?}
\item $v$
\item $2v$\correct
\item $4v$
\item it depends on the coefficient of friction between the present and the plane.
\end{MCquestion}
\end{checkpoint}


\subsection{Modelling situations where forces change magnitude}
So far, the models that we have considered involved forces that remained constant in magnitude. In many cases, the forces exerted on an object can change magnitude and direction. For example, the force exerted by a spring changes as the spring changes length or the force of drag changes as the object changes speed. In these case, even if the object undergoes linear motion, we need to break up the motion into many small segments over which we can assume that the forces are constant. If the forces change continuously, we will need to break up the motion into an infinite number of segments and use calculus.

Consider the block of mass $m$ that is shown in Figure \ref{fig:applyingnewtonslaws:blockvaryingforce}, which is sliding along a frictionless horizontal surface and has a  horizontal force $\vec F(x)$ exerted on it. The force has a different magnitude in the three segments of length $\Delta x$ that are shown. If the block starts at position $x=x_0$ axis with speed $v_0$, we can find, for example, its speed at position $x_3=3\Delta x$, after the block travelled through the three segments.

\capfig{0.7\textwidth}{figures/ApplyingNewtonsLaws/blockvaryingforce.png}{\label{fig:applyingnewtonslaws:blockvaryingforce} A block being pushed along a frictionless horizontal surface with a force that changes.}

The horizontal force, $\vec F$, exerted on the block can be written as:
\begin{align*}
  \vec F (x)=
  \begin{cases}
    F_1\hat x & x<\Delta x \quad \text{(segment 1)}\\
    F_2\hat x & \Delta x \leq x< 2\Delta x \quad \text{(segment 2)}\\
    F_3\hat x & 2\Delta x \leq x\quad \text{(segment 3)}
  \end{cases}
\end{align*}
as it depends on the location of the block. To find the speed of the block at the end of the third segment, we can model each segment separately. The forces exerted on the block are the same in each segment:
\begin{enumerate}
\item $\vec F_g$, its weight, with magnitude $mg$.
\item $\vec N$, a normal force exerted by the ground.
\item $\vec F(x)$, an applied force that changes magnitude with position and is different in the three different segments.
\end{enumerate}
The forces are illustrated in the free-body diagram show in Figure \ref{fig:applyingnewtonslaws:blockvaryingforce_fbd}.

\capfig{0.2\textwidth}{figures/ApplyingNewtonsLaws/blockvaryingforce_fbd.png}{\label{fig:applyingnewtonslaws:blockvaryingforce_fbd} Free-body diagram for the block shown in Figure \ref{fig:applyingnewtonslaws:blockvaryingforce}.}

Newton's Second Law can be used to determine the acceleration of the block for each of the three segments, since the forces are constant within one segment. For all three segments, the $y$ component of Newton's Second Law just tells us that the normal force exerted by the ground is equal in magnitude to the weight of the block. The $x$ component of Newton's Second Law gives the acceleration:
\begin{align*}
\sum F_x = F_i = ma_i
\end{align*}
where we have used the index $i$ to indicate which segment the block is in ($i$ can be 1, 2 or 3). The acceleration of the block in segment $i$ is given by:
\begin{align*}
a_i = \frac{F_i}{m}
\end{align*}
If the speed of the block is $v_0$ at the beginning of segment 1 ($x=x_0$), we can find its speed at the end of segment 1 ($x=x_1$), $v_1$, using kinematics and the fact that the acceleration in segment 1 is $a_1$:
\begin{align*}
v_1^2-v_0^2 &= 2a_1(x_1 - x_0)\\
v_1^2 &=v_0^2+ 2a_1\Delta x\\
\therefore v_1^2 &=v_0^2+2\frac{F_1}{m}\Delta x
\end{align*}
We can now easily find the speed at the end of segment 2 ($x=x_2$), $v_2$, since we know the speed at the beginning of segment 2 ($x_1$,$v_1$) and the acceleration $a_2$:
\begin{align*}
v_2^2 -v_1^2 &= 2a_2(x_2 - x_1)\\
\therefore v_2^2 &= v_1^2 + 2a_2\Delta x\\
&=v_0^2+ 2\frac{F_1}{m}\Delta x + 2\frac{F_2}{m}\Delta x
\end{align*}
It is easy to show that the speed at the end of the third segment is:
\begin{align*}
v_3^2 = v_0^2+ 2\frac{F_1}{m}\Delta x + 2\frac{F_2}{m}\Delta x +2\frac{F_3}{m}\Delta x
\end{align*}
If there were $N$ segments, with the force being different in each segment, we could use the summation notation to write:
\begin{align*}
v_N^2 &= v_0^2 + 2\sum_{i=1}^{i=N} \frac{F_i}{m}\Delta x
\end{align*}
Finally, if the magnitude of the force varied continuously as a function of $x$, $\vec F(x)$, we would model this by taking segments whose length, $\Delta x$, tends to zero (and we would need an infinite number of such segments). For example, if we wanted to know the speed of the object at position $x=X$ along the $x$ axis, with a force that was given by $\vec F(x)=F(x)\hat x$, if the object started at position $x_0$ with speed $v_0$, we would take the following limit:
\begin{align*}
v^2 = v_0^2 + \lim_{\Delta x \to 0} 2\sum_{i=1}^{i=N} \frac{F(x)}{m}\Delta x
\end{align*}
where $\Delta x = \frac{X}{N}$ so that as $\Delta x\to 0$, $N\to\infty$. Of course, integrals are the exact tool that allow us to evaluate the sum in this limit:
\begin{align*}
\lim_{\Delta x \to 0} 2\sum_{i=1}^{i=N} \frac{F_i}{m}\Delta x =2 \int_{x_0}^{X}\frac{F(x)}{m}dx 
\end{align*}
and the speed at position $x=X$ is given by:
\begin{align*}
v^2 = v_0^2 + 2 \int_{x_0}^{X}\frac{F(x)}{m}dx 
\end{align*}
Naturally, we can find the above result starting directly from calculus. If the component of the (net) force in the $x$ direction is given by $F(x)$, then the  acceleration is given by $a(x) = \frac{F(x)}{m}$. The velocity is related to the acceleration:
\begin{align*}
a(x) &= \frac{dv}{dt}\\
\therefore dv &= a(x)dt\\
\end{align*}
We cannot simply integrate the last equation to find that $v=\int a(x)dt$ because the acceleration is given as a function of position, $a(x)$, and not a function of time, $t$. Thus, we cannot simply take the integral over $t$ and must instead ``change variables'' to take the integral over $x$. $x$ and $t$ are related through velocity:
\begin{align*}
v &= \frac{dx}{dt}\\
\therefore dt &= \frac{1}{v}dx
\end{align*}
We can thus write:
\begin{align*}
dv &= a(x)dt = a(x)\frac{1}{v}dx \\
\end{align*}
The equation above is called a ``separable differential equation'', which can also be written:
\begin{align*}
\frac{dv}{dx}=\frac{1}{v}a(x)
\end{align*}
This is called a differential equation because it relates the derivative of a function (the derivative of $v$ with respect to $x$, on the left) to the function itself ($v$ appears on the right as well). The differential equation is ``separable'', because we can separate out all of the quantities that depend on $v$ and on $x$ on different sides of the equation:
\begin{align*}
vdv = a(x)dx
\end{align*} 
This last equation says that $vdv$ is equal to $a(x)dx$. Remember that $dx$ is the length of a very small segment in $x$, and that $dv$ is the change in velocity over that very small segment. Since the terms on the left and right are equal, if we sum (integrate) the quantity $vdv$ over many segments, that sum must be equal to the sum (integral) of the quantity $a(x)dx$ over the same segments. Let us choose those segment such that for the beginning of the first interval the position and speed are $x_0$ and $v_0$, respectively, and the position and speed at the end of the last segment are $X$ and $V$, respectively. We then must have that:
\begin{align*}
\int_{v_0}^{V}vdv&=\int_{x_0}^{X}a(x)dx\\
 \frac{1}{2}V^2 - \frac{1}{2}v_0^2 &= \int_{x_0}^{X}a(x)dx\\
\therefore V^2 &= v_0^2 + 2\int_{x_0}^{X}a(x)dx\\
\end{align*}
which is the same as we found earlier. If the acceleration is constant, we recover our formula from kinematics:
\begin{align*}
V^2 &= v_0^2+ 2\int_{x_0}^{X}adx\\
&=v_0^2+ 2a(X-x_0)\\
\therefore V^2- v_0^2 &= 2a(X-x_0)
\end{align*}

%Example with spring, example with drag and terminal velocity
\begin{example}{\label{ex:applyingnewtonslaws:blockspring}\capfig{0.4\textwidth}{figures/ApplyingNewtonsLaws/blockspring.png}{\label{fig:applyingnewtonslaws:blockspring} A block is launched along a frictionless surface by compressing a spring by a distance $D$. The top panel shows the spring when at rest, and the bottom panel shows the spring compressed by a distance $D$ just before releasing the block.}
A block of mass $m$ can slide freely along a frictionless surface. A horizontal spring, with spring constant, $k$, is attached to a wall on one end, while the other end can move freely, as shown in Figure \ref{fig:applyingnewtonslaws:blockspring}. A coordinate system is defined such that the $x$ axis is horizontal and the free end of the spring is at $x=0$ when the spring is at rest. The block is pushed against the spring so that the spring is compressed by a distance $D$. The block is then released. What speed will the block have when it leaves the spring?}

As you recall, the force exerted by a spring depends on the compression or extension of the spring and is given by Hooke's Law:
\begin{align*}
\vec F(x) = -kx\hat x
\end{align*}
where $x$ is the position of the free end of the spring and $x=0$ corresponds to the spring being at rest. In our case, when the edge of the block is located at $x_0=-D$ (the spring is compressed), the force is thus in the positive $x$ direction (since $x_0$ is a negative number). 

The forces on the block are:
\begin{enumerate}
\item $\vec F_g$, its weight, with magnitude $mg$.
\item $\vec N$, a normal force exerted by the ground.
\item $\vec F(x)$, the spring force.
\end{enumerate}
Since the block is not moving vertically, the magnitude of the normal force must equal the weight $N=mg$, since these are the only forces with components in the vertical direction. The $x$ component of Newton's Second Law gives us the acceleration of the block (which depends on $x$):
\begin{align*}
\sum F_x = -kx &= ma(x)\\
\therefore a(x)&=-\frac{k}{m}x
\end{align*}
Again, recall that if $x$ is negative, then the acceleration will be in the positive direction. Since this scenario is exactly the same that we described above in the text, namely a force that varies continuously with position, we can apply the formula that we found earlier for determining the velocity after a varying force has been applied from position $x=x_0$ to position $x=X$:
\begin{align*}
V^2 &= v_0^2 + 2\int_{x_0}^{X}a(x)dx
\end{align*}
$V$ is the final speed that we would like to find, $v_0=0$ because the block starts at rest, and $x_0=-D$ is the starting position of the block. $X$ is the position along the $x$ axis where the block leaves the spring.

We have to think a little about what the value of $X$ should be: when the spring is compressed and the block accelerating, the spring is pushing the block in the positive $x$ direction. Once the block reaches $x=0$ the spring would want to pull the block backwards, but since it is not attached to the block, it stops exerting a force on the block at that point. The block thus leaves the spring at $x=0$, so that the final position is $X=0$. The speed of the block when it leaves the spring is thus:
\begin{align*}
V^2 &= v_0^2 + 2\int_{x_0}^{X}a(x)dx\\
&= 0 + 2\int_{-D}^{0}a(x)dx\\
&= 2\int_{-D}^{0}-\frac{k}{m}xdx\\
&= 2\left[ - \frac{k}{m}\frac{1}{2}x^2\right]_{-D}^{0}\\
&= \frac{k}{m}D^2\\
\therefore V &= \sqrt{\frac{k}{m}}D
\end{align*}
\textbf{Discussion:} This model for the speed of the block when it leaves the spring makes sense because:
\begin{itemize}
\item The dimension for the expression for $V$ is correct (you should check this!).
\item If the spring is compressed more (bigger value of $D$), then the speed will be higher.
\item If the mass is bigger (more inertia), then the final speed will be lower.
\item If the spring is stiffer (bigger value of $k$), then the final speed will be higher.
\end{itemize}
If you have studied physics before, you may have realized that the speed is easily found by conservation of energy:
\begin{align*}
\frac{1}{2}mV^2=\frac{1}{2}kD^2
\end{align*}
which gives the same value for $V$. As we will see in a later chapter, kinetic and potential energy are defined as they are, precisely because it makes using conservation of energy equivalent to using forces as we just did.
\end{example}

\begin{example}{An object of mass $m$ is released from rest out of a helicopter. The drag (air-resistance) on the object can be modelled as having a magnitude given by $bv$, where $v$ is the speed of the object and $b$ is a constant of proportionality. How does the velocity of the object depend on time?}

\label{ex:applyingnewtonslaws:drag}
As the object falls through the air, the forces exerted on the object are:
\begin{enumerate}
\item $F_g$, its weight, with magnitude $mg$, exerted downwards.
\item $F_d$, the force of drag, with magnitude $bv$, exerted upwards. 
\end{enumerate}
Since the object will fall in a straight line, this is a one-dimensional problem, and we can choose the $x$ axis to be vertical, with positive $x$ pointing downwards, and the origin located where the object was released. The object will thus have a positive acceleration and move in the positive $x$ direction with this choice of coordinate system. This is illustrated in the free-body diagram in Figure \ref{fig:applyingnewtonslaws:drag_fbd}.
\capfig{0.2\textwidth}{figures/ApplyingNewtonsLaws/drag_fbd.png}{\label{fig:applyingnewtonslaws:drag_fbd} Free-body diagram for a block free-falling with drag.}

Newton's Second Law for the object gives:
\begin{align*}
\sum F_x = F_g - F_d &= ma\\
mg - bv &= -ma\\
\therefore a &= g-\frac{b}{m}v 
\end{align*}
In this case, the acceleration depends explicitly on velocity rather than position, as we had before. However, we can use the same methodology to find how the velocity changes with time. First, we can note that the acceleration is zero if:
\begin{align*}
g-\frac{b}{m}v &=0\\
\therefore v = \frac{mg}{b}
\end{align*}
That is, once the object reaches a speed of $v_{term}=mg/b$, it will stop accelerating, i.e. it will reach ``terminal velocity''. Note that this is the same condition as requiring that the drag force ($bv$) have the same magnitude as the weight ($mg$).

Writing the acceleration as $a=\frac{dv}{dt}$, we can write:
\begin{align*}
\frac{dv}{dt} &= \left(g-\frac{b}{m}v \right)
\end{align*}
which again, is a separable differential equation, in which we can write the terms that depend on $v$ and those that depend on $t$ on separate sides of the equal sign:
\begin{align*}
\frac{dv}{g-\frac{b}{m}v}&= dt\\
\frac{dv}{v-\frac{mg}{b}}&= -\frac{b}{m}dt\\
\end{align*}
where we re-arranged the equation in the second line so that it would be easier to integrate in the next step. We can find the velocity, $v(t)$, at some time, $t$, by stating that $v=0$ at $t=0$ and taking the integrals (sum) on both sides. Again, we are modelling the motion as being made up of a large number of very small segments where the quantities on both sides of the equation are the same. Thus, if we sum (integrate) those quantities over all of the same segments, the left and right hand side of the equations will still be equal to each other:
\begin{align*}
\int_0^{v(t)}\frac{dv}{v-\frac{mg}{b}} &= -\int_0^t\frac{b}{m} dt\\
\left[\ln\left(v-\frac{mg}{b} \right)\right]_0^{v(t)} &=-\frac{b}{m}t\\
\ln\left(v(t)-\frac{mg}{b} \right)-\ln\left(-\frac{mg}{b} \right)&=-\frac{b}{m}t\\
\ln\left( \frac{v(t)-\frac{mg}{b}}{-\frac{mg}{b}} \right)&=-\frac{b}{m}t\\
\end{align*}
where, in the last line, we used the property that $\ln(a)-\ln(b)=\ln(a/b)$. By taking the exponential on either side of the equation ($e^{\ln(x)}=x$), we can find an expression for the velocity as a function of time:
\begin{align*}
\frac{v(t)-\frac{mg}{b}}{-\frac{mg}{b}} &= e^{-\frac{b}{m}t}\\
v(t)-\frac{mg}{b} &= -\frac{mg}{b}e^{-\frac{b}{m}t}\\
\therefore v(t) &= \frac{mg}{b}-\frac{mg}{b}e^{-\frac{b}{m}t}\\
&=\frac{mg}{b}\left(1-e^{-\frac{b}{m}t}\right)
\end{align*}
\textbf{Discussion:} This equation tells us that the velocity increases as a function of time, but the rate of increase decreases exponentially with time. At time $t=0$, the velocity is zero, as expected. As $t$ approaches infinity, $v$ approaches, $\frac{mg}{b}$, which is the terminal velocity. The time dependence of the velocity is illustrated in Figure \ref{fig:applyingnewtonslaws:drag_vt}.

\capfig{0.7\textwidth}{figures/ApplyingNewtonsLaws/drag_vt.png}{\label{fig:applyingnewtonslaws:drag_vt} Velocity as a function of time for an object of mass $m=\SI{10}{kg}$ which is free-falling from rest with a drag coefficient $b=\SI{0.5}{Ns/m}$.}

 
\end{example}

\section{Uniform circular motion}
As we saw in Chapter \ref{chap:describingmotioninnd}, ``uniform circular motion'' is defined to be motion along a circle with constant speed. This may be a good time to review Section \ref{sec:describingmotioninnd:circular} for the kinematics of motion along a circle. In particular, for the uniform circular motion of an object around a circle of radius $R$, you should recall that:
\begin{itemize}
\item The velocity vector, $\vec v$, is always tangent to the circle.
\item The acceleration vector, $\vec a$, is always perpendicular to the velocity vector, because the magnitude of the velocity vector does not change.
\item The acceleration vector, $\vec a$, always points towards the centre of the circle.
\item The acceleration vector has magnitude $a=v^2/R$.
\item The angular velocity, $\omega$, is related to the magnitude of the velocity vector by $v=\omega R$ and is constant.
\item The angular acceleration, $\alpha$, is zero for uniform circular motion, since the angular velocity does not change.
\end{itemize}
In particular, you should recall that even if the speed is constant, the acceleration vector is always non-zero in uniform circular motion because the \textbf{velocity changes direction}. According to Newton's Second Law, this implies that there \textbf{must be a net force on the object that is directed towards the centre of the circle}\footnote{The sum of the forces is often called the ``net force'' on an object, and in the specific case of uniform circular motion, that net force is sometimes called the ``centripetal force'' - however, it is not a force in and of itself and it is always the sum of the forces that points towards the centre of the circle.} (parallel to the acceleration):
\begin{align*}
\sum \vec F = m\vec a
\end{align*} 
where the acceleration has a magnitude $a=v^2/R$. Because the acceleration is directed towards the centre of the circle, we sometimes call it a ``radial'' acceleration (parallel to the radius), $a_R$, or a ``centripetal'' acceleration (directed towards the centre), $a_c$.

Consider an object in uniform circular motion in a horizontal plane on a frictionless surface, as depicted in Figure \ref{fig:applyingnewtonslaws:circleH}.
\capfig{0.3\textwidth}{figures/ApplyingNewtonsLaws/circleH.png}{\label{fig:applyingnewtonslaws:circleH} An object undergoing uniform circular motion on a frictionless surface, as seen from above.}
The only way for the object to undergo uniform circular motion as depicted is if the net force on the object is directed towards the centre of the circle. One way to have a force that is directed towards the centre of the circle is to attach a string between the center of the circle and the object, as shown in Figure \ref{fig:applyingnewtonslaws:circleH}. If the string is under tension, the force of tension will always be towards the centre of the circle. The forces on the object are thus:
\begin{enumerate}
\item $\vec F_g$, its weight with magnitude $mg$.
\item $\vec N$, a normal forced exerted by the surface.
\item $\vec T$, a force of tension exerted by the string.
\end{enumerate}
The forces are depicted in the free-body diagram shown in Figure \ref{fig:applyingnewtonslaws:circleH_fbd} (as viewed from the side), where we also drew the acceleration vector. Note that this free-body diagram is only ``valid'' at a particular instant in time since the acceleration vector continuously changes direction and would not always be lined up with the $x$ axis. 
\capfig{0.25\textwidth}{figures/ApplyingNewtonsLaws/circleH_fbd.png}{\label{fig:applyingnewtonslaws:circleH_fbd} Free-body diagram (side view) for the object from Figure \ref{fig:applyingnewtonslaws:circleH_fbd} undergoing uniform circular motion.}
Writing out the $x$ and $y$ components of Newton's Second Law:
\begin{align*}
\sum F_x &= T = ma_R\\
\sum F_y &= N - F_g =0
\end{align*}
The $y$ component just tells us that the normal force must have the same magnitude as the weight because the object is not accelerating in the vertical direction. The $x$ component tells us the relation between the magnitudes of the tension in the string and the radial acceleration. Using the speed of the object, we can also write the relation between the tension and the speed:
\begin{align*}
T &= ma_R=m\frac{v^2}{R}\\
\end{align*}
Thus, we find that the tension in the string increases with the square of the speed, and decreases with the radius of the circle.

\begin{checkpoint}
\begin{MCquestion}{\capfig{0.9\textwidth}{figures/ApplyingNewtonsLaws/trajectoryABCD.png}{\label{fig:applyingnewtonslaws:trajectoryABCD} Possible trajectories (in red) that the block will follow if the string breaks.}
An object is undergoing uniform circular motion in the horizontal plane, when the string connecting the object to the centre of rotation suddenly breaks. What path will the block take after the string broke?}
\item A
\item B \correct
\item C
\item D
\end{MCquestion}
\end{checkpoint}
\vspace{-0.45cm}
\begin{example}{\capfig{0.25\textwidth}{figures/ApplyingNewtonsLaws/car.png}{\label{fig:applyingnewtonslaws:car} A car going around a curve that can be approximated as the arc of a circle of radius $R$.}
\label{ex:applyingnewtonslaws:car}A car goes around a curve which can be approximated as the arc of a circle of radius $R$, as shown in Figure \ref{fig:applyingnewtonslaws:car}. The coefficient of static friction between the tires of the car and the road is $\mu_s$. What is the maximum speed with which the car can go around the curve without skidding? }

If the car is going at constant speed around a circle, then the sum of the forces on the car must be directed towards the centre of the circle. The only force on the car that could be directed towards the centre of the circle is the force of friction between the tires and the road. If the road were perfectly slick (think driving in icy conditions), it would not be possible to drive around a curve since there could be no force of friction. The forces on the car are:
\begin{enumerate}
\item $\vec F_g$, its weight with magnitude $mg$.
\item $\vec N$, a normal force exerted upwards by the road.
\item $\vec f_s$, a force of static friction between the tires and the road. This is static friction, because the surface of the tire does not move relative to the surface of the road if the car is not skidding. The force of static friction has a magnitude that is at most $f_s\leq\mu_sN$.
\end{enumerate}
The forces on the car are shown in the free-body diagram in Figure \ref{fig:applyingnewtonslaws:car_fbd}.
\capfig{0.25\textwidth}{figures/ApplyingNewtonsLaws/car_fbd.png}{\label{fig:applyingnewtonslaws:car_fbd} Free-body diagram for the car as seen looking at the car from the back (the centre of the curve is towards the left).}
The $y$ component of Newton's Second Law tells us that the normal force exerted by the road must equal the weight of the car:
\begin{align*}
\sum F_y = N-F_g&=0\\
\therefore N &=mg
\end{align*}
The $x$ component relates the force of friction to the radial acceleration (and thus to the speed):
\begin{align*}
\sum F_x = f_s =ma_R&=m\frac{v^2}{R}\\
\therefore f_s &= m\frac{v^2}{R}
\end{align*}
The force of friction must be less than or equal to $f_s\leq\mu_sN=\mu_smg$ (since $N=mg$ from the $y$ component of Newton's Second Law), which gives us a condition on the speed:
\begin{align*}
f_s = m\frac{v^2}{R}&\leq\mu_smg\\
v^2 &\leq \mu_s g R\\
\therefore v &\leq \sqrt{\mu_s g R}
\end{align*}
Thus, if the speed is less than $\sqrt{\mu_s g R}$, the car will not skid and the magnitude of the force of static friction, which results in an acceleration towards the centre of the circle, will be smaller or equal to its maximal possible value.

\textbf{Discussion:} The model for the maximum speed that the car can travel around the curve makes sense because:
\begin{itemize}
\item The dimension of $\sqrt{\mu_s g R}$ is speed.
\item The speed is larger if the radius of the curve is larger (one can go faster around a wider curve without skidding).
\item The speed is larger if the coefficient of friction is large (if the force of friction is larger, a larger radial acceleration can be sustained).
\end{itemize}
\end{example}
\vspace{-0.25cm}
\begin{example}{\capfig{0.27\textwidth}{figures/ApplyingNewtonsLaws/circleV.png}{\label{fig:applyingnewtonslaws:circleV} A ball attached to a string undergoing circular motion in a vertical plane.}
A ball is attached to a mass-less string and executing circular motion along a circle of radius $R$ that is in the vertical plane, as depicted in Figure \ref{fig:applyingnewtonslaws:circleV}. Can the speed of the ball be constant? What is the minimum speed of the ball at the top of the circle if it is able to make it around the circle?}

The forces that are acting on the ball are:
\begin{enumerate}
\item $\vec F_g$, its weight with magnitude $mg$.
\item $\vec T$, a force of tension exerted by the string.
\end{enumerate}
Figure \ref{fig:applyingnewtonslaws:circleV_fbd} shows the free-body diagram for the forces on the ball at three different locations along the path of the circle.
\capfig{0.4\textwidth}{figures/ApplyingNewtonsLaws/circleV_fbd.png}{\label{fig:applyingnewtonslaws:circleV_fbd} A ball attached to a string undergoing circular motion in a vertical plane.}
In order for the ball to go around in a circle, there must be at least a component of the net force on the ball that is directed towards the centre of the circle at all times. In the bottom half of the circle (positions 1 and 2), only the tension can have a component directed towards the centre of the circle.

Consider in particular the position labelled 2, when the string is horizontal and the tension is equal to $\vec T_2$. The free-body diagram in Figure \ref{fig:applyingnewtonslaws:circleV_fbd} also shows the vector sum of the weight and tension at position 2 (the red arrow labelled $\sum \vec F$), which points downwards and to the left. It is thus clearly impossible for the acceleration vector to point towards the centre of the circle, and the acceleration will have components that are both tangential ($a_T$) to the circle and radial ($a_R$), as shown by the vector $\vec a_2$ in Figure \ref{fig:applyingnewtonslaws:circleV_fbd}.

The radial component of the acceleration will change the direction of the velocity vector so that the ball remains on the circle, and the tangential component will reduce the magnitude of the velocity vector. According to our model, it is thus impossible for the ball to go around the circle at constant speed, and the speed must decrease as it goes from position 2 to position 3, no matter how one pulls on the string (you can convince yourself of this by drawing the free-body diagram at any point between points 2 and 3). 

The minimum speed for the ball at the top of the circle is given by the condition that the tension in the string is zero just at the top of the trajectory (position 3). The ball can still go around the circle because, at position 3, gravity is towards the centre of the circle and can thus give an acceleration that is radial, even with no tension. The $y$ component of Newton's Second Law, at position 3 gives:
\begin{align*}
\sum F_y = -F_g &= ma_y\\
\therefore a_y &=-g
\end{align*}
The magnitude of the acceleration is the radial acceleration, and is thus related to the speed at the top of the trajectory:
\begin{align*}
a_R&=-a_y=g = m\frac{v^2}{R}\\
\therefore v_{min}&=\sqrt{\frac{gR}{m}}
\end{align*}
which is the minimum speed at the top of the trajectory for the ball to be able to continue along the circle. The tension in the string would change as the ball moves around the circle, and will be highest at the bottom of the trajectory, since the tension has to be bigger than gravity so that the net force at the bottom of the trajectory is upwards (towards the centre of the circle).

\textbf{Discussion:} The model for the minimum speed of the ball at the top of the circle makes sense because:
\begin{itemize}
\item $\sqrt{\frac{gR}{m}}$ has the dimension of speed.
\item The minimum velocity is larger if the circle has a larger radius (try this with a mass attached at the end of a string). 
\item The minimum velocity is larger if the mass is bigger (again, try this at home!). 
\end{itemize}
\end{example}

\begin{checkpoint}
\begin{MCquestion}{Consider a ball attached to a string, being spun in a vertical circle (such as the one depicted in figure \ref{fig:applyingnewtonslaws:circleV}). If you shortened the string, how would the minimum angular velocity (measured at the top of the trajectory) required for the ball to make it around the circle change? }
\item It would decrease \correct
\item It would stay the same
\item It would increase
\end{MCquestion}
\end{checkpoint}

\subsection{Banked curves}
As we saw in Example \ref{ex:applyingnewtonslaws:car}, there is a maximum speed with which a car can go around a curve before it starts to skid. You may have noticed that roads, highways especially, are banked where there are curves. Racetracks for cars that go around an oval (the boring kind of car races) also have banked curves. As we will see, this allows the speed of vehicles to be higher when going around the curve; or rather, it makes the curves safer as the speed at which vehicles \textit{would} skid is higher. In Example \ref{ex:applyingnewtonslaws:car}, we saw that it was the force of static friction between the tires of the car and the road that provided the only force with a component towards the centre of the circle. The idea of using a banked curve is to change the direction of the normal force between the road and the car tires so that it, too, has a component in the direction towards the centre of the circle. 

Consider the car depicted in Figure \ref{fig:applyingnewtonslaws:carbank} which is seen from behind making a left turn around a curve that is banked by an angle $\theta$ with respect to the horizontal and can be modelled as an arc from a circle of radius $R$.
\capfig{0.7\textwidth}{figures/ApplyingNewtonsLaws/carbank.png}{\label{fig:applyingnewtonslaws:carbank} A car moving into the page and going around a banked curved so that it is turning towards the left (the centre of the circle is to the left). }
The forces exerted on the car are the same as in Example \ref{ex:applyingnewtonslaws:car}, except that they point in different directions. The forces are:
\begin{enumerate}
\item $\vec F_g$, its weight with magnitude $mg$.
\item $\vec N$, a normal force exerted by the road, perpendicular to the surface of the road.
\item $\vec f_s$, a force of static friction between the tires and the road. This is static friction, because the surface of the tire does not move relative to the surface of the road if the car is not skidding. The force of static friction has a magnitude that is at most $f_s\leq\mu_sN$ and is perpendicular to the normal force. The force could be either upwards or downwards, \textit{depending on the other forces on the car}.
\end{enumerate} 
A free-body diagram for the forces on the car is shown in Figure \ref{fig:applyingnewtonslaws:carbank_fbd}, along with the acceleration (which is in the radial direction, towards the centre of the circle), and our choice of coordinate system (choosing $x$ parallel to the acceleration). The direction of the force of static friction is not known \textit{a priori} and depends on the speed of the car:
\begin{itemize}
\item If the speed of the car is zero, the force of static friction is upwards. With a speed of zero, the radial acceleration is zero, and the sum of the forces must thus be zero. The impeding motion of the car would be to slide down the banked curve (just like a block on an incline).
\item If the speed of the car is very large, the force of static friction is downwards, as the impeding motion of the car would be to slide up the bank. The natural motion of the car is to go in a straight line (Newton's First Law). If the components of the normal force and of the force of static friction directed towards the centre of the circle are too small to allow the car to turn, then the car would slide up the bank (so the impeding motion is up the bank and the force of static friction is downwards).
\end{itemize}
\capfig{0.3\textwidth}{figures/ApplyingNewtonsLaws/carbank_fbd.png}{\label{fig:applyingnewtonslaws:carbank_fbd} Free-body diagram for the forces on the car. The direction of the force of static friction cannot be determined, as it depends on the acceleration of the car, so it is shown twice (with dotted lines). }
There is thus an ``ideal speed'' at which the force of static friction is precisely zero, and the $x$ component of the normal force is responsible for the radial acceleration. At higher speeds, the force of static friction is downwards and increases in magnitude to keep the car's acceleration towards the centre of the circle. At some maximal  speed, the force of friction will reach its maximal value, and no longer be able to keep the car's acceleration pointing towards the centre of the circle. At speeds lower than the ideal speed, the force of friction is directed upwards to prevent the car from sliding down the bank. If the coefficient of static friction is too low, it is possible that at low speeds, the car would start to slide down the bank (so there would be a minimum speed below which the car would start to slide down). 

Let us model the situation where the force of static friction is identically zero so that we can determine the ideal speed for the banked curve. The only two forces on the car are thus its weight and the normal force. The $x$ and $y$ component of Newton's Second Law give:
\begin{align}
\label{eq:applyingnewtonslaws:carbank_x}
\sum F_x &= N\sin\theta = ma_R=m\frac{v^2}{R}\nonumber\\
\therefore N\sin\theta &= m\frac{v^2}{R}
\end{align}
\begin{align}
\label{eq:applyingnewtonslaws:carbank_y}
\sum F_y &= N\cos\theta-F_g = 0\nonumber\\
\therefore N\cos\theta&=mg
\end{align}
We can divide Equation \ref{eq:applyingnewtonslaws:carbank_x} by Equation \ref{eq:applyingnewtonslaws:carbank_y}, noting that $\tan\theta=\sin\theta/\cos\theta$, to obtain:
\begin{align*}
\tan\theta &= \frac{v^2}{gR}\\
\therefore v_{ideal} &=\sqrt{gR\tan\theta}
\end{align*}
At this speed, the force of static friction is zero. In practice, one would use this equation to determine which bank angle to use when designing a road, so that the ideal speed is around the speed limit or the average speed of traffic. We leave it as an exercise to determine the maximal speed that the car can go around the curve before sliding out.


\subsection{Inertial forces in circular motion}
As you sit in a car that is going around a curve, you will feel pushed outwards, away from the centre of the circle that the car is going around. This is because of your inertia (Newton's First Law), and your body would go in a straight line if the car were not exerting a net force on you towards the centre of the circle. You are not so much feeling a force that is pushing you outwards as you are feeling the effects of the car seat pushing you inwards; if you were leaning against the side of the car that is on the outside of the curve, you would feel the side of the car pushing you inwards towards the centre of the curve, even if it ``feels'' like you are pushing outwards against the side of the car.

If we model your motion looking at you from the ground, we would include a force of friction between the car seat (or the side of the car, or both) and you that is pointing towards the centre of the circle, so that the sum of the forces exerted on you is towards the centre of the circle. We can also model your motion from the non-inertial frame of the car. As you recall, because this is a non-inertial frame of reference, we need to include an additional inertial force, $\vec F_I$, that points opposite of the acceleration of the car, with magnitude $F_I=ma_R$ (if the net acceleration of the car is $a_R$). Inside the non-inertial frame of reference of the car, your acceleration (relative to the reference frame, i.e. the car) is zero. This is illustrated by the diagrams in Figure \ref{fig:applyingnewtonslaws:carinertial}.
\capfig{0.6\textwidth}{figures/ApplyingNewtonsLaws/carinertial.png}{\label{fig:applyingnewtonslaws:carinertial}(Left:) A person sitting on a car seat in a car turning towards the left. (Centre:) Free-body diagram for the person as modelled in the inertial reference frame of the ground. (Right:) Free-body diagram for the person as modelled in the non-inertial frame of reference of the car, including an additional inertial force. }
The $y$ component of Newton's Second Law in both frames of reference is the same:
\begin{align*}
\sum F_y&=N-F_g=0\\
\therefore N&=mg
\end{align*}
and simply tells us that the normal force is equal to the weight. In the reference frame of the ground, the $x$ component of Newton's Second Law gives:
\begin{align*}
\sum F_x &= f_s = ma_R\\
\therefore f_s &= m\frac{v^2}{R}
\end{align*}
In the frame of reference of the car, where your acceleration is zero and an inertial force of magnitude $F_I=mv^2/R$ is exerted on you, the $x$ component of Newton's Second Law gives:
\begin{align*}
\sum F_x &= f_s-F_I = 0\\
\therefore f_s - m\frac{v^2}{R} &= 0
\end{align*}
which of course, mathematically, is exactly equivalent. The inertial force is not a real force in the sense that it is not exerted by anything. It only comes into play because we are trying to use Newton's Laws in a non-inertial frame of reference. However, it does provide a good model for describing the sensation that we have of being pushed outwards when the car goes around a curve. Sometimes, people will refer to this force as a ``centrifugal'' force, which means ``a force that points away from the centre''. You should however remember that this is not a real force exerted on the object, but is the result of modelling motion in a non-inertial frame of reference.

\begin{checkpoint}
\begin{MCquestion}{Jamie is driving his tricycle around a circular pond. Jamie feels a centrifugal force with magnitude $F_I$. If Jamie pedals twice as fast, what will be the magnitude of the centrifugal force that he experiences?}
\item $\sqrt{2}F_I$
\item $\frac{1}{2}F_I$
\item $2F_I$
\item $4F_I$\correct
\end{MCquestion}
\end{checkpoint}

\section{Non-uniform circular motion}
In non-uniform circular motion, an object's motion is along a circle, but the object's speed is not constant. In particular, the following will be true
\begin{itemize}
\item The object's velocity vector is always tangent to the circle.
\item The speed and angular speed of the object are not constant.
\item The angular acceleration of the object is not zero.
\item The acceleration vector will not point towards the centre of the circle. 
\end{itemize}
Since the acceleration vector does not point towards the centre of the circle, it is usually convenient to break up the acceleration vector into two components: $a_R$, a component that is radial (towards the centre of the circle), and $a_T$, a component that is tangent to the circle (and perpendicular to to the radial component). The \textbf{radial component is ``responsible'' for the change in direction of the velocity} such that the object goes in a circle. the magnitude of the radial acceleration is the same as it is for uniform circular motion:
\begin{align*}
a_R=\frac{v^2}{r}
\end{align*}
where the speed is no longer constant in time. The tangential component of the acceleration is responsible for changing the magnitude of the velocity of the object:
\begin{align*}
a_T = \frac{dv}{dt}
\end{align*}


\begin{example}{
\capfig{0.25\textwidth}{figures/ApplyingNewtonsLaws/ant.png}{\label{fig:applyingnewtonslaws:ant}An ant on a horizontal turntable that is starting to spin, as seen from above.}
A small ant is sleeping on a turntable just as the turntable starts to spin from rest, with an angular acceleration $\alpha=\SI{1}{rad/s}$ that is small enough so that, initially, the ant remains on the turntable. The ant is a distance $R=\SI{0.1}{m}$ from the centre of the turntable, as shown in Figure \ref{fig:applyingnewtonslaws:ant}  and the coefficient of static friction between the ant's ``feet'' and the turntable is $\mu_s=0.5$. After how much time will the ant slide off from the turntable?}
As the turntable accelerates, the force of static friction between the turntable and the ant will keep the ant moving with the turntable. Once the turntable is going fast enough, the force of friction will no longer be large enough to provide the total acceleration that is required to keep the ant moving with the turntable (with a constant tangential component of the acceleration and an increasing radial component of the acceleration). 

The forces on the ant are:
\begin{enumerate}
\item $\vec F_g$, its weight, with magnitude $mg$.
\item $\vec N$, a normal force exerted by the turntable on the ant.
\item $\vec f_s$, a force of static friction exerted by the turntable on the ant. The force of friction will be such that it has both radial and tangential components.
\end{enumerate}
A free-body diagram for the forces on the ant is shown in Figure \ref{fig:applyingnewtonslaws:ant_fbd}, as seen from above and from the side, for some point in time. We have chosen the point in time to be just when the ant is about to slide off of the turntable, when the force of static friction makes an unknown angle $\theta$ with the $x$ axis. We have placed the origin of the coordinate system at the centre of the turntable and chosen the $x$ axis such that the ant is located on the positive $x$ axis with its velocity in the positive $y$ direction. We used a three dimensional coordinate system where the weight and normal force are exerted in the $z$ (vertical) direction since the acceleration vector of the ant will have both radial ($x$) and tangential ($y$) components.
\capfig{0.5\textwidth}{figures/ApplyingNewtonsLaws/ant_fbd.png}{\label{fig:applyingnewtonslaws:ant_fbd}(Left:) Forces on the ant as seen from above. The normal force is out of the page ($\odot$), whereas the weight is into the page ($\times$). (Right:) Forces on the ant as seen from the side. Note that the acceleration vector and force of static friction also have components in the $y$ direction, which is why their magnitude is shown as being smaller than in the top view.}
Newton's Second Law has to be written out in three components. The $z$ component relates the weight and normal force:
\begin{align*}
\sum F_z &= N - F_g = 0\\
\therefore N&=mg
\end{align*}
The $x$ component of Newton's Second Law is such that the $x$ component of the acceleration is its radial component:
\begin{align*}
\sum F_x &= -f_s\cos\theta = -ma_R = -m\frac{v^2}{R}\\
\therefore f_s\cos\theta &= m\frac{v^2}{R}
\end{align*}
The $y$ component of Newton's Second relates the tangential component of the force of static friction to the tangential component of the acceleration:
\begin{align*}
\sum F_y &= f_s\sin\theta = ma_T \\
\therefore f_s\sin\theta &= m\alpha R
\end{align*}
where we used the fact that the (linear) tangential acceleration, $a_T$, is related to the angular acceleration, $\alpha$, by:
\begin{align*}
a_T = \alpha R
\end{align*}
Summarizing the three equations that we obtained from the three components of Newton's Second Law:
\begin{align*}
f_s\cos\theta &= m\frac{v^2}{R}\\
f_s\sin\theta &= m\alpha R\\
N&=mg
\end{align*}
Also, note that the speed, $v(t)$ at some time $t$ is given by simple kinematics:
\begin{align*}
v(t)=v_0+a_Tt=(0)+\alpha R t
\end{align*}
The ant will start to slip when the force of friction reaches its maximal amplitude, $f_s=\mu_sN=\mu_Smg$. The $x$ of Newton's Second Law can be used to find an expression for the time at which force of friction reaches its maximal value (in terms of the unknown angle $\theta$):
\begin{align*}
f_s\cos\theta &= m\frac{v^2}{R}\\
\mu_sg\cos\theta &= R\alpha^2t^2\\
\therefore t &= \sqrt{\frac{\mu_sg\cos\theta}{R\alpha^2}}
\end{align*}
We can use the $y$ component to determine the angle $\theta$:
\begin{align*}
f_s\sin\theta &= m\alpha R\\
\mu_sg\sin\theta &= \alpha R\\
\therefore \sin\theta &= \frac{\alpha R}{\mu_s g}\\
\therefore \theta &= \sin^{-1}\left( \frac{\alpha R}{\mu_s g}  \right)=\sin^{-1}\left( \frac{(\SI{1}{rad/s^2})(\SI{0.1}{m})}{(0.5)(\SI{9.8}{N/kg})}  \right)\\
&=\SI{1.17}{\degree}
\end{align*}
The angle is very small, and we see that the force of friction is mostly directed towards the centre of the circle. The radial acceleration is thus much larger than the tangential acceleration. We can then use the angle to find the time using the expression we derived above:
\begin{align*}
t &= \sqrt{\frac{\mu_sg\cos\theta}{R\alpha^2}}= \sqrt{\frac{(0.5)(\SI{9.8}{N/kg})\cos(\SI{1.17}{\degree})}{(\SI{0.1}{m})(\SI{1}{rad/s^2})^2}}\\
&=\SI{7.0}{s}
\end{align*}
\end{example}


%%%%%%%%%%%%%%%%%%%%%%%%%%%%%%%%%%%%
%% End of chapter content
%%%%%%%%%%%%%%%%%%%%%%%%%%%%%%%%%%%%

\newpage
\section{Summary}
\vspace{1cm}
\begin{chapterSummary}

When the velocity of an object does not change direction continuously (``linear motion''), we can model its motion independently over several segments in such a way that the motion is one dimensional in each segment. This allows us to choose a coordinate system in each segment where the acceleration vector is co-linear with one of the axes.

When the forces on an object changes continuously, we need to use calculus to determine the motion of the object. If the velocity vector for an object changes direction continuously, we need to model the motion in each dimension independently.

If an object undergoes uniform circular motion, the acceleration vector and the sum of the forces always point towards the centre of the circle. In the radial direction, Newton's Second Law gives
\begin{align*}
\sum \vec F = ma_R = m\frac{v^2}{R}
\end{align*}
If an object's speed is changing as it moves around a circle the acceleration vector will have a component that is towards the centre of the circle (the radial component) and a component that is tangential to the circle. The tangential component is responsible for the change in speed, whereas the radial component is responsible for the change in direction of the velocity.

In a reference frame that is rotating about a circle, an inertial force, sometimes called the centrifugal force, appears to push all objects co-moving with the reference frame towards the outside of the circle.

\end{chapterSummary}

\newpage
\section{Thinking about the material}
\begin{chapteractivity}{Reflect and research}
{
\item Is there a maximum speed with which an object can spin? (Something about the thing eventually flying apart if it rotates too fast, as the atoms can not be held together at some point - maybe there is a cool video to look up?)
}
\end{chapteractivity}

\begin{chapteractivity}{To try at home}
{
\item Spin a mass on a string in a vertical circle, what is the tension in the string when the mass is at the top for it to barely make it around?
\item Spin a mass on a string in a vertical circle, how does the minimum speed at the top of the circle to barely make it around  depend on the radius of the circle or the mass?
\item Spin a mass on a string in a vertical circle, describe the motion if the mass does not have the minimum speed to make it around the circle. If it makes it to the top, does it automatically make it all the way around the circle?
}
\end{chapteractivity}

\begin{chapteractivity}{To try in the lab}
{
\item Build a conical pendulum and determine whether the opening angle of the cone is related to the speed of the bob, in the way that you expect it to be.
}
\end{chapteractivity}

\newpage
\subsection{Problems and Solutions}
%Conical pendulum, roller coaster (speed at the top) 

\begin{problemParts}{soln:applyingnewtonslaws:pendulum}{\label{prob:applyingnewtonslaws:pendulum} Consider a conical pendulum with a mass $m$, attached to a string of length $L$. The mass executes uniform circular motion in the horizontal plane, about a circle of radius $R$, as shown in Figure \ref{fig:applyingnewtonslaws:conicalpendulum}. One can think of the horizontal circle and the point where the string is attached to as forming a cone. The circular motion is such that the (constant) angle between the string and the vertical is $\theta$.}{
\item Derive an expression for the tension in the string.
\item Derive an expression for the speed of the mass.
\item Derive an expression for the period of the motion.
}
\end{problemParts}

\capfig{0.30\textwidth}{figures/ApplyingNewtonsLaws/conicalpendulum.png}{\label{fig:applyingnewtonslaws:conicalpendulum} The conical pendulum.}

\begin{problem}{soln:applyingnewtonslaws:rollercoaster}{\label{prob:applyingnewtonslaws:rollercoaster} Barb and Kenny are going to the amusement park. Barb insists on riding the giant roller coaster, but Kenny is scared that they will fall out of the roller coaster at the top of the loop. Barb reassures Kenny by asking the roller coaster technician for more information. The technician says that they will be travelling at $\SI{15}{m/s}$ when upside down, and that the roller coaster loop has a radius of $\SI{22}{m}$. Kenny is still sceptical. Is he correct in being sceptical?}
\capfig{0.6\textwidth}{figures/ApplyingNewtonsLaws/rollercoaster.png}{\label{fig:applyingnewtonslaws:rollercoaster} The roller coaster}
\end{problem}

\newpage
\subsection{Solutions}
\begin{solution}{prob:applyingnewtonslaws:pendulum}\label{soln:applyingnewtonslaws:pendulum} 
\begin{enumerate}[label=\alph*)]
\item We start by identifying the forces that are acting on the mass. These are:
\begin{itemize}
\item $\vec F_g$, its weight, with a magnitude $mg$.
\item $\vec F_T$, a force of tension exerted by the string.
\end{itemize}
The forces are illustrated in Figure \ref{fig:applyingnewtonslaws:conicalpendulumfbd}, along with our choice of coordinate system and the direction of the acceleration of the mass (towards the centre of the circle).
\capfig{0.2\textwidth}{figures/ApplyingNewtonsLaws/conicalpendulumfbd.png}{\label{fig:applyingnewtonslaws:conicalpendulumfbd} Forces acting on the conical pendulum}
The $y$ component of Newton's Second law gives the relation between the tension in the string, the weight, and the angle $\theta$
\begin{align*}
\sum F_y&=0 \\
F_T\cos\theta -F_g&=0 \\
F_T\cos\theta&=mg \\
\therefore F_T&=\frac{mg}{\cos\theta} \\
\end{align*}
\item In order for the mass to move in a circle, the net force must be directed towards the centre of the circle at all times. The $x$ component of Newton's Second Law, combined with our expression for the magnitude of the tension, $F_T$, allows us to determine the speed of the mass:
\begin{align*}
\sum F_x&=ma_r \\
F_T\sin\theta&=m\frac{v^2}{R}\\
\left(\frac{mg}{\cos\theta}\right)\sin\theta &=m\frac{v^2}{R}\\
g\tan\theta&=\frac{v^2}{R}\\
\therefore v &= \sqrt{gR\tan\theta}
\end{align*}
\item Now that we know the speed, we can easily find the period, $T$, of the motion:
\begin{align*}
T&=\frac{2\pi R}{v} \\
&=\frac{2\pi R}{\sqrt{gR\tan\theta }}=2\pi\sqrt{\frac{R}{g\tan\theta}}
\end{align*}

\end{enumerate}
\end{solution}

\begin{solution}{prob:applyingnewtonslaws:rollercoaster}\label{soln:applyingnewtonslaws:rollercoaster} 
We need to determine if the speed of Barb and Kenny is large enough for them to go around the circle. The minimum speed that they must have at the top of the loop is such that their weight (the only force acting on them) provides the centripetal (net) force required to go around the loop. 

Writing Newton's Second Law in the vertical direction, for the case where only the weight acts on Barb or Kenny (mass $m$), when they are going at speed $v$
\begin{align*}
mg &= ma_R = m\frac{v^2}{R}\\
\therefore v &= \sqrt{gR} = \sqrt{(\SI{9.8}{m/s^2})(\SI{22}{m})}=\SI{14.68}{m/s}
\end{align*}
This corresponds to the minimum speed that they must have at the top of the loop to make it around. If they go faster, the normal force from their seat (downwards, since they are upside-down), would result in a larger net force towards the centre of the circle. This situation corresponds to the normal force from their seat just barely reaching 0 at the top of the loop. Since the roller coaster is quoted as having a speed of $\SI{15}{m/s}$ at the top of the loop, they will just barely make it. However, this is way too close to the minimal speed to not fall out of the roller coaster, so Kenny is correct in being sceptical! The engineers designing the roller coaster should include a much bigger safety margin! 
\end{solution}




%\chapter{Work and energy}
\label{chap:workenergy}
In this chapter, we introduce a new way to build models derived from Newton's theory of classical physics. We will introduce the concepts of work and energy in order to allow us to model situations using scalar quantities, such as energy, instead of vector quantities, such as forces. It is important to remember that even if we use energy and work, these are tools that are derived from Newton's Laws; that is, we may not be using Newton's Second Law explicitly, but the models that we develop are still based on the same theory of classical physics. 

\begin{learningObjectives}{
 \item Understand the concept of work and how to calculate the work done by a force.
 \item Understand the concept of the net work done on an object and how that relates to a change in speed of the object.
 \item Understand the concept of kinetic energy and where it comes from.
 \item Understand the concept of power.
 }
\end{learningObjectives}

\begin{opening}
You are holding a heavy book. The book does not move, even though it is difficult for you to keep it from falling to the ground. Do your arms do work on the book? If you start walking to class while holding the book, do your arms do work on the book? 

\begin{answer}
Your arms do no work on the book. There is no displacement (the book does not move up or down), so you do no work, even if its tiring! If you are walking, the displacement is perpendicular to the force applied by your arms, your arms do no work.
\end{answer}
\end{opening}

\newpage
\section{Work}
We introduce the concept of work as the starting point for using energy instead of forces. Work is a scalar quantity that is meant to represent how a force that was exerted onto an object over a given distance has resulted in a change in the speed of the object. We will first introduce the concept of work done by a force on an object, and then look at how work can change the kinematics of the object. This is analogous to how we first defined the concept of force, and then looked at how force affects motion (by using Newton's Second Law, which connected the concept of force to the acceleration of the object).

The work done by a force, $\vec F$, on an object over a displacement, $\vec d$, is defined to be:
\begin{align}
\Aboxed{W = \vec F \cdot \vec d = Fd\cos\theta = F_xd_x+F_yd_y+F_zd_z}
\end{align}
where $\theta$ is the angle between the vectors when these are placed tail to tail, as in Figure \ref{fig:workenergy:fddotproduct}. The dimension of work, force times displacement, is also called ``energy''. The S.I. unit for energy is the Joule (abbreviated $\si{J}$) which is equivalent to $\si{Nm}$ or $\si{kg m^2/s^2}$ in base units.
\capfig{0.4\textwidth}{figures/WorkEnergy/fddotproduct.png}{\label{fig:fddotproduct}When finding the dot product $Fd\cos\theta$, $\theta$ is the angle between the vectors when they are placed tail to tail.}

The work done by the force is the scalar product of the force vector and the displacement vector of the object. That is, the force ``does work'' if it is exerted while the object moves (has a displacement vector) and in such a way that the scalar product of the force and displacement vectors is non-zero. A force that is perpendicular to the displacement vector of an object does no work (since the scalar product of two perpendicular vectors is zero).  A force exerted in the same direction as the displacement will do positive work ($\cos\theta$ positive), and a force in the opposite direction of the displacement will do negative work $\cos\theta$ negative). As we will see, positive work corresponds to increasing the speed of the object, whereas negative work corresponds to decreasing its speed.

\begin{checkpoint}
\begin{MCquestion}
{A pendulum of length $R$ consists of a mass connected to a string (Figure \ref{fig:workenergy:pendulumtension}). The string exerts a force of tension $\vec F_T$ on the mass. What is the work done by tension when the pendulum swings through an angle $\theta$?
\capfig{0.2\textwidth}{figures/WorkEnergy/pendulumworktension.png}{\label{fig:workenergy:pendulumtension}A pendulum swings through an angle $\theta$.}} 
\item $W=F_TR\theta$
\item $W=F_TR(1-\cos\theta)$
\item Tension does no work on the pendulum. \correct
\end{MCquestion}
\end{checkpoint}

You may be tempted, as you will be when we introduce other quantities, to ask, ``Why work? Why not something else? Why that scalar product in particular? How could we possible have thought of that?''. In general, it seems arbitrary that we introduce this quantity (work) and then find that it leads to a convenient way of building models. We did not just pull this quantity ``out of thin air''! Many theorists, over many years, defined all sorts of quantities, and tried different ways to rephrase Newton's Theory, that were not useful. The ones that make it into the textbooks are those that turned out to be useful! You should thus accept that, when we present quantities, like work, that seem to come out of nowhere, there was in fact a lot of work done to evaluate different ways to do things, and the one we chose to present is the one that turned out to be most useful! So let's talk about work, and see why it's useful!


You should also keep in mind that, just like force, work is a ``made-up'' mathematical tool that we find to be useful in describing the world around us. There is no such thing as work or energy; they are just useful mathematical tools.

\subsection{Work in one dimension.}
Since work involves vectors, we first examine the concept in one dimension, before extending this to two and three dimensions. If we choose $x$ as that dimension, then all vectors only have an $x$ component, in one dimension. We can write a force vector as $\vec F=F\hat x$, where $F$ is the $x$ component of the force (which could be positive or negative). A displacement vector can be written as $\vec d = d \hat x$, where again, $d$ is the $x$ component of the displacement, and can be positive or negative. In one dimension, work is thus:
\begin{align*}
W = \vec F \cdot \vec d = (F\hat x) \cdot ( d\hat x ) = Fd (\hat x\cdot\hat x)=Fd
\end{align*}
where $\hat x \cdot \hat x = 1$. Consider, for example, the work done by a force, $\vec F$, on a box, as the box moves along the $x$ axis from position $x=x_0$ to position $x=x_1$, as shown in Figure \ref{fig:workenergy:work1d}.
\capfig{0.4\textwidth}{figures/WorkEnergy/work1d.png}{\label{fig:workenergy:work1d}A force, $\vec F$, exerted on an object as it moves from position $x=x_0$ to position $x=x_1$.}
We can write the length of the displacement vector as $||\vec d|| =d= \Delta x = x_1-x_0$. The work done by the force is given by:
\begin{align*}
W = \vec F \cdot \vec d = F\hat x\cdot \Delta x\hat x =F\Delta x =F(x_1-x_0) 
\end{align*}
which is a positive quantity, since $x_1 > x_0$, with our choice of coordinate system. 

\begin{checkpoint}
\begin{MCquestion}
{A constant force $\vec F$ acts on a box, as in Figure \ref{fig:workenergy:work1d}. Consider the work done by $\vec F$ as the box moves from $x_1$ to $x_0$. How does it compare to the work done by $\vec F$ when moving from $x_0$ to $x_1$ (that we calculated above)?}
\item $\vec F$ does no work on the box when it moves from $x_0$ to $x_1$. 
\item The work has the same magnitude as before, but the work is now negative. \correct
\item The work done by $\vec F$ is the same in both cases.
\end{MCquestion}
\end{checkpoint}

\subsection{Work in one dimension - varying force}
Suppose that instead of a constant force, $\vec F$, we have a force that changes with position, $\vec F(x)$, and can take on three different values between $x=x_0$ and $x=x_1$:
\begin{align*}
  \vec F (x)=
  \begin{cases}
    F_1\hat x & x<\Delta x \\
    F_2\hat x & \Delta x \leq x< 2\Delta x \\
    F_3\hat x & 2\Delta x \leq x
  \end{cases}
\end{align*}
as illustrated in Figure \ref{fig:workenergy:work1d}, which shows the force on an object as it moves from position $x=x_0$ to position $x=x_3$, along three (equal) displacement vectors, $\vec d_1=\vec d_2=\vec d_3=\Delta x \hat x$. 
\capfig{0.7\textwidth}{figures/WorkEnergy/work1dvarying.png}{\label{fig:workenergy:work1dvarying}A varying force, $\vec F(x)$, exerted on an object as it moves from position $x=x_0$ to position $x=x_3$.}
The total work done by the force over the three separate displacements is the sum of the work done over each displacement:
\begin{align*}
W^{tot}&=W_1+W_2+W_3\\
&=\vec F_1\cdot \vec d_2+\vec F_1\cdot \vec d_2+\vec F_3\cdot \vec d_3\\
&= F_1\Delta x +F_2\Delta x + F_3\Delta x
\end{align*} 

If instead of 3 segments we had $N$ segments and the $x$ component of the force had the $N$ corresponding values $F_i$ in the $N$ segments, the total work done by the force would be:
\begin{align*}
W^{tot} = \sum_{i=0}^N\vec F_i \cdot \Delta \vec x
\end{align*}
where we introduced a vector $\Delta \vec x$ to be the vector of length $\Delta x$ pointing in the positive $x$ direction. In the limit where $\vec F(x)$ changes continuously as a function of position, we take the limit of an infinite number of infinitely small segments of length $dx$, and the sum becomes an integral:
\begin{align}
\Aboxed{W^{tot} = \int_{x_0}^{x_f}\vec F(x) \cdot d\vec x}
\end{align}
where the work was calculated in going from $x=x_0$ to $x=x_f$, and $d\vec x=dx\hat x$ is an infinitely small displacement vector in the positive $x$ direction.

\begin{example}{\label{ex:workenergy:spring} A block is pressed against the free end of a horizontal spring with spring constant, $k$, so as to compress the spring by a distance $D$ relative to its rest length, as shown in Figure \ref{fig:workenergy:spring}. The other end of the spring is fixed to a wall. What is the work done by the spring force on the block in going from $x=x_0-D$ to $x=0$? What is the work done by the block on the spring over the same displacement?
\capfig{0.4\textwidth}{figures/WorkEnergy/spring.png}{\label{fig:workenergy:spring}A block is pressed against a horizontal spring so as to compress the spring by a distance $D$ relative to its rest length.}}
The force exerted by the spring on the block changes continuously with position, according to Hooke's law:
\begin{align*}
\vec F(x) = -kx \hat x
\end{align*}
and points in the positive $x$ direction when the end of the spring has a negative $x$ position (with our coordinate choice illustrated in Figure \ref{fig:workenergy:spring}, where the origin is located at the rest length of the spring). To calculate the work done by the force, we sum the work done by the force over many small displacements $d\vec x$:
\begin{align*}
W &= \int_{-D}^0 \vec F(x) \cdot d\vec x\\
&=\int_{-D}^0 (-kx \hat x) \cdot (dx \hat x)\\
&=\int_{-D}^0 -kxdx (\hat x \cdot \hat x)\\
&=-\int_{-D}^0 kx dx\\
&=-\left[\frac{1}{2}kx^2  \right]_{-D}^0\\
&=\frac{1}{2}kD^2
\end{align*}
In order to determine the work that was done by the block on the spring, we need to determine the force, $\pvec F'(x)$, exerted by the block on the spring. By Newton's Third Law, this is equal in magnitude but opposite in direction to the force exerted by the spring on the block:
\begin{align*}
\pvec F'(x) = -\vec F(x) = kx \hat x
\end{align*}
The work done by block on the spring over the same displacement is thus:
\begin{align*}
W' &= \int_{-D}^0 \pvec F'(x) \cdot d\vec x\\
&=\int_{-D}^0 (kx \hat x) \cdot (dx \hat x)\\
&=\int_{-D}^0 kx dx=-\frac{1}{2}kD^2\\
\end{align*}
which is negative. Indeed, the force exerted by the block onto the spring is in the direction opposite of the displacement, so the work will be negative. 
\end{example}

\subsection{Work in multiple dimensions}
First, consider the work done by a force $\vec F$ in pulling a crate over a displacement $\vec d$, in the case where the force is directed at an angle $\theta$ above the horizontal, as shown in Figure \ref{fig:workenergy:workangle}, and the displacement is along the $x$ axis (or rather, we chose the $x$ axis to be parallel to the displacement).
\capfig{0.4\textwidth}{figures/WorkEnergy/workangle.png}{\label{fig:workenergy:workangle}A force, $\vec F$, exerted on an object as it moves from position $x=x_0$ to position $x=x_1$.}
The work done by the force is given by:
\begin{align*}
W = \vec F \cdot \vec d &= Fd\cos\theta\\
&= F_{\parallel}d\\
&= Fd_{\parallel}\\
\end{align*}
where we highlighted the fact that the scalar product ``picks out'' components of vectors that are parallel to each other. $F_{\parallel} = F\cos\theta$ is thus the component of $\vec F$ that is parallel to $\vec d$, and $d_{\parallel}=d\cos\theta$ is the component of $\vec d$ that is parallel to $\vec F$. These are also shown in Figure \ref{fig:workenergy:workangle}.

\begin{checkpoint}
\begin{MCquestion}{
Brent and Dean pull two crates at the same angle above the horizontal and with the same force. The magnitude of the crates' displacement is the same, but Brent's crate moves horizontally on the ground and Dean's crate moves up a frictionless ramp. Who did more work on the crate?}
\item Brent
\item Dean \correct
\item They did the same amount of work.
\end{MCquestion}
\end{checkpoint}

In general, if an object is moving along an arbitrary path, we cannot choose the $x$ axis to be parallel to the displacement or to the force. If the path can be sub-divided into straight segments over which the force is constant, as in Figure \ref{fig:workenergy:workd2d}, we can calculate the work done by the force over each segment and add the work done in each segment together to obtain the total work done by the force. Note that, in general, the work done by a force as an object moves from one position to another depends on the particular path that was taken between the two positions.
\capfig{0.3\textwidth}{figures/WorkEnergy/work2d.png}{\label{fig:workenergy:work2d}A arbitrary two dimensional path of an object from $A$ to $B$ broken into three straight segments.}

\begin{example}{\label{ex:workenergy:workfriction}
Compare the work done by the force of kinetic friction in sliding a crate along a horizontal surface from position $A$ (coordinates $x_A, y_A$) to position $B$ (coordinates $x_B, y_B$) using the two different paths depicted in Figure \ref{fig:workenergy:workfriction}. Assume that the mass of the crate is $m$ and that the coefficient of kinetic friction between the crate and the ground is $\mu_k$.
\capfig{0.3\textwidth}{figures/WorkEnergy/workfriction.png}{\label{fig:workenergy:workfriction}Two possible paths to slide a crate from position $A$ to position $B$, as seen from above.}}
The force of kinetic friction is always in the direction opposite to that of motion. Thus, regardless of the path taken, the force of friction will do negative work. 

Let us first calculate the work done by the force of kinetic friction along the first path (the straight line). The force of kinetic friction will have a magnitude:
\begin{align*}
f_k = \mu_k N = \mu_k mg
\end{align*}
since the normal force will have the same magnitude as the weight because the crate is not moving (accelerating) in the direction perpendicular to the $xy$ plane.  The displacement vector from $A$ to $B$ can be written as:
\begin{align*}
\vec d &= (x_B-x_A)\hat x + (y_B-y_A) \hat y\\
\therefore ||\vec d|| &=d= \sqrt{(x_B-x_A)^2 - (y_B-y_A)^2}
\end{align*}  
The force of kinetic friction will be in the opposite direction of the displacement vector, and the angle between the two vectors will thus be $\SI{180}{\degree}$ ($\cos\theta=-1$). The work done by the force of kinetic friction is thus:
\begin{align*}
W = \vec f_k \cdot\vec d = f_k d \cos\theta = -\mu_k mg\sqrt{(x_B-x_A)^2 - (y_B-y_A)^2}
\end{align*}
and is negative, as expected.

For path 2, we break up the motion into two segments, with displacements vectors $\vec d_1$ (along $y$) and $\vec d_2$ (along $x$). We can write the two displacement vectors as:
\begin{align*}
\vec d_1 &= 0\hat x + (y_A-y_B) \hat y\\
\therefore ||\vec d_1||&=d_1=(y_A-y_B)\\
\vec d_2 &= (x_A-x_B)\hat x + 0 \hat y\\
\therefore ||\vec d_2||&=d_2=(x_A-x_B)\\
\end{align*}

Along each segment, the force of kinetic friction is anti-parallel to the displacement (note that the force of friction changes direction over the two segments), but the magnitude is $f_k=\mu_kmg$. The work done along the first segment is thus:
\begin{align*}
W_1 = \vec f_k \cdot \vec d_1 = f_k d_1 \cos\theta = -\mu_k mg(y_A-y_B)
\end{align*}
The work done along the second segment is:
\begin{align*}
W_2 = \vec f_k \cdot \vec d_2 = f_k d_2 \cos\theta = -\mu_k mg(x_A-x_B)
\end{align*}
And the total work done by the force of kinetic friction over the second path is:
\begin{align*}
W^{tot} = W_1 + W_2 = -\mu_k mg \left((x_A-x_B) + (y_A-y_B)\right)
\end{align*}
which is more work than was done along path 1. This makes sense because for both paths, the force of friction has the same magnitude and is always in the opposite direction of motion; thus, the longer the path, the more work will be done by the force.
\end{example}

\begin{example}{\label{ex:workenergy:workgravity}
A box of mass $m$ is moved from the floor onto a table using two different paths, as shown in Figure \ref{fig:workenergy:workgravity}. The table is a horizontal distance $L$ away from where the box starts and a height $H$ above the floor. Compare the work done by the weight of the box along the two possible paths.\capfig{0.5\textwidth}{figures/WorkEnergy/workgravity.png}{\label{fig:workenergy:workgravity}Two possible paths to move a box from the floor onto a table.}}
We can use a coordinate system such that the origin coincides with the initial position of the box. $x$ is horizontal and $y$ is vertical, as shown in Figure \ref{fig:workenergy:workgravity}. The weight of the box can be written as:
\begin{align*}
\vec F_g = -mg \hat y
\end{align*}
and points in the negative $y$ direction with a magnitude of $mg$. To calculate the work done by the weight along the first path, we first determine the corresponding displacement vector, $\vec d$:
\begin{align*}
\vec d = L\hat x + H\hat y
\end{align*}
and we can then determine the work:
\begin{align*}
W &= \vec F_g \cdot \vec d = (-mg \hat y) \cdot (L\hat x + H\hat y)\\
&=(F_xd_x+F_yd_y)=( (0)(L) + (-mg)(H))\\
&= -mgH
\end{align*}
Along path 1, the work from the weight is negative, and does not depend on the horizontal distance $L$. Let us now calculate the work done along the second path, which we break up into two segments with displacement vectors $\vec d_1$ (vertical) and $\vec d_2$ (horizontal). The displacement vectors are:
\begin{align*}
\vec d_1 &= H\hat y\\
\vec d_2 &= L\hat x
\end{align*}
The work done along the vertical segment is:
\begin{align*}
W_1 &= \vec F_g \cdot \vec d_1 = (-mg \hat y) \cdot (H\hat y)\\
&=-mgH
\end{align*}
The work done along the horizontal segment is:
\begin{align*}
W_2 &= \vec F_g \cdot \vec d_2 = (-mg \hat y) \cdot (L\hat x)\\
&=0
\end{align*}
which is zero, because the force of gravity is always vertical and thus perpendicular to the displacement vector of the horizontal segment. The total work done by the weight along the second path is:
\begin{align*}
W^{tot} = W_1 + W_2 = -mgH
\end{align*}
which is the same as the work done along path 1. As we will see, when a force is constant in magnitude and direction, the work that it does on an object in going from one position to another is independent of the path taken. This was not the case in Example \ref{ex:workenergy:workfriction}, because the direction of the force of kinetic friction depends on the direction of the displacement. 
\end{example} 

\begin{checkpoint}
Clare and Amelia go down two different slides, shown in Figure \ref{fig:workenergy:slidecheckpoint}. Clare and Amelia have the same mass and the slides have the same non-zero coefficients of friction. 
\capfig{0.5\textwidth}{figures/WorkEnergy/slidecheckpoint.png}{\label{fig:workenergy:slidecheckpoint} Clare ($C$) and Amelia ($A$) go down two different slides of the same height.}
For each of the following forces, decide whether the force: does more work on Clare, does more work on Amelia, or does the same amount of work on both.
\begin{enumerate}
\item The force of gravity... 
\item The force of friction... 
\item The normal force... 
\end{enumerate}
\begin{answer}
Gravity does the same amount of work on both, friction does more work on Amelia, and the normal force does the same amount of work on both (the normal force does zero work). 
\end{answer}
\end{checkpoint}

The most general case in which we can calculate the work done by a force is the case when the force changes continuously along a path where the displacement also changes direction continuously. This is illustrated in Figure \ref{fig:workenergy:workgeneral} which shows a force, $\vec F(\vec r)$, that depends on position ($\vec r$), and an arbitrary path between two points $A$ and $B$. In general, the work done by the force on an object that goes from $A$ to $B$ will depend on the actual path that was taken.
\capfig{0.5\textwidth}{figures/WorkEnergy/workgeneral.png}{\label{fig:workenergy:workgeneral}An arbitrary path between two points $A$ and $B$ with a force that depends on position, $\vec F(\vec r)$. }
The strategy for calculating the work in the general case is the same: we break up the path into small straight segments with displacement vectors $d\vec l$ (Figure \ref{fig:workenergy:dldiagram}) where we assume that the force is constant over the segment. The total work is then sum of the work over each segment:
\begin{align}
\Aboxed{W = \int_A^B \vec F(\vec r) \cdot d\vec l}
\end{align}
As usual, we use the integral symbol to indicate that you need to take an infinite number of infinitely small segments $d\vec l$ in order to calculate the sum.
\capfig{0.3\textwidth}{figures/WorkEnergy/elementoflengthdl.png}{\label{fig:workenergy:dldiagram}We divide the path into infinitesimally small segments with displacement vectors $d\vec l$.}
You should note that this is not an integral like any other that we have seen so far: the integral is not over a single integration variable (usually we use $x$), but it is the integral (the sum!) over the specific path that we have chosen in going from $A$ to $B$. This is called a ``path integral'', and is generally difficult to evaluate. 

\newpage
\begin{example}{
A force, $\vec F(\vec r) = \vec F(x,y) = F_x\hat x + F_y \hat y$, is exerted on an object. The object starts at position $A$ and ends at position $B$, along a parabolic path, $y(x) = a+bx^2$, as depicted in Figure \ref{fig:workenergy:workparabola}. What is the work done by the force, $\vec F$, along this trajectory?\capfig{0.4\textwidth}{figures/WorkEnergy/workparabola.png}{\label{fig:workenergy:workparabola}A parabolic path between $A$ and $B$. }}
In this case, the force is constant in magnitude and direction, but because the path is curved, the displacement vector changes direction continuously. When we break up the path into small segments $d\vec l$, we need to incorporate the equation of the parabola to include the fact that $d\vec l$ must always be tangent to the parabola. Consider one small segment along the trajectory and the infinitesimal displacement vector $d\vec l$ at that point, as in Figure \ref{fig:workenergy:workparabola_dr}.
\capfig{0.2\textwidth}{figures/WorkEnergy/workparabola_dr.png}{\label{fig:workenergy:workparabola_dr}The infinitesimal displacement vector, $d\vec l$. }

We can write the $x$ and $y$ components of the vector as infinitesimal distances, $dx$ and $dy$, along the $x$ and $y$ axes, respectively. The vector $d\vec l$ can thus be written:
\begin{align*}
d\vec l = dx \hat x + dy \hat y
\end{align*}
The total work done by the force is then:
\begin{align*}
W &= \int_A^B \vec F(\vec r) \cdot d\vec l\\
&=\int_A^B (F_x\hat x + F_y \hat y) \cdot (dx \hat x + dy \hat y)\\
&=\int_A^B (F_x dx + F_ydy)\\
\therefore W&= \int_A^B F_x dx + \int_A^B F_ydy
\end{align*}
where in the last line, we simply used the property that the integral of a sum is the sum of the corresponding integrals. At this point, we have two integrals over integration variables ($x$ and $y$) that are meaningful. However, we have not yet used the fact that our path is a parabola, and in general, we expect that the shape of the path is important. By saying that we are integrating (or calculating the work) over a specific path, we are really saying that $x$ and $y$ are not independent; that is, if we know the value of $x$ at some point on the path, we know the corresponding value of $y$ ($y = a+bx^2$). 

Since $x$ and $y$ are not independent, we can use a ``substitution of variables'' in order to express $y$ in terms of $x$, and $dy$ in terms of $dx$:
\begin{align*}
y(x) &= a + bx^2\\
\frac{dy}{dx} &= 2bx\\
\therefore dy &= 2bxdx
\end{align*} 
This allows us to convert the integral over $y$ to an integral over $x$, which also allows us to be explicit for the limits of the integral (in our example, the integral goes from $x=0$ to $x=x_1$):
\begin{align*}
W&= \int_A^B F_x dx + \int_A^B F_ydy\\
&=\int_0^{x_1} F_x dx + \int_0^{x_1} F_y(2bxdx)\\
&=\int_0^{x_1} (F_x + 2bxF_y)dx
\end{align*}
where we would need to know how $F_x$ and $F_y$ depends on $x$ and $y$ in order to actually evaluate the integral.

For example, if the force were constant ($F_x$ and $F_y$ constant), then the work done along the parabolic path would be:
\begin{align*}
W &= \int_0^{x_1} (F_x + 2bxF_y)dx\\
&=\left[F_x x + bF_yx^2  \right]_0^{x_1}\\
&=F_x x_0 + bF_yx_0^2
\end{align*}
As we mentioned earlier, \textbf{if the force is constant in magnitude and direction}, then the work done is independent of path. We can easily check this, using the displacement vector $\vec d = x_1\hat x + bx_1^2 \hat y$:
\begin{align*}
W &= \vec F \cdot \vec d = (F_x\hat x+ F_y\hat y) \cdot (x_1\hat x + bx_1^2 \hat y)\\
&=F_x x_1 + bF_yx_1^2
\end{align*}
as we found above. 
\end{example}

\subsection{Net work done}
So far, we have considered the work done on an object by a single force. If more than one force is exerted on an object, then each force can do work on the object, and we can calculate the ``net work'' done on the object by adding together the work done by each force. We will show that this is equivalent to first calculating the net force on the object, $F^{net}$ (i.e. the vector sum of the forces on the object), and then calculating the work done by the net force.
 
Suppose that three forces, $\vec F_1$, $\vec F_2$, and $\vec F_3$ are exerted on an object as it moves such that its displacement vector is $\vec d$. The net work done on the object is:
\begin{align*}
W^{net} &= W_1 + W_2 + W_3 \\
&= \vec F_1 \cdot \vec d + \vec F_2 \cdot \vec d  + \vec F_3 \cdot \vec d \\
&=(F_{1x}d_x+F_{1y}d_y+F_{1z}d_z)+ (F_{2x}d_x+F_{2y}d_y+F_{2z}d_z) + (F_{3x}d_x+F_{3y}d_y+F_{3z}d_z)\\
&=(F_{1x} + F_{2x} + F_{3x})d_x+(F_{1y} + F_{2y} + F_{3y})d_y+(F_{1z} + F_{2z} + F_{3z})d_z\\
&=\vec F^{net} \cdot \vec d
\end{align*}
where $\vec F^{net} = \vec F_1 + \vec F_2 + \vec F_3$. The result is easily generalized to any number of forces, including if those forces change as a function of position:
\begin{align*}
W^{net} = \int_A^B F^{net}(\vec r) \cdot d\vec l
\end{align*} 

\begin{example}{\label{ex:workenergy:networkramp}
You push with an unknown horizontal force, $\vec F$, against a crate of mass $m$ that is located on an inclined plane that makes an angle $\theta$ with respect to the horizontal, as shown in Figure \ref{fig:workenergy:workincline}. The coefficient of kinetic friction between the crate and the incline is $\mu_k$. You push in such a way that that crates moves at a constant speed up the incline. What is the net work done on the crate if it moves up the incline by a distance $d$?\capfig{0.3\textwidth}{figures/WorkEnergy/workincline.png}{\label{fig:workenergy:workincline} A crate being pushed up an incline.}}
Although the answer may be obvious, let's go the long way about it and calculate the work done by each force, and then sum them together to get the total work done. We start by identifying the forces exerted on the crate:
\begin{enumerate}
\item $\vec F$, the applied force, of unknown magnitude, $\vec F$.
\item $\vec F_g$, the weight of the crate, with magnitude $mg$. 
\item $\vec N$, a normal force exerted by the incline.
\item $\vec f_k$, a force of kinetic friction, with magnitude $\mu_k N$, that points in the direction opposite of $\vec d$. 
\end{enumerate}
These are shown in the free-body diagram in Figure \ref{fig:workenergy:workincline_fbd}, along with our choice of coordinate system, and the displacement vector. 
\capfig{0.25\textwidth}{figures/WorkEnergy/workincline_fbd.png}{\label{fig:workenergy:workincline_fbd} Free-body diagram for the crate on the incline.}
With our choice of coordinate system, the displacement vector is given by:
\begin{align*}
\vec d = d (\cos\theta \hat x + \sin\theta \hat y)
\end{align*}
Before calculating the work done by each force, we need to determine the magnitude of the normal force (and thus of the force of kinetic friction). Since the crate is moving at constant velocity, its acceleration is zero, so the sum of the forces must be zero. Writing out the $y$ component of Newton's Second Law allows us to find the magnitude of the normal force:
\begin{align*}
\sum F_y &= N\cos\theta -F_g - f_k\sin\theta = 0\\
\therefore mg &= N\cos\theta-\mu_kN\sin\theta = N(\cos\theta-\mu_k\sin\theta)\\
\therefore N &= \frac{mg}{\cos\theta-\mu_k\sin\theta}
\end{align*}
Writing out the $x$ component of Newton's Second Law allows us to find the magnitude of the unknown force $F$:
\begin{align*}
\sum F_x &= F - N\sin\theta - f_k\cos\theta = 0\\
\therefore F &= N\sin\theta+\mu_kN\cos\theta = N(\sin\theta+\mu_k\cos\theta)\\
&=mg\frac{\sin\theta+\mu_k\cos\theta}{\cos\theta-\mu_k\sin\theta}
\end{align*}
We now proceed to calculate the work done by each force. The work done by the normal force is identically zero, since it is perpendicular to the displacement vector. The work done by the applied force, $\vec F = F\hat x$, is:
\begin{align*}
W_F &= \vec F \cdot \vec d = (F\hat x)\cdot(d (\cos\theta \hat x + \sin\theta \hat y))\\
&=Fd\cos\theta=mg\frac{\sin\theta+\mu_k\cos\theta}{\cos\theta-\mu_k\sin\theta}d\cos\theta
\end{align*}
The work done by the force of gravity, $\vec F_g = -mg \hat y$, is:
\begin{align*}
W_g &= \vec F_g \cdot \vec d = (-mg \hat y)\cdot(d (\cos\theta \hat x + \sin\theta \hat y))\\
&=-mgd\sin\theta
\end{align*}
The work done by the force of friction, $\vec f_k$, noting that $\vec f_k$ and $\vec d$ are antiparallel:
\begin{align*}
W_f &= \vec f_k \cdot \vec d = -f_kd = -\mu_kNd\\
&=-\mu_k\frac{mg}{\cos\theta-\mu_k\sin\theta}d
\end{align*}
The net work done on the crate is thus:
\begin{align*}
W^{net}&=W_F + W_g + W_f\\
&=mg\frac{\sin\theta+\mu_k\cos\theta}{\cos\theta-\mu_k\sin\theta}d\cos\theta-mgd\sin\theta -\mu_k\frac{mg}{\cos\theta-\mu_k\sin\theta}d\\
&=mgd \left(  \frac{\sin\theta+\mu_k\cos\theta}{\cos\theta-\mu_k\sin\theta}\cos\theta - \sin\theta - \mu_k\frac{1}{\cos\theta-\mu_k\sin\theta} \right)\\
&=mgd \left(  \frac{(\sin\theta+\mu_k\cos\theta)\cos\theta - \sin\theta(\cos\theta-\mu_k\sin\theta) - \mu_k}{\cos\theta-\mu_k\sin\theta} \right)\\
&=mgd \left(  \frac{\sin\theta\cos\theta+\mu_k\cos^2\theta - \sin\theta\cos\theta+\mu_k\sin^2\theta - \mu_k}{\cos\theta-\mu_k\sin\theta} \right)\\
&=mgd \left(  \frac{\mu_k(\cos^2\theta+\sin^2\theta) - \mu_k}{\cos\theta-\mu_k\sin\theta} \right)\\
&=0
\end{align*}
where we used the fact that $\cos^2\theta+\sin^2\theta=1$. Thus we find that the net work done on the crate is zero!

\textbf{Discussion:} Of course, this makes sense, because the net force on the crate is zero, since it is not accelerating, so the net work done is also zero. As a consequence, or rather, by construction, we have the condition that if the net work done on an object is zero, then that object does not accelerate. We thus have scalar quantity (work) that can tell us something about whether an object is changing speed. In the next section, we introduce a new quantity, ``kinetic energy'', to describe how an object's speed changes when the work done is not zero.
\end{example}

\newpage
\begin{studentOpinion}{Olivia} Pay close attention to the words ``on'' and ``by.'' There are a few things about this that can be tricky:
\begin{enumerate}
\item In Example \ref{ex:workenergy:networkramp}, we were asked to find the \textbf{net work} done \textbf{on} the crate. Sometimes, the question won't specify that it wants you to find the net work, and will just say ``What is the work done \textbf{on} the crate?'' When you are just asked for the work done ``on'' an object, the question is implicitly asking for the net work done on the object. \\

\item Just because the net work done \textbf{on} an object is zero doesn't mean that the work done \textbf{by} each of the forces is zero. This may seem obvious, but it's easy to get tripped up on a test or exam. If you are reading a question about work and it says that the object is moving at a constant speed, it's tempting to just jump ahead and say that the work must be equal to zero. However, you can only say this if it's asking you for the net work done on the object. For instance, in example \ref{ex:workenergy:networkramp}, we concluded that since the crate was moving at a constant speed, the net work was equal to zero. But if the question asked you to find the work done on the crate \textbf{by gravity}, that would mean something different. The work done \textbf{by gravity} in this case is not equal to zero (it's actually negative). \\

\item The work done ``on'' an object is not the same as the net work done ``by'' that object. For example, say you are in a tug-of-war and you pull the other team towards you, but you yourself do not move. The net work done \textbf{on} you is zero, but the work done \textbf{by} you is not zero. So, when you are talking about work, you should always state explicitly whether the work is being done ``on'' the object or ``by'' the object. 
\end{enumerate}
\textbf{Note}: The wording won't always be like this - sometimes it will say ``How much work do you do on the box?'' instead of ``How much work is done \textbf{by} you on the box,'' so always be careful. Still, looking for key words like ``by'' and ``on'' is a good place to start. 
\end{studentOpinion}

\begin{checkpoint}
\begin{MCquestion}{A \SI{2}{kg} box sits on a horizontal surface. A constant horizontal force of \SI{6}{N} is applied to the box. The box moves with a constant acceleration of \SI{2}{m/s^2}. Which of the following has the greatest magnitude?}
\item The work done by the applied force. \correct
\item The work done by friction.
\item The net work done on the box.
\end{MCquestion}
\end{checkpoint}

\section{Kinetic energy and the work energy theorem}
At this point, you should be comfortable calculating the net work done on an object upon which several forces are exerted. As we saw in the previous section, the net work done on an object is connected to the object's acceleration; if the net force on the object is zero, then the net work done and acceleration are also zero. In this section, we derive a new quantity, kinetic energy, which allows us to connect the work done on an object with its change in speed. Like the derivation for work, the following derivation appears to ``come out of thin air''. Remember, though, that theorists have tried all sorts of mathematical tricks to reformulate Newton's Theory, and this is the one that worked.

Consider the most general case of an object of mass $m$ acted upon by a net force, $\vec F^{net}(\vec r)$, which can vary in magnitude and direction. We wish to calculate the  net work done on the object as it moves along an arbitrary path between two points, $A$ and $B$, in space, as shown in Figure \ref{fig:workenergy:kepath}. The instantaneous acceleration of the object, $\vec a$, is shown along with an ``element of the path'', $d\vec l$. 
\capfig{0.4\textwidth}{figures/WorkEnergy/kepath.png}{\label{fig:workenergy:kepath} An object moving along an arbitrary path between points $A$ and $B$ that is acted upon by a net force $\vec F^{net}$.}
The net work done on the object can be written:
\begin{align*}
W^{net} = \int_A^B F^{net}(\vec r) \cdot d\vec l
\end{align*} 
and is in general a difficult integral to evaluate for an arbitrary path. Our goal is to find a way to evaluate this integral by finding a function, $K$, with the property that:
\begin{align*}
\int_A^B F^{net}(\vec r) \cdot d\vec l =K_B - K_A
\end{align*}
That is, we will only have to evaluate $K$ at the end points of the path in order to determine the value of the integral. In this way, the function $K$ is akin to an anti-derivative.  

In order to determine the form for the function $K$, we start by noting that, by using Newton's Second Law, we can write the integral for work in terms of the acceleration of the object:
\begin{align*}
\sum \vec F &= \vec F^{net} = m\vec a\\
\therefore \int_A^B F^{net}(\vec r) \cdot d\vec l &= \int_A^B m\vec a\cdot d\vec l =m\int_A^B \vec a\cdot d\vec l
\end{align*}
where we assumed that the mass of the object does not change along the path and can thus be factored out of the integral. Consider the scalar product of the acceleration, $\vec a$, and the path element, $d\vec l=dx\hat x  +dy\hat y + dz\hat z$, written in terms of the velocity vector:
\begin{align*}
\vec a & = \frac{d\vec v}{dt}\\
\therefore \vec a\cdot d\vec l &= \frac{d\vec v}{dt}\cdot d\vec l\\
&=\left(\frac{dv_x}{dt}\hat x+ \frac{dv_y}{dt}\hat y + \frac{dv_z}{dt}\hat z\right) \cdot (dx\hat x  +dy\hat y + dz\hat z)\\
&=\frac{dv_x}{dt}dx+\frac{dv_y}{dt}dy+\frac{dv_z}{dt}dz
\end{align*}
Any of the terms in the sum can be re-arranged so that the time derivative acts on the element of path ($dx$, $dy$, or $dz$) instead of the velocity, for example:
\begin{align*}
\frac{dv_x}{dt}dx = \frac{dx}{dt}dv_x
\end{align*}
where we recognize that $\frac{dx}{dt} = v_x$. We can thus write the scalar product between the acceleration vector and the path element as:
\begin{align*}
\vec a\cdot d\vec l&= \frac{dv_x}{dt}dx+\frac{dv_y}{dt}dy+\frac{dv_z}{dt}dz\\
&=\frac{dx}{dt}dv_x + \frac{dy}{dt}dv_y+\frac{dz}{dt}dv_z
\end{align*}
We can thus write the integral for the net work done as:
\begin{align*}
W^{net} &= \int_A^B F^{net}(\vec r) \cdot d\vec l =m \int_A^B (v_xdv_x + v_ydv_y + v_zdv_z)\\
&=m\int_A^B v_xdv_x +m\int_A^B  v_ydv_y + m\int_A^B v_zdv_z
\end{align*}
which corresponds to the sum of three integrals over the three independent components of the velocity vector. The components of the velocity vector are functions that change over the path and have fixed values at either end of the path. Let the velocity vector of the object at point $A$ be $\vec v_A=(v_{Ax}, v_{Ay}, v_{Az})$ and the velocity vector at point $B$ be $\vec v_B=(v_{Bx}, v_{By}, v_{Bz}$. The integral over, say, the $x$ component of velocity is then:
\begin{align*}
m\int_A^B v_xdv_x &= m\int_{v_{Ax}}^{v_{Bx}} v_xdv_x= m\left[\frac{1}{2}v_x^2  \right]_{v_{Ax}}^{v_{Bx}}\\
&=\frac{1}{2}m(v_{Bx}^2-v_{Ax}^2)
\end{align*} 
We can thus write the net work integral as:
\begin{align*}
W^{net} &=m\int_A^B v_xdv_x +m\int_A^B  v_ydv_y + m\int_A^B v_zdv_z\\
&=\frac{1}{2}m(v_{Bx}^2-v_{Ax}^2) + \frac{1}{2}m(v_{By}^2-v_{Ay}^2) +\frac{1}{2}m(v_{Bz}^2-v_{Az}^2)\\
&=\frac{1}{2}m(v_{Bx}^2+v_{By}^2+v_{Bz}^2)-\frac{1}{2}m(v_{Ax}^2+v_{Ay}^2+v_{Az}^2)\\
&=\frac{1}{2}mv_B^2 - \frac{1}{2}mv_A^2
\end{align*}
where we recognized that the magnitude (squared) of the velocity is given by $v_A^2 = v_{Ax}^2+v_{Ay}^2+v_{Az}^2$. We have thus arrived out our desired result, namely, we have found a function of speed, $K(v)$, that when evaluated at the endpoints of the path allows us to calculate the net work done on the object:
\begin{align}
\Aboxed{K(v) = \frac{1}{2}mv^2}
\end{align}
That is, if you know the speed at the start of the path, $v_A$, and the speed at the end of the path, $v_B$, then the net work done on the object is given by:
\begin{align}
\Aboxed{W^{net} = \Delta K = K(v_B) - K(v_a)}
\end{align}
We call $K(v)$ the ``kinetic energy'' of the object. We can say that the net work done on an object in going from $A$ to $B$ is equal to its change in kinetic energy (final kinetic energy minus initial kinetic energy). It is important to note that we defined kinetic energy in a way that it is equal to the net work done. You may have already heard of kinetic energy from past introductions to physics as a quantity that is just given; here, we instead derived a function that has the desired property of being equal to the net work done and called it ``kinetic energy''. 

The relation between the net work done and the change in kinetic energy is called the ``Work-Energy Theorem'' (or Work-Energy Principle). It is the connection that we were looking for between the dynamics (the forces from which we calculate work) and the kinematics (the change in kinetic energy). Unlike Newton's Second Law, which relates two vector quantities (the vector sum of the forces and the acceleration vector), the Work-Energy Theorem relates two scalar quantities to each other (work and kinetic energy). Although we introduced the kinetic energy as a way to calculate the integral for the net work, if you know the value of the net work done on an object, then the Work-Energy Theorem can be used to calculate the change in speed of the object.

Most importantly, the Work-Energy theorem introduces the concept of ``energy''. As we will see in later chapters, there are other forms of energy in addition to work and kinetic energy. The Work-Energy Theorem is the starting point for the idea that you can convert one form of energy into another. The Work-Energy Theorem tells us how a force, by doing work, can provide kinetic energy to an object or remove kinetic energy from an object.  

\begin{example}{A net work of $W$ was done on an object of mass $m$ that started at rest. What is the speed of the object after the work has been done on the object?}
Using the Work-Energy Theorem:
\begin{align*}
W = \frac{1}{2}mv_f^2 - \frac{1}{2}mv_i^2
\end{align*}
where $v_i$ is the initial speed of the object and $v_f$ is its final speed. Since the initial speed is zero, we can easily find the final speed:
\begin{align*}
v_f = \sqrt{\frac{2W}{m}}
\end{align*}
\end{example}


\begin{example}{A block is pressed against the free end of a horizontal spring with spring constant, $k$, so as to compress the spring by a distance $D$ relative to its rest length, as shown in Figure \ref{fig:workenergy:spring2}. The other end of the spring is fixed to a wall. 
\capfig{0.4\textwidth}{figures/WorkEnergy/spring.png}{\label{fig:workenergy:spring2}A block is pressed against a horizontal spring so as to compress the spring by a distance $D$ relative to its rest length.}
If the block is released from rest and there is no friction between the block and the horizontal surface, what is the speed of the block when it leaves the spring?
}

This is the same problem that we presented in Chapter \ref{chap:ApplyingNewtonsLaws} in Example \ref{ex:applyingnewtonslaws:blockspring}, where we solved a differential equation to find the speed. 

Our first step is to calculate the net work done on the object in going from $x=-D$ to $x=0$ (which corresponds to when the object leaves the spring, as discussed in Example \ref{ex:applyingnewtonslaws:blockspring}. The forces on the object are:
\begin{enumerate}
\item $\vec F_g$, its weight, with magnitude $mg$.
\item $\vec N$, the normal force exerted by the ground.
\item $\vec F(x)$, the force from the spring, with magnitude $kx$. 
\end{enumerate}
Both the normal force and weight are perpendicular to the displacement, so they will do no work. The net work done is thus the work done by the spring, which we calculated in Example \ref{ex:workenergy:spring} to be:
\begin{align*}
W^{net} = W_F = \frac{1}{2}kD^2
\end{align*}
By the Work-Energy Theorem, this is equal to the change in kinetic energy. Noting that the object started at rest ($v_i=0$), the final speed $v_f$ is found to be:
\begin{align*}
W^{net} &=  \frac{1}{2}mv_f^2 - \frac{1}{2}mv_i^2 =  \frac{1}{2}mv_f^2 - 0\\
\frac{1}{2}kD^2 &=\frac{1}{2}mv_f^2\\
\therefore v_f &=\sqrt{\frac{kD^2}{m}}
\end{align*}
\end{example}

\begin{example}{A block is pressed against the free end of a horizontal spring with spring constant, $k$, so as to compress the spring by a distance $D$ relative to its rest length, as shown in Figure \ref{fig:workenergy:spring3}. The other end of the spring is fixed to a wall. 
\capfig{0.4\textwidth}{figures/WorkEnergy/spring.png}{\label{fig:workenergy:spring3}A block is pressed against a horizontal spring so as to compress the spring by a distance $D$ relative to its rest length.}
If the block is released from rest and the coefficient of kinetic friction between the block and the horizontal surface is $\mu_k$, what is the speed of the block when it leaves the spring?} This is the same example as the previous one, but with kinetic friction. The forces on the block are:
\begin{enumerate}
\item $\vec F_g$, its weight, with magnitude $mg$.
\item $\vec N$, the normal force exerted by the ground on the block.
\item $\vec F(x)$, the force from the spring, with magnitude $kx$. 
\item $\vec f_k$, the force of kinetic friction, with magnitude $\mu_kN$.
\end{enumerate}
Both the normal force and weight are perpendicular to the displacement, so they will do no work. Furthermore, since the acceleration in the vertical direction is zero, the normal force will have the same magnitude as the weight ($N=mg$). The magnitude of the force of kinetic friction is thus $f_k = \mu_k mg$. The net work done will be the sum of the work done by the spring, $W_F$, and the work done by friction, $W_f$:
\begin{align*}
W^{net} = W_F + W_f
\end{align*}
We have already determined the work done by the spring:
\begin{align*}
W_F = \frac{1}{2}kD^2
\end{align*}
The work done by the force of kinetic friction will be negative (since it is in the direction opposite of the motion) and is given by:
\begin{align*}
W_f = \vec f_k \cdot \vec d = -f_kD = -\mu_kmgD
\end{align*}
Applying the work energy theorem, and noting that the block started at rest ($v_i=0$), the final speed $v_f$ is found to be:
\begin{align*}
W^{net} =W_F + W_f&= \frac{1}{2}mv_f^2 - \frac{1}{2}mv_i^2 \\
\frac{1}{2}kD^2-\mu_kmgD  &=\frac{1}{2}mv_f^2\\
\therefore v_f &=\sqrt{\frac{kD^2}{m}-2\mu_kgD}
\end{align*}
\textbf{Discussion:} We can think of this in terms of the concept of energy. The spring does positive work on the block, and so it increases its kinetic energy. Friction does negative work on the block, decreasing its kinetic energy. Only the spring is ``introducing'' energy into the block, as friction is removing that energy by doing negative work. Another way to think about it is that the spring is inputting energy; some of that energy goes into increasing the kinetic energy of the block, and some of it is lost by friction. The energy that is lost to friction, can be thought of as ``thermal energy'' (heat) that goes up into heating the block and the surface. Indeed, if you rub your hand against the table, you will notice that it gets warmer; you are losing some of the energy introduced to your hand by the work done by your arm into heating up the table and your hand! 
\end{example}

\section{Power}
We finish the chapter by introducing the concept of ``power'', which is the rate at which work is done on an object, or more generally, the rate at which energy is being converted from one form to another. If an amount of work, $\Delta W$, was done in a period of time $\Delta t$, then the work was done at a rate of:
\begin{align}
\Aboxed{P = \frac{\Delta W}{\Delta t}}
\end{align}
where $P$ is called the power. The SI unit for power is the ``Watt'', abbreviated $\si{W}$, which corresponds to $\si{J/s}=\si{kg m^2/s^3}$ in base SI units. If the rate at which work is being done changes with time, then the instantaneous power is defined as:
\begin{align}
\Aboxed{P = \frac{dW}{dt}}
\end{align}
You have probably already encountered power in your everyday life. For example, your $\SI{1000}{W}$ hair dryer consumes ``electrical energy'' at a rate of $\SI{1000}{J}$ per second and converts it into the kinetic energy of the fan as well as the thermal energy to heat up the air. Horsepower ($\si{hp}$) is an imperial unit of power that is often used for vehicles, the conversion being $\SI{1}{hp} = \SI{746}{W}$. A $\SI{100}{hp}$ car thus has an engine that consumes the chemical energy released by burning gasoline at a rate of $\SI{7.46e4}{J}$ per second and converts it into work done on the car as well as into heat. 

\begin{checkpoint}
\begin{MCquestion}
{Two cranes lift two identical boxes off the ground. One crane is twice as powerful as the other. Both cranes do the same amount of work on the boxes. Which of the following statements is true of the boxes, once the cranes have done work on them?}
\item One box has been lifted twice as high as the other.
\item The boxes are lifted the same height in the same amount of time.
\item The boxes are lifted the same height, but it takes one of the boxes twice as long to get there. \correct
\item One box is lifted twice as high as the other, but it takes twice as long to get there. 
\end{MCquestion}
\end{checkpoint}
\newpage
\begin{example}{If a car engine can do work on the car with a power of $P$, what will be the speed of the car at some time $t$ if the car was at rest at time $t=0$?}
First, we need to calculate how much total work was done on the car:
\begin{align*}
W = P t
\end{align*}
Then, using the Work-Energy Theorem, we can find the speed of the car at some time $t$:
\begin{align*}
W &= \frac{1}{2}mv_f^2 - \frac{1}{2}mv_i^2\\
Pt &= \frac{1}{2}mv_f^2 \\
\therefore v_f &= \sqrt{\frac{2Pt}{m}}
\end{align*}
\textbf{Discussion:} The model for the final speed of the car makes sense because:
\begin{itemize}
\item The dimension of the expression for $v_f$ is speed (you should check this!).
\item The speed is greater if either the time or power are greater (so the speed is larger if more work is done on the car).
\item The speed is smaller if the mass of the car is greater (the acceleration of the car will be less if the mass of the car is larger).
\end{itemize}
\end{example}

\begin{example}{\label{ex:workenergy:powerconstantv}You are pushing a crate along a horizontal surface at constant speed, $v$. You find that you need to exert a force of $\vec F$ on the crate in order to overcome the friction between the crate and the ground. How much power are you expending by pushing on the crate?}
We need to calculate the rate at which the force, $\vec F$, that you exert on the crate does work. If the crate is moving at constant speed, $v$, then in a time $\Delta t$, it will cover a distance, $d=v\Delta t$. Since you exert a force in the same direction as the motion of the crate, the work done over that distance $d$ is:
\begin{align*}
\Delta W = \vec F \cdot \vec d = Fd\cos(0) = Fv\Delta t
\end{align*}
The power corresponding to the work done in that period of time is thus:
\begin{align*}
P = \frac{\Delta W}{\Delta t} = Fv
\end{align*}
This is quite a general result for the rate at which a force does work when it is exerted on an object moving at constant speed. 
\end{example}

\begin{studentOpinion}{Olivia} Example \ref{ex:workenergy:powerconstantv} ties into what I brought up earlier. If you think to yourself: ``The velocity is constant, so the work must be zero'', the formula,
\begin{align*}
P = \frac{\Delta W}{\Delta t} = Fv
\end{align*}
wouldn't make any sense. Since $v$ is a constant velocity, the power would always be equal to zero, which of course isn't right. Again, remember that when the velocity is constant, it is only the \textbf{net work} that is equal to zero. In Example \ref{ex:workenergy:powerconstantv}, it's asking for the power that \textbf{you} are expending by pushing on the crate (which is the same as asking for the rate of the work done \textbf{by} you \textbf{on} the crate). So, the formula does indeed make sense. 
\end{studentOpinion}


\newpage
\section{Summary}
\begin{chapterSummary}{
The work, $W$, done on an object by a force, $\vec F$, while the object has moved through a displacement $\vec d$ is defined as the scalar product:
\begin{align*}
W = \vec F \cdot \vec d &= Fd\cos\theta\\
&= F_xd_x+F_yd_y+F_zd_z
\end{align*}
If the force changes with position and/or the object moves along an arbitrary path in space, the work done by that force over the path is given by:
\begin{align*}
W =\int_A^B \vec F(\vec r) \cdot  d\vec l
\end{align*}
%Work allows us to quantify how a force acting on an object changes the speed of that object. 
If multiple forces are exerted on an object, then one can calculate the net force on the object (the vector sum of the forces), and the total work done on the object will be equal to the work done by the net force:
\begin{align*}
W^{net} = \int_A^B \vec F^{net}(\vec r) \cdot d\vec l
\end{align*}
If the net work done on an object is zero, that object does not accelerate.

We can define the kinetic energy, $K(v)$ of an object of mass $m$ that has speed $v$ as:
\begin{align*}
K(v) = \frac{1}{2} mv^2
\end{align*}

The Work-Energy Theorem states that the net work done on an object in going from position $A$ to position $B$ is equal to the object's change in kinetic energy:
\begin{align*}
W^{net} = \Delta K = \frac{1}{2} mv_B^2 - \frac{1}{2} mv_A^2
\end{align*}
where $v_A$ and $v_B$ are the speed of the object at positions $A$ and $B$, respectively.

The rate at which work is being done is called power and is defined as:
\begin{align*}
P = \frac{dW}{dt}
\end{align*}
If a constant force $\vec F$ is exerted on an object that has a constant velocity $\vec v$, then the power that corresponds to the work being done by that force is:
\begin{align*}
P &= \frac{d}{dt} W = \frac{d}{dt}(\vec F \cdot \vec d)\\
&= \vec F \cdot \frac{d}{dt}\vec d = \vec F \cdot \vec v
\end{align*}
}
\end{chapterSummary}


\begin{importantEquations}
\begin{multicols}{2}
\textbf{Work:}
\begin{align*}
W &= \vec F \cdot \vec d = Fd\cos\theta\\
W &= F_xd_x+F_yd_y+F_zd_z\\
W &=\int_A^B \vec F(\vec r) \cdot  d\vec l\\
W^{net} &= \int_A^B \vec F^{net}(\vec r) \cdot d\vec l
\end{align*}
\textbf{Kinetic Energy:}
\begin{align*}
K(v) = \frac{1}{2} mv^2
\end{align*}
\columnbreak

\textbf{Work-Energy Theorem:}
\begin{align*}
W^{net} = \Delta K = \frac{1}{2} mv_B^2 - \frac{1}{2} mv_A^2
\end{align*}
\textbf{Power:}
\begin{align*}
P &= \frac{dW}{dt}\\
P &= \vec F \cdot \vec v
\end{align*}
\end{multicols}
\end{importantEquations}


\newpage
\section{Thinking about the material}
\subsection{Reflect and research}

\begin{enumerate}
\item When was the concept of work first introduced?
\item To construct the pyramids, the ancient Egyptians used simple machines, like levers, to accomplish tasks that would not be possible otherwise. Apply what we know about work to find out how levers help people lift incredibly heavy objects. 
\item After an accident, investigators use skid marks to figure out how fast the cars were going before the crash. How do they do this?
\item The Tesla Model S can accelerate from 0-100 \si{km/h} in as little as 2.7 seconds. Calculate the power of the car in horsepower. Why is it unusual for a 7 seat sedan, like the Model S, to have such a short acceleration time? Investigate how it's possible for the Tesla to accelerate so quickly. 
\end{enumerate}
\subsection{To try at home}

\begin{tQuestion}Find the power of your legs. How you do this is totally up to you! \end{tQuestion}

\subsection{To try in the lab}

\newpage
\section{Sample problems and solutions}
\subsection{Problems}
\begin{problemParts}{soln:workenergy:skijump}{\label{prob:workenergy:skijump} A ski jump can be modelled by a ramp of height $h$, as shown in Figure \ref{fig:workenergy:skijumpprob}. A skier of mass $m$ is moving at a speed $v_0$ when they start going up the ramp. When the skiier lands the jump, they are moving at a speed $v_f$. Ignore air resistance. 
\capfig{0.5\textwidth}{figures/WorkEnergy/skijumpprob.png}{\label{fig:workenergy:skijumpprob} A person of mass $m$ goes off a ski jump of height $h$.}
}
\item What is the speed of the skier the instant they leave the ski jump?
\item Use the answer from part (a) to find the work done by friction along the ramp?
\end{problemParts}


\begin{problemParts}{soln:workenergy:swingwork}{\label{prob:workenergy:swingwork}
A child of mass $m$ sits on a swing of length $L$, as in Figure \ref{fig:workenergy:swingprob}. You push the child with a horizontal force $\vec F$. You apply the force in such a way that the child moves at a constant speed (note that $\vec F$ will not have a constant magnitude).}
\item How much work do you do to move the child from $\theta=0$ to $\theta=\theta_1$? 
\item Use a detailed diagram to show that the work done by $\vec F$ is equal to $mgh$, where $h$ is the change in height of the child. 
\capfig{0.4\textwidth}{figures/WorkEnergy/swingprob.png}{\label{fig:workenergy:swingprob}A child on a swing is pushed from $\theta=0$ to $\theta=\theta_1$}
\end{problemParts}

\newpage
\subsection{Solutions}
\begin{solution}{prob:workenergy:skijump}\label{soln:workenergy:skijump}
\begin{enumerate}[label=\alph*)]
\item Let's start by setting up our axes. We let $x$ be horizontal, $y$ be vertical, and set positive $y$ to be ``up''. \\

We can solve this problem using the Work-Energy Theorem. This theorem states that the net work done on an object is equal to its change in kinetic energy:
\begin{align*}
W^{net}=\frac{1}{2}mv_b^2-\frac{1}{2}mv_A^2
\end{align*}
For this part of the problem, we are only interested in what happens to the skier after they leave the jump. In order to use the Work-Energy Theorem, we need to know the net work done on the skier when they are in the air. Once the skier leaves the jump, the only force acting on them is gravity. The net force acting on the skier is thus $F=mg$, which points in the $-y$ direction. We can then use the formula: 
\begin{align*}
W=\vec F \cdot \vec d = F_xd_x+F_yd_y
\end{align*}
Since $F$ has no $x$ component, we get:
\begin{align*}
W=F_yd_y
\end{align*}
We want to multiply $F$ by the component of the skier's displacement that is parallel to the force. This component is simply equal to $h$, and it points in the $-y$ direction (Figure \ref{fig:workenergy:skijumpprobdisplacement}).
\capfig{0.4\textwidth}{figures/WorkEnergy/skiprobdisplacement.png}{\label{fig:workenergy:skijumpprobdisplacement} The displacement of the skier after they leave the jump.}
 We find that the work is equal to:
\begin{align*}
W^{net}&=-mg\hat y \cdot -h\hat y\\
&=mgh
\end{align*}
We can now find the speed of the skier when they leave the jump (let's call this $v_j$) using the Work-Energy theorem:
\begin{align*}
W^{net}&=\frac{1}{2}mv_f^2-\frac{1}{2}mv_j^2\\
\frac{1}{2}mv_j^2&=\frac{1}{2}mv_f^2-W^{net}\\
\frac{1}{2}mv_j^2&=\frac{1}{2}mv_f^2-mgh\\
\frac{1}{2}v_j^2&=\frac{1}{2}v_f^2-gh\\
v_j&=\sqrt{v_f^2-2gh}
\end{align*}
$\therefore$ the speed of the skier the instant they leave the jump is $\sqrt{v_f^2-2gh}$.
\item Now we are asked to find the work done by friction on the skier, when the skier is going up the jump. We are again going to use the Work-Energy Theorem. We know that the speed of the skier at the bottom of the ramp is $v_i$, and we just found that the speed of the skier at the top of the ramp is $v_j=\sqrt{v_f^2-2gh}$. We can use this to find the net work done on the skier, and from there find the work done by friction. The net work is equal to:
\begin{align*}
W^{net}&=\frac{1}{2}mv_j^2-\frac{1}{2}mv_i^2\\
&=\frac{1}{2}m(v_j^2-v_i^2)\\
&=\frac{1}{2}m(v_f^2-v_i^2-2gh)
\end{align*}
The net work done is the sum of the work done by each of the three forces acting on the skier. These forces are gravity, friction, and the normal force, as shown in Figure \ref{fig:workenergy:skijumpprobfbd}.
\capfig{0.2\textwidth}{figures/WorkEnergy/skijumpprobfbd.png}{\label{fig:workenergy:skijumpprobfbd} Free body diagram for the skier as they go up the jump.}
The normal force is perpendicular to the skier's displacement, so it does no work on the skier. We then have:
\begin{align*}
W^{net}=W^{friction}+W^{gravity}
\end{align*}
If we were to solve for the work done by friction directly, using $W^{friction}=f_kd\cos\theta$, we would need to know the coefficient of kinetic friction and the length of the ramp - but we don't know either of these! Luckily, we don't need to solve for the work done by friction directly. We know the net work, and we can easily solve for the work done by gravity, so we can just solve for the work done by friction by rearranging the above formula:
\begin{align*}
W^{friction}=W^{net}-W^{gravity}
\end{align*}
We can solve for the work done by gravity exactly as we did in part (a), using $W=\vec F \cdot \vec d$. The component of the skier's displacement that is parallel to gravity is $h$, but this time it is in the $+y$ direction, so we get:
\begin{align*}
W^{gravity}&=-mg\hat y \cdot h\hat y\\
&=-mgh
\end{align*}
The work done by gravity is the same as in part (a), except that its negative, which makes sense. We now have everything we need to solve for the work done by friction:
\begin{align*}
W^{friction}&=W^{net}-W^{gravity}\\
&=\frac{1}{2}m(v_f^2-v_i^2-2gh)+mgh\\
&=\frac{1}{2}m(v_f^2-v_i^2)
\end{align*}
$\therefore$ the work done by friction is equal to $\frac{1}{2}m(v_f^2-v_i^2)$. \\

\textbf{Discussion}: This equation tells us that only friction is responsible for the difference between the skier's initial speed and their final speed. If the work done by friction were equal to zero, we would have $0=\frac{1}{2}m(v_f^2-v_i^2)$, so $v_f$ would have to be equal to $v_i$. Does this make sense? If there were no friction, the only force doing work on the skier would be gravity. We found that as the skier goes up the jump, the work done by gravity is equal to $-mgh$, and when the skier comes down from the jump, the work done by gravity is equal to $mgh$. So, the total work done by gravity is equal to zero ($-mgh+mgh=0$). It makes sense, then, that if there were no friction, the initial and final speeds (and the kinetic energy) would be the same. We can also see that the total work done by gravity is zero just by looking at the initial and final positions of the skier. The total vertical displacement is equal to zero, so the total work done by gravity must be zero. This problem leads into the next chapter, where we will discuss conservative and non-conservative forces. \\
\end{enumerate}
\end{solution}

\begin{solution}{prob:workenergy:swingwork}\label{soln:workenergy:swingwork}
\begin{enumerate}[label=\alph*)]
\item We want to find the work done by the applied force $F$. We first need to find an expression for the magnitude of $\vec F$. Figure \ref{fig:workenergy:swingprobfbd} shows the forces acting on the child. 
\capfig{0.3\textwidth}{figures/WorkEnergy/swingprobfbd.png}{\label{fig:workenergy:swingprobfbd}A free body diagram for the child on the swing}

The child is moving at a constant speed, so the net force is equal to zero. We set the sum of the forces in the $x$ and $y$ directions equal to zero:
\begin{align*}
x: \quad F-F_T\sin\theta &=0\\
y: \quad F_T\cos\theta -mg &= 0
\end{align*} 
Rearranging these equations gives:
\begin{align*}
F&=F_T\sin\theta\\
mg&=F_T\cos\theta
\end{align*}
We want an expression for $F$ that does not depend on $F_T$ (since $F_T$ is unknown), so we divide the above equations then multiply each side by $mg$:
\begin{align*}
\frac{F}{mg}\cdot mg &= \frac{F_T\sin\theta}{F_T\cos\theta}\cdot mg\\
F&=mg\tan\theta
\end{align*}
Now we have an expression for $F$. Since the force is not constant, we can calculate the work done by $\vec F$ using the formula:
\begin{align*}
W=\int_A^B\vec F(\vec r) \cdot d\vec l
\end{align*} 
Remember that $dl$ is the ``element of the path.'' We can take advantage of the fact that the child is moving along a circular path. Instead of using cartesian coordinates, it makes sense to describe the position of the child in terms of $s$ and $\theta$. An infinitesimally small section of the path can thus be written as $ds$, which is equal to $Ld\theta$. This gives us the expression:
\begin{align*}
W=\int_0^{\theta_1}\vec F \cdot Ld\theta
\end{align*}
Where we set the limits of integration to be $\theta=0$ and $\theta=\theta_1$. Now we can find the scalar product:
\begin{align*}
W&=\int_0^{\theta_1}(F\cos\theta)Ld\theta\\
&=\int_0^{\theta_1}(mg\tan\theta)(\cos\theta)Ld\theta\\
&=\int_0^{\theta_1}(mgL)\frac{\sin\theta}{\cos\theta}\cos\theta d\theta\\
&=\int_0^{\theta_1}(mgL)\sin\theta d\theta\\
\end{align*}
Now we can pull $mgL$ in front of the integral (because they are constants) and integrate: 
\begin{align*}
W&=mgL\int_0^{\theta_1}\sin\theta d\theta\\
&=mgL(-\cos\theta\vert_0^{\theta_1})\\
W&=mgL(1-\cos\theta)
\end{align*}
$\therefore$ the work done by you on the child is equal to $mgL(1-\cos\theta)$. 
\item We know that the work done by $\vec F$ is $W=mgL(1-\cos\theta)$. So, we want to prove that $L(1-\cos\theta)$ is equal to $h$. Expanding $L(1-\cos\theta)$ gives:
\begin{align*}
L(1-\cos\theta)&=L-L\cos\theta
\end{align*}
Your diagram will look like the one shown in Figure \ref{fig:workenergy:swingprobgeometry}. 
\capfig{0.4\textwidth}{figures/WorkEnergy/swingprobgeometry.png}{\label{fig:workenergy:swingprobgeometry}A diagram showing the geometry of the problem}
This clearly shows that $h$ is equal to $L-L\cos\theta$.\\

\textbf{Discussion:} The net force acting on the mass is equal to zero, so the net work must be equal to zero. The two forces that do work on the mass are the applied force $\vec F$, and gravity. The work done by gravity is $-mgh$, so the work done by $\vec F$ is $mgh$. 
\end{enumerate}
\end{solution}


%\include{PotentialEcons}
%
\chapter{Gravity}
\label{chapter:gravity}
In previous chapters, we have so far learned about Newton's Theory of Classical Mechanics, which allowed us to model the motion of an object based on the forces acting on the object. In this chapter, we present the theories that describe the force of gravity itself. We will see several theories of gravity and focus primarily on Newton's Universal Theory of Gravity. 

\begin{learningObjectives}{
 \item Understand Kepler's Laws.
 \item Understand Newton's Universal Theory of Gravity. 
 \item Understand Gauss' Law and the gravitational field.
 \item Understand how to use energy to describe orbits.
 \item Understand how Einstein's General Theory of Relativity differs from Newton's theory of gravity.
 }
\end{learningObjectives}

\begin{opening}
\begin{MCquestion}{A question}
\item a choice
\item another choice %correct
\end{MCquestion}
\end{opening}

\section{Kepler's Laws}
Although humans have long been fascinated by the motion of objects in the sky, it was Johannes Kepler, in the early seventeenth century, that was the first to write down quantitative rules that described the motions of planets around the Sun. His theory was based on the extensive and detailed observations recorded by Tycho Brahe in the late sixteenth century. 

Kepler proposed three laws that describe all of the data that Tycho Brahe had collected about planetary motion:
\begin{enumerate}
\item The path of a planet around the Sun is described by an ellipse with the Sun at once of its foci.
\item All planets move in such a way that the area swept by a line connecting the planet and the Sun in a given period of time is constant, independent of the planet.
\item The ratio between the orbital periods, $T$, of two planets squared is equal to the ratio of the semi-major axes, $s$, of their orbits cubed:
\begin{align*}
\left(\frac{T_1}{T_2}\right)^2=\left(\frac{s_1}{s_2}\right)^3
\end{align*}
\end{enumerate}
We examine these three laws in more detail in the sections that follow. It should also be noted that, even though Kepler's Laws were derived for planets orbiting the Sun, they apply to any body that is orbiting any other body under the influence of gravity\footnote{In fact, they apply for any two bodies orbiting each other if the force between them is an ``inverse-square'' law, such as the gravitational and electric forces.}.

\subsection{Kepler's First Law}
Kepler noticed that the motion of all planets followed the path of an ellipse with the Sun located at one of its foci. Figure \ref{fig:gravity:ellipse} shows a diagram of an ellipse, along with its two foci, its semi-major axis, $s$, its semi-minor axis, $b$, and its eccentricity, $e$. The eccentricity is a measure of how far a focus is from the centre of the ellipse. A larger eccentricity thus corresponds to a ``flatter'' ellipse. Note that a circle is just a special case of an ellipse, with both foci located at the centre of the circle.
\capfig{0.5\textwidth}{figures/Gravity/ellipse.png}{\label{fig:gravity:ellipse}A ellipse, showing its two foci, its semi-major axis, $s$, its semi-minor axis, $b$, and its eccentricity, $e$.}
The sun is located at one of the foci. The point of closest approach to the Sun is called the ``perihelion'' of the orbit (or ``perigee'' if the orbit is not around the Sun), and the point furthest from the Sun is called the ``aphelion'' of the orbit (or ``apogee'' if the orbit is not around the Sun), as shown in Figure \ref{fig:gravity:perigeeapogee}.
\capfig{0.7\textwidth}{figures/Gravity/periapogee.png}{\label{fig:gravity:perigeeapogee}The orbit of the Earth around the Sun, showing the perihelion and aphelion, and the orbit of the Moon around the Earth, showing the perigee and the apogee. (Not to scale.)}


\begin{checkpoint} Order the ellipses from smallest eccentricity to largest eccentricity.
\capfig{0.7\textwidth}{figures/Gravity/eccellipses.png}{\label{fig:gravity:eccellipses}Three ellipses, each with an eccentricity $e$.}
\begin{answer}
$A<C<B$
\end{answer}
\end{checkpoint}

\subsection{Kepler's Second Law}
Kepler's Second Law is really a statement about the speed of a planet in an elliptical orbit. It states that the area swept by a line connecting the planet and the Sun in a given period of time is fixed. This is illustrated in Figure \ref{fig:gravity:ellipse2}, which shows the elliptical orbit of a planet around the Sun located at one of the foci, and the area swept out when the planet goes from $A$ to $B$ and from $C$ to $D$. 
\capfig{0.5\textwidth}{figures/Gravity/ellipse2.png}{\label{fig:gravity:ellipse2}Illustration of Kepler's Second Law, showing the area that is ``swept'' by a planet in a fixed period of time. }

Kepler's Second Law states that the two areas that are shown by the greyed out sections in the figure are the same if the planet took the same amount of time to travel between points $A$ and $B$ as it did to travel between points $C$ and $D$. Because the points $C$ and $D$ are further away from the Sun than points $A$ and $B$, the distance between points $C$ and $D$ must be smaller than the distance between points $A$ and $B$ for the two areas to be the same. This, in turn, implies that the planet must be moving slower between $C$ and $D$ than between points $A$ and $B$. The speed of a planet is thus greatest at the perihelion and smallest at the aphelion. As we will see in a later chapter, Kepler's Second Law is equivalent to the statement that the angular momentum of the planet is conserved.

\begin{checkpoint}
Based on Kepler's second law, what can you say about the speed of a planet in a \textbf{circular} orbit? 
\begin{answer} {The speed of the planet is constant.}
\end{answer}
\end{checkpoint} 

\subsection{Kepler's Third Law}
Kepler's Third Law is quantitative and relates the orbital periods ($T$) and the semi-major axes ($s$) between any two planets in orbit around the Sun:
\begin{align*}
\left(\frac{T_1}{T_2}\right)^2=\left(\frac{s_1}{s_2}\right)^3
\end{align*}
We can re-arrange this relation so that all of the quantities related to one planet are on the same side of the equal sign:
\begin{align*}
\frac{T_1^2}{s_1^3}=\frac{T_2^2}{s_2^3}=\text{constant}
\end{align*}
In other words, the ratio between the orbital period squared and the semi-major axis cubed is a constant, independent of the particular planet. In Example \ref{ex:gravity:keplerconstant}, we will use Newton's Universal Theory of Gravity to evaluate the constant.

\begin{checkpoint}
\begin{MCquestion}{If you double the radius of a circular orbit, what happens to the orbital speed?}
\item It doubles.
\item It increases by a factor of 8.
\item It increases by a factor of $\sqrt{8}$.
\item It decreases by a factor of $\sqrt{8}$.
\item It decreases by a factor of $\frac{\sqrt{8}}{2}$. \correct
\end{MCquestion}
\end{checkpoint}
TODO: Check answer

\section{Newton's Universal Theory of Gravity}
Newton supposedly gained insight into the gravitational force by observing an apple falling from a tree and concluding that if it is the same force that makes apples fall at sea level and at the top of a mountain, perhaps that force can be exerted all the way up to the moon. It is rather remarkable that Newton was able to make the connection between falling apples and the motion of the moon around the Earth to find a single theory to describe both situations.

We should be clear that the theory of gravity is a different theory than Newton's ``Laws of Motion'' (Newton's Three Laws). The Laws of Motion introduce the concepts of force and inertial mass, and prescribe how to use those concepts in order to model motion using kinematics. Newton's Universal Theory of Gravity is a theory that describes the force of gravity that two bodies with (gravitational) mass exert on each other.

Newton's Universal Theory of Gravity states that if two bodies with masses $M_1$ and $M_2$, located at positions $\vec r_1$ and $\vec r_2$, respectively, are separated by a distance, $r$, then $M_2$ will exert an attractive force on $M_1$, $\vec F_{12}$, given by:
\begin{align}
\vec F_{12}=-G\frac{M_1M_2}{r^2}\hat r_{21}
\end{align}
where $\hat r_{21}$ is the unit vector from $M_2$ to $M_1$:
\begin{align*}
\vec r_{21} &= \vec r_2 - \vec r_1\\
\hat r_{21} &= \frac{1}{r} \vec r_{21}
\end{align*}
as shown in Figure \ref{fig:gravity:gvectors}. $\vec F_{12}$ should be read as ``the force on body 1 from body 2''. $G=\SI{6.67e-11}{Nm^2/kg^2}$ is Newton's Universal Constant of Gravity. It should be noted that Newton's theory for the force of gravity written in this form only applies to either point masses (separated by a distance $r$) or spherical bodies whose centres are separated by a distance $r$ that is larger than the radius of either sphere.
\capfig{0.4\textwidth}{figures/Gravity/gvectors.png}{\label{fig:gravity:gvectors}Illustration of the vectors involved in Newton's Universal Theory of Gravity.}

Originally, Newton argued that the force of gravity would be proportional to the masses of the bodies, and inversely proportional to the square of the distance between them:
\begin{align*}
F_{12}\propto \frac{M_1M_2}{r^2}
\end{align*}
and $G$ is simply the constant of proportionality.

Because of Newton's Third Law, body 1 exerts a force on body 2 that is equal in magnitude but opposite in direction:
\begin{align*}
\vec F_{12} = -\vec F_{21}
\end{align*} 

\begin{example}{Calculate the magnitude of the force of gravity between yourself and a person standing $\SI{50}{cm}$ from you and compare that to your weight at the surface of the Earth (the force of gravity exerted by the Earth on you).}
If we assume that the two people have a mass of $\SI{50}{kg}$, the force of gravity exerted by one on the other, if they are separated by $\SI{50}{cm}$, is given by:
\begin{align*}
F=G\frac{M_1M_2}{r^2}=(\SI{6.67e-11}{Nm^2/kg^2})\frac{(\SI{50}{kg})(\SI{50}{kg})}{(\SI{0.5}{m})^2}=\SI{6.67e-7}{N}
\end{align*}
This is a very small force, compared to their weight, $F_g$:
\begin{align*}
F_g=M_1g=(\SI{50}{kg})(\SI{9.8}{N/kg})=\SI{490}{N}
\end{align*}
which is approximately 700 million times bigger. 

\textbf{Discussion:} The force of gravity is a very weak force when compared to other forces in Nature, such as the electric force between two charges. Newton's Universal Constant of Gravity is very small, so the force of gravity between two objects is very small unless either of the masses involved are very large, or the distance between them is very small. In general, when modelling the motion of objects on the Earth, it is generally safe to ignore the forces of gravity between objects and only include their weight (the force of gravity from the Earth). 
\end{example}

\begin{checkpoint}
\begin{MCquestion}{The radius of the Earth is \SI{6371}{km}. What is the order of magnitude of the Earth's mass?}
\item $10^{24}$
\item $10^{18}$
\item $10^{19}$
\item $10^{21}$
\item Not enough information.
\end{MCquestion}
\end{checkpoint}

\begin{example}{\label{ex:gravity:keplerconstant}Determine the constant in Kepler's Third Law for planets orbiting the Sun, namely the value of the ratio:
\begin{align*}
\frac{s^3}{T^2}
\end{align*}
where $s$ is the semi-major axis and $T$ is the orbital period.
}
Since Kepler's Third Law holds for any body orbiting the sun, we can easily determine the ratio by considering a planet of mass $m$ in a circular orbit of radius $R$ centred about the Sun. The semi-major axis of the orbit is equal to the radius of the orbit for a circular orbit ($s=R$).

If the planet is in a circular orbit about the Sun, its speed, $v$, will be constant, by Kepler's Second Law, and it will thus be executing uniform circular motion. The only force exerted on the planet is the force of gravity exerted by the Sun. Thus the force of gravity must be equal to the mass of the planet times its radial (centripetal) acceleration, $a_R$, which is given by:
\begin{align*}
a_R=\frac{v^2}{R}
\end{align*}
Newton's Second Law for the planet can be written as:
\begin{align*}
\sum F = F_g &= ma_R\\
G\frac{Mm}{R^2}&=m\frac{v^2}{R}\\
G\frac{M}{R}&=v^2
\end{align*}
where $M$ is the mass of the Sun. The speed of the planet is given by the circumference of the orbit divided by the orbital period $T$, since it is constant:
\begin{align*}
v=\frac{2\pi R}{T}
\end{align*}
Re-arranging the equation from Newton's Second Law:
\begin{align*}
G\frac{M}{R}&=v^2\\
G\frac{M}{R}&=\frac{4\pi^2 R^2}{T^2}\\
\therefore \frac{R^3}{T^2}&=G\frac{M}{4\pi^2}
\end{align*}
Thus, we find that the ratio of the cube of the radius orbit to the period squared is a constant that depends only on the mass of the Sun, which will then be the same for all planets (as it does not depend on, say, the mass of the planet that we considered).

\textbf{Discussion}: The relation above can, for example, be used to determine the mass of the Sun, since we can use geometry to determine the semi-major axis for the orbit of a planet, as Kepler did with the data from Tycho Brahe.
\end{example}

\begin{example}{\label{ex:gravity:geosyncorbit}Geosynchronous satellites are satellites that are placed in a circular orbit around the Earth in such a way that their orbital period is synchronized with the $\SI{24}{h}$ rotation period of the Earth. The advantage of geosynchronous satellites is that they are always above the same point on Earth, which makes them useful for establishing communication networks. At what altitude must geosynchronous satellites be placed?}
When a satellite orbits the Earth, the only force on the satellite is the force of gravity from the Earth. Since the satellite is in a circular orbit, that force of gravity must point towards the centre of the Earth in such a way that the satellite has the correct radial acceleration, $a_R$, to stay in uniform circular motion:
\begin{align*}
a_r=\frac{v^2}{R}
\end{align*}
where $v$ is the speed of the satellite, and $R$ is the distance between the satellite and the centre of the Earth (i.e. the centre of the circular orbit). The magnitude of the force of gravity on the satellite of mass $m$ is given by:
\begin{align*}
F = G\frac{Mm}{R^2}
\end{align*}
where $M$ is the mass of the Earth. Newton's Second Law applied to the satellite is:
\begin{align*}
\sum F_r = F &= ma_r\\
\therefore G\frac{Mm}{R^2}&=m\frac{v^2}{R}
\end{align*}
The speed of the satellite can be found from the fact that it must travel a distance of $2\pi R$ (the circumference of the orbit) in a period $T=\SI{24}{h}$:
\begin{align*}
v=\frac{2\pi R}{T}
\end{align*}
which we can substitute into the equation from Newton's Second Law to find the distance $R$ (i.e. the radius of the circular orbit):
\begin{align*}
G\frac{Mm}{R^2}&=m\frac{v^2}{R}\\
G\frac{M}{R^2}&=\frac{(2\pi R)^2}{T^2R}\\ 
G\frac{M}{R^2}&=\frac{4\pi^2 R}{T^2}\\ 
\therefore R&=\sqrt[3]{G\frac{MT^2}{4\pi^2}}\\
&=\sqrt[3]{(\SI{6.67e-11}{Nm^2/kg^2})\frac{(\SI{5.97e24}{kg})(\SI{86400}{s})^2}{4\pi^2}}\\
&=\SI{42.2e6}{m}
\end{align*}
which corresponds to the distance between the satellite and the centre of the Earth. To obtain the ``altitude'', $h$, namely the distance from the surface of the Earth to the satellite, we must subtract the radius of the Earth, $R_\oplus=\SI{6.371e6}{m}$ from this distance:
\begin{align*}
h = R-R_\oplus = \SI{35.9e6}{m}
\end{align*}
Thus, geosynchronous satellites are located at an altitude of approximately $\SI{36000}{km}$.

\textbf{Discussion}: Note that we could have also easily used Kepler's Third Law to determine the radius of the orbit, since we already know the period ($\SI{24}{h}$), and we know the value of the constant for Kepler's Third Law from Example \ref{ex:gravity:keplerconstant}.
\end{example}

\begin{example}{\label{ex:gravity:gofr}The acceleration due to Earth's gravity depends on the force that the Earth exerts on an object. Using the Earth's mass and radius, determine the acceleration of falling objects due to Earth's gravity at the surface of the Earth. Also, determine the altitude where the acceleration due to Earth's gravity is half of that at the Earth's surface.}
We can find the acceleration due to Earth gravity by determining the acceleration of a mass $m$ upon which gravity is the only acting force. In other words, we model an object that is in free-fall a distance $R$ away from the centre of the Earth. Newton's Second Law can be used in one dimension (corresponding to the direction that connects the falling mass to the centre of the Earth):
\begin{align*}
\sum F &= G\frac{Mm}{R^2}=ma\\
\therefore a&=G\frac{M}{R^2}
\end{align*}
where $M=\SI{5.97e24}{kg}$ is the mass of the Earth. At the surface of the Earth, $R=R_\oplus=\SI{6.371e6}{m}$:
\begin{align*}
a&=G\frac{M}{R_\oplus^2}=(\SI{6.67e-11}{Nm^2/kg^2})\frac{(\SI{5.97e24}{kg})}{(\SI{6.371e6}{m})^2}\\
&=\SI{9.81}{m/s^2}
\end{align*}
which, of course, is the value of $g$ that we have been using so far for the acceleration due to gravity near the surface of the Earth. To find the altitude at which this is reduced by half, we first find the value of $R$ that corresponds to this acceleration:
\begin{align*}
\frac{1}{2}G\frac{M}{R_\oplus^2}&=G\frac{M}{R^2}\\
\therefore R &=\sqrt{2}R_\oplus = \SI{9.0e6}{m}
\end{align*}
which corresponds to an altitude of $h=R-R_\oplus=\SI{2640}{km}$, well above the Earth's atmosphere.

\textbf{Discussion:} The acceleration of falling objects depends on the distance to the centre of the Earth, and decreases as one moves further from the centre of the Earth. It is thus an approximation to assume that $g$ is a constant, although in most cases this is a very good approximation. In practice, the value of $g$ will depend both on the distance from the centre of the Earth and the composition (density) of the material in the Earth's crust below where $g$ is being measured. Precise measurements of $g$ have bee used to measure the composition of the Earth's crust, as well as to search for mineral and oil deposits.
\end{example}

\subsection{Weight and apparent weight}
You have probably seen images of astronauts floating around the International Space Station (ISS) or other orbiting vessels, and heard of the term ``weightlessness''  to describe their motion. The ISS is in orbit at an altitude of approximately $\SI{400}{km}$, where the force of Earth's gravity is far from negligible (in Example \ref{ex:gravity:gofr} we showed that one needs to go to an altitude of $\SI{2640}{km}$ for the force to be reduced by half of that at the surface of the Earth). The contradiction between being weightless and the fact that weight is not zero is resolved by understanding that the popular term ``weightless'' is imprecise from a physics perspective.

The correct term to use from a physics perspective is to say that the \textit{apparent weight} of the astronauts is zero when they are floating around. Weight is the magnitude of the force of gravity exerted by the Earth. Apparent weight is, for example, the force that one measures when standing on a spring scale, which is equal to the normal force exerted by the spring scale on the person. Apparent weight could also be determined by the tension in a string from which a person is suspended. The apparent weight is the sum of the forces exerted on a person minus the force of gravity. If gravity is the only force exerted on a person (or object), that person's apparent weight is zero, which is what is popularly called being weightless.

Consider a person standing on a spring scale at the North pole of the Earth (top free-body diagram in Figure \ref{fig:gravity:apparentweight}). The only two forces exerted on the person are their weight, $\vec F_g$, and the normal force, $\vec N$, exerted by the spring scale. Since the person is not accelerating, the normal force and the weight have the same magnitude and opposite directions. The scale will thus read the actual weight of the person\footnote{The weight that is displayed on the scale is equal in magnitude to the normal force exerted by the scale on the person. It is the reaction force to that normal force.}.

\capfig{0.4\textwidth}{figures/Gravity/apparentweight.png}{\label{fig:gravity:apparentweight}The apparent weight, given by the normal force, is different at the Earth's equator because a person's acceleration is non-zero as they rotate with the Earth.}

Consider, instead, a person standing on a spring scale at the equator of the Earth (Figure \ref{fig:gravity:apparentweight}). That person is accelerating, because they are rotating with the Earth in uniform circular motion (since the Earth is rotating). Again, the only forces exerted on the person are their weight and the normal force exerted by the scale. The sum of the forces must now be equal to the person's mass, $m$, times the radial acceleration, $a_r$, that is necessary for them to follow the surface of the Earth as the Earth rotates about its axis. Newton's Second Law allows us to find the magnitude of the normal force acting on the person:
\begin{align*}
\sum F &= F_g-N=ma_r=m\frac{v^2}{R}\\
\therefore N &= F_g - m\frac{v^2}{R}\\
&=G\frac{Mm}{R^2} -  m\frac{v^2}{R}\\
&=m\left(G\frac{M}{R^2} - \frac{v^2}{R}  \right)\\
&=m\left(g - \frac{v^2}{R}  \right)
\end{align*}
where $M$ is the mass of the Earth, $R$ is the radius of the Earth, $v$ is the speed at the surface of the Earth due to the Earth's rotation, and, in the last line, we used the result from Example \ref{ex:gravity:gofr} where we determined the value of $g$ in terms of the mass and radius of the Earth.

We see that the normal force is reduced compared to what it would be if the Earth were not rotating ($v=0$) or if one is standing at one of the poles. Your apparent weight, which you can measure by standing on a spring scale, is thus smaller at the equator than it is at the poles. The quantity in parenthesis can be thought of as a modified or ``effective'' value of $g$ at the equator. To an observer at the equator, unaware of the Earth's rotation, the acceleration of objects as they fall towards the ground is smaller than at the poles. 

\begin{checkpoint}
\begin{MCquestion}{What is the effective value of $g$ at the equator?}
\item \SI{9.80}{m/s^2}
\item \SI{9.78}{m/s^2} \correct
\item \SI{9.51}{m/s^2}
\item \SI{9.70}{m/s^2}
\end{MCquestion}
\end{checkpoint}
TODO: check answer


If you are circling the Earth a distance $R$ from the centre of the Earth at a constant speed $v$, it is possible for your apparent weight to be zero. Imagine standing on a scale in an aircraft that is circling the Earth and measuring your apparent weight with the spring scale. As the speed of the aircraft increases, your apparent weight, $N$, decreases according to the formula that we just found:
\begin{align*}
N=m\left(G\frac{M}{R^2} - \frac{v^2}{R}  \right)
\end{align*}
At a certain speed, $v$, your apparent weight is zero and you feel weightless:
\begin{align*}
G\frac{M}{R^2} &= \frac{v^2}{R}\\
\therefore v&= \sqrt{G\frac{M}{R} }
\end{align*}
That is, your acceleration is exactly that which is due to gravity, since gravity is the only force that is acting on you (since the normal force is now zero).

Another way to feel weightless is when you are in free-fall (e.g. the first few seconds of a parachute jump from an airplane). One can think of being in orbit as continuously falling towards the centre of the Earth, but with an initial velocity in a direction such that you never actually collide with the Earth. The feeling of weightlessness will occur any time that the only force exerted on you is the force of gravity, which does not require you to be in a circular orbit. If you are in a spacecraft in any orbit and the only force on the spacecraft is from gravity (i.e. no rockets or wings are exerting any forces), then everything in the spacecraft will have the same acceleration, since gravity is the only force acting on anything in the spacecraft, and it will appear that everything is just floating in the spacecraft. To an outside observer, it would obviously be clear that the spacecraft and its contents are all accelerating.

At any position on Earth that is not at the equator or the poles, the sum of the forces on an object that is stationary relative to the Earth's rotating surface cannot be zero. This is because the object must rotate along with the Earth, and so the net force must point toward the centre of the circle about which that location on Earth is rotating. 

Take, for example, a plumb line, which consists of a mass hanging from a string. The two forces acting on the mass are gravity and tension. 
If the plumb line is located on Earth, some angle $\theta$ from the nearest pole (where $0^{\circ}<\theta<90^{\circ}$) as in Figure \ref{fig:gravity:apparentweight2}, then the string will point slightly away from the centre of the Earth. In order for the mass to remain stationary relative to the ground, it must rotate along with the Earth (radius $R$) around a circle of radius $R\sin\theta$. Thus, the tension from the string cannot point away from the centre of the Earth, because the net force must point towards the centre of the circle.

\capfig{0.7\textwidth}{figures/Gravity/apparentweight2.png}{\label{fig:gravity:apparentweight2}Away from the equator and poles, a plumb line (right) does not point towards the centre of the Earth, because a component of the tension must provide acceleration towards the centre of the circle (of radius $R\sin\theta$) about which the plumb line rotates due to the Earth's rotation. Note that the deflection of the plumb line is exaggerated}

\begin{checkpoint}
\begin{MCquestion}{You cut the string of the plumb line. Where does the mass land?}
\item At the same latitude.
\item Closer to the nearest pole.
\item Closer to the equator. \correct
\end{MCquestion}
\end{checkpoint}




TODO: Question library question about finding the angle between the plumbline and the true vertical at some location (give latitude so that they have to figure out how $\theta$ relates to latitude (it is latitude)

\subsection{The gravitational field}
The gravitational force exerted on a mass $m$ by a mass $M$ can be written as:
\begin{align*}
\vec F(\vec r) = -G\frac{Mm}{r^2}\hat r
\end{align*}
if we define a coordinate system with the origin located at the centre of mass $M$ so that $\vec r$ is the position of $m$ relative to $M$. We can define the ``gravitational field'' at position, $\vec g(\vec r)$, of mass $M$ as the gravitational force per unit mass exerted by $M$:
\begin{align}
\Aboxed{\vec g(\vec r) = - \frac{GM}{r^2}\hat r}
\end{align}
The word ``field'' is just a mathematical term for a function that depends on position. Since $\vec g(\vec r)$ is a vector, we would refer to it as a ``vector field''.

Defining the gravitational field makes it easy to calculate the force of gravity from $M$ on any mass $m$:
\begin{align*}
\vec F_g = m\vec g(\vec r)
\end{align*}

At the surface of the Earth, the magnitude of the gravitational field is given by:
\begin{align*}
g(R_\oplus)=\frac{GM}{R_\oplus^2}=\SI{9.81}{N/kg}
\end{align*}
where $R_\oplus$ is the radius of the Earth. Of course, this also corresponds to the acceleration of objects in free-fall near the surface of the Earth, which we can find from Newton's Second Law:
\begin{align*}
\sum \vec F &= \vec F_g = m\vec a\\
m\vec g(R_\oplus)&= m\vec a\\
\therefore \vec a &= \vec g(R_\oplus)
\end{align*}
but we see here why it more precise to refer to $g$ as the ``magnitude of the gravitational field at the surface of the Earth'' rather than ``the acceleration due to Earth's gravity''. It is also worth noting that the two are only equal if the gravitational mass (on the left of the equation in the second line) is the same as the inertial mass (on the right of the equation). The gravitational mass is the mass that appears in the gravitational force defined by Newton, whereas the inertial mass is the mass that appears with the acceleration in Newton's Second Law.

Suppose that there are two large mass bodies, $M_1$ and $M_2$, and a smaller mass body, $m$. We can calculate the net gravitational force on $m$ by summing the gravitational force vectors from $M_1$ and $M_2$ separately. If the gravitational field from $M_1$ and $M_2$ are given by $\vec g_1(\vec r)$ and $\vec g_2(\vec r)$, respectively, then the total gravitational force on $m$ is given by:
\begin{align*}
\vec F &= m\vec g_1(\vec r) + m\vec g_2(\vec r)=m(\vec g_1(\vec r)+\vec g_2(\vec r))\\
&=m \vec g(\vec r)
\end{align*}
where we have introduced the total gravitational field:
\begin{align*}
\vec g(\vec r) = \vec g_1(\vec r)+\vec g_2(\vec r)
\end{align*}
In other words, if there are multiple bodies that result in a non-negligible force of gravity, we can calculate their gravitational fields independently and sum them together to define a net gravitational field, $\vec g(\vec r)$, that models the net force of gravity from all of the bodies. The net gravitational force on a new body of mass $m'$ is then simply given by $m'\vec g(\vec r)$, and we do not need to add any more vectors together. For example, when calculating the motion of satellites that can be influenced by the force of gravity from both the Earth and the Moon, we simply need to calculate the net gravitational field from the Earth and Moon, and the motion of any satellite can then be modelled using that net gravitational field.

TODO: checkpoint?
TODO: problem about the gravitational field between multiple bodies

TODO: Question Library question to find L1 Lagrange point (it requires a binomial expansion), see e.g. example 6-10 in Giancolli


\subsection{Gauss' Law}
Newton's Universal Theory of Gravity postulates that the force of gravity between two bodies decreases as the squared of the distance between those two bodies. Using the terminology of a field, we would say that the strength of the gravitational field from an object decreases as the inverse of the square of the distance to that object. This is an example of a what we generally call an ``inverse-square law''. The electric force between two charges is also given by an inverse-square law, and we now understand that these forces behave as if they were ``transmitted'' by waves or particles.

If a force is given by an inverse-square law, then Gauss' Law gives a way to determine the strength of the field that is associated with that force. In the case of gravity, Gauss' Law states that:
\begin{align*}
\oint \vec g(\vec r) \cdot d\vec A = 4\pi G M^{enc}
\end{align*}
where the integral on the left is an integral over a ``closed surface''. That is, imagine taking a closed surface, $S$, and then taking a small element of that surface and drawing a vector $d\vec A$ that points outwards from the closed surface and has a magnitude equal to the area of that small surface, as illustrated in Figure \ref{fig:gravity:gauss}. You can then take the scalar product of that vector with the gravitational field, $\vec g(\vec r)$, at that point on the surface. If you sum all of those scalar products, you get the value of the integral on the left. Gauss' Law states that the value of that integral is equal $4 \pi G$ times the total mass that is enclosed by the surface. 

We will go into more detail about Gauss' Law when we cover electromagnetism, but it is worth seeing how to use it in a simple scenario. Figure \ref{fig:gravity:gauss} shows a spherical body of mass $M$ and radius $R$ for which we would like to determine the value of the gravitational field at a distance $r$ from the centre of the body.
\capfig{0.3\textwidth}{figures/Gravity/gauss.png}{\label{fig:gravity:gauss}Example of a spherical Gaussian surface, $S$, of radius $r$ centred about a body of mass $M$ and radius $R$. An element of the surface, $d\vec A$ is also shown along with the gravitational field, $\vec g(\vec r)$, at that point.}
To do so, we draw a ``Gaussian surface'', $S$, that is a sphere with a radius $r$, and centred about the body. At any point on the surface, the area element vector $d\vec A$ points away from the centre of the spherical surface. The gravitational field vector will always point towards the centre of the spherical surface. Furthermore, by symmetry, the magnitude of $\vec g(\vec r)$ is constant along the whole Gaussian surface. Thus, the scalar product $\vec g(\vec r) \cdot d\vec A=-g(r)dA$ everywhere along the surface (it is negative because the two vectors are anti-parallel). The integral is thus given by:
\begin{align*}
\oint \vec g(\vec r) \cdot d\vec A = -g(r)\oint dA 
\end{align*}
where we factored $g(r)$ out of the integral, since the magnitude of $\vec g(\vec r)$ is constant for all of the area elements $dA$ on the sphere. Remember that an integral is a sum. The integral $\oint dA$ thus means ``sum all of the area elements $dA$ over the entire surface $S$''. Thus, the integral is the total area of the spherical surface $S$\footnote{The surface area of a sphere of radius $r$ is $4\pi r^2$.}:
\begin{align*}
\oint \vec g(\vec r) \cdot d\vec A = -g(r)\oint dA =-g(r)(4\pi r^2)
\end{align*}
Inserting this into Gauss' Law, we find:
\begin{align*}
\oint \vec g(\vec r) \cdot d\vec A &= 4\pi G M^{enc}\\
-g(r)(4\pi r^2) &= 4\pi G M^{enc}\\
\therefore g(r) &= - \frac{GM}{r^2}
\end{align*}
where $M^{enc}=M$ is the total mass enclosed by the Gaussian surface (in this case, the entire mass $M$ is enclosed). This is of course the result that we expected. Note that Gauss' Law is only easy to use if the system is highly symmetric (e.g. spherically symmetric), and that it does not give the direction of the field vector, which must be obtained from symmetry arguments. 

\begin{studentOpinion}{Olivia}
Here's an analogy to describe Gauss's Law for gravity: A famous celebrity is doing an event, and they attract a certain number of fans. The fans will try to get as close to the celebrity as possible, so you have to put up a barricade around the celebrity. The gravitational field is represented by how crowded it is somewhere along the barricade. If a second celebrity is at the event, they will attract their own fans, so there will be more people around the barricade. The number of celebrities (or how famous they are) is kind of like the enclosed mass $M^{enc}$; the more mass that is enclosed, the stronger the field will be. 

Now imagine that you haven't put up a barricade yet. A photographer is coming to the event, and you told him to stand at some location that is a distance $r$ from one of the celebrities. The photographer wants to know how crowded it will be when he is standing behind the barricade at that location. Gauss's law gives us a way to figure this out. If you know which celebrities are at the event ($M^{enc}$), you can determine how many people will be there (this is like finding $4\pi GM^{enc}$). Then, if you can build a barricade such that the fans are evenly distributed around it, and you know how long that barricade is ($\oint dA$), you can easily calculate how crowded it will be at some point along the barricade (you can just divide the number of people by the length of the barricade). 

The barricade represents our Gaussian surface and, like a Gaussian surface, it can be whatever shape we want as long as it encloses the celebrities and passes through the point we are interested in. If we want to make sure the people are spread out evenly, the shape of the barricade is going to depend on the specific case. The shape of the barrier is going to depend on how many celebrities there are, how famous each one is, and their locations. For example, consider the case shown in Figure \ref{fig:gravity:barricadeanalogy}, which depicts two celebrities that are far apart from each other, who attract the same number of fans (this is analogous to having two spherical bodies of the same mass).

\capfig{0.7\textwidth}{figures/Gravity/barricadeanalogy.png}{\label{fig:gravity:barricadeanalogy}There are two celebrities (two dots) enclosed by a barricade. They each have fans that want to be as close to them as possible, and a photographer located at $P$. We want to build a barricade so that the fans are spread out evenly. We can technically make the barricade on the left, but it isn't useful to us. The barricade on the right is useful because all of the fans are spread out evenly.}

In this section, we used Gauss's law to find the gravitational field due to a spherical body. This is the simplest case, and in our analogy it's like having just one celebrity at the event the photographer is attending. In this case, the gravitational field is the same so long as the distance to the centre of the sphere is the same. This means that if we built a circular barricade with a radius $r$, all the fans would be evenly spread out, which makes sense. This is analogous to using a sphere of radius $r$ as our Gaussian surface to find the gravitational field due to the spherical body at a distance $r$.

(Remember though that the Gaussian surface isn't a physical thing, so it won't affect the gravitational field. Its' just a mathematical tool we use where we take advantage of the field's geometry.)
\end{studentOpinion}

We can also use Gauss' Law to determine the gravitational field \textbf{inside} of the body of mass $M$ and radius $R$. This is illustrated in Figure \ref{fig:gravity:gauss2}, which shows a spherical Gaussian surface of radius $r$ that is \textbf{inside} of the body of mass $M$.
\capfig{0.3\textwidth}{figures/Gravity/gauss2.png}{\label{fig:gravity:gauss2}Example of a spherical Gaussian surface, $S$, of radius $r$ centred inside a body of mass $M$ and radius $R$.}
The gravitational field inside of the body of mass $M$ is also symmetric and constant in magnitude across the whole surface, so that the integral is the same as before:
\begin{align*}
\oint \vec g(\vec r) \cdot d\vec A=-g(r)(4\pi r^2)
\end{align*}
However, in order to use Gauss' Law, we need to determine the mass of the body that is enclosed within the spherical surface, which will be less than $M$. If we assume that the mass density, $\rho$, of the object is constant (the body is made of a uniform material), then the density is simply the mass of the object over its volume:
\begin{align*}
\rho = \frac{M}{\frac{4}{3}\pi R^3}
\end{align*}
The amount of mass enclosed by the spherical surface of radius $r$ is the density multiplied by the volume of a sphere of radius $r$:
\begin{align*}
M^{enc} = \rho \frac{4}{3}\pi r^3 = M\frac{r^3}{R^3}
\end{align*}
Applying Gauss' Law, we can now find the magnitude of the gravitational field inside of the spherical body at a distance $r$ from the centre:
\begin{align*}
\oint \vec g(\vec r) \cdot d\vec A &= 4\pi G M^{enc}\\
-g(r)(4\pi r^2) &= 4\pi G M\frac{r^3}{R^3}\\
\therefore g(r) &= - \frac{G M}{R^3}r
\end{align*}
And we find that, inside a uniform spherical body of mass $M$, the gravitational field increases linearly with radius as one moves out from the centre. At the centre of the body, the gravitational field is zero. 

\section{Gravitational potential energy}
Consider a large spherical body of mass $M$ with a coordinate system whose origin coincides with the centre of the spherical body (for example, the large body could be the Earth). The force, $\vec F(\vec r)$ on a body of mass $m$ (for example, a satellite), located at a position $\vec r$ is then given by:
\begin{align*}
\vec F(\vec r) = - G\frac{Mm}{r^2}\hat r=- G\frac{Mm}{r^3}\vec r
\end{align*}
where in the second equality, we use the fact that the unit vector in the direction of $\vec r$ is simply the vector $\vec r$, divided by its magnitude. We can write the force out in Cartesian coordinates:
\begin{align*}
\vec r &= x\hat x + y \hat y + z\hat z\\
r &= \sqrt{x^2+y^2+z^2} =(x^2+y^2+z^2)^\frac{1}{2} \\
\therefore \vec F(x,y,z) &= - G\frac{Mm}{(x^2+y^2+z^2)^\frac{3}{2} }(x\hat x + y \hat y + z\hat z)
\end{align*}
Mathematically, this is equivalent to the force that we considered in Example \ref{ex:potentialecons:gravitycons} of Chapter \ref{chap:potentialecons}, which we showed was a conservative force. The force of gravity in Newton's theory is thus a conservative force, for which we can determine a potential energy function.

In order to determine the gravitational potential energy function for the mass $m$ in the presence of a mass $M$, we calculate the work done by the force of gravity on the mass $m$ over a path where the integral for work will be ``easy'' to evaluate, namely a straight line. Figure \ref{fig:gravity:potential} shows such a path in the radial direction, $r$, over which it will be easy to calculate the work done by the force of gravity from mass $M$ when mass $m$ moves from being a distance $r_A$ to a distance $r_B$ from the centre of mass $M$.
\capfig{0.5\textwidth}{figures/Gravity/potential.png}{\label{fig:gravity:potential}Calculating the work done on a mass $m$ by the force of gravity exerted by mass $M$ when mass $m$ moves from a distance $r_A$ to a distance $r_B$ from the center of mass $M$.}
The work done by the force of gravity is given by:
\begin{align*}
W &= \int_{r_A}^{r_B}\vec F(r) \cdot d\vec r = \int_{r_A}^{r_B} \left(- G\frac{Mm}{r^2}\hat r \right)\cdot d\vec r =\int_{r_A}^{r_B} - G\frac{Mm}{r^2}dr\\
&=\left[G\frac{Mm}{r} \right]_{r_A}^{r_B} =G\frac{Mm}{r_B} - G\frac{Mm}{r_A}
\end{align*}
The difference in potential energy in going from position $A$ to position $B$ is given by the negative of the work done by the force:
\begin{align*}
\Delta U = U(r_B) - U(r_A) = -W = G\frac{Mm}{r_A} - G\frac{Mm}{r_B}
\end{align*}
By inspection, we can identify the potential energy function for gravity:
\begin{align}
\Aboxed{U(r) = -G\frac{Mm}{r} + C}
\end{align}
which is determined only up to a constant, $C$. 

A particularly useful choice of constant is $C=0$. This corresponds to choosing the potential energy to be zero only when $r$ goes to infinity. That is, the potential energy of mass $m$ is zero only when it is infinitely far away from mass $M$. The choice of constant $C$ corresponds to the (arbitrary) value of the potential energy when mass $m$ is infinitely far from mass $M$. When mass $m$ is not infinitely far away, it has \textbf{negative} potential energy (if $C=0$). This is not a problem! Remember, the only thing that is meaningful is a difference in potential energy, so the specific value of the potential energy has no meaning. The kinetic energy of an object, on the other hand, has to be positive.

Recall that if there are no other forces acting on an object, that object will move in such a way to reduce its potential energy. If the object of mass $m$ is located at some distance $r$ from the object of mass $M$, the force of gravity will attract $m$ so that $r$ decreases. As $r$ decreases in magnitude, the potential energy becomes more negative (larger in magnitude, but further away from zero), and the potential energy will indeed decrease under the influence of gravity. The potential energy that we have found does have the correct properties.

TODO: Checkpoint

\subsection{Mechanical energy with gravity}
Unless noted otherwise, we will continue our discussion of gravitational potential energy with the particular choice of constant $C=0$:
\begin{align}
\Aboxed{U(r) = -G\frac{Mm}{r}}
\end{align}
Furthermore, we will assume that $M$ is a large body, such as the Earth, which we can consider as fixed, and focus our discussion on describing the motion of mass $m$ (e.g. a satellite). If $M$ is much bigger than $m$, they will both experience a force of gravity from each other of the same magnitude (Newton's Third Law), but because $M$ is so much larger, its acceleration will be much smaller (Newton's Second Law). Thus, it is a good approximation to assume that $M$ is stationary and that only $m$ moves when $M>>m$. 

We can define the total mechanical energy of mass $m$ when it has a speed $v$ (relative to $M$) and is located at a distance $r$ from the centre of mass $M$:
\begin{align*}
E = U + K = -G\frac{Mm}{r}+\frac{1}{2}mv^2
\end{align*}
where the kinetic energy term is always positive. If gravity is the only force exerted on mass $m$, then the mechanical energy, $E$, as defined above, will be a constant. The mechanical energy of an object can give us insight into the possible motion of the object.

Imagine launching a rocket upwards from the surface of the Earth; once all of the fuel has burnt up, the rocket's mechanical energy becomes constant as the rocket engine stops doing work on the rocket. As soon as the engine stops providing thrust, the rock will start to slow down as the force of gravity attracts the rocket back to Earth. If the rocket is going fast enough, it will be able to completely escape the Earth's gravitational pull and travel to infinity (we assume that there are no other planets or the Sun, just the Earth exists!). If, on the other hand, the rocket's speed is too low, it will eventually stop and fall back to Earth. This is the same thing that happens to you when you try to jump vertically. If you could jump hard enough, you would be able to escape the Earth's gravitational pull!

In terms of mechanical energy, we can ask ourselves if the mechanical energy of the rocket is large enough to escape the Earth's gravitational pull. Specifically, we can ask ourselves what the value of the rocket's kinetic energy would be when it reaches infinity. The kinetic energy of the rocket is given by:
\begin{align*}
K = E - U
\end{align*} 
If the rocket is infinitely far from the Earth, then its potential energy is zero, and the kinetic energy is equal to $E$.

If the mechanical energy, $E$, is negative, it is not possible for the rocket to ever make it to infinity because its kinetic energy would have to be negative. In other words, if the mechanical energy is negative, then the object of mass $m$ can never escape the gravitational pull of object $M$. We say that $m$ is ``gravitationally bound'' to $M$.

If the mechanical energy, $E$, is exactly zero, then the object's kinetic energy will become zero just as it reaches infinity. In other words, it will just barely be able to escape the gravitational pull from mass $M$. The condition for this to happen is:
\begin{align*}
E &= 0\\
K & = -U\\
\frac{1}{2}mv^2 &= G\frac{Mm}{r}\\
\therefore v_{esc} &= \sqrt{\frac{2GM}{r}}
\end{align*}
which we can interpret as a condition for the speed of the rocket. If at some distance $r$ from $M$, the rocket has the speed given by the condition above, then it will have enough kinetic energy to escape the gravitational pull of $M$. We call this speed the ``escape velocity''. 

Finally, if the mechanical energy is greater than zero, then the rocket will have enough energy to escape the gravitational pull of $M$ and have a non-zero speed when it reaches infinity. 

TODO: Checkpoint MC question: What is the escape velocity from the surface of the Earth?

\begin{example}{How much energy must be expended in order to place a satellite of mass $m=\SI{1000}{kg}$ in a geosynchronous circular orbit around the Earth, if the satellite is launched from the North Pole of the Earth? How much energy is this per kilogram of satellite placed in orbit?}
We need to calculate how much work must be done for the satellite to go from being at rest at the surface of the Earth to being in a geosynchronous orbit. That work will be done by a non-conservative force (a rocket engine). The work done by the non-conservative force, $W$, is equal to the satellite's change in energy:
\begin{align*}
W = \Delta E = E_B -E_A
\end{align*}
The initial energy of the satellite, $E_A$, is given by:
\begin{align*}
E_A = K + U = 0 - G\frac{Mm}{R_\oplus}
\end{align*}
where $M=\SI{5.97e24}{kg}$ is the mass of the Earth, and $R_\oplus=\SI{6.731e6}{m}$ is the radius of the Earth. Initially, the rocket is at rest, and so all of the mechanical energy is potential energy. 

In orbit, the energy of the rocket, $E_B$, is given by:
\begin{align*}
E_B = K + U = \frac{1}{2}mv^2 - G\frac{Mm}{R}
\end{align*}
where $R=\SI{42.2e6}{m}$ is the radius of the geosynchronous orbit (Example \ref{ex:gravity:geosyncorbit}) and $v$ is the speed of the satellite in orbit. The speed is given by:
\begin{align*}
v = \frac{2\pi R}{T}
\end{align*}
where $T=\SI{24}{h}$ is the orbital period. The net work that must be done to place the satellite in orbit is thus given by:
\begin{align*}
W &= E_B - E_A = \frac{1}{2}mv^2 - G\frac{Mm}{R} - \left(- G\frac{Mm}{R_\oplus}\right)\\
&=\frac{1}{2}m\frac{4\pi^2 R^2}{T^2}+GMm\left(\frac{1}{R_\oplus}-\frac{1}{R}\right)\\
&=\frac{1}{2}(\SI{1000}{kg})\frac{4\pi^2 (\SI{42.2e6}{m})^2}{(\SI{86400}{s})^2}\\
&+(\SI{6.67e-11}{Nm^2/kg^2})(\SI{5.97e24}{kg})(\SI{1000}{kg})\left(\frac{1}{(\SI{6.731e6}{m})}-\frac{1}{(\SI{42.2e6}{m})}\right)\\
&=\SI{5.78e10}{J}
\end{align*}
This corresponds to the energy that must be imparted to a $\SI{1000}{kg}$ satellite for it to end up in a geosynchronous orbit. This corresponds to $\SI{5.78e7}{J/kg}$ as the energy required per kilogram of payload placed in geosynchronous orbit. Although we calculated work as if it were work done by a force, we can think of this work coming from stored chemical potential energy in the fuel of the rocket carrying the satellite. 

\textbf{Discussion:} The energy that we found above is the minimum energy that one must provide to the satellite. In practice, in order to place a satellite in orbit, one will also need to provide enough energy to accelerate the rocket that carries the satellite up into orbit, which is typically much heavier than the satellite. If the satellite were instead launched from the equator of the Earth, the satellite would already have some initial kinetic energy due to the rotation of the Earth, and one would need to provide less energy to place it in orbit. This is the reason that most rockets are launched from near the equator (think French Guyana, Florida, Kazakhstan) in a direction that is roughly parallel with the Earth's rotation.
\end{example}

\subsubsection{Types of orbits}
The mechanical energy of a body of mass $m$ determines whether it is gravitationally bound to (i.e. cannot escape) the body of mass $M$. The path (orbit) that $m$ will take depends on its velocity with respect to $M$. Clearly, if the velocity of $m$ is directed at the centre of $M$, then $m$ will just collide with $M$. In all other cases, the orbit that $m$ will take depends on the mechanical energy of $m$ as well as the speed of $m$ at the point of closest approach to $M$ (see Figure \ref{fig:gravity:conical}). The velocity of $m$ at the point of closest approach will always be perpendicular to the line joining the centres of $m$ and $M$. The different possible orbits are:
\begin{enumerate}
\item A \textbf{circular orbit} of radius $R$ (where $R$ is the distance of closest approach) if the \textbf{mechanical energy is negative} (i.e. it is bound) and the speed is exactly equal to the value necessary for the gravitational force to provide the required centripetal acceleration for uniform circular motion:
\begin{align*}
\sum F = G\frac{Mm}{R^2} &= m\frac{v^2}{R}\\
\therefore v_{circ}=\sqrt{\frac{GM}{R}}
\end{align*}
\item An \textbf{elliptical orbit} if the \textbf{mechanical energy is negative} and the speed at the point of closest approach is different than that required for a circular orbit.
\item A \textbf{parabolic orbit} if the \textbf{mechanical energy is exactly zero}.
\item A \textbf{hyperbolic orbit} if the \textbf{mechanical energy is bigger than zero}.
\end{enumerate}
The possible orbits are illustrated in Figure \ref{fig:gravity:conical}, and are curves in the family of ``conical sections'', as they can be found by the intersection of a plane and a cone. All conical sections have at least one ``focus'' point (ellipses have two) that corresponds to the location of $M$. 
\capfig{0.7\textwidth}{figures/Gravity/conical.png}{\label{fig:gravity:conical}The different possible orbits of $m$ due to the gravitational force of $M$ depend on the mechanical energy, $E$, of $m$. The orbits are drawn in a frame of reference where $M$ is at rest. }

TODO: I think the figure with the orbits could be better. Note that I lifted the curves from actual conical curves with a common focus and traced out the unbound curves using powerpoint so that they would be correct....

\section{Einstein's Theory of General Relativity}
Newton's Universal Theory of Gravity was extremely successful at describing the motion of planets in the solar system, and allowed for high precision astronomy. For example, precision measurements of the orbit of Uranus showed that it appeared to be inconsistent with Newton's theory, unless the gravitational influence of another planet was included in the model. This led to the discovery of the planet Neptune. 

Some issues were however uncovered with Newton's theory. The orbit of Mercury was also noticed to be different than what Newton's theory could describe, but searches for another planet (Vulcan) proved unfruitful. In addition, Albert Einstein's theory of Special Relativity, published in 1905, was found to be incompatible with Newton's theory of gravity. One of the consequences of Special Relativity is that nothing can propagate faster than the speed of light. However, Newton's Universal Theory of Gravity implies that the gravitational force is transmitted instantaneously. In Newton's theory, if the Sun suddenly disappeared, Earth would immediately ``fall out'' of its orbit, and we would immediately know that the Sun has disappeared. This would violate Special Relativity, since there cannot exist any mechanism for us to know that the Sun has disappeared until a signal has propagated from the Sun to the Earth at the speed of light. In other words, for the $\SI{8}{min}$ that are required for light to travel from the Sun to the Earth, we cannot know that the Sun has disappeared: only when we literally see the Sun disappear would the Earth be ``allowed'' to fall out of its orbit.

Einstein's Theory of General Relativity is a theory developed by Einstein in order to describe gravity in a way that is consistent with Special Relativity and the properties of how light propagates. Einstein was famous for his ``thought experiments'' that are designed to think about some of the implications of a theory, even if they would be very difficult experiments to carry out in practice. One such thought experiment is to consider what an observer would observe in an accelerating frame of reference.

Consider an observer in an elevator, as illustrated in Figure \ref{fig:gravity:elevator}. If the elevator is stationary at the surface of the Earth (left panel), and the observer is standing on a scale, they could measure their weight, $mg$, on the scale. The two forces on the observer are their weight and the normal force, which would be equal in magnitude since the observer is not accelerating. The normal force, read out by the scale, would thus correspond to their weight. To be more precise, the normal force would be equal to $m_gg$, where $m_g$ is the gravitational mass of the observer (that mass which is related to the force of gravity experienced by a mass).

If instead, the elevator was placed in empty space, and the elevator was accelerating upwards with an acceleration of $g$ (right panel), the observer would still be able to measure their weight by stepping on the scale. The only force on the observer is the normal force from the scale, which must be equal to its mass times their acceleration $N=ma=mg$, since the observer is accelerating with the elevator. In this case, it is the inertial mass of the observer, $m_i$, that comes into play, so the normal read on the scale is $m_ig$.  


\capfig{0.8\textwidth}{figures/Gravity/elevator.png}{\label{fig:gravity:elevator} (Left:) A person standing on a scale in an elevator at rest at the surface of the Earth. (Right:) A person in an elevator that is accelerating in empty space with the same acceleration as that due to gravity at the Earth's surface. }


TODO: Make above figure much better!


Einstein postulated that it would be impossible for the observer to distinguish whether they are at rest on the surface of the Earth, or in empty space accelerating with an acceleration of $g$. In other words, he postulated that the inertial and gravitational masses are exactly equivalent. This is what is called the ``Equivalence Principle''.

This simple statement has dramatic implications. Special Relativity requires that light will travel in a straight line in empty space. If a beam of light enters and then exits the elevator, the observer on Earth and the one accelerating in empty space must observe the same thing, since they cannot distinguish between being on Earth or accelerating in space. The observer in space, who is accelerating, will observe that the beam of light bends as it crosses the elevator (the beam travels in a straight line as observed in an inertial reference frame, so the person in the accelerating elevator would see it follow a parabolic path). The observer on Earth must thus observe the same thing, namely that light will follow a curved path in the presence of a gravitational field.

But... light travels in a straight line, so if the path of a beam of light is curved, it is space itself that is curved in the presence of a gravitational field! In other words, Einstein's Theory of General Relativity describes how the presence of mass (or energy) results in a curvature of space (and time). Thus, objects that are moving in a gravitational field are actually following Newton's First Law (they are moving at constant velocity in a straight line and no force is exerted on them) - they appear to follow a curved path, because it is space that is curved! It is strange, unexpected, but high precision measurements confirm that this correctly describes everything that we have measured!

Einstein's theory was able to describe the orbit of Mercury, and the prediction that gravity leads to light following a curved path was confirmed by Eddington within five years of Einstein's theory being published. Another implication of the theory is that time goes by slower in the presence of a gravitational field. Clocks on Earth run slower than clocks in orbit (where the gravitational field is weaker). This effect is taken into account when using GPS to determine your position on Earth, since this is based on comparing the time that it takes signals to arrive to your position on Earth from different satellites.

\newpage
\section{Summary}

\begin{chapterSummary}
Kepler was the first to synthesize a large amount of data to quantitatively describe gravity with his three laws:
\begin{enumerate}
\item The path of a planet around the Sun is described by an ellipse with the Sun at once of its foci.
\item Planets move in such a way that the area swept by a line connecting the planet and the Sun in a given period of time is constant, independent of the location of the planet.
\item The ratio between the orbital periods, $T$, squared of two planets is equal to the ratio of the semi-major axes, $s$, of their orbits cubed:
\begin{align*}
\left(\frac{T_1}{T_2}\right)^2=\left(\frac{s_1}{s_2}\right)^3
\end{align*}
\end{enumerate}

Newton described the attractive force of gravity exerted between two bodies of mass $M_1$ and $M_2$ (which must be point masses) as:
\begin{align*}
\vec F_{12}=-G\frac{M_1M_2}{r^2}\hat r_{21}
\end{align*}
where $\vec F_{12}$ is the force on body 1 from body 2, $r$ is the distance between the two bodies, and $\vec r_{21}$ is the vector from body 1 to body 2. The motion of a body under the influence of only this force will satisfy all of Kepler's Laws.

The gravitational field, $\vec g(\vec r)$, from a body of mass $M$, is defined as the gravitational force that another body would experience per unit mass:
\begin{align*}
\vec g(\vec r)=-G\frac{M}{r^2}\hat r
\end{align*}
The field can be used to determine the corresponding gravitational force, $\vec F_g$, that a body of mass $m$ would experience if located at a position $\vec r$ relative to the $M$:
\begin{align*}
F_g = m \vec g(\vec r)
\end{align*}
When describing the motion of objects near the surface of the Earth, it is thus more precise to refer to $g=\SI{9.8}{N/kg}$ as the magnitude of the Earth's gravitational field at the surface of the Earth, then to refer to $g=\SI{9.8}{m/s^2}$ as the acceleration due to Earth's gravity. The two are only equal if gravitational mass (the $m$ in the above equation) and inertial mass (the $m$ in Newton's Second Law) are the same. 

Gauss' Law, which applies to all inverse-square force laws, can be used to determine the magnitude of the gravitational field from a body of mass $M$, even it if it not a point mass:
\begin{align*}
\oint \vec g(\vec r) \cdot d\vec A = 4\pi G M^{enc}
\end{align*}

Since the force described by Newton's theory is conservative, we can define a potential energy function. The gravitational potential energy of a mass $m$ located a distance $r$ away from a mass $M$ is:
\begin{align*}
U(r) = -G\frac{Mm}{r} + C
\end{align*}
A convenient choice of the constant is $C=0$, as this corresponds to the gravitational potential energy being equal to zero when $m$ is infinitely far away from $M$.

The mechanical energy, $E$, of an object of mass $m$ that is located at a distance $r$ from an object of mass $M$, if gravity is the only conservative force exerted on $m$, is given by:
\begin{align*}
E = K + U = \frac{1}{2}mv^2 - G\frac{Mm}{r}
\end{align*}
where we have explicitly chosen $C=0$, and $v$ is the speed of $m$ relative $M$ (considered to be at rest). Furthermore, if no non-conservative forces do work on the body of mass $m$, the mechanical energy, $E$ is constant.

If the mechanical energy of $m$ is negative, it is gravitationally bound to $M$. Depending on the mechanical energy of $m$ and its velocity at the point of closest approach to $M$, the orbit of $m$ will be described by one of four conical sections (circle, ellipse, parabola, hyperbola).

Einstein's Theory of General Relativity describes gravitation as the bending of space and time caused by the presence of mass and energy. In Einstein's theory, objects follow straight (inertial) paths and do not feel a force of gravity. The curvature of space is what results in their apparent motion not being a straight line. Einstein's theory is based on the Equivalence Principle (inertial and gravitational mass are exactly equal) and the properties of how light propagates according to the Theory of Special Relativity.

\end{chapterSummary}

\newpage
\begin{importantEquations}
This is an important equation
\begin{align*}
E = mc^2
\end{align*}

\end{importantEquations}


\newpage
\section{Thinking about the material}
\subsection{Reflect and research}

\begin{enumerate}
\item When you look at the night sky, how can you tell the difference between a planet and a star?
\item What was the relationship between Tycho Brahe and Johannes Kepler?
\item How did Tycho Brahe collect all the data that Kepler used?
\item How much time elapsed between Kepler publishing his laws and Newton publishing his Universal Theory of Gravity?
\item What was Kepler's original intention when he synthesized Tycho Brahe's observations? What was he hoping to show?
\item What was Ptolemy's theory of gravity based upon?
\item Who was the first to suggest that planets revolved around the Sun instead of the Earth?
\item Explain how the force of gravity from the moon results in tides on both sides of the Earth.
\item Explain what an L1 Lagrange point is, and how it does not violate Kepler's Third Law.
\item How did Eddington confirm that light follows a curved path in a gravitational field?
\end{enumerate}
\subsection{To try at home}

\begin{tQuestion}Try doing this \end{tQuestion}

\subsection{To try in the lab}
Theory project: Prove, based on Newton's Universal Theory of Gravity, that the motion of orbiting bodies is given by a conical section. 

\newpage
\section{Sample problems and solutions}
\subsection{Problems}
\begin{problemParts}{A question\label{Q:chaptertitle:q1}}
\item How close can he get to the hurdle before he has to jump?
\item What maximum height does he reach?
\end{problemParts}

\newpage
\subsection{Solutions}
\begin{solution}{\ref{Q:chaptertitle:q1}}
{
the solution
}
\end{solution}


%
\chapter{Linear momentum and the centre of mass}
\label{chapter:momentumandcm}
In this chapter, we introduce the concepts of linear momentum and of centre of mass. Momentum is a quantity that, like energy, can be defined from Newton's Second Law, to facilitate building models. Since momentum is often a conserved quantity within a system, it can make calculations much easier than using forces. The concepts of momentum and of centre of mass will also allow us to apply Newton's Second Law to systems comprised of multiple particles including solid objects. 

\begin{learningObjectives}{
 \item Understand how to calculate linear momentum.
 \item Understand how to calculate impulse and that it corresponds to a change in momentum.
 \item Understand when and how to apply conservation of linear momentum to model situations.
 \item Understand the difference between elastic and inelastic collisions, and when mechanical energy is conserved.
 \item Understand how to calculate the centre of mass of an object.
}
\end{learningObjectives}

\begin{opening}
\begin{MCquestion}{You hit a pool ball square on with the cue ball. If both balls have the same mass, and you can neglect any ``english'' on the cue ball, what happens to the cue ball?}
\item It stops.
\item It continues, with half of its original speed.
\item It continues, with its original speed.
\item It rebounds, with its original speed.
\end{MCquestion}
\end{opening}


\section{Momentum}
\subsection{Momentum of a point particle}
We can define the momentum, $\vec p$, of a particle of mass $m$ and velocity $\vec v$ as the vector quantity:
\begin{align}
\Aboxed{\vec p = m\vec v}
\end{align}
Since this is a vector equation, it corresponds to three equations, one for each component of the momentum vector. It should be noted that the numerical value for the momentum of a particle is arbitrary, as it depends in which frame of reference the velocity of the particle is defined. For example, your velocity with respect to the surface of the Earth is zero, so your momentum relative to the surface of the Earth is zero. However, relative to the surface of the Sun, your velocity, and momentum, are not zero. As we will see, forces are related to a \textit{changes} in momentum, just as they are related to a change in velocity (acceleration). 

If the particle has a constant mass, then the time derivative of its momentum is given by:
\begin{align*}
\frac{d}{dt}\vec p = \frac{d}{dt}m\vec v = m\frac{d}{dt}\vec v=m\vec a
\end{align*}
and we can write this as Newton's Second Law, since $m\vec a$ must be equal to the vector sum of the forces on the particle of mass $m$:
\begin{align}
\Aboxed{\frac{d}{dt}\vec p = \sum \vec F = \vec F^{net}}
\end{align}
The equation above is the original form in which Newton first developed his theory. It says that the net force on an object is equal to the rate of change of its momentum. \textbf{If the net force on the object is zero, then its momentum is constant} (as is its velocity). In terms of components, Newton's Second Law written for the rate of change of momentum is given by:
\begin{align*}
\frac{dp_x}{dt} =& \sum F_x\\
\frac{dp_y}{dt} =& \sum F_y\\
\frac{dp_z}{dt} =& \sum F_z
\end{align*}

\begin{example}{A particle of mass $m$ is released from rest and allowed to fall freely under the influence of gravity near the Earth's surface (assume that drag is negligible). Is the mechanical energy of the particle conserved? Is the momentum of the particle conserved? If momentum is not conserved, how does momentum change with time? Do your answers change if the force of drag cannot be ignored?}
First, we model the falling particle assuming that there is no force of drag. The only force exerted on the particle is thus its weight. 

The mechanical energy of the particle will be conserved only if there are no non-conservative forces doing work on the particle. Since the force of gravity is the only force acting on the particle, its mechanical energy is conserved.

The total momentum of the particle is not conserved, because the sum of the forces on the particle is not zero. Choosing the $z$ axis to be vertical and positive upwards, Newton's Second Law in the $z$ direction is given by:
\begin{align*}
\sum F_z = -mg=\frac{dp_z}{dt}
\end{align*}
Note that the $x$ and $y$ components of momentum are conserved, since there are no forces with components in that direction. We can find how the $z$ component of the momentum changes with time by taking the anti-derivative of the force with respect to time (from $t=0$ to $t=T$):
\begin{align*}
\frac{dp_z}{dt} &= -mg\\
\int dp_z &= \int_0^T (-mg) dt\\
p_z(T) - p_z(0) &= -mgT\\
\therefore p_z(T) &= p_z(0) - mgT
\end{align*}
where the $z$ component of momentum, $p_z(T)$ at some time $T$, is given by its value at time $t=0$ plus $-mgT$. If the object started at rest ($\vec v=0$), then the magnitude of the momentum, as a function of time, is given by:
\begin{align*}
p(t) = p_z(t) = -mgt
\end{align*}
and indeed changes with time.

If the force of drag were not negligible, there would be a non-conservative force acting on the particle, so its mechanical energy would no longer be conserved. The particle will accelerate until it reaches terminal velocity. During that phase of acceleration, the net force on the particle is not zero (it is accelerating), so its momentum is not conserved. Once the particle reaches terminal velocity, the net force on the particle is zero, and its momentum is conserved from then on.

\textbf{Discussion:} This simple example highlights the fact that mechanical energy and momentum are conserved under different conditions. Just because one is conserved does not mean that the other is conserved. It also shows that Newton's Second Law is a statement about change in momentum, not momentum itself (just like it is a statement about acceleration, change in velocity, not velocity).
\end{example}

\subsection{Impulse}
When we introduced the concept of energy, we started by calculating the ``work'', $W$, done by a force exerted on an object over a specific path between two points:
\begin{align*}
W = \int_A^B \vec F \cdot d\vec l
\end{align*}
We then introduced kinetic energy, $K$, to be that quantity whose change is equal to the net work done on the particle
\begin{align*}
W^{net} = \int_A^B \vec F^{net}\cdot d\vec l = \Delta K
\end{align*}
where the net force, $\vec F^{net}$, is the vector sum of the forces on the particle.

We can do the same thing, but instead of integrating the force over distance, we can integrate it over time. We thus introduce the concept of ``impulse'', $\vec J$, of a force, as that force integrated from an initial time, $t_A$, to a final time, $t_B$:
\begin{align}
\vec J = \int_{t_A}^{t_B}\vec F dt
\end{align}
where it should be clear that impulse is a vector quantity (and the above vector equation thus corresponds to one integral per component). Impulse is, in general, defined as an integral because the force, $\vec F$, could change with time. If the force is constant in time (magnitude and direction), then we can define the impulse without using an integral:
\begin{align*}
\vec J = \vec F \Delta t
\end{align*}
where $\Delta t$ is the amount of time over which the force was exerted. Although the force might never be constant, we can sometimes use the above formula to calculate impulse using an average value of the force.

\begin{checkpoint}
\begin{MCquestion}{What is the SI unit for impulse?}
\item kg per $m/s^2$
\item g per $m/s^2$
\item kg per m/s
\item kg per s
\end{MCquestion}
\end{checkpoint}

\begin{example}{Estimate the impulse that is given to someone's head when they are slapped in the face.}
When we slap someone's face with our hand, our hand exerts a force on their face during the period of time, $\Delta t$, over which our hand is in contact with their face. During that period of time, the force on their face goes from being 0, to some unpleasantly high value, and then back to zero, so the force cannot be considered constant. 

Let us estimate the average magnitude of the slapping force by considering the deceleration of our slapping hand and modelling the motion as one-dimensional. Let us assume that our slapping hand has a mass $m=\SI{1}{kg}$ and that it is has a speed of $\SI{2}{m/s}$ just before it makes contact. Furthermore, let us assume that it is contact with the face for a period of time $\Delta t$. This allows us to find the average acceleration of our hand and thus the average force exerted by the face on our hand to stop it:
\begin{align*}
a &= \frac{\Delta v}{\Delta t}\\
\therefore F &= ma = m  \frac{\Delta v}{\Delta t}
\end{align*}
By Newton's Third Law, the force decelerating our hand has the same magnitude as the force that our hand exerts on the face, allowing us to calculate the impulse given to the person's head:
\begin{align*}
J &= F\Delta t =  \left(m  \frac{\Delta v}{\Delta t}\right) \Delta t = m\Delta v\\
&=(\SI{1}{kg})(\SI{2}{m/s})=\SI{2}{kgm/s}
\end{align*}
\textbf{Discussion:} Note that the impulse given to the head corresponds exactly to the change in momentum of the hand ($\Delta p=m\Delta v$).
\end{example}

So far, we calculated the impulse that is given by a single force. We can also consider the net impulse given to an object by the net force exerted on the object:
\begin{align*}
\vec J^{net} = \int_{t_A}^{t_B}\vec F^{net} dt
\end{align*}
Compare this to Newton's Second Law written out using momentum:
\begin{align*}
\frac{d}{dt}\vec p &= \vec F^{net}\\
\int_{\vec p_A}^{\vec p_B} d\vec p &=  \int_{t_A}^{t_B}\vec F^{net} dt\\
\vec p_B - \vec p_A &=  \int_{t_A}^{t_B}\vec F^{net}dt\\
\therefore \Delta \vec p &= \int_{t_A}^{t_B}\vec F^{net}	 dt
\end{align*}
and we find that the net impulse received by a particle is precisely equal to its change in momentum:
\begin{align}
\Aboxed{\Delta \vec p = \vec J^{net}}
\end{align}
This is similar to the statement that the net work done on an object corresponds to its change in kinetic energy, although one should keep in mind that momentum is a vector quantity, unlike kinetic energy.

\begin{example}{A car moving with a speed of $\SI{100}{km/h}$ collides with a building and comes to a complete stop. The driver and passenger each have a mass of $\SI{80}{kg}$. The driver wore a seat belt that extended during the collision, so that the force exerted by the seatbelt on the driver acted for about $\SI{2.5}{s}$. The passenger did not wear a seat belt and instead was slowed down by the force exerted by the dashboard, over a much smaller amount of time, $\SI{0.2}{s}$. Compare the average decelerating force experienced by the driver and the passenger.}
We can calculate the change in momentum of both people, which will be equal to the impulse they received as they collided with the seatbelt or with the dashboard. Since we know the duration in time that the forces were exerted, we can calculate the average force involved in order to give the required impulse. We can assume that this all happens in one dimension, so we use scalar quantities instead of vectors.

The change in momentum along the direction of motion for either the driver or passenger is given by:
\begin{align*}
\Delta p = p_B - p_A = (0)-p_A=-mv_A
\end{align*}
where $v_A$ is the initial speed of the car, and the final momentum of either person is zero. 

The change in momentum is equal to the impulse received by either person during a period of time $\Delta t$:
\begin{align*}
J=F\Delta t &= \Delta p = -mv_A\\
F&=-m \frac{v_A}{\Delta t}
\end{align*}
For the driver, this corresponds:
\begin{align*}
F=(\SI{80}{kg})\frac{(\SI{27.8}{m/s})}{(\SI{2.5}{s})}=\SI{890}{N}
\end{align*}
and for the passenger:
\begin{align*}
F=(\SI{80}{kg})\frac{(\SI{27.8}{m/s})}{(\SI{0.2}{s})}=\SI{11120}{N}
\end{align*}
The force on the driver is thus comparable to their weight, whereas the passenger experiences an average force that is more than 10 times their weight.

\textbf{Discussion:} Any mechanism that results in a longer collision time will help to reduce the forces that are involved. This is why cars are designed to crumple in head-on collisions. We can understand this in terms of the crumpling of the car absorbing some of the kinetic energy of the car, as well as lengthening the time of the collision so that the forces involved are smaller. Note that we did not need to use impulse to calculate the average force, since we could have just used kinematics to determine the acceleration and Newton's Second Law to calculate the corresponding force. Using impulse is equivalent by construction, but sometimes, it is easier mathematically.
\end{example}

TODO: Question Library question: Give a function F(t), calculate the change in momentum resulting from that force over a certain range of time (i.e. they need to take an integral). 

\subsection{Systems of particles: internal and external forces}
So far, we have only used Newton's Second Law to describe the motion of a single point mass particle or to describe the motion of an object whose orientation we did not need to describe (e.g. a block sliding down a hill). In this section, we consider what happens when there are multiple point particles that form a ``system''.

In physics, we loosely define a system as the ensemble of objects/particles that we wish to describe. So far, we have only described systems made of one particle, so describing the motion of the system was equivalent to describing the motion of that single particle. A  system of two particles could be, for example, two billiard balls on a pool table. To describe that system, we would need to provide functions that describe the positions, velocities, and forces exerted on both balls. We can also define functions/quantities that describe the system as a whole, rather than the details. For example, we can define the total kinetic energy of the system, $K$, corresponding to the sum of kinetic energies of the two balls. We can also define the total momentum of the system, $\vec P$, given by the vector sum of the momenta of the two balls.

When considering a system of multiple particles, we distinguish between \textbf{internal} and \textbf{external} forces. Internal forces are those forces that the particles on the system exert on each other. For example, if the two billiard balls in the system collide with each other, they will each exert a force on the other during the collision; those forces are internal. External forces are all other forces exerted on the particles of the system. For example, the force of gravity and the normal force from the pool table are both external forces exerted on the balls in the system (exerted by the Earth, or by the pool table, neither of which we considered to be part of the system). The force exerted by a person hitting one of the balls with a pool queue is similarly an external force. What we consider to be a system is arbitrary; we could consider the pool table and the Earth to be part of the system along with the two balls; in that case, the normal force and the weight of the balls would become internal forces. The classification of whether a force is internal or external to a system of course depends on what is considered part of the system.

\begin{checkpoint}
\begin{MCquestion}{Two pool balls crash against each other. Is this (minuscule) force of gravity exerted from one ball to the other an internal or external force?}
\item Internal, because it is exerted by a particle in the system on another. %correct
\item External, because one of the particles is not in the system.
\end{MCquestion}
\end{checkpoint}

The key property of internal forces is that \textbf{the vector sum of the internal forces in a system is zero}. Indeed, Newton's Third Law states that for every force exerted by object A on object B, there is a force that is equal in magnitude and opposite in direction exerted by object B on object A. If we consider both objects to be in the same system, then the sum of the internal forces between objects A and B must sum to zero. It is important to note that this is quite different than what we have discussed so far about summing forces. The forces that sum to zero are exerted on \textit{different} objects. Thus far, we had only ever considered summing forces that are exerted on the same object in order to apply Newton's Second Law. We have never encountered a situation where ``action'' and ``reaction'' forces are summed together, because they act on different objects.

\subsection{Conservation of momentum}
Consider a system of two particles with momenta $\vec p_1$ and $\vec p_2$.  Newton's Second Law must hold for each particle:
\begin{align*}
\frac{d\vec p_1}{dt}&=\sum_k \vec F_{1k}\\
\frac{d\vec p_2}{dt}&=\sum_k \vec F_{2k}
\end{align*}
where $F_{ik}$ is the $k$-th force that is acting on particle $i$.  We can sum these two equations together:
\begin{align*}
\frac{d\vec p_1}{dt}+\frac{d\vec p_2}{dt} &= \sum_k \vec F_{1k} + \sum_k \vec F_{2k}
\end{align*}
The quantity on the right is the sum of the forces exerted on particle 1 plus the sum of the forces exerted on particle 2. In other words, it is the sum of all of the forces exerted on all of the particles in the system, which we can write as a single sum. On the left hand side, we have the sum of the two time derivatives of the momenta, which is equal to the time-derivative of the sum of the momenta. We can thus re-write the equation as:
\begin{align*}
\frac{d}{dt}(\vec p_1 + \vec p_2) = \sum \vec F
\end{align*}
where, again, the sum on the right is the sum over all of the forces exerted on the system. Some of those forces are external (e.g. gravity exerted by Earth on the particles), whereas some of the forces are internal (e.g. a contact force between the two particles). Dividing up the sum into a sum over all external forces ($\vec F^{ext}$) and a sum over internal forces ($\vec F^{int}$):
\begin{align*}
\sum \vec F = \sum \vec F^{ext} + \sum \vec F^{int} 
\end{align*}
The sum of the internal forces is zero:
\begin{align*}
\sum \vec F^{int} = 0
\end{align*}
because for every force that particle 1 exerts on particle 2, there will be an equal and opposite force exerted by particle 2 on particle 1. We thus have:
\begin{align*}
\frac{d}{dt}(\vec p_1 + \vec p_2) = \sum \vec F^{ext}
\end{align*}
Furthermore, if we introduce the ``total momentum of the system'', $\vec P=\vec p_1 + \vec p_2$, as the sum of the momenta of the individual particles, we find:
\begin{align*}
\frac{d\vec P}{dt} &= \sum \vec F^{ext}
\end{align*}
which is the equivalent of Newton's Second Law for a system where, $\vec P$, is the total momentum of the system, and the sum of the forces is only over external forces to the system.

Note that the derivation above easily extends to any number, $N$, of particles, even though we only did it with $N=2$. In general, for the ``ith particle'', with momentum $\vec p_i$, we can write Newton's Second Law:
\begin{align*}
\frac{d\vec p_i}{dt}=\sum_k \vec F_{ik}
\end{align*} 
where the sum is over only those forces exerted on particle $i$. Summing the above equation for all $N$ particles in the system:
\begin{align*}
\frac{d}{dt}\sum_i \vec p_i=\sum \vec F^{ext} + \sum \vec F^{int}
\end{align*}
where the sum over internal forces will vanish for the same reason as above. Introducing the total momentum of the system, $\vec P$:
\begin{align*}
\vec P = \sum_i \vec p_i\\
\end{align*}
We can write an equation for the time-derivative of the total momentum of the system:
\begin{align}
\Aboxed{\frac{d\vec P}{dt} &= \sum \vec F^{ext}}
\end{align}
where the sum of the forces is the sum over all forces external to the system. Thus, \textbf{if there are no external forces on a system, then the total momentum of that system is conserved} (if the time-derivative of a quantity is zero then that quantity is constant).

We already argued in the previous section that we can make all forces internal if we choose our system to be large enough. If we make the system be the Universe, then there are no forces external to the Universe, and the total momentum of the Universe must be constant:
\begin{align*}
\frac{d\vec P^{Universe}}{dt} &= \sum_{Universe} \vec F^{ext} = 0 \\
\therefore \vec P^{Universe}&=\text{constant}
\end{align*}

In summary, we saw that:
\begin{itemize}
\item If no forces are exerted on a single particle, then the momentum of that particle is constant (conserved).
\item In a system of particles, the total momentum of the system is conserved if there are no external forces on the system.
\item If there are no non-conservative forces exerted on a particle, then that particle's mechanical energy is constant (conserved).
\item In a system of multiple particles, the total mechanical energy of the system will be conserved if there are no non-conservative forces exerted on the system.
\end{itemize} 
When we refer to a force being ``exerted on a system'', we mean exerted on one or more of the particles in the system. In particular, the sum of the work done by internal forces is not necessarily zero, so \textbf{energy and momentum are thus conserved under different conditions}.

\begin{example}{\capfig{0.8\textwidth}{figures/MomentumAndCM/train.png}{\label{fig:momentumandcm:train}A train with $N$ cars of mass $m$ about to collide with a car of mass $m$ that is at rest on the track.} 
Consider a train made of $N$ cars of equal mass $m$ that is travelling at constant speed $v$ along a straight piece of track where friction and drag are negligible, as depicted in Figure \ref{fig:momentumandcm:train}. An empty car of mass $m$ was left at rest on the track in front of the train. The train collides with the empty car which stays attached to the front of the train. What is the speed of the train after the collision? Is the total mechanical energy of the system conserved?}
When the train collides with the car, it will exert a ``collision'' force on the car, and the car will exert an opposite force on the train. If we consider both of the train and the car as being part of the same system, then those collision forces will be internal, and the momentum of the system (train + car) will be conserved. The train and car both experience external forces from Earth's gravity and the normal force from the train tracks. However, those two sets of forces cancel each other out, since neither the train nor the car have any acceleration in the vertical direction (the sum of the forces on each object has no net vertical component). Thus, there are no net external forces on the car+train system, and the total momentum of the system is conserved through the collision.

We can model this system in one dimension (along the track), which we call the $x$ axis. We choose the ground as a frame of reference, the positive direction to correspond to the initial velocity of the train, and the origin to be located where the car initially starts. Before the collision, the $x$ component of the momenta of the train (mass $Nm$) and car (mass $m$) are:
\begin{align*}
p_{train}&=Nmv\\
p_{car}&=0
\end{align*}
After the collision, the car is attached to the train (and thus has the same speed, $v'$), so the momenta of the train and car are:
\begin{align*}
p'_{train}&=Nmv'\\
p'_{car}&=mv'
\end{align*}
where the primes $'$ denote quantities after the collision. Applying conservation of momentum to the system, the total momentum before and after the collision must be equal:
\begin{align*}
p_{train}+p_{car}&=p'_{train}+p'_{car}\\
\therefore Nmv &= Nmv' +mv'\\
\therefore v' &=\frac{N}{N+1}v
\end{align*}
and the speed of the train with the additional car attached is reduced by a factor $N/(N+1)$ compared to what it was before the collision.

We can check to see if the mechanical energy of the system is conserved, since we know the speeds of the train and car before and after the collision. Since all of the motion is horizontal, gravity and the normal force do no work on either the train or car, so their mechanical energy can be taken as their kinetic energy (their gravitational potential energy does not change after the collision). The total mechanical energy of the system, $E$, before the collision is the kinetic energy of the train:
\begin{align*}
E= \frac{1}{2}Nmv^2
\end{align*}
The total mechanical energy of the system, $E'$, after the collision is:
\begin{align*}
E' &= \frac{1}{2}Nmv'^2 + \frac{1}{2}mv'^2 = \frac{1}{2}(N+1)mv'^2 \\
&=\frac{1}{2}(N+1)m \left( \frac{N}{N+1}v \right)^2\\
&=\frac{1}{2}m\frac{N^2}{N+1}v^2
\end{align*}
and we see that $E'<E$, and thus that the total mechanical energy of the system is not conserved (it is reduced after the collision).

\textbf{Discussion: }We could have solved this problem by carefully modelling the force exerted by the car on the train during the collision, which would have allowed us to find the speed of the train after the collision using its acceleration. This would have required a detailed model for that force, which we do not have. However, by realizing that the train and car could be considered as a system with no net external forces exert on it, we were able to easily find the speed of the train after the collision using conservation of momentum.

We also found that mechanical energy was not conserved. This makes physical sense because, for the car to remain attached to the train, there presumably had to be some significant forces in play that ``crushed'' the car into the train. Some of the initial kinetic energy of the train was used to deform the train and the car during the collision. We can also think of deforming a material as giving it energy. Sometimes that energy is recoverable (e.g. compressing a spring), sometimes, it is not (e.g. crushing a car).

If the car and train were equipped with large springs to absorb the energy of the impact, the collision could have conserved mechanical energy, as the springs compress and then expand back. The speed of the car and train would then be different after the collision in this case (see example \ref{ex:momentumandcm:1delastic}). It is a feature of collisions where the two bodies remain attached to each other that mechanical energy is not conserved.
\end{example}

\section{Collisions}
In this section we go through a few examples of applying conservation of momentum to model collisions. Collisions can loosely be defined as events where the momenta of individual particles in a system are different before and after the event.

We distinguish between two types of collisions: \textbf{elastic} and \textbf{inelastic} collisions. Elastic collisions are those for which the total mechanical energy of the system is conserved during the collision (i.e. it is the same before and after the collision). Inelastic collisions are those for which the total mechanical energy of the system is not conserved. In either case, to model the system, one chooses to define the system such that there are no external forces on the system so that total momentum is conserved.

\subsection{Inelastic collisions}
In this section, we give a few examples of modelling inelastic collisions. Inelastic collisions are usually easier to handle mathematically, because one only needs to consider conservation of momentum and does not use conservation of energy (which usually involves equations that are quadratic in the speeds because of the kinetic energy term). 
\begin{example}{
\capfig{0.4\textwidth}{figures/MomentumAndCM/skaters.png}{\label{fig:momentumandcm:skaters}One skater pushing another on a frictionless horizontal surface.}
You (mass $m_s$) and your friend (mass $m_f$) face each other on ice skates on an ice surface that is slippery enough that friction can be considered negligible, as shown in Figure \ref{fig:momentumandcm:skaters}. You shove your friend away from you so that he moves with velocity $\vec v_f$ away from you (the velocity is measured relative to the ice). Is the collision elastic? What is your speed relative to the ice after you shoved your friend?}
We can consider the system as being comprised of you and your friend. There are no net external forces on the system (gravity and normal forces cancel each other), so the momentum of the system will be conserved. 

The mechanical energy will not be conserved. You had to use chemical potential energy stored in your muscles to shove your friend. Thus, external energy (i.e. not mechanical energy from you or your friend) was injected into the system, and we should expect the total mechanical energy to be larger after the collision. 

Before the collision, both you and your friend have zero speed, and thus zero kinetic energy and zero momentum. After the collision, your friend as a velocity $\vec v_f$. We can use conservation of total momentum, $\vec P$, to determine your velocity, $\vec v_s$, after the collision. 
\begin{align*}
\vec P &=\pvec P'\\
0 &= m_s\vec v_s + m_f\vec v_f\\
\therefore \vec v_s &= -\frac{m_f}{m_s}\vec v_f
\end{align*}
where primes ($'$) denote a quantity after the collision. We find that your velocity is in the opposite direction from that of your friend. Before the collision, the mechanical energy, $E$, of the system is zero (we can ignore gravitational potential energy, since everything is in the horizontal plane). After the collision, the mechanical energy, $E'$, is:
\begin{align*}
E' = \frac{1}{2}m_sv_s^2+\frac{1}{2}m_fv_f^2
\end{align*}
which is clearly bigger than the mechanical energy before the collision (i.e. 0), as we suspected it would be.

\textbf{Discussion:} We find that you recoil in the opposite direction, which makes sense. If you push your friend in one direction, Newton's Third Law says that your friend pushes you in the opposite direction. Your speed furthermore depends on the ratio of your friend's mass to yours. This also makes sense, because if you both feel the same force, the person with the smallest mass will have the highest speed; if your mass is higher than your friend's, then your speed after the collision will be smaller than your friend's.

We also saw that mechanical energy was not conserved. In terms of energy, we can explain this by saying that you burned up chemical potential energy stored in your muscles in order to shove your friend. Because we included both you and your friend in the system, the shove was an internal force and momentum is conserved. Of course, if we had considered only you as the system, then your momentum would not have been conserved during the collision. 

The type of collision that we described here is also sometimes called an ``explosion''. You can imagine all of the parts that make up a bomb as small particles. When the bomb explodes, chemical potential energy is converted into the kinetic energy of the bomb fragments. If you consider all of the particles/fragments of the bomb as a system, then the total momentum of all of the bomb fragments is conserved (and equal to zero if the bomb was initially at rest). Again, mechanical energy would not be conserved (and would increase) as the chemical potential energy is converted into mechanical energy.
\end{example}

\begin{example}{
\capfig{0.4\textwidth}{figures/MomentumAndCM/protonnucleus.png}{\label{fig:momentumandcm:protonnucleus}A proton of mass $m_p$ colliding inelastically with a nucleus of mass $m_N$.}
A proton of mass $m_p$ and initial velocity $\vec v_p$ collides inelastically with a nucleus of mass $m_N$ at rest, as shown in Figure \ref{fig:momentumandcm:protonnucleus}. A coordinate system is set up as shown, such that the initial velocity of the proton is in the $x$ direction. After the collision, the proton's speed is measured to be $v'_p$ and its velocity vector is found to make an angle $\theta$ with the $x$ axis as shown. What is the velocity vector of the nucleus after the collision? Assume that the collision takes place in vacuum.}
As a system, we consider the proton and the nucleus together, so that the total momentum of the system is conserved during the collision, as no other external forces are exerted on the two particles (since they are in vacuum). Because momentum is a vector, each component of the total momentum, $\vec P$, is conserved during the collision:
\begin{align*}
\vec P &= \pvec P'\\
\therefore P_x &= P'_x\\
\therefore P_y &= P'_y
\end{align*}
where, as usual, primes ($'$) denote quantities after the collision. After the collision, both particles will have velocity vectors that have $x$ and $y$ components. Let the velocity vector of the nucleus after the collision be $\pvec v'_N$ and let $\phi$ be the angle that it makes with the $x$ axis, as shown in Figure \ref{fig:momentumandcm:protonnucleus}. 

We can start by considering the conservation of the $x$ component of the total momentum. The initial and final momenta in the $x$ direction are given by:
\begin{align*}
P_x &= m_p v_p\\
P'_x &= m_p v'_p\cos\theta + m_N v'_N\cos\phi\\
\therefore m_p v_p &= m_p v'_p\cos\theta + m_N v'_N\cos\phi
\end{align*}
which gives us a first equation to determine the final velocity of the nucleus.

The $y$ component of the total momentum before the collision is zero since we chose the coordinate system such that the initial velocity of the proton is in the $x$ direction. The initial and final momenta in the $y$ direction are given by:
\begin{align*}
P_y &= 0\\
P'_y &= m_p v'_p\sin\theta - m_N v'_N\sin\phi\\
\therefore m_p v'_p\sin\theta &= m_N v'_N\sin\phi
\end{align*}
which gives us a second equation to solve for the velocity of the nucleus. With the two equations from momentum conservation, we can solve for the magnitude and direction of the velocity of the nucleus. From the $y$ component of momentum conservation, we can find an expression for the speed of the nucleus:
\begin{align*}
m_p v'_p\sin\theta &= m_N v'_N\sin\phi\\
\therefore v'_N &= \frac{m_p}{m_N}v'_p\sin\theta \frac{1}{\sin\phi}
\end{align*}
which we can substitute into the $x$ equation for momentum conservation to solve for the angle $\phi$:
\begin{align*}
m_p v_p &= m_p v'_p\cos\theta + m_N v'_N\cos\phi\\
m_p v_p &= m_p v'_p\cos\theta + m_N\frac{m_p}{m_N}v'_p\sin\theta \frac{\cos\phi}{\sin\phi} \\
v_p &= v'_p\cos\theta + v'_p\sin\theta \frac{1}{\tan\phi}\\
\therefore \tan\phi &=  \frac{v'_p\sin\theta}{v_p-v'_p\cos\theta}
\end{align*}
If we were given numbers for the initial and final speed of the proton, as well as the angle $\theta$, we would be able to find a value for the angle $\phi$, which we could then use to determine the final speed of the nucleus:
\begin{align*}
 v'_N &= \frac{m_p}{m_N}v'_p\sin\theta \frac{1}{\sin\phi}
\end{align*}
\textbf{Discussion:} By using the conservation of momentum equation and writing out the $x$ and $y$ components, we were able to find two equations to determine the magnitude and direction of the nucleus' velocity after the collision. In the limit where $m_N >> m_p$, the final speed of the nucleus would be very small (close to zero). 
\end{example}

\begin{example}{
\capfig{0.7\textwidth}{figures/MomentumAndCM/ballistic.png}{\label{fig:momentumandcm:ballistic}A bullet of mass $m$ strikes and embeds itself into a ballistic pendulum of mass $M$.}
A ballistic pendulum is a device that can be built to measure the speed of a projectile. The pendulum is constructed such that the projectile is fired at the bob of the pendulum (typically a block of wood) which then swings as illustrated in Figure \ref{fig:momentumandcm:ballistic}, with the projectile embedded within. By measuring the height that is reached by the pendulum's bob, one can determine the speed of the projectile before it collided with the pendulum. If a ballistic pendulum with a mass $M$ suspended at the end of strings of length $L$ is observed to rise by a height $h$ after being struck by a bullet of mass $m$, how fast was the bullet moving?}
We can model this situation by dividing it into three phases:
\begin{enumerate}
\item Before the bullet collides with the pendulum, only the bullet has momentum in the $x$ direction.
\item Immediately after the \textbf{inelastic} collision, the bullet and pendulum form a combined object of mass $M+m$ that has the same momentum as the bullet, in the $x$ direction, before the pendulum starts to swing upwards.
\item The pendulum with the embedded bullet swings upwards until its kinetic energy is zero.
\end{enumerate}
The collision between the bullet and pendulum is inelastic, because some of the kinetic energy of the bullet is used to deform the bullet and the pendulum. In general, any collision where two objects end up ``stuck together'' is inelastic.

In order to model the pendulum's motion we first apply conservation of momentum to determine the speed, $v'$, of the pendulum and embedded bullet just after the collision. Applying conservation of momentum in the $x$ direction to the system formed by the pendulum and the bullet, just before and after the collision, we have:
\begin{align*}
P &= mv\\
P' &= (M+m)v'\\
\therefore mv &= (M+m)v'\\
\therefore v' &= \frac{m}{m+M}v
\end{align*}
where $P$ and $P'$ are the initial and final momenta of the system, respectively. The pendulum with the bullet embedded in it will thus have a speed of $v'$ at the bottom of the pendulum's motion, before it swings upwards. 

We can now use conservation of energy to model the swinging motion since, at that point, only tension and gravity act on the pendulum, and there are no non-conservative forces. If we choose the origin to be the location of the pendulum at the bottom of its trajectory, its initial gravitational potential energy is zero and its initial mechanical energy, $E$, is given by:
\begin{align*}
E = \frac{1}{2}(m+M) v'^2 
\end{align*}
At the top of the trajectory, the pendulum with the embedded bullet will stop and have no kinetic energy. The mechanical energy at the top of the trajectory, $E'$, is thus equal to the gravitational potential energy of the pendulum at a height $h$ above the origin:
\begin{align*}
E' = (m+M)gh
\end{align*}
Applying conservation of mechanical energy allows us to find the initial speed of the bullet:
\begin{align*}
E &= E'\\
\frac{1}{2}(m+M) v'^2 &= (m+M)gh\\
v'^2 &= 2gh\\
\left( \frac{m}{m+M}v\right)^2&= 2gh\\
\therefore v &= \frac{m+M}{m} \sqrt{2gh}
\end{align*}
where is the second last line we used the expression for $v'$ that we obtained from conservation of momentum.

\textbf{Discussion: }This example showed a situation in which momentum and energy were both conserved, but not at the same time. This example also highlighted how, by using conservation laws, one can derive models that are much easier to solve mathematically than if one had to model all of the forces involved.
\end{example}


\subsection{Elastic collisions}
In this section, we give a few examples of modelling elastic collisions. Even though it is mechanical energy that is conserved in an elastic collision, one can almost always simplify this to only kinetic energy being conserved. If a collision takes place in a well localized position in space (i.e. before and after the collision are the same point in space), then the potential energies of the objects involved will not change, thus any change in their mechanical energy is due to a change in kinetic energy.


\begin{example}{\label{ex:momentumandcm:1delastic}
\capfig{0.6\textwidth}{figures/MomentumAndCM/1delastic.png}{\label{fig:momentumandcm:1delastic}Two blocks about to collide elastically.}
A block of mass $M$ moves with velocity $\vec v_M$ in the $x$ direction, as shown in Figure \ref{fig:momentumandcm:1delastic}. A block of mass $m$ is moving with velocity $\vec v_m$ also in the $x$ direction and collides elastically with block $M$. Both blocks slide with no friction on the horizontal surface. What are the velocities of the two blocks after the collision?}
Because this is an elastic collision, both the total momentum and total mechanical energy are conserved. Equating the total momentum before and after the collision, and considering only the $x$ component gives the following equation:
\begin{align*}
\vec P &=\pvec P'\\
Mv_M+mv_m&=Mv'_M+mv'_m
\end{align*}
where the primes ($'$) correspond to the quantities after the collision. Note that, in principle, the $x$ components of the velocities ($v_M$, $v'_M$, $v_m$, $v'_m$) could be negative numbers if the corresponding block is moving in the negative $x$ direction.

For the mechanical energy of the two blocks, we only need to consider their kinetic energy since their gravitational potential energies are the same before and after the collision on the horizontal surface.
\begin{align*}
E &=E'\\
\frac{1}{2}Mv_M^2+\frac{1}{2}mv_m^2&=\frac{1}{2}Mv'^2_M+\frac{1}{2}mv'^2_m\\
\therefore Mv_M^2+mv_m^2&=Mv'^2_M+mv'^2_m
\end{align*}
where we cancelled the factor of one half in the last line. This gives two equations (conservation of energy and momentum) and two unknowns (the two speeds after the collision). This is not a linear system of equations, because the equation from conservation of energy is quadratic in the speeds.

The following method allows many models for elastic collisions between two particles to be solved easily by converting the quadratic equation from energy conservation into an equation that is linear in the speeds. First, write both equations so that the quantities related to each particle are on opposite sides of the equation. For momentum, this gives:
\begin{align}
\label{eq:momentumandcm:exptemp}
Mv_M+mv_m&=Mv'_M+mv'_m\nonumber\\
\therefore M(v_M-v'_M) &= m(v'm-v_m)
\end{align}
For conservation of energy, this gives:
\begin{align*}
Mv_M^2+mv_m^2&=Mv'^2_M+mv'^2_m\\
\therefore  M(v_M^2-v'^2_M)&= M(v'^2_m-v^2_m)
\end{align*}
which we can re-write as:
\begin{align*}
M(v_M^2-v'^2_M)&= M(v'^2_m-v^2_m)\\
M(v_M-v'_M)(v_M+v'_M)&= M(v'_m-v_m)(v'_m+v_m)
\end{align*}
We can then divide this by the momentum equation (Equation \ref{eq:momentumandcm:exptemp}):
\begin{align*}
\frac{M(v_M-v'_M)(v_M+v'_M)}{M(v_M-v'_M)}&= \frac{M(v'_m-v_m)(v'_m+v_m)}{m(v'm-v_m)}\\
\therefore v_M+v'_M&=v'_m+v_m
\end{align*}
which gives us an equation that is much easier to work with, since it is linear in the speeds. If we re-arrange this last equation back so that quantities before and after the collision are on different sides of the equality:
\begin{align*}
\Aboxed{v_M-v_m &= - (v'_M-v'_m)}
\end{align*}
we can see that the relative speed between $M$ and $m$ is the same before and after the collision. That is, if block $M$ ``saw'' block $m$ approaching with a speed of $\SI{3}{m/s}$ before the collision, it would ``see'' block $m$ moving \textit{away} with speed $\SI{3}{m/s}$ after the collision, regardless of the actual directions and velocities of the block, if the collision was elastic.

By using this equation with the original conservation of momentum equation, we now have two equations and two unknowns that are easy to solve:
\begin{align*}
v_M-v_m &= - (v'_M-v'_m)\\
Mv_M+mv_m&=Mv'_M+mv'_m
\end{align*}
Solving for $v'_m$ in both equations gives:
\begin{align*}
v_M-v_m &= - (v'_M-v'_m)\\
\therefore v'_m &= v_M+v'_M-v_m\\
Mv_M+mv_m&=Mv'_M+mv'_m\\
\therefore v'_m&=\frac{1}{m}(Mv_M+mv_m-Mv'_M)
\end{align*}
Equating the two expressions for $v'_m$ allows us to solve for $v'_M$:
\begin{align*}
\frac{1}{m}(Mv_M+mv_m-Mv'_M)&=v_M+v'_M-v_m\\
Mv_M+mv_m-Mv'_M&=mv_M+mv'_M-mv_m\\
(M-m)v_M+2mv_m&=(M+m)v'_M\\
\therefore v'_M&=\frac{M-m}{M+m}v_M+\frac{2m}{M+m}v_m
\end{align*}
One can easily solve for the other speed, $v'_m$:
\begin{align*}
\therefore v'_m &= \frac{m-M}{M+m}v_m+\frac{2M}{M+m}v_M
\end{align*}
And writing these together:
\begin{align*}
v'_M&=\frac{M-m}{M+m}v_M+\frac{2m}{M+m}v_m\\
v'_m &= \frac{m-M}{M+m}v_m+\frac{2M}{M+m}v_M
\end{align*}
\textbf{Discussion:} The formulas that we obtained above are valid for any one dimensional elastic collision. 
\end{example}

\begin{checkpoint}
\begin{MCquestion}{Two trains of equal masses collide elastically on a track. If train A was at rest and train B had a speed v, what are the speeds of the trains after the collision?}
\item Both trains A and B travel away from each other with speeds $\frac{1}{2}$v.
\item Train A will be at rest and train B will move away with a speed v.
\item Both trains A and B will stick together and move at a speed of v.
\item Train B will be at rest and train A will move away at a speed of v. %correct
\end{MCquestion}
\end{checkpoint}

\begin{example}{
\capfig{0.5\textwidth}{figures/MomentumAndCM/protonproton.png}{\label{fig:momentumandcm:protonproton}A proton elastically collides with a proton at rest.}
 A proton of mass $m$ and initial velocity $\vec v_1$ collides elastically with a second proton that is at rest. After the collision, the two protons have velocities $\pvec v'_1$ and $\pvec v'_2$, as shown in Figure \ref{fig:momentumandcm:protonproton}. Show that the velocity vectors of the two protons are perpendicular after the collision.}
This example highlights a particular feature of elastic collisions when the two objects have the same mass and one of the objects is initially at rest. The conservation of momentum for the system comprised of the two protons can be written as:
\begin{align*}
m\vec v_1 &= m\pvec v'_1 + m\pvec v'_2\\
\vec v_1 &= \pvec v'_1 + \pvec v'_2
\end{align*}
where the left hand side corresponds to the initial total momentum and the right hand side to the total momentum after the collision. In the second line, we cancelled out the mass, and obtained a vector relation between the velocity vectors. We can graphically illustrate the vector relation as in Figure \ref{fig:momentumandcm:vsum} which shows the triangle that is formed by adding the two outgoing velocity vectors to obtain the initial velocity vector.
\capfig{0.2\textwidth}{figures/MomentumAndCM/vsum.png}{\label{fig:momentumandcm:vsum}Graphical illustration of the relation between the initial and final velocity vectors as a vector sum.}
Conservation of kinetic energy for the collision can be written as:
\begin{align*}
\frac{1}{2}mv_1^2 &= \frac{1}{2}mv'^2_1+\frac{1}{2}mv'^2_2\\
v_1^2 &= v'^2_1+ v'^2_2
\end{align*}
where the left hand side corresponds to the initial kinetic energy and the right hand side to the final kinetic energy. We cancelled the mass and factor of one half in the second line. This last equation gives a relation between the magnitudes of the velocity vectors. By comparing the equation above to Pythagoras' theorem, and by inspecting the triangle in Figure \ref{fig:momentumandcm:vsum}, it is clear that the triangle must be a right angle triangle, and thus that $\pvec v'_1$ and $\pvec v'_2$ must be perpendicular.
\end{example}

\subsection{Frames of reference}
\begin{review}
Before proceeding, you may wish to review sections \ref{sec:DescribingMotionIn1D:Relative motion} and  \ref{sec:DescribingMotionInND:Motion in two dimensions:Using vectors to describe motion in two dimensions} on expressing velocities in different frames of reference.
\end{review}

Because the momentum of a particle is defined using the velocity of the particle, its value depends on the reference frame that we chose to measure that velocity. In some cases, it is useful to apply momentum conservation in a frame of reference where the total momentum of the system is zero. For example, consider two particles of mass $m_1$ and $m_2$, moving towards each other with velocities $\vec v_1$ and $\vec v_2$, respectively, as measured in a frame of reference $S$, as illustrated in Figure \ref{fig:momentumandcm:2particles}.
\capfig{0.3\textwidth}{figures/MomentumAndCM/2particles.png}{\label{fig:momentumandcm:2particles}Two particles moving towards each other.}
In the frame of reference $S$, the total momentum, $\vec P$, of the two particles can be written:
\begin{align*}
\vec P = m_1\vec v_1 + m_2\vec v_2
\end{align*}
Consider a frame of reference, $S'$, that is moving with velocity $\vec v_{CM}$ relative to the frame of reference $S$. In that frame of reference, the velocities of the two particles are different and given by:
\begin{align*}
\pvec v'_1&=\vec v_1- \vec v_{CM}\\
\pvec v'_2&=\vec v_2- \vec v_{CM}
\end{align*}
The total momentum, $\pvec P'$, in the frame of reference $S'$ is then given by\footnote{Note that we are using primes ($'$) to denote quantities in a different reference frame, not after a collision.}:
\begin{align*}
\vec P' &= m_1\pvec v'_1 + m_2 \pvec v'_2\\
&=m_1(\vec v_1- \vec v_{CM})+m_2(\vec v_2- \vec v_{CM})\\
&= m_1\vec v_1 + m_2\vec v_2 - (m_1+m_2) \vec v_{CM}
\end{align*}
We can choose the velocity of the frame, $\vec v_{CM}$ such that the total momentum in that frame of reference is zero:
\begin{align*}
\vec P' &= 0\\
m_1\vec v_1 + m_2\vec v_2 - (m_1+m_2) \vec v_{CM} &=0\\
\therefore \vec v_{CM} &= \frac{m_1\vec v_1 + m_2\vec v_2 }{m_1+m_2}
\end{align*}
This ``special'' frame of reference, in which the total momentum of the system is zero, is called the ``centre of mass frame of reference''. The velocity of centre of mass frame of reference can easily be obtained if there are $N$ particles involved instead of two:
\begin{align}
\Aboxed{\therefore \vec v_{CM} = \frac{m_1\vec v_1 + m_2\vec v_2 + m_3 \vec v_3 + \dots }{m_1+m_2+m_3+\dots}=\frac{\sum m_i\vec v_i}{\sum m_i}}
\end{align}
Again, you should note that because the above equation is a vector equation, it represents one equation per component of the vectors. For example, the $x$ component of the velocity of the centre of mass frame of reference is given by:
\begin{align*}
\therefore  v_{CMx} = \frac{m_1 v_{1x} + m_2v_{2x} + m_3 v_{3x} + \dots }{m_1+m_2+m_3+\dots}=\frac{\sum m_iv_{ix}}{\sum m_i}
\end{align*}


\begin{example}{\capfig{0.4\textwidth}{figures/MomentumAndCM/labframe.png}{\label{fig:momentumandcm:labframe}One block approaching another identical block at rest, as seen in the lab frame of reference.} In the frame of reference of a lab, a block of mass $m$ has a velocity $\vec v_1$ directed along the positive $x$ axis and is approaching a second block of mass $m$ that is at rest ($\vec v_2=0$), as shown in Figure \ref{fig:momentumandcm:labframe}. What is the velocity of the centre of mass frame? What is the velocity of each block in the centre of mass frame? Verify that the total momentum is zero in the centre of mass frame.}
Since this is a one dimensional situation, we only need to evaluate the $x$ component of the velocity of the centre of mass:
\begin{align*}
\vec v_{CM} &= \frac{m_1\vec v_1 + m_2\vec v_2 }{m_1+m_2}\\
\therefore v_{CMx} &= \frac{m_1 v_{1x} + m_2 v_{2x}}{m_1+m_2}\\
&=\frac{mv_1 + m(0) }{m+m}\\
&=\frac{1}{2}v_1
\end{align*}
The centre of mass frame of reference is thus also moving along the positive direction of the $x$ axis, but with a speed that is half of that of the moving block. In the centre of mass frame of reference, it appears that the block on the left is slower than in the lab frame and that the block on the right is moving in the negative $x$ direction. The velocities of the two blocks in the centre of mass frame of reference are given by:
\begin{align*}
v'_1&=v_1-v_{CMx}=\frac{1}{2}v_1\\
v'_2&=(0)-v_{CMx}=-\frac{1}{2}v_1
\end{align*}
Thus, in the reference frame of the centre of mass, the two block are approaching each other with the same speed ($v_1/2$), which is only the case because the two blocks have the same mass. The blocks, as viewed in the centre of mass frame of reference, are shown in Figure \ref{fig:momentumandcm:cmframe}.
\capfig{0.4\textwidth}{figures/MomentumAndCM/cmframe.png}{\label{fig:momentumandcm:cmframe}In the centre of mass frame of reference, the block approach each other with the same speed, because they have the same mass.} 
Clearly, the total momentum is zero in the centre of mass frame of reference:
\begin{align*}
\vec P' = m\pvec v'_1+ m\pvec v'_2 = m \left(\frac{1}{2}\vec v_1 - \frac{1}{2}\vec v_1\right) = 0
\end{align*}

\textbf{Discussion:} As we have seen, in the centre of mass frame of reference the total momentum is zero. If there are only two particles, and they have the same mass, then, in the centre of mass frame of reference, they both have the same speed and move either towards or away from each other. 
\end{example}

\begin{example}{\capfig{0.6\textwidth}{figures/MomentumAndCM/springcollision.png}{\label{fig:momentumandcm:springcollision}One block attached to a spring about to collide with another block.} A block of mass $M$ with a spring of spring constant $k$ attached to it is sliding on a frictionless surface with velocity $\vec v_M$ in the $x$ direction. A second block of mass $m$ has velocity $\vec v_m$ also in the $x$ direction (shown above in the negative $x$ direction, but let us assume that we do not necessarily know the direction, only  that the two blocks will collide). During the collision between the blocks, what is the maximum amount by which the spring is compressed?
}
The collision is elastic because the energy used to compress the spring is ``given back'' when the spring extends again, since the spring force is conservative. 

They key to modelling the compression of the spring is to identify the condition under which the spring is maximally compressed. This will occur at the point during the collision where the two masses will have exactly the same velocity, momentarily moving in unison as the spring is maximally compressed. Because, instantaneously, the masses have the same velocity, there is a frame of reference in which the two masses are at rest, and the momentum is zero. Of course, that frame of reference is the centre of mass frame of reference. 

Because the collision is one-dimensional, we can calculate the velocity of the centre of mass as:
\begin{align*}
v_{CM} = \frac{Mv_M+mv_m}{m+M}
\end{align*}
where we note that $v_m$ is a negative number, since the block of mass $m$ is moving in the negative $x$ direction. The total momentum, $\vec P^{CM}$, in the centre of mass frame of reference must be zero. Writing this out for the $x$ component and transforming the velocities of the two blocks into the centre of mass frame of reference:
\begin{align*}
P^{CM}_x = M(v_M-v_{CM})+m(v_m-v_{CM})&=0\\
\therefore (v_m-v_{CM}) &= -\frac{M}{m}(v_M-v_{CM})
\end{align*}
Also note that we can write the velocity difference $v_M-v_{CM}$ without using the centre of mass velocity:
\begin{align*}
v_M-v_{CM} &= v_M-\frac{Mv_M+mv_m}{m+M}=\frac{1}{m+M}(v_M(m+M)-Mv_M-mv_m)\\
&=\frac{m}{m+M}(v_M-v_m)
\end{align*}
We can then use conservation of energy in the centre of mass frame to determine the maximal compression of the spring. Before the collision, the total mechanical energy in the system, $E$, is the sum of the kinetic energies of the two blocks (as the spring is not compressed):
\begin{align*}
E&=\frac{1}{2}m(v_m-v_{CM})^2+\frac{1}{2}M(v_M-v_{CM})^2\\
&=\frac{1}{2}\frac{M^2}{m}(v_M-v_{CM})^2+\frac{1}{2}M(v_M-v_{CM})^2\\
&=\frac{1}{2}M \left( 1 + \frac{M}{m}\right) (v_M-v_{CM})^2\\
&=\frac{1}{2}M \left(\frac{m+M}{m} \right)(v_M-v_{CM})^2\\
&=\frac{1}{2}M \left(\frac{m+M}{m} \right)\left(\frac{m}{m+M}(v_M-v_m)\right)^2\\
&=\frac{1}{2} \left(\frac{mM}{m+M}\right)(v_M-v_m)^2
\end{align*}
where we used our expressions above to simplify the expression. When the spring is maximally compressed, the two blocks are at rest and the mechanical energy of the system, $E'$, is all ``stored'' as spring potential energy:
\begin{align*}
E'&=\frac{1}{2}kx^2
\end{align*}
where $x$ is the distance by which the spring is compressed. Equating the two allows us to determine the maximal compression of the spring:
\begin{align*}
E &= E' \\
\frac{1}{2} \left(\frac{mM}{m+M}\right)(v_M-v_m)^2 &= \frac{1}{2}kx^2\\
\therefore x &= \sqrt{\frac{1}{k} \left(\frac{mM}{m+M}\right)}(v_M-v_m)
\end{align*}
\textbf{Discussion:} By modelling the collision in the centre of mass frame of reference, we were easily able to determine the maximal compression of the spring. This would have been more difficult in the lab frame of reference, because the two blocks would still be moving when the spring is maximally compressed, so we would have needed to determine their speeds to determine the total mechanical energy when the spring is compressed.

When we calculated the initial kinetic energy, we found that it was given by:
\begin{align*}
E=\frac{1}{2} \left(\frac{mM}{m+M}\right)(v_M-v_m)^2 &=\frac{1}{2}M_{red}(v_M-v_m)^2
\end{align*}
The combination of masses in parentheses is called the ``reduced mass'' of the system, and is a sort of effective mass that can be used to model the system as a whole. 
\end{example}

\section{The centre of mass}
In this section, we show how to generalize Newton's Second Law so that it may describe the motion of an object that is not a point particle. Any object can be described as being made up of point particles; for example, those particles could be the atoms that make up regular matter. We can thus use the same terminology as in the previous sections to describe a complicated object as a ``system'' comprised of many point particles, themselves described by Newton's Second Law. A system could be a rigid object where the point particles cannot move relative to each other, such as atoms in a solid\footnote{In reality, even atoms in a solid can move relative to each other, but they do not move by large amounts compared to the object.}. Or, the system could be a gas, made of many atoms moving around, or it could be a combination of many solid objects moving around. 

In the previous section, we saw how the total momentum and the total mechanical energy of the system could be used to describe the system as a whole. In this section, we will define the centre of mass which will allow us to describe the position of the system as a whole.

Consider a system comprised of $N$ point particles. Each point particle $i$, of mass $m_i$, can be described by a position vector, $\vec r_i$, a velocity vector, $\vec v_i$, and an acceleration vector, $\vec a_i$, relative to some coordinate system. Newton's Second Law can be applied to any one of the particles in the system:
\begin{align*}
\sum_k \vec F_{ik} = m_i \vec a_i
\end{align*}
where $\vec F_{ik}$ is the k-th force exerted on particle $i$. We can write Newton's Second Law once for each of the $N$ particles, and we can sum those $N$ equations together:
\begin{align*}
\sum_k \vec F_{1k} + \sum_k \vec F_{2k} + \sum_k \vec F_{3k} +\dots &= m_1\vec a_1 + m_2 \vec a_2 + m_3 \vec a_3 + \dots\\
\sum \vec F = \sum_i m_i \vec a_i 
\end{align*}
where the sum on the left is the sum of all of the forces exerted on all of the particles in the system\footnote{Again, we are summing together forces that are acting on \textbf{different} particles} and the sum over $i$ on the right is over all of the $N$ particles in the system. As we have already seen, the sum of all of the forces exerted on the system can be divided into separate sums over external and internal forces:
\begin{align*}
\sum \vec F = \sum \vec F^{ext} + \sum \vec F^{int} 
\end{align*}
and the sum over the internal forces is zero\footnote{Recall, the internal forces are those forces that particles in the system are exerting on one another. Because of Newton's Third Law, these will sum to zero.}. We can thus write that the sum of the external forces exerted on the system is given by:
\begin{align}
\label{eqn:momentumandcm:cmtemp1}
\sum \vec F^{ext}&= \sum_i m_i \vec a_i
\end{align}
We would like this equation to resemble Newton's Second Law, but for the system as a whole. Suppose that the system has a total mass, $M$:
\begin{align*}
M = m_1 + m_2 + m_3 +\dots = \sum_i m_i
\end{align*}
we would like to have an equation of the form:
\begin{align}
\label{eqn:momentumandcm:cmtemp2}
\sum \vec F^{ext}&=M\vec a_{CM}
\end{align}
to describe the system as a whole. However, it is not clear what the acceleration, $\vec a_{CM}$ refers to, since the particles in the system could all be moving in different directions. Suppose that there is a point in the system, whose position is given by the vector, $\vec r_{CM}$, in such a way that the acceleration above is the second time-derivative of that position vector:
\begin{align*}
\vec a_{CM} = \frac{d^2 }{dt^2}\vec r_{CM}
\end{align*}
We can compare Equations \ref{eqn:momentumandcm:cmtemp1} and \ref{eqn:momentumandcm:cmtemp2} to determine what the position vector $\vec r_{CM}$ corresponds to:
\begin{align*}
\sum \vec F^{ext}&= \sum_i m_i \vec a_i = \sum_i m_i \frac{d^2 }{dt^2}\vec r_i \\
\sum \vec F^{ext}&=M\vec a_{CM} = M \frac{d^2 }{dt^2}\vec r_{CM}\\
\therefore M \frac{d^2 }{dt^2}\vec r_{CM}&= \sum_i m_i \frac{d^2 }{dt^2}\vec r_i
\end{align*}
Re-arranging, and noting that the masses are constant in time, and so they can be factored into the derivatives:
\begin{align*}
\frac{d^2 }{dt^2}\vec r_{CM} &= \frac{1}{M}\sum_i m_i \frac{d^2 }{dt^2}\vec r_i\\
\frac{d^2 }{dt^2}\vec r_{CM} &= \frac{d^2 }{dt^2}\left(\frac{1}{M}\sum_i m_i\vec r_i \right)\\
\therefore \vec r_{CM} &=\frac{1}{M}\sum_i m_i\vec r_i
\end{align*}
where in the last line we set the quantities that have the same time derivative equal to each other\footnote{Technically, the terms in the derivatives are only equal to within two constants of integration, $\vec r_{CM} =\frac{1}{M}\sum_i m_i\vec r_i + at + b$, which we can set to zero}. $\vec r_{CM}$ is the vector that describes the position of the ``centre of mass'' (CM). The position of the centre of mass is described by Newton's Second Law applied to the system as a whole:
\begin{align}
\Aboxed{\sum \vec F^{ext}&=M\vec a_{CM}}
\end{align}
where $M$ is the total mass of the system, and the sum of the forces is the sum over only external forces on the system.

Although we have formally derived Newton's Second Law for a system of particles, we really have been using this result throughout the text. For example, when we modelled a block sliding down an incline, we never worried that the block was made of many atoms all interacting with each other and the surroundings. Instead, we only considered the external forces on the block, namely, the normal force from the incline, any frictional forces, and the total weight of the object (the force exerted by gravity). Technically, the force of gravity is not exerted on the block as a whole, but on each of the atoms. However, when we sum the force of gravity exerted on each atom:
\begin{align*}
m_1\vec g+ m_2 \vec g + m_3\vec g + \dots = (m_1+m_2+m_3+\dots)\vec g = M\vec g
\end{align*}
we find that it can be modelled by considering the block as a single particle of mass $M$ upon which gravity is exerted. The centre of mass is sometimes described as the ``centre of gravity'', because it \textbf{corresponds to the location where we can model the total force of gravity, $M\vec g$, as being exerted}. When we applied Newton's Second Law to the block, we then described the motion of the block as a whole (and not the motion of the individual atoms). Specifically, we modelled the motion of the centre of mass of the block. 

The position of the centre of mass is a vector equation that is true for each coordinate:
\begin{align}
\vec r_{CM} &=\frac{1}{M}\sum_i m_i\vec r_i\nonumber\\
\therefore x_{CM} &= \frac{1}{M}\sum_i m_i x_i\nonumber\\
\therefore y_{CM} &= \frac{1}{M}\sum_i m_i y_i\nonumber\\
\therefore z_{CM} &= \frac{1}{M}\sum_i m_i z_i
\end{align}
The centre of mass is that \textbf{position in a system that is described by Newton's Second Law when it is applied to the system as a whole}. The centre of mass can be thought of as an average position for the system (it is the average of the positions of the particles in the system, weighted by their mass). By describing the position of the centre of mass, we are not worried about the detailed positions of the all of the particles in the system, but rather only the average position of the system as a whole. In other words, this is equivalent to viewing the whole system as a single particle of mass $M$ located at the position of the centre of mass. 

Consider, for example, a person throwing a dumbbell that is made from two spherical masses connected by a rod, as illustrated in Figure \ref{fig:momentumandcm:cmparabola}. The dumbbell will rotate in a complex manner as it moves through the air. However, the centre of mass of the dumbbell will travel along a parabolic trajectory (projectile motion), because the only external force exerted on the dumbbell during its trajectory is gravity.
\capfig{0.6\textwidth}{figures/MomentumAndCM/cmparabola.png}{\label{fig:momentumandcm:cmparabola} The motion of the centre of mass of a dumbbell is described by Newton's Second Law, even if the motion of the rotating dumbbell is more complex.}


If we take the derivative with respect to time of the centre of mass position, we obtain the velocity of the centre of mass, and its components, which allow us to describe how the system is moving as a whole:
\begin{align}
\vec v_{CM} &= \frac{d}{dt}\vec r_{CM} = \frac{1}{M}\sum_i m_i\frac{d}{dt}\vec r_i=  \frac{1}{M}\sum_i m_i\vec v_i\nonumber\\
\therefore v_{CMx} &= \frac{1}{M}\sum_i m_i v_{ix}\nonumber\\
\therefore v_{CMy} &= \frac{1}{M}\sum_i m_i v_{iy}\nonumber\\
\therefore v_{CMz} &= \frac{1}{M}\sum_i m_i v_{iz}
\end{align}
Note that this is the same velocity that we found earlier for the velocity of the centre of mass frame of reference. In the centre of mass frame of reference, the total momentum of the system is zero. This makes sense, because the centre of mass represents the average position of the system; if we move ``with the system'', then the system appears to have zero momentum.

We can also define the total momentum of the system, $\vec P$, in terms of the total mass, $M$, of the system and the velocity of the centre of mass:
\begin{align*}
\vec P &= \sum m_i \vec v_i = \frac{M}{M}\sum m_i \vec v_i\\
&=M\vec v_{CM}
\end{align*}
which we can also use in Newton's Second Law:
\begin{align*}
\frac{d}{dt}\vec P = \sum \vec F^{ext}
\end{align*}
and again, see that the total momentum of the system is conserved if the net external force on the system is zero. In other words, the centre of mass of the system will move with constant velocity when momentum is conserved.

Finally, we can also define the acceleration of the centre of mass by taking the time derivative of the velocity:
\begin{align}
\vec a_{CM} &= \frac{d}{dt}\vec v_{CM} = \frac{1}{M}\sum_i m_i\frac{d}{dt}\vec v_i=  \frac{1}{M}\sum_i m_i\vec a_i\nonumber\\
\therefore a_{CMx} &= \frac{1}{M}\sum_i m_i a_{ix}\nonumber\\
\therefore a_{CMy} &= \frac{1}{M}\sum_i m_i a_{iy}\nonumber\\
\therefore a_{CMz} &= \frac{1}{M}\sum_i m_i a_{iz}
\end{align}

\begin{example}{
\capfig{0.5\textwidth}{figures/MomentumAndCM/sunearthmars.png}{\label{fig:momentumandcm:sunearthmars} A syzygy between the Sun, Earth, and Mars.}
In astronomy, a syzygy is defined as the event in which three bodies are all lined up along a straight line. For example, a syzygy occurs when the Sun (mass $M_S=\SI{2.00e30}{kg}$), Earth (mass $M_E=\SI{5.97e24}{kg}$), and Mars (mass $M_M=\SI{6.39e23}{kg}$) are all lined up, as in Figure \ref{fig:momentumandcm:sunearthmars}. How far from the centre of the Sun is the centre of mass of the Sun, Earth, Mars system during a syzygy?}
Since this is a one-dimensional problem, we can define an $x$ axis that is co-linear with the three bodies, and find only the $x$ coordinate of the position of the centre of mass. We are free to choose the origin of the coordinate system, so we choose the origin to be located at the centre of the Sun. This way, the position of the centre of mass along the $x$ axis will directly correspond to its distance from the centre of the Sun.

The Sun, Earth, and Mars are not point particles. However, because they are spherically symmetric, their centres of mass correspond to their geometric centres. We can thus model them as point particles with the mass of the body located at the corresponding geometric centre. If $r_E=\SI{1.50e11}{m}$ ($r_M=\SI{2.28e11}{m}$) is the distance from the centre of the Earth (Mars) to the centre of the Sun, then the position of the centre of mass is given by:
\begin{align*}
x_{CM} &= \frac{1}{M}\sum_i m_i x_i\\
&=\frac{M_S(0)+M_Er_E+M_Mr_M}{M_S+M_E+M_M}\\
&=\frac{(\SI{2.00e30}{kg})(0)+(\SI{5.97e24}{kg})(\SI{1.50e11}{m})+(\SI{6.39e23}{kg})(\SI{2.28e11}{m})}{(\SI{2.00e30}{kg})+(\SI{5.97e24}{kg})+(\SI{6.39e23}{kg})}\\
&=\SI{5.21e5}{m}
\end{align*}
The centre of mass of the Sun-Earth-Mars system during a syzygy is located approximately $\SI{500}{km}$ from the centre of the Sun.

\textbf{Discussion:} The radius of the Sun is approximately $\SI{700000}{km}$, so the centre of mass of the system is well inside of the Sun. The Sun is so much more massive than either of the Earth or Mars, that the two planets hardly contribute to shifting the centre of mass away from the centre of the Sun. We would generally consider the masses of the two planets to be negligible if one wanted to model how the solar system itself moves around the Milky Way galaxy.
\end{example}

\begin{example}{\capfig{0.5\textwidth}{figures/MomentumAndCM/cmraft.png}{\label{fig:momentumandcm:cmraft}Three people on rafts on a lake.}
Alice (mass $m_A$), Brice (mass $m_B$), and \chloe (mass $m_C$) are stranded on individual rafts of negligible mass on a lake, off of the coast of Nyon. The rafts are located at the corners of a right-angle triangle, as illustrated in Figure \ref{fig:momentumandcm:cmraft}, and are connected by ropes. The distance between Alice and Brice is $r_{AB}$ and the distance between Alice and \chloe is $r_{AC}$, as illustrated. Alice decides to pull on the rope that connects her to \chloe, while Brice decide to pull on the rope that connects him to Alice. Where will the three rafts meet?}
We consider the system comprised of the three people and their rafts and model each person and their raft as a point particle with the mass concentrated at the centre of the raft. The forces exerted by pulling on the ropes are internal forces (one particle on the other), and will thus have no impact on the motion of the centre of mass of the system. There are no net external forces exerted on the system (the forces of gravity are balanced out by the forces of buoyancy from the rafts). The centre of mass of the system does not move when the people are pulling on the ropes, so they must ultimately meet at the centre of mass.

We can define a coordinate system such that the origin is located where Alice is initially located, the $x$ axis is in the direction from Alice to Brice, and the $y$ axis is in the direction from Alice to Chlo\"e. The initial positions of Alice, Brice, and \chloe are thus:
\begin{align*}
\vec r_A &= 0\hat x + 0\hat y\\
\vec r_B &= r_{AB}\hat x + 0\hat y\\
\vec r_C &= 0\hat x + r_{AC}\hat y
\end{align*}
respectively. The $x$ and $y$ coordinates of the centre of mass are thus:
\begin{align*}
x_{CM} &= \frac{1}{M}\sum_i m_i x_i = \frac{m_A(0) + m_Br_{AB} + m_C(0)}{m_A + m_B + m_C}=\left(\frac{m_B}{m_A + m_B + m_C}\right)r_{AB}\\
y_{CM} &= \frac{1}{M}\sum_i m_i y_i = \frac{m_A(0) + m_B(0) + m_Cr_{AC}}{m_A + m_B + m_C}=\left(\frac{m_C}{m_A + m_B + m_C}\right)r_{AC}\\
\end{align*}
which corresponds to the position where the three rafts will meet, relative to the initial position of Alice.

\textbf{Discussion: }By using the centre of mass, we easily found where the three rafts would meet. If we had used Newton's Second Law on the three rafts individually, the model would have been complicated by the fact that the forces exerted by Alice and Brice on the ropes change direction as the rafts begin to move, which would have required the use of integrals to determine the motion of each person.
\end{example}
TODO: Question library problem, projectile that splits in flight, where does it land? See Giancolli example 9-18.

\subsection{The centre of mass for a continuous object}
So far, we have considered the centre of mass for a system made of point particles. In this section, we show how one can determine the centre of mass for a ``continuous object''\footnote{In reality, there are of course no continuous objects since, at the atomic level, everything is made of particles.}. We previously argued that if an object is uniform and symmetric, its centre of mass will be located at the centre of the object. Let us show this explicitly for a uniform rod of total mass $M$ and length $L$, as depicted in Figure \ref{fig:momentumandcm:rod}.
\capfig{0.5\textwidth}{figures/MomentumAndCM/rod.png}{\label{fig:momentumandcm:rod} A rod of length $L$ and mass $M$.}
In order to determine the centre of mass of the rod, we first model the rod as being made of $N$ small ``mass elements'' each of equal mass, $\Delta m$, and of length $\Delta x$, as shown in Figure \ref{fig:momentumandcm:rod}. If we choose those mass elements to be small enough, we can model them as point particles, and use the same formulas as above to determine the centre of mass of the rod.

We define the $x$ axis to be co-linear with the rod, such that the origin is located at one end of the rod. We can define the ``linear mass density'' of the rod, $\lambda$, as the mass per unit length of the rod:
\begin{align*}
\lambda = \frac{M}{L}.
\end{align*}

A small mass element of length $\Delta x$, will thus have a mass, $\Delta m$, given by:
\begin{align*}
\Delta m = \lambda \Delta x 
\end{align*}

If there are $N$ mass elements that make up the rod, the $x$ position of the centre of mass of the rod is given by:
\begin{align*}
x_{CM} &= \frac{1}{M}\sum_i^N m_i x_i = \frac{1}{M}\sum_i^N \Delta m x_i \\
&=\frac{1}{M}\sum_i^N \lambda \Delta x x_i\\
\end{align*}
where $x_i$ is the $x$ coordinate of the $i$-th mass element. Of course, we can take the limit over which the length, $\Delta x$, of each mass element goes to zero to obtain an integral:
\begin{align*}
x_{CM} = \lim_{\Delta x \to 0} \frac{1}{M}\sum_i^N \lambda \Delta x x_i = \frac{1}{M} \int_0^L \lambda x dx
\end{align*} 
where the discrete variable $x_i$ became the continuous variable $x$, and $\Delta x$ was replaced by $dx$ (which is the same, but indicates that we are taking the limit of $\Delta x \to 0$). The integral is easily found:
\begin{align*}
x_{CM} &= \frac{1}{M} \int_0^L \lambda x dx = \frac{1}{M}\lambda \left[ \frac{1}{2} x^2\right]_0^L\\
&=\frac{1}{M}\lambda \frac{1}{2} L^2 = \frac{1}{M}\left( \frac{M}{L}\right) \frac{1}{2} L^2\\
&=\frac{1}{2}L
\end{align*}
where we substituted the definition of $\lambda$ back in to find, as expected, that the centre of mass of the rod is half its length away from one of the ends.

Suppose that the rod was instead not uniform and that its linear density depended on the position $x$ along the rod:
\begin{align*}
\lambda(x) = 2a + 3bx
\end{align*}

We can still find the centre of mass by considering an infinitesimally small mass element of mass $dm$, and length $dx$. In terms of the linear mass density and length of the mass element, $dx$, the mass $dm$ is given by:
\begin{align*}
dm = \lambda(x) dx
\end{align*}
The $x$ position of the centre of mass is thus found the same way as before, except that the linear mass density is now a function of $x$:
\begin{align*}
x_{CM} &= \frac{1}{M} \int_0^L \lambda(x) x dx =\frac{1}{M} \int_0^L (2a + 3bx) x dx=\frac{1}{M} \int_0^L (2ax + 3bx^2) dx\\
&=\frac{1}{M}  \left[  ax^2 + bx^3  \right]_0^L\\
&=\frac{1}{M} (aL^2 + bL^3 )
\end{align*}

In general, for a continuous object, the position of the centre of mass is given by:
\begin{align}
\vec r_{CM} &=\frac{1}{M}\int \vec r dm\nonumber\\
\therefore x_{CM} &= \frac{1}{M}\int x dm\nonumber\\
\therefore y_{CM} &=  \frac{1}{M}\int y dm\nonumber\\
\therefore z_{CM} &=  \frac{1}{M}\int z dm\\
\end{align}
where in general, one will need to write $dm$ in terms of something that depends on position (or a constant) so that the integrals can be evaluated over the spatial coordinates ($x$,$y$,$z$) over the range that describe the object. In the above, we wrote $dm = \lambda dx$ to express the mass element in terms of spatial coordinates.
\begin{example}{\capfig{0.5\textwidth}{figures/MomentumAndCM/cmbowl.png}{\label{fig:momentumandcm:cmbowl} A symmetric bowl with parabolic sides is completely filled with water. The bowl has a height $h$.}
A bowl of height $h$ has parabolic sides and a circular cross-section, as illustrated in Figure \ref{fig:momentumandcm:cmbowl}. The bowl is filled with water. The bowl itself has a negligible mass and thickness, so that the mass of the full bowl is dominated by the mass of the water. Where is the centre of mass of the full bowl?}
We can define a coordinate system such that the origin is located at the bottom of the bowl and the $z$ axis corresponds to the axis of symmetry of the bowl. Because the bowl is full of water, and the bowl itself has negligible mass, we can model the full bowl as a uniform body of water with the same shape as the bowl and (volume) mass density $\rho$ equal to the density of water. Furthermore, by symmetry, the centre of mass of the bowl will be on the $z$ axis. 

Because the bowl has a circular cross-section, we can divide it up into disk-shaped mass elements, $dm$, that have an infinitesimally small height $dz$, and a radius $r(z)$, that depends on their $z$ coordinate (Figure \ref{fig:momentumandcm:cmbowl}). 
\capfig{0.5\textwidth}{figures/MomentumAndCM/cmbowlsoln.png}{\label{fig:momentumandcm:cmbowlsoln} The parabolic bowl divided up into disk-shaped mass elements, $dm$, that have an infinitesimally small height $dz$, and a radius $r(z)$, that depends on their $z$ coordinate.}
The centre of mass of each disk-shaped mass element will be located where the corresponding disk intersects the $z$ axis. The mass of one disk element is given by:
\begin{align*}
dm = \rho dV = \rho \pi r^2(z) dz
\end{align*}
where $dV = \pi r(z)^2 dz$ is the volume of the disk with radius $r(z)$ and thickness $dz$. The radius of the infinitesimal disk depends on its $z$ position, since the radii of the different disks must describe a parabola:
\begin{align*}
z(r) &= \frac{1}{a^2}r^2\\
r(z) &= a\sqrt z\\
\therefore dm &= \rho \pi r^2(z) dz= \rho \pi a^2  z dz
\end{align*}
where we introduced the constant $a$ so that the dimensions are correct. The constant $a$ determines how ``steep'' the parabolic sides are. The $z$ coordinate of the centre of mass is thus given by:
\begin{align*}
z_{CM} &=  \frac{1}{M}\int z dm =\frac{1}{M}\int_0^h z  (\rho \pi a^2 z dz)=\frac{\rho \pi a^2}{M}\int_0^h z^2dz \\
&=\frac{\rho \pi a^2}{M}\left[ \frac{1}{3}z^3 \right]_0^h\\
&=\frac{\rho \pi a^2}{3M}h^3
\end{align*}
However, we are not quite done, since we do not know the total mass, $M$, of the water. To find the total mass of water, $M$, we proceed in an analogous way, and determine the value of the sum (integral) of all of the mass elements:
\begin{align*}
M = \int dm = \int_0^h \rho \pi a^2 z dz = \rho \pi a^2 \left[ \frac{1}{2}z^2 \right]_0^h= \frac{1}{2}\rho \pi a^2 h^2
\end{align*}
Substituting this value for $M$, we can determine the $z$ coordinate of the centre of mass of the full bowl:
\begin{align*}
z_{CM} &=\frac{\rho \pi a^2}{3M}h^3 = \frac{2\rho \pi a^2}{3\rho \pi a^2 h^2}h^3=\frac{2}{3}h
\end{align*}
Regardless of the actual shape of the parabola (the parameter $a$), the centre of mass will always be two thirds of the way up from the bottom of the bowl.

\textbf{Discussion: }In determining the centre of mass of a three dimensional object, we used symmetry to argue that the $x$ and $y$ coordinates would be zero. We then found the $z$ position of the centre of mass by dividing up the bowl into infinitesimally small mass elements (disks) along the direction in which we needed to find the centre of mass coordinate.
\end{example}

\begin{checkpoint}
\begin{MCquestion}{True or False: The centre of mass of a continuous object is always located within the object.}
\item True
\item False %correct
\end{MCquestion}
\end{checkpoint}

TODO: For above, redo the figure, and split into a figure with just the bowl in the question and a figure showing, dm, dz, etc, in the solution.

TODO: Question Library find the CM of a uniform half disk.

\newpage
\section{Summary}

\begin{chapterSummary}{
The momentum vector, $\vec p$, of a point particle of mass $m$ with velocity $\vec v$ is defined as:
\begin{align*}
\vec p = m\vec v
\end{align*}
We can write Newton's Second Law for a point particle in term of its momentum:
\begin{align*}
\frac{d}{dt}\vec p = \sum \vec F = \vec F^{net}
\end{align*}
where the net force on the particle determines the rate of change of its momentum. In particular, if there is no net force on a particle, its momentum will not change.

The net impulse vector, $\vec J^{net}$, given over a period of time is defined as the net force exerted on a particle integrated from a time $t_A$ to a time $t_B$:
\begin{align*}
\vec J^{net} = \int_{t_A}^{tB} \vec F^{net} dt
\end{align*}
The net impulse vector is also equal to the change in momentum of the particle in that same period of time:
\begin{align*}
\vec J^{net} = \Delta \vec p = \vec p_B - \vec p_A
\end{align*}

When we define a system of particles, we can distinguish between internal and external forces. Internal forces are those forces exerted by the particles in the system on each other. External forces are those forces on the particles in the system that are not exerted by the particles on each other. The sum over all of the forces on all of the particles in the system will be equal to the sum over the external forces, because the sum over all internal forces on a system is always zero (Newton's Third Law).

The total momentum of a system, $\vec P$, is the sum of the momenta, $\vec p_i$, of all of the particles in the system:
\begin{align*}
\vec P = \sum \vec p_i
\end{align*}

The rate of change of the momentum of a system is equal to the sum of the external forces on the system:
\begin{align*}
\frac{d}{dt}\vec P = \sum \vec F^{ext}
\end{align*}
which can be thought of as an equivalent description as Newton's Second Law, but for the system as a whole. If the net (external) force on a system is zero, then the total momentum of the system is conserved. 

Collisions are those events when the particles in a system interact (e.g. by colliding) and change their momenta. When modelling collisions, it is usually beneficial to first define a system for which the total momentum is conserved before and after the collision. One can then write the total momentum of the system in terms of the momenta of the individual particles before and after the collision and equate them to determine the momenta of the individual particles.

Collisions can be elastic or inelastic. Elastic collisions are those where, in addition to the total momentum, the total mechanical energy of the system is conserved. The total mechanical energy can usually be taken as the sum of the kinetic energies of the particles in the system.

Inelastic collisions are those in which the total mechanical energy of the system is not conserved. One can usually identify if mechanical energy was introduced or removed from the system and determine if the collision is elastic. It is important to identify when momentum and mechanical energy are conserved. Momentum is conserved if no net force is exerted on the system, whereas mechanical energy is conserved if no net work was done on the system by non-conservative forces.

We can always choose in which frame of reference to model a collision. In some cases, it is convenient to use the frame of reference of the centre of mass of the system, because in that frame of reference, the total momentum of the system is zero.

If a system has a total mass $M$, then one can use Newton's Second Law to describe its motion:
\begin{align*}
\sum \vec F^{ext} &= M \vec a_{CM}\\
\sum \vec F^{ext} &=\frac{d}{dt} \vec P
\end{align*}
where the sum of the forces is over all of the external forces on the system. The acceleration vector, $\vec a_{CM}$, describes the motion of the ``centre of mass'' of the system. $\vec P=M\vec v_{CM}$ is the total momentum of the system.

The centre of mass of a system is a mass-weighted average of the positions of all of the particles of mass $m_i$ and position $\vec r_i$ that comprise the system:
\begin{align*}
\vec r_{CM} &=\frac{1}{M}\sum_i m_i\vec r_i
\end{align*}
The vector equation can be broken into components to find the $x$, $y$, and $z$ component of the position of the centre of mass. Similarly, one can also define the velocity of the centre of mass of the system, in terms of the individual velocities, $\vec v_i$, of the particles in the system:
\begin{align*}
\vec v_{CM} &= \frac{1}{M}\sum_i m_i\vec v_i
\end{align*}
Finally, one can define the acceleration of the centre of mass of the system, in terms of the individual accelerations, $\vec a_i$, of the particles in the system:
\begin{align*}
\vec a_{CM} &=  \frac{1}{M}\sum_i m_i\vec a_i
\end{align*}

If the system is a continuous object, we can find its centre of mass using a sum (integral) of infinitesimally small mass elements, $dm$, weighted by their position:
\begin{align*}
\vec r_{CM} &=\frac{1}{M}\int \vec r dm\\
\therefore x_{CM} &= \frac{1}{M}\int x dm\\
\therefore y_{CM} &=  \frac{1}{M}\int y dm\\
\therefore z_{CM} &=  \frac{1}{M}\int z dm
\end{align*}
The strategy to set up the integrals above is usually to express the mass element, $dm$, in terms of the position and density of the material of which the object is made. One can then integrate over position in the range defined by the dimensions of the object.
}
\end{chapterSummary}

\newpage
\begin{importantEquations}
\medskip
\begin{multicols}{2}
\textbf{Momentum of a point particle:}
\begin{align*}
\vec p = m\vec v \\
\frac{d}{dt}\vec p = \sum \vec F = \vec F^{net}
\end{align*}
\textbf{Impulse:}
\begin{align*}
\vec J^{net} = \int_{t_A}^{tB} \vec F^{net} dt \\
\vec J^{net} = \Delta \vec p = \vec p_B - \vec p_A
\end{align*}
\textbf{Momentum of a system:}
\begin{align*}
\vec P = \sum \vec p_i \\
\frac{d}{dt}\vec P = \sum \vec F^{ext}
\end{align*}
\textbf{Newton's Second Law for a \\ system:}
\begin{align*}
\sum \vec F^{ext} &= M \vec a_{CM}\\
\sum \vec F^{ext} &=\frac{d}{dt} \vec P
\end{align*}
\columnbreak
\textbf{Position of the Centre of Mass \\ of a system:}
\begin{align*}
\vec r_{CM} &=\frac{1}{M}\sum_i m_i\vec r_i 
\end{align*}
\textbf{Velocity of the Centre of Mass \\ of a system:}
\begin{align*}
\vec v_{CM} &= \frac{1}{M}\sum_i m_i\vec v_i \\
\end{align*}
\textbf{Acceleration of the Centre of Mass \\ of a system:}
\begin{align*}
\vec a_{CM} &=  \frac{1}{M}\sum_i m_i\vec a_i \\
\end{align*}
\textbf{Position of the Centre of Mass for a \\ continuous object:}
\begin{align*}
\vec r_{CM} &=\frac{1}{M}\int \vec r dm\\
\therefore x_{CM} &= \frac{1}{M}\int x dm\\
\therefore y_{CM} &=  \frac{1}{M}\int y dm\\
\therefore z_{CM} &=  \frac{1}{M}\int z dm
\end{align*}
\medskip
\end{multicols}
\end{importantEquations}


\newpage
\section{Thinking about the material}
\subsection{Reflect and research}

\begin{enumerate}
\item Explain how Newton's Cradle illustrates the conservation of momentum. Are the collisions in Newton's Cradle elastic? Explain! 
\end{enumerate}
\subsection{To try at home}

\begin{tQuestion}Try doing this \end{tQuestion}

\subsection{To try in the lab}

\newpage
\section{Sample problems and solutions}
\subsection{Problems}


\newpage
\subsection{Solutions}



%
\chapter{Rotational dynamics}
\label{chapter:rotationaldynamics}
In this Chapter, we use Newton's Second Law to develop a formalism to describe how objects rotate. In particular, we will introduce the concept of torque which plays a similar role to that of force in non-rotational dynamics. We will also introduce the concept of moment of inertia to describe how objects resist rotational motion. 

\begin{learningObjectives}{
 \item Understand how to use vector quantities for describing the kinematics of rotations.
 \item Understand how to use torque to determine the angular acceleration of an object.
 \item Understand conditions for static and dynamic equilibrium. 
 \item Understand how to determine the moment of inertia of an object.
 }
\end{learningObjectives}

\begin{opening}
\begin{MCquestion}{A question}
\item a choice
\item another choice %correct
\end{MCquestion}
\end{opening}

\section{Rotational kinematic vectors}
TODO: Review box to rotational kinematics, and vector product

\subsection{Scalar rotational kinematic quantities}
Recall that we can describe the motion of a particle along a circle of radius $R$ by using its angular position, $\theta$, its angular velocity, $\omega$, and its angular acceleration, $\alpha$. With a suitable choice of coordinate systen, the angular position can be defined as the angle made by the position vector of the particles, $\vec r$, and the $x$ axis of a coordinate system whose origin is the centre of the circle and for which the motion is around the $z$ axis, as shown in Figure \ref{fig:rotationaldynamics:vcircle}. 
\capfig{0.4\textwidth}{figures/RotationalDynamics/vcircle.png}{\label{fig:rotationaldynamics:vcircle} Angular position for a particle moving around the $z$ axis (out of the page), along a circle of radius $R$ centre at the origin.}

The angular velocity, $\omega$, is the rate of the change of the angular position, and the angular acceleration, $\alpha$, is the rate of change of the angular velocity:
\begin{align*}
\omega &= \frac{d}{dt}\theta \\
\alpha &= \frac{d}{dt}\omega
\end{align*}
If the angular acceleration is constant, then angular velocity and position as a function of time are given by:
\begin{align*}
\omega(t) = \omega_0+\alpha t\\
\theta(t) = \theta_0+\omega_0 t+\frac{1}{2}\alpha t^2
\end{align*}
where $\theta_0$ and $\omega_0$ are the angular position and velocity, respectively, at $t=0$.

We can also describe the motion of the particle in terms of ``linear'' quantities (as opposed to ``angular'' quantities) along a one-dimensional axis that is curved along the circle. If $s$ is the distance along the circumference of the circle, measured counter-clockwise from where the circle intersects the $x$ axis, then it is related to the angular displacement:
\begin{align*}
s = R\theta
\end{align*}
if $\theta$ is expressed in radians. Similarly, the linear velocity along the $s$ axis, $v_s$, and the corresponding acceleration, $a_s$, are given by:
\begin{align*}
v_s &= \frac{ds}{dt} =\frac{d}{dt}R\theta = R\omega\\
a_s&= \frac{dv}{dt} =\frac{d}{dt}R\omega = R\alpha
\end{align*}
where the radius of the circle, $R$, is a constant that can be taken out of the time derivatives. For motion along a circle, the velocity vector, $\vec v$, of the particle is always tangent to the circle (Figure \ref{fig:rotationaldynamics:vcircle}), so $v_s$ corresponds to the speed of the particle. The acceleration vector, $\vec a$, is in general not tangent to the circle; $a_s$ represents the component of the acceleration vector that is tangent to the circle. If $a_s=0$, then $\alpha=0$, and the particle is moving with a constant speed (uniform circular motion), and the acceleration vector points towards the centre of the circle.

TODO: Checkpoint question: Show a rotating disk, two points at different radii, which is correct? a) both have the same angular and linear speeds. b)both have the same linear speed but different angular speed. c) both have the same angular speed but different linear speed (correct)


\subsection{Vector rotational kinematic quantities}
In the previous section, we defined angular quantities to describe the motion of a particle about the $z$ axis along a circle of radius $R$ that lies in the $xy$ plane. By using vectors, we can define the angular quantities for rotation about an \textbf{axis that can point in any direction}. Given an axis of rotation, the path of any particle rotating about that axis can be described by a circle that lies in the plane perpendicular to that axis of rotation, as illustrated in Figure \ref{fig:rotationaldynamics:axis}.
\capfig{0.5\textwidth}{figures/RotationalDynamics/axis.png}{\label{fig:rotationaldynamics:axis} Defining the vector $\vec r$ and the angular velocity, $\vec \omega$ for a particle with velocity $\vec v$ rotating about an axis in a general direction.}

We define the vector, $\vec r$, for a particle to be the vector that goes from the axis of rotation to the particle and is in a plane perpendicular to the axis of rotation, as in Figure \ref{fig:rotationaldynamics:axis}. Given the velocity vector of the particle, $\vec v$, we define its angular velocity vector, $\vec\omega$, \textbf{about the axis of rotation}, as:
\begin{align}
\Aboxed{\vec \omega = \frac{1}{r^2} \vec r \times \vec v}
\end{align}
The angular velocity vector is perpendicular to both the velocity vector and the vector $\vec r$, since it is defined as their cross-product. Thus, the \textbf{angular velocity vector is co-linear with the axis of rotation}. By using the angular velocity vector, we can specify \textbf{the direction of the axis of rotation as well as the direction in which the particle is rotating about that axis}. The direction of rotation is given by the right hand rule for rotational quantities: the direction of rotation is that obtained by curling the fingers of the right-hand when the thumb points in the same direction as the angular velocity vector, as illustrated in Figure \ref{fig:rotationaldynamics:hand}.
\capfig{0.5\textwidth}{figures/RotationalDynamics/hand.png}{\label{fig:rotationaldynamics:hand} The right-hand rule for rotational quantities.}

TODO: Need to update the figure to correspond to what we need here, and likely update the Vectors appendix to talk about this. We should perhaps also find a better name than "right hand rule for rotational quantities".

This definition of the angular velocity is consistent with the description from the previous section for motion about a circle of radius $R$ that lies in the $xy$ plane, as in Figure \ref{fig:rotationaldynamics:vcircle}. In that case, the magnitude of the angular velocity is given by:
\begin{align*}
\omega &=\frac{1}{r^2} || \vec r \times \vec v||= \frac{1}{r^2}r v\sin\phi= \frac{v}{R}\\
\therefore v &= R\omega
\end{align*}
where $\phi$ is the angle between the vectors $\vec r$ and $\vec v$ ($\SI{90}{\degree}$ for motion around a circle). The direction of the angular velocity in Figure \ref{fig:rotationaldynamics:vcircle} is in the positive $z$ direction, which corresponds to counter-clockwise rotation about the $z$ axis. 

TODO: Checkpoint MC. You push on the right-hand side of a door to open it, as the door's hinges are on the left. The angular velocity of the door is: upwards (correct), downwards, forwards, backwards.

One can always define an angular velocity vector \textbf{relative to a point}, even if the particle is not moving along a circle. If we define the vector $\vec r$ to be the vector from the point of rotation to the particle, then the angular velocity vector describes the motion of the particle as if it were instantaneously moving in a circle centred at the point of rotation, in a plane given by the vectors $\vec r$ and $\vec v$. 

Consider, for example, the particle in Figure \ref{fig:rotationaldynamics:vline} which is moving in a straight line with a velocity vector in the $xy$ plane at a position $\vec r$ relative to the origin. We can define its angular velocity vector relative to the origin, which will be in the positive $z$ direction. 
\capfig{0.4\textwidth}{figures/RotationalDynamics/vline.png}{\label{fig:rotationaldynamics:vline} Angular position for a particle moving in a straight line.}
The angular velocity describes the motion of the particle as if it were \textbf{instantaneously moving along a circle of radius $r$ centred about the origin}. The angular velocity is related to the component of $\vec v$, $v_\perp$, that is perpendicular to $\vec r$ (which is the component tangent to the circle of radius $r$, in Figure \ref{fig:rotationaldynamics:vline}):
\begin{align}
||\vec \omega|| = \frac{1}{r^2} || \vec r \times \vec v||=\frac{v\sin\phi}{r}= \frac{v_\perp}{r}
\end{align}
where $\phi$ is the angle between $\vec r$ and $\vec v$.

Similarly, we can define the angular acceleration vector, $\vec \alpha$, about an axis of rotation:
\begin{align}
\vec \alpha = \frac{1}{r^2}\vec r \times \vec a
\end{align}
where $\vec a$ is the particle's acceleration vector,  and $\vec r$ is the vector from the axis of rotation to the particle. The direction of the angular acceleration is co-linear with the axis of rotation and the right-hand rule gives the rotational direction of the angular acceleration. We can also define the angular acceleration about a point; in that case, the direction of the vector will define an instantaneous axis of rotation about a circle of radius $r$ centred at the point as well as the direction of the angular acceleration about that axis.

Finally, we can define an angular displacement vector, $\vec \theta$, relative to an axis of rotation. The direction of the angular displacement vector will be co-linear with the axis of rotation, its direction will indicate the direction of rotation about that axis, and its magnitude (in radians) will correspond to the angular displacement (as shown in Figure \ref{fig:rotationaldynamics:axis}). We can only relate the angular displacement vector to an infinitesimal linear displacement vector, $d\vec s$, since the position vector $\vec r$ from the axis of rotation will be different at each end of the displacement vector, if the displacement is large:
\begin{align*}
d\vec \theta &= \frac{1}{r^2} \vec r \times d\vec s\\
\end{align*}

TODO: Checkpoint question: An ant on a disk that is rotating slower and slower as illustrated. MC: the angular velocity vector is into the page and the angular acceleration is out of the page, etc... (they will be in opposite directions) 


The angular velocity vector is the rate of change of the angular displacement vector:
\begin{align*}
\vec\omega &= \frac{d\vec \theta}{dt} = \frac{d}{dt} \frac{1}{r^2} \vec r \times d\vec s = \frac{1}{r} \frac{d\vec s}{dt} = \frac{1}{r}\vec v_s\\
\therefore v_s &= r\omega
\end{align*}
where $\vec v_s$ is the (instantaneous) tangential velocity around the circle. The angular acceleration vector is the rate of change of the angular velocity vector:
\begin{align*}
\vec\alpha = \frac{d}{dt} \vec \omega
\end{align*}

Given the angular kinematic quantities, the related linear quantities at a position $\vec r$ from the axis of rotation are given by:
\begin{align}
d\vec s &= d\vec\theta \times \vec r\nonumber\\
\vec v_s &= \vec \omega \times \vec r\nonumber\\
\vec a_s&= \vec \alpha \times \vec r
\end{align}
where the linear quantities are always in the direction perpendicular to $\vec r$ (tangent to the circle, for motion around a circle). In other words, one cannot, say, take the acceleration vector, obtain the angular acceleration vector, and then get back the original acceleration vector - one will only get back the component of the acceleration vector that is perpendicular to $\vec r$.  

TODO: Checkpoint MC: A particle has an angular velocity in the negative $z$ direction. In which way is the particle's velocity vector at a point in its trajectory when it is on the positive $y$ axis? (positive x direction)

\section{Rotational dynamics for a single particle}
Suppose that a single force, $\vec F$, is acting on a particle of mass $m$.  Newton's Second Law for the particle is then given by:
\begin{align*}
\vec F = m \vec a
\end{align*}
We can define a point of rotation such that $\vec r$ is the position of the particle relative to that point. We can take the cross-product of $\vec r$ with both sides of the equation in Newton's Second Law:
\begin{align*}
\vec r \times \vec F &= m \vec r \times \vec a
\end{align*}
The left hand-side of the equation is called ``the torque of $\vec F$ relative to the point of rotation'', and is usually denoted by $\vec \tau$:
\begin{align}
\Aboxed{\vec \tau = \vec r \times \vec F}
\end{align}
The right-hand side of the equation is related to the angular acceleration vector, $\vec \alpha$, about that point of rotation:
\begin{align*}
 m \vec r \times \vec a = mr^2\vec\alpha
\end{align*}
Putting this altogether, we get:
\begin{align*}
\vec\tau = mr^2 \vec\alpha
\end{align*}

If more than one force is exerted on the particle, it is easy to show that the \textbf{net torque} from the net force on the particle \textbf{is equal to the sum of the torques on the particle}:
\begin{align*}
\vec r \times (\vec F_1 + \vec F_2 + \vec F_3 + \dots) &=  (\vec r \times \vec F_1 + \vec r \times \vec F_2 + \vec r \times \vec F_3 + \dots) \\
\therefore \vec r \times \sum \vec F &= \sum \vec \tau = \pvec \tau^{net}
\end{align*}

We can write ``Newton's Second Law for the rotational dynamics of a particle'':
\begin{align}
\Aboxed{\sum \vec \tau = \pvec \tau ^{net} = mr^2 \vec \alpha}
\end{align}
This equation provides us an alternate formulation to Newton's Second Law that is useful for describing the motion of a particle that is rotating. The left-hand side of the equation corresponds to the ``causes of motion'' (much like the sum of the forces in Newton's Second Law), and the right-hand side of the equation to the inertia and the kinematics. A few things to note when comparing to Newton's Second Law:
\begin{enumerate}
\item The rotational quantities, torque and angular acceleration, \textbf{are only defined with respect to a point or axis of rotation} (as this determines the vector $\vec r$). If one chooses a different point of rotation, then the torque and angular acceleration will be different.
\item The angular acceleration of a particle is proportional to the net torque exerted on it, much like the linear acceleration is proportional to the net force exerted on the particle.
\item Torque about a centre of rotation can be thought of as the equivalent of a force that causes things rotate about an axis that goes through the point of rotation and that is parallel to the torque/angular acceleration vectors.
\item Instead of mass, it is mass times $r^2$ that plays the role of inertia and determines how large of an angular acceleration a particle will experience for a given net torque.  
\end{enumerate}

\begin{example}{\capfig{0.3\textwidth}{figures/RotationalDynamics/rocket.png}{\label{fig:rotationaldynamics:rocket} A toy rocket accelerating around a circle of radius $R$, as seen from above.} A toy rocket is attached to a string on a horizontal frictionless table, as shown in Figure \ref{fig:rotationaldynamics:rocket}. The rocket has a mass $m$ and produces a constant force of thrust with a magnitude $F$ that accelerates the rocket along a circle of radius $R$ (the length of the string). If the rocket starts at rest, what distance along the circumference of the circle will the rocket have travelled after a time, $t$?}
We can model the rocket as a point particle of mass $m$ with the following forces exerted on it:
\begin{enumerate}
\item $\vec F$, the thrust of the rocket, always acting tangent to the circle.
\item $\vec T$, the force of tension in the string, always acting towards the centre of the circle.
\item $\vec F_g$, the rocket's weight, acting into the page, with magnitude $mg$.
\item $\vec N$, a normal force exerted by the table, out of the page, with magnitude $mg$.
\end{enumerate}
Because the normal force and the weight are equal in magnitude and opposite in direction, the net force will be the sum of the force of thrust and the force of tension, which are always perpendicular to each other. Thinking about this with Newton's Second Law, we could model the force of thrust as increasing the speed of the particle, while the force of tension keeps the rocket moving in a circle (it can do no work to increase the speed, since it is always perpendicular to the motion).

We can also think about this in terms of torques and angular acceleration about the centre of the circle. The thrust will result in a net torque about the centre of rotation, which will lead to the rocket having an angular acceleration. By determining the angular acceleration, we can then model the displacement at some time, $t$, using kinematics. The force of tension will create no torque about the centre of the circle because the force of tension is always co-linear with the position vector, $\vec r$ (the cross-product of co-linear vectors is always zero).

We introduce a coordinate system whose origin coincides with the centre of the circle, as shown in Figure \ref{fig:rotationaldynamics:rocket_fbd}, so that $\vec r$ corresponds to the position of the rocket relative to the origin. We have not shown the weight and normal forces on the diagram. 
\capfig{0.4\textwidth}{figures/RotationalDynamics/rocket_fbd.png}{\label{fig:rotationaldynamics:rocket_fbd} Coordinate system to describe the motion of the rocket.}

The net torque on the rocket about the point of rotation is given by the cross-product between the thrust force, $\vec F$, and the position vector, $\vec r$:
\begin{align*}
\vec\tau^{net} = \vec r\times\vec F
\end{align*}
and will point in the positive $z$ direction (as given by the right hand rule). $\vec r$ and $\vec F$ are perpendicular, so the magnitude of the net torque is given by:
\begin{align*}
\tau^{net} = rF \sin(\SI{90}{\degree}) = RF
\end{align*}
where $R$ is the magnitude of $\vec r$. The net torque vector is thus:
\begin{align*}
\vec\tau^{net} = RF \hat z
\end{align*}
Applying the rotational version of Newton's Second Law allows us to determine the angular acceleration:
\begin{align*}
\vec \tau ^{net} &= mr^2\vec\alpha\\
RF \hat z&= mR^2\vec\alpha\\
\therefore \vec \alpha &= \frac{F}{mR}\hat z
\end{align*}
The angular acceleration vector points in the positive $z$ direction (as does the net torque), and indicates that the rocket is accelerating in the counter-clockwise direction about the $z$ axis. 

After a period of time $t$, the rocket will have covered an angular displacement, $\Delta \theta$, given by:
\begin{align*}
\Delta \theta &= \theta(t)-\theta_0 = \omega_0t + \frac{1}{2}\alpha t^2\\
&=\frac{1}{2}\frac{F}{mR} t^2
\end{align*}
The linear displacement, $\Delta s$, that corresponds to this angular displacement is:
\begin{align*}
\Delta s = R \Delta\theta = \frac{1}{2}\frac{F}{m} t^2
\end{align*}
\textbf{Discussion:} The formula that we found for the total linear displacement is the same that we would have found if the particle were moving in a straight line with a net force $F$ applied to it (as the particle would have a constant acceleration given by $F/m$).
\end{example}

%%%%%%%%%%%%%%%%%%%%%%%%%%%%%%%%%%%%%%%%%%%%%%%%%%%%%%%%%%%%%%%%%

\section{Torque}
The torque associated with a force is a mathematical tool to describe how much a particular force will cause a particle (or solid) object to rotate about a given point or a given axis of rotation. A torque is \textbf{only defined relative to an axis or point of rotation}. It never makes sense to say ``the torque is ...'', and one should always say ``the torque about this axis/point of rotation is ... ''. Angular quantities (torque, angular velocity, angular displacement, etc) are only ever defined relative to a specific axis or point of rotation.

Mathematically, the torque vector from a force $\vec F$ exerted at a position $\vec r$ relative to the axis or point of rotation is defined as:
\begin{align*}
\vec \tau = \vec r \times \vec F
\end{align*}
Note that the torque from a given force increases if that force is further from the axis of rotation (if $\vec r$ has a bigger magnitude). 

Consider the solid disk of radius $\vec r$ depicted in Figure \ref{fig:rotationaldynamics:disk}. The disk can rotate about an axis that passes through the centre of the disk and that is perpendicular to the plane of the disk. A force, $\vec F$ is exerted on the edge of the disk as shown.
\capfig{0.7\textwidth}{figures/RotationalDynamics/disk.png}{\label{fig:rotationaldynamics:disk} A force exerted on the perimeter of a disk that can rotate about an axis that is perpendicular to the disk and passes through its centre. We can determine the resulting torque by considering either the component of $\vec F$ that is perpendicular to $\vec r$ (left panel) or the component of $\vec r$ that is perpendicular to $\vec F$ (right panel). The torque vector, $\vec \tau$, is out of the page, as illustrated in the centre. }
Intuitively, that force will cause the disk to rotate in the counter-clockwise direction. The torque from the force $\vec F$ about the axis as rotation is given by:
\begin{align*}
\vec \tau = \vec r \times \vec F
\end{align*}
where the vector $\vec r$ is perpendicular to the axis of rotation and goes from the axis of rotation to the point where $\vec F$ is exerted. The direction of the torque vector is out of the page (right hand rule, see Figure \ref{fig:rotationaldynamics:disk}), and will thus lead to an angular acceleration that is also out of the page, which corresponds to the counter-clockwise direction, as anticipated. 

We can break up the force into components that are parallel ($F_\parallel$) and perpendicular ($F_\perp$) to the vector $\vec r$, as shown on the left panel of Figure \ref{fig:rotationaldynamics:disk}. Only the component of the force that is perpendicular to $\vec r$ will contribute to rotating the disk. Imagine that the force is from a string that you have attached to the perimeter of the disk; if you pull on the string such that the force is parallel to $\vec r$, the disk would not rotate. The magnitude of the torque is given by:
\begin{align}
\label{eq:rotationaldynamics:taumag}
\tau = rF\sin\phi
\end{align}
where $\phi$ is the angle between $\vec r$ and $\vec F$, as shown in Figure \ref{fig:rotationaldynamics:disk}. $F \sin \phi$ is precisely the component of $\vec F$ that is perpendicular to $\vec r$, so we could also write the magnitude of the torque as:
\begin{align*}
\tau =rF_\perp
\end{align*}
which highlights that only the component of the force that is perpendicular to $\vec r$ contributes to the torque. Instead of combining the $\sin \phi$ with $F$ to obtain $F_\perp$, the component of $\vec F$ perpendicular to $\vec r$, we can instead combine the $\sin \phi$ with $r$ in Equation \ref{eq:rotationaldynamics:taumag} to obtain $r_\perp$, the component of $\vec r$ that is perpendicular to $\vec F$. This is illustrated in the right panel of Figure \ref{fig:rotationaldynamics:disk}. The magnitude of the torque is thus also given by:
\begin{align*}
\tau =r_\perp F
\end{align*}
The quantity $r_\perp$ is called the ``lever arm'' of the force about a specific axis of rotation. 

\begin{checkpoint}\label{cp:rotationaldynamics:door}
\begin{MCquestion}{Why is the handle of a door placed on the side of the door that is opposite to the hinges?}
\item Because it increases the lever arm of a force used to rotate the door about the handle.
\item Because it increases the perpendicular component of force used to rotate the door about the hinges.
\item \label{correct:rotationaldynamics:door}  Because it increases the lever arm of a force used to rotate the door about the hinges.
\item Because it would be inconvenient if the handle were next to the hinges.
\end{MCquestion}
\end{checkpoint}

\section{Rotation about an axis versus rotation about a point}
When defining angular quantities (torque, angular acceleration, etc.), it is important to identify whether these are defined relative to an axis or to a point of rotation. This, in turn, determines the vector $\vec r$ that is involved in the definition of the angular quantities.

Consider a disk of radius $r$ with a force, $\vec F$ exerted on its perimeter, as illustrated in Figure \ref{fig:rotationaldynamics:fplane}. The disk can only rotate about an axis that is perpendicular to the disk and that goes through the centre of the disk, like a wheel mounted on an axle. The force has a component, $\vec F_{plane}$, that lies in the plane perpendicular to the axis of rotation, and a component, $\vec F_{axis}$, that is parallel to axis of rotation.
\capfig{0.6\textwidth}{figures/RotationalDynamics/fplane.png}{\label{fig:rotationaldynamics:fplane} A force exerted on disk that can only rotate about an axis through its centre and perpendicular to its plane. Only the component of $\vec F$ that is in the plane perpendicular to the axis of rotation, $\vec F_{plane}$, will contribute to the torque about the axis of rotation.}

The vector $\vec r$ is  \textbf{always defined to be perpendicular to the axis of rotation and to go from the axis of rotation to the point where the force $\vec F$ is exerted}, as illustrated. The torque obtained by taking the cross product:
\begin{align*}
\vec \tau = \vec r \times \vec F
\end{align*}
will be perpendicular to both $\vec r$ and $\vec F$, and will thus not be parallel to the axis of rotation. \textbf{Only the component of the torque that is parallel to the axis of rotation} will contribute to rotating the disk about the axis. Only the component of the force that lies in the plane perpendicular to the axis of rotation, $\vec F_{plane}$, will contribute to the component of the torque about that axis of rotation. Thus, when we need to determine the torque about an axis of rotation, we can \textbf{consider vectors $\vec r$ and $\vec F$ that lie in the plane perpendicular to the axis of rotation}. The torque of $\vec F$ relative to the axis of rotation is thus:
\begin{align*}
\vec \tau_{axis} = \vec r \times \vec F_{plane}
\end{align*}
Furthermore, only the component of $\vec F_{plane}$ that is perpendicular to $\vec r$ will contribute to that torque, as we saw in the previous section. 

In general, solid objects such as a disk can only rotate about an axis. In that case, one can consider only the components of forces that lie in the plane perpendicular to the axis of rotation in order to calculate the components of the torques about that axis that are parallel to that axis. 

A point particle may be able to rotate about any axis that goes through a point of rotation. The net torque vector on the particle about that point will indicate the direction of the axis about which the particle would rotate. This is illustrated in the left panel of Figure \ref{fig:rotationaldynamics:pointaxis}.

Instead, if the particle were constrained to rotate about the $z$ axis (e.g. if the particle is on a track), then we would use the component of the torque vector that is parallel to the $z$ axis to describe its motion, as illustrated in the right panel. The $z$ component of the torque could be determined by using only the components of the forces that lie in the plane perpendicular to the axis, and defining the vector $\vec r$ from the axis to the particle rather than from the point of rotation to the particle.

\capfig{0.7\textwidth}{figures/RotationalDynamics/pointaxis.png}{\label{fig:rotationaldynamics:pointaxis} Left panel: a particle rotating about a circle centred at the origin with an axis determined from the net torque vector. Right panel: a particle that is constrained to rotate about the $z$ axis.}

\begin{example}{A force given by $\vec F=F_x\hat x + F_y \hat y + F_z \hat z$ is exerted at a position $\vec r=r_x \hat x + r_y \hat y + r_z\hat z$. Calculate the torque about the $z$ axis as well as the torque about the origin.}
To calculate the torque about the $z$ axis, we need to take the components of the vectors $\vec r$ and $\vec F$ that lie in the $x-y$ plane, since that is the plane perpendicular to the axis of rotation (the $z$ axis). This gives:
\begin{align*}
\vec\tau_z =(r_x \hat x + r_y \hat y) \times (F_x\hat x + F_y \hat y) =(r_xF_y-r_yF_y)\hat z
\end{align*}
If instead we want to calculate the torque about the origin, we take the cross-product between the two vectors:
\begin{align*}
\vec\tau &=(r_x \hat x + r_y \hat y+ r_z\hat z) \times (F_x\hat x + F_y \hat y+ F_z \hat z)\\
&=(r_yF_z-r_zF_y)\hat x+(r_zF_x-r_xF_z)\hat y+(r_xF_y-r_yF_y)\hat z
\end{align*}
If a particle were located at the given position, the force would cause the particle to (instantaneously) rotate about an axis that goes through the origin and is parallel to the torque vector. 

\textbf{Discussion:} This example highlights the difference between calculating the torque about an axis of rotation that goes through the origin and determining the torque about the origin. When calculating the torque about an axis that goes through the origin, we only consider the components of the vectors $\vec r$ and $\vec F$ that are in the plane perpendicular to the axis of rotation. This would correspond to a situation in which the particle is constrained to move in a plane that is perpendicular to the axis of rotation. Instead, if we calculate the torque about the origin, then the torque vector determines the axis of rotation through the origing about which the particle would rotate. 
\end{example}


\section{Rotational dynamics for a solid object}
We now consider the rotational dynamics for a solid object about a specific axis of rotation. Just as we did in Chapter \ref{chap:momentumandcm}, we model a solid object as a system made of many particles of mass $m_i$. Because all of the points in a solid must move in unison, they all \textbf{rotate about an axis of rotation instead of a point}. We describe the position of each particle $i$ by a vector $\vec r_i$ that is \textbf{perpendicular to the axis of rotation and goes from the axis to the corresponding particle}, as shown in Figure \ref{fig:rotationaldynamics:blob}.
\capfig{0.3\textwidth}{figures/RotationalDynamics/blob.png}{\label{fig:rotationaldynamics:blob} Two point particles that are part of a large solid object and their position vectors relative to an axis of rotation.}

We wish to model the motion of the object as it rotates about a specific axis. Thus, when considering the net torque on any particle $i$, we only consider the component of the particle's net torque that is parallel to the axis of rotation (that component of torque that comes from forces that are in the plane perpendicular to the rotation axis).

We can write the rotational version of Newton's Second Law for particle, $i$, with mass $m_i$, and position vector $\vec r_i$ relative to the rotation axis:
\begin{align*}
\sum_k \vec\tau_{ik} = \pvec\tau_i^{net} &= m_ir_i^2\vec\alpha_i
\end{align*}
where $\vec\tau_{ik}$ is the $k$-th torque on particle $i$. $\pvec\tau_i^{net}$ is the net torque on the particle \textbf{about the axis of rotation} and $\vec\alpha_i$ is the particle's angular acceleration about that axis.

We can divide the torques exerted on a particle into internal and external torques. Internal torques are those exerted by another particle in the system, whereas external torques are exerted by something external to the system. If particle 1 exerts a torque $\vec tau$ on particle 2, particle 2 will exert an equal and opposite torque, $-\vec\tau$ on particle 1.

Indeed, consider the two particles that exert an equal and opposite force (Newton's Third Law), $\vec F$, on each other, and an arbitrary point/axis of rotation, as illustrated in Figure \ref{fig:rotationaldynamics:internaltau}. The torque on particle 1 from the force exerted by particle 2 will have the same magnitude as the torque on particle 2 from the force by particle 1. This is because both forces have the same magnitude and they are co-linear, which results in them having the same lever arm. The torque vector from each force will be in opposite directions, because the forces are in opposite direction. Newton's Third Law thus also holds for torques.
\capfig{0.3\textwidth}{figures/RotationalDynamics/internaltau.png}{\label{fig:rotationaldynamics:internaltau} Two particles will exert equal and opposite torques on each other due to Newton's Third Law; the forces exerted by each particle on the other are co-linear and will thus have the same lever arm relative to any point/axis of rotation.}

We can sum together the equations for each particle $i$:
\begin{align*}
\pvec\tau_1^{net} + \pvec\tau_2^{net} +\pvec\tau_3^{net} + \dots &= m_1r_1^2\vec\alpha_1 + m_2r_2^2\vec\alpha_2 +m_3r_3^2\vec\alpha_3 +\dots\\
\sum_i \pvec\tau_i^{net} &= \sum_i  m_ir_i^2\vec\alpha_i
\end{align*}
where the sum over all of the torques exerted on each particle will be equal to the net external torque exerted on all of the particles, since the sum of the internal torques, $\pvec\tau_i^{int}$, will be zero:
\begin{align*}
\sum_i \pvec\tau_i^{net} = \sum_i \pvec\tau_i^{int} + \sum_i \pvec\tau_i^{ext} = \sum_i \pvec\tau_i^{ext} = \pvec\tau^{ext}
\end{align*}
where $ \pvec\tau^{ext}$ is the net external torque on the system.

All of the particles are part of the same rigid body, and cannot move relative to each other. Furthermore, they must all move around circles that are centred about the axis of rotation and in a plane perpendicular to that axis. They must thus all have the same angular acceleration\footnote{They will have different linear accelerations, but the angular acceleration (and velocity) will be the same for all particles if they are moving in unison.}, $\vec\alpha_i = \vec \alpha_1 = \vec \alpha_2 =\dots=\vec\alpha$. We can thus factor the angular acceleration, $\vec \alpha$, out of the sum.

We can thus write Newton's Second Law for rotational dynamics of a solid object as:
\begin{align*}
\sum_i \pvec\tau_i^{net} &= \sum_i  m_ir_i^2\vec\alpha_i\\
\therefore \pvec\tau^{ext}&= \left(\sum_i  m_ir_i^2\right)\vec\alpha
\end{align*}
The term in parentheses describes how the various masses are distributed relative to the axis of rotation. The term in parenthesis is called the \textbf{moment of inertia of the object}, and usually denoted with the letter, $I$:
\begin{align}
\Aboxed{I = \sum_i  m_ir_i^2}
\end{align}
The moment of inertia is a property of the object \textbf{relative to a specific axis of rotation}. Re-writing Newton's Second Law for the rotational dynamics of solid objects using the moment of inertia:
\begin{align}
\Aboxed{\pvec\tau^{ext}&= I\vec\alpha}
\end{align}
The net torque exerted on an object in the direction of the axis of rotation is thus equal to its moment of inertia about that axis multiplied by its angular acceleration about that axis. In other words, the moment of inertia describes how the object will resist rotational motion given a net torque. An object with a smaller moment of inertia will have a larger angular acceleration for a given torque. Again, this is analogous to the linear case, where the acceleration of an object given a net force is determined by its inertial mass.
\begin{example}{\capfig{0.3\textwidth}{figures/RotationalDynamics/dumbbell.png}{\label{fig:rotationaldynamics:dumbbell} A dumbbell made of two small identical masses separated by a distance $L$.} Two small point masses, $m$, are connected by a mass-less rod of length $L$ to form a dumbbell, as illustrated in Figure \ref{fig:rotationaldynamics:dumbbell}. A net force of magnitude $F$ is exerted on each mass, in opposite directions, as illustrated in the Figure. What is the linear acceleration of the centre of mass of the dumbbell? What is the angular acceleration of the dumbbell relative to an axis that goes through its centre of mass and is perpendicular to the page? What is the angular acceleration of the dumbbell relative to an axis that goes through one of the masses and is perpendicular to the page? }
We model the dumbbell as a rigid body made of two point masses held at a fixed distance. The linear acceleration of the centre of mass must be zero, because the net force on the dumbbell is zero. However, just because the centre of mass does not move does not mean that all parts of the dumbbell are immobile.

First, we calculate the angular acceleration relative to an axis that is perpendicular the page and goes through the centre of mass. The centre of mass is located midway between the two masses, as illustrated in Figure \ref{fig:rotationaldynamics:dumbbell_CM}. We also define a coordinate system as shown, such that the $z$ axis is out of the page.
\capfig{0.3\textwidth}{figures/RotationalDynamics/dumbbell_CM.png}{\label{fig:rotationaldynamics:dumbbell_CM} The dumbbell rotating about the centre of mass.}
The vector from the axis of rotation to each mass will have the same magnitude, $r$, but different directions. The net external torque on the dumbbell relative to the axis that goes through the centre of mass, $\pvec\tau^{ext}$, which is equal to the sum of the torques from each force:
\begin{align*}
\pvec\tau^{ext}&= \vec r \times \vec F + (-\vec r) \times (-\vec F) \\
&= 2 (\vec r \times \vec F)=2 (r\hat x \times F\hat y) = 2rF (\hat x \times \hat y)=2rF\hat z\\
&=LF\hat z
\end{align*}
where we used the fact that $2r = L$. The net torque is thus non zero and in the positive $z$ direction; the dumbbell will have an angular acceleration that is parallel to the net torque, and thus will accelerate in the counter-clockwise direction.

The moment of inertia of the dumbbell relative to the axis through the centre of mass is given by:
\begin{align*}
I = \sum_i  m_ir_i^2 = mr^2 +mr^2 = 2mr^2 = \frac{1}{2}mL^2
\end{align*}
Using Newton's Second Law for rotational dynamics, we find the angular acceleration to be:
\begin{align*}
\pvec\tau^{ext}&= I\vec\alpha\\
LF\hat z&=\frac{1}{2}mL^2\vec\alpha\\
\therefore \vec\alpha &= \frac{2F}{mL}\hat z
\end{align*}
Because the centre of mass is fixed (the sum of the forces is zero), the two ends of the dumbbell will rotate about an axis that goes through the centre of mass. This is a feature of all situations in which the net force on an object is zero and the net torque about an axis that goes through the centre of mass is non-zero.

Let us now calculate the angular acceleration of the dumbbell about an axis that goes through one of the masses, as illustrated in Figure \ref{fig:rotationaldynamics:dumbbell_end}.
\capfig{0.3\textwidth}{figures/RotationalDynamics/dumbbell_end.png}{\label{fig:rotationaldynamics:dumbbell_end} The dumbbell rotating about one of its ends.}

We first calculate the net torque on the dumbbell. The vector that goes from the axis of rotation to the force exerted on the mass that coincides with the rotation axis is zero. Thus, only the force exerted on the mass that is not at the rotation axis contributes to the net torque:
\begin{align*}
\pvec\tau^{ext}&= \vec r \times \vec F = LF\hat z
\end{align*}
The moment of inertia of the dumbbell about this axis is:
\begin{align*}
I = \sum_i  m_ir_i^2 = m(0)^2 + m(r^2) = mL^2
\end{align*}
which is larger than it was about the centre of mass. Again, the angular acceleration is found using Newton's Second Law for rotational dynamics:
\begin{align*}
\pvec\tau^{ext}&= I\vec\alpha\\
LF\hat z&=mL^2\vec\alpha\\
\therefore \vec\alpha &= \frac{F}{mL}\hat z
\end{align*}
We find that the angular acceleration is smaller about an axis that goes through one of the mass than it is about an axis through the centre of mass. Because the centre of mass of the dumbbell is fixed, we can only think of the dumbbell as instantaneously rotating about one of its ends; that is, the motion of the dumbbell will not be such that one mass rotates about the other; this is only true instantaneously.

\textbf{Discussion: }This simple example illustrates several key features about rotational dynamics:
\begin{itemize}
\item If the sum of the forces on an object is zero, it does not mean that the entire object is stationary; it only implies that the centre of mass is stationary (or rather, moving with a constant velocity, but we can always choose to model the system in a frame of reference where the centre of mass is stationary).
\item If the sum of the forces on an object is zero, and the sum of the external torques is non-zero, the object will rotate about an axis that goes through the centre of mass. That is, all points on the object will move along circles that are centred on an axis that goes through the centre of mass. 
\item We can model the rotating object about any axis that we choose. In general, the net external torque and the moment of inertia will depend on the choice of axis, as will the resulting angular acceleration. 
\item When determining the motion of the centre of mass, we can draw a free-body diagram, and the location of where the forces are exerted do not matter.
\item When determining how the object rotates, we cannot use a free-body diagram, because it matters where the forces are applied (as the torque from a given force depends on the location where the force is applied relative to the axis of rotation).
\end{itemize}
\end{example}

\section{Equilibrium}
In this section, we consider the conditions under which an object is in static or dynamic equilibrium. An object is in equilibrium if it does not rotate when viewed in a frame of reference where the object's centre of mass is stationary (or moving at constant velocity).
\subsection{Static equilibrium}
An object is in static equilibrium, if \textbf{both the sum of the external forces exerted on the object and the sum of the external torques (about any axis) are zero}. If the object is in static equilibrium the centre of mass will have no acceleration and the object will have no angular acceleration. In the centre of mass frame of reference, the object is immobile. 
\begin{example}{\capfig{0.6\textwidth}{figures/RotationalDynamics/scale.png}{\label{fig:rotationaldynamics:scale} Two masses on a balance.}
 Two masses, $m_1$ and $m_2$ are placed on a balance as shown in Figure \ref{fig:rotationaldynamics:scale}. The balance is made of a plank of mass $M$ and length $L$ that is placed on a fulcrum that is a distance $d$ from one of the edges of the plank. If mass $m_1$ is placed at a distance $r_1$ from the fulcrum, how far should mass $m_2$ be placed on the other side of the plank in order for the balance to be in equilibrium?}
We can consider the plank as the object that is in static equilibrium. Thus, the sum of the forces and the sum of the torques on the plank must be zero. We first start by identifying the forces that are exerted on the plank; these are:
\begin{enumerate}
\item $\vec F_g$, the weight of the plank, exerted at the centre of mass of the plank.
\item $\vec F_1$, a force equal to the weight of mass $m_1$, exerted at the location of $m_1$. 
\item $\vec F_2$, a force equal to the weight of mass $m_2$, exerted at the location of $m_2$.
\item $\vec N$, a normal force exerted by the fulcrum.
\end{enumerate} 
The forces are illustrated in Figure \ref{fig:rotationaldynamics:scale_fbd} along with our choice of coordinate system. The $z$ axis is not illustrated, and comes out of the page. 
\capfig{0.6\textwidth}{figures/RotationalDynamics/scale_fbd.png}{\label{fig:rotationaldynamics:scale_fbd} Forces exerted on the plank.}

All of the forces are in the $y$ direction, so we only write the $y$ component of Newton's Second Law (with zero acceleration), which allows us to determine the magnitude of the normal force:
\begin{align*}
\sum F_y = N - Mg -m_1g - m_2 g &=0\\
\therefore N &= (M+m_1+m_2) g
\end{align*}

Because the plank is in static equilibrium, the sum of the torques must also be zero. We can choose the axis of rotation about which to calculate the torques. We choose an axis that is parallel to the $z$ axis (out of the page) and goes through the fulcrum. In general, since we can choose the axis of rotation, it is usually convenient to choose an axis that goes through a point where at least one force is being exerted, because the torque from that force will be zero (its lever arm will be zero).  Furthermore, since all of the forces are in the $xy$ plane, the net torque on the plank will be in the $z$ direction, so it makes sense to choose an axis in that direction.

The torques from the weight of the plank and from the force exerted by mass $m_2$ will be in the negative $z$ direction, and the torque from the force exerted by mass $m_1$ will be in the positive $z$ direction. The normal force will not result in any torque, because it is exerted at the axis of rotation and has a lever arm of zero. 

We define $\vec r_1$ as the vector from the fulcrum to mass $m_1$. The torque, $\vec \tau_1$, from the force exerted by mass $m_1$ is given by:
\begin{align*}
\vec \tau_1 &= \vec r_1 \times \vec F_1 = (-r_1 \hat x) \times (-F_1 \hat y) \\
&= r_1F_1(\hat x \times \hat y) = r_1F_1\hat z=r_1m_1g\hat z
\end{align*}
where we used the fact that the magnitude of $\vec F_1$ is $m_1 g$. Similarly, the torques from the force exerted by $m_2$, $\vec\tau_2$, and by the weight, $\vec\tau_g$, are given by:
\begin{align*}
\vec \tau_2 &=\vec r_2 \times \vec F_2 = -m_2 g r_2 \hat z\\
\vec \tau_g &=\vec r \times \vec F_g=-rMg\hat z = -\left(\frac{L}{2}-d\right)Mg\hat z
\end{align*}
where $\frac{L}{2}-d$ is the distance between the fulcrum and where the weight of the plank is exerted. We require that the $z$ component of the net torque be equal to zero (since all of the torques are in the $z$ direction), which allows us to determine $r_2$:
\begin{align*}
\sum \tau_z = \tau_{1z} + \tau_{2z} + \tau_{gz} &=0\\
m_1 g r_1 -m_2 g r_2 -\left(\frac{L}{2}-d\right)Mg &=0\\
\therefore r_2 = \frac{1}{m_2} \left(m_1r_1-\left(\frac{L}{2}-d\right)M\right)
\end{align*}
Note that because we chose to calculate the torques about a point that goes through the fulcrum, in this case, we did not need to determine the value of the normal force which we obtained from Newton's Second Law.

\textbf{Discussion: }This example highlights the fact that when an object is in static equilibrium, we can choose a convenient axis about which to calculate the torques. In this case, by calculating the torques about the fulcrum, we did not need to consider the torque from the normal force. If we had chosen a different point, then the torque from the normal force would have been non-zero, and we would have used Newton's Second Law to express the normal force in terms of the other quantities. Physically, if we had placed the fulcrum at the centre of the plank $d = L/2$, then we would have found that $m_1r_1 = m_2r_2$, the well known equation for a balance. This equation, of course, comes from requiring that the torques from the forces exerted by $m_1$ and $m_2$ are equal in magnitude and opposite in direction.
\end{example}

\begin{example}{\capfig{0.4\textwidth}{figures/RotationalDynamics/sign.png}{\label{fig:rotationaldynamics:sign} A sign is suspended on a horizontal bar of mass $M$ and length $L$.} A sign holder is built by attaching a bar of mass $M$ and length $L$ to a wall using a hinge that allows the bar to rotate in the vertical plane. The sign of mass $m$ is attached to the end of the  bar that is opposite fo the wall. The bar is held up by a rope that is attached to the wall on one end and to the bar on the other end, two thirds of the length of the bar from the wall, as illustrated in Figure \ref{fig:rotationaldynamics:sign}. The rope makes an angle $\theta$ with respect to the horizontal bar. Find the tension in the rope and the magnitude of the force exerted by the hinge onto the bar.}
The whole system does not move and so it is in static equilibrium. In order to determine the forces exerted on the bar by the rope and the hinge, we model the bar as being in static equilibrium. The forces exerted on the bar are:
\begin{enumerate}
\item $\vec F_g$, the weight of the bar, with magnitude $Mg$, exerted at the bar's centre of mass.
\item $\vec F_m$, a downwards forced exerted by the sign at the end of the bar, with magnitude $mg$.
\item $\vec T$, a force of tension exerted by the rope at a distance $2/3 L$ from the wall.
\item $\vec R$, a force exerted by the hinge on the bar at the end next to the wall\footnote{We chose the letter R for ``Reaction'', as this is the force of reaction from the hinge as the bar pushes against the hinge.}. We expect that the force from the hinge will have both a horizontal component, $R_x$, and a vertical component, $R_y$, in order for the net force on the bar to be zero.
\end{enumerate}
The forces are illustrated in Figure \ref{fig:rotationaldynamics:sign_fbd} along with our choice of coordinate system (and the $z$ axis, not shown, points out of the page).
\capfig{0.4\textwidth}{figures/RotationalDynamics/sign_fbd.png}{\label{fig:rotationaldynamics:sign_fbd} Forces on the bar that is holding the sign of mass $m$.}
We start by writing out the $x$ and $y$ components of Newton's Second Law (with zero acceleration):
\begin{align*}
\sum F_x &= R_x - T\cos\theta =0\\
\sum F_y &= R_y + T\sin\theta - Mg - mg=0
\end{align*}
We can choose the axis about which to calculate the torques. Since all of the forces are in the $xy$ plane, we choose to calculate the torques about an axis parallel to the $z$ axis that goes through the hinge on the wall. The force from the hinge, $\vec R$, will thus result in a torque of zero (since it has a lever arm of zero). The torque from each force about the hinge is given by:
\begin{align*}
\vec \tau_M &= \vec r_M \times \vec F_g = \left(\frac{L}{2}\hat x\right) \times (-Mg \hat y) =-Mg\frac{L}{2} \hat z\\
\vec \tau_T &= \vec r_T \times \vec T = \left(\frac{L}{3}\hat x\right) \times (-T\cos\theta \hat x + T\sin\theta \hat y) =T\sin\theta\frac{L}{3} \hat z\\
\vec \tau_m &= \vec r_m \times \vec F_m = (L\hat x) \times (-mg \hat y) =-mgL\hat z\\
\end{align*}
The sum of the torques in the $z$ direction must be zero for static equilibrium, which allows us to determine the magnitude of the force of tension:
\begin{align*}
\sum \tau_z = \tau_{Mz} + \tau_{Tz}+ \tau_{mz} &=0\\
-Mg\frac{L}{2} + T\sin\theta\frac{L}{3} -mgL &=0\\
-Mg\frac{1}{2} + T\sin\theta\frac{1}{3} -mg &=0\\
\therefore T&= \frac{3g}{\sin\theta} \left( m + \frac{M}{2}\right)
\end{align*}
Using the $x$ and $y$ components of Newton's Second Law, we can now use the tension to determine the $x$ and $y$ components of the force exerted by the hinge:
\begin{align*}
R_x &= T\cos\theta = \frac{3g}{\tan\theta} \left( m + \frac{M}{2}\right)\\
R_y &= (M+m)g - T\sin\theta = (M+m)g - 3g \left( m + \frac{M}{2}\right)=  -\left( 2m + \frac{M}{2}\right)g
\end{align*}
We find that the $y$ component from the hinge is in the negative $y$ direction, so \textbf{our diagram in Figure \ref{fig:rotationaldynamics:sign_fbd} is wrong}! If you removed the hinge on the wall and instead held that end of the bar with your hand, you would feel that the end of the bar is trying to go into the wall and upwards, as the bar tries to rotate with the opposite end moving downwards due to the weight of the sign. You would have to push in the positive $x$ and negative $y$ direction to keep the bar from moving. 

\textbf{Discussion: }In this example, we saw that we needed to use both the sum of the forces and the sum of the torques in order to determine the forces on the bar. 
\end{example}

\subsection{Dynamic equilibrium}
TODO: Review box on inertial forces

When an object is in dynamic equilibrium, its centre of mass is accelerating, but the object is not rotating when viewed from its centre of mass frame of reference. Thus, the sum of the external forces exerted on the object is not zero, while the net external torque exerted on the object is zero, in the frame of reference of the centre of mass.


Consider, for example, a speed skater going around a circular track or radius $R$, and leaning into the centre making an angle $\theta$ with the ice, as depicted in Figure \ref{fig:rotationaldynamics:skater}. The skater's centre of mass is accelerating, because she is going around a circle, so the net force on the skater is not zero. However, in the reference frame of the skater, the skater is not rotating; she is thus in dynamic equilibrium.
\capfig{0.3\textwidth}{figures/RotationalDynamics/skater.png}{\label{fig:rotationaldynamics:skater} A speed skater leaning in as she goes around a circle.}

The forces on the skater are:
\begin{enumerate}
\item $\vec F_g$, her weight, exerted at her centre of mass with magnitude $Mg$.
\item $\vec N$, a normal force, exerted by the ice upwards on her skates.
\item $\vec f_s$, a force of static friction, exerted towards the centre of the circle, by the ice on her skates.
\end{enumerate}
The forces are illustrated in Figure \ref{fig:rotationaldynamics:skater_fbd} along with our choice of coordinate system.
\capfig{0.5\textwidth}{figures/RotationalDynamics/skater_fbd.png}{\label{fig:rotationaldynamics:skater_fbd} Forces on the speed skater from Figure \ref{fig:rotationaldynamics:skater}.}

The sum of the forces exerted on the skater must be towards the centre of the circle and equal to the mass of the skater times her centripetal acceleration (which is the acceleration of her centre of mass, $\vec a_{CM}$). The $x$ and $y$ components of Newton's Second Law are thus given by:
\begin{align*}
\sum F_x &= -f_s = -ma_{CM}- m\frac{v^2}{R}\\
\sum F_y &= N-mg = 0
\end{align*}

All of the forces exerted on the skater are in the $xy$ plane, so we consider torques about an axis that is co-linear with the $z$ axis. Consider the torques about an axis through the point of contact between the skates and the ice; there is a net torque in the counter-clockwise direction due to the weight of the skater (the weight is the only force that can result in a torque about the point of contact with the ice). We expect that the skater would topple over, however, this must not be a correct model for the skater, since we know that it is possible for her to lean in without falling. 

Consider, instead, the sum of the torques about an axis through her centre of mass. If the skater has a length $L$ and the centre of mass is in the middle of the skater, the sum of the torques about the centre of mass is given by the torques from the normal forces and the force of friction:
\begin{align*}
\sum \tau = \tau_{Nz} + \tau_{f_sz} = \frac{L}{2}\cos\theta N - \frac{L}{2}\sin\theta f_s
\end{align*}
About the centre of mass, the torques must be zero for the skater not to rotate, and this would give a relation between the force of static friction and the normal force.

Why do we get an incorrect model when we take the torques about the point of contact between the ice and the skater? In order to determine if the skater is rotating, we need to be in the same reference frame as the skater. However, the frame of reference of the skater is not an inertial frame of reference, since the skater is accelerating. We can still model the forces on the skater in the non-accelerating frame of reference, \textbf{as long as we include the inertial force $-m\vec a_{CM}$} in that frame of reference. In the frame of reference of the skater, there is an additional inertial force $-m\vec a_{CM}$ in order for the sum of the forces to be zero (in the frame of reference of the skater, the sum of the forces must be zero since the skater is not accelerating in that frame of reference). The additional inertial force is exerted at the centre of mass, as illustrated in Figure \ref{fig:rotationaldynamics:skater_fbdcm}.
\capfig{0.5\textwidth}{figures/RotationalDynamics/skater_fbdcm.png}{\label{fig:rotationaldynamics:skater_fbdcm} Forces on the speed skater from Figure \ref{fig:rotationaldynamics:skater} as seen in the accelerating frame of reference of the centre of mass.}

The reason that our model worked when taking the torques about the centre of mass is that the inertial force, exerted at the centre of mass, does not result in a torque (since it has a lever arm of zero). Our model was technically wrong, but if we take the torques about the centre of mass, then we do not need to worry about the inertial force. If we include the additional inertial force, then we can take the torques about any point, just as in the static equilibrium case.

\section{Moment of inertia}
In order to model how an object rotates about an axis, we can use Newton's Second Law for rotational dynamics:
\begin{align*}
\pvec\tau^{ext} = I \vec \alpha
\end{align*}
where $\pvec\tau^{ext}$ is the net external torque exerted on the object about the axis of rotation, $\vec \alpha$ is the angular acceleration of the object, and $I$ is the moment of inertia of the object (about the axis). If we consider the object as being made of many particles of mass $m_i$ each located at a position $\vec r_i$ relative to the axis of rotation, the moment of inertia is defined as:
\begin{align*}
I = \sum_i m_i r_i^2
\end{align*}
Consider, for example, the moment of inertia of a uniform rod of mass $M$ and length $L$ that is rotated about an axis perpendicular to the rod that pass through one of the ends of the rod, as depicted in Figure \ref{fig:rotationaldynamics:rod}.
\capfig{0.5\textwidth}{figures/RotationalDynamics/rod.png}{\label{fig:rotationaldynamics:rod} A rod of length $L$ and mass $M$ being rotated about an axis perpendicular to the rod that goes through one of its ends.}
We introduce the linear mass density of the rod, $\lambda$, as the mass per unit length:
\begin{align*}
\lambda = \frac{M}{L}
\end{align*}
We model the rod as being made of many small mass elements of mass $\Delta m$, of length $\Delta r$, at a location $r_i$, as illustrated in Figure \ref{fig:rotationaldynamics:rod}. Using the linear mass density, the mass element has a mass of:
\begin{align*}
\Delta m = \lambda \Delta r
\end{align*}
The rod is made of many such mass elements, and the moment of inertia of the rod is thus given by:
\begin{align*}
I &= \sum_i \Delta m r_i^2 =\sum_i \lambda \Delta r r_i^2
\end{align*}
If we take the limit in which the length of the mass element is infinitesimally small ($\Delta r \to dr$) the sum can be written as an integral over the dimension of the rod:
\begin{align*}
I &= \int_0^L\lambda r_i^2dr = \frac{1}{3}\lambda L^3 = \frac{1}{3}\left( \frac{M}{L} \right)L^3 \\
&=\frac{1}{3} ML^2
\end{align*}
where we re-expressed the linear mass density in terms of the mass and length of the rod. In general, we can write the moment of inertia of a continuous object as:
\begin{align*}
I = \int r^2 dm 
\end{align*}
where $dm$ is a small mass element that makes up the object, $r$ is the distance from that mass element to the axis of rotation, and the integral is over the dimension of the object. As we did above, we would usually set up this integral so that $dm$ is expressed in terms of $r$ so that we can take an integral over $r$. 

\begin{example}{Calculate the moment of inertia of a uniform thin ring of mass $M$ and radius $R$, rotated about an axis that goes through its centre and is perpendicular to the disk.}
We take a small mass element $dm$ of the ring, as shown in Figure \ref{fig:rotationaldynamics:ring}. 
\capfig{0.3\textwidth}{figures/RotationalDynamics/ring.png}{\label{fig:rotationaldynamics:ring} A small mass element on a ring.}
The moment of inertia is given by:
\begin{align*}
I = \int dm r^2
\end{align*}
In this case, each mass element around the ring will be the same distance away from the axis of rotation. The value $r^2$ in the integral is a constant over the whole ring, and so can be taken out of the integral:
\begin{align*}
I = \int dm r^2 = R^2\int dm
\end{align*}
where we used the fact that the ring has a radius $R$, so the distance $r$ of each mass element to the axis of rotation is $R$. The integral:
\begin{align*}
\int dm
\end{align*}
just means ``sum all of the mass elements, $dm$'', and is thus equal to $M$, the total mass of the ring. The moment of inertia of the ring is thus:
\begin{align*}
I = R^2\int dm = MR^2
\end{align*}
\end{example}

\begin{example}{Calculate the moment of inertia of a uniform disk of mass $M$ and radius $R$, rotated about an axis that goes through its centre and is perpendicular to the disk.}
We need to split up the disk into mass elements, $dm$, that we can sum together to obtain the moment of inertia of the disk. We can choose a ring of radius $r$ and radial thickness $dr$ for the shape of our mass element, as depicted in Figure \ref{fig:rotationaldynamics:diskI}.
\capfig{0.3\textwidth}{figures/RotationalDynamics/diskI.png}{\label{fig:rotationaldynamics:diskI} A mass element, $dm$, in the shape of a ring of radius $r$ and radial thickness $dr$.}
We can define a surface mass density, $\sigma$, equal to the mass per unit area of the disk:
\begin{align*}
\sigma = \frac{M}{\pi R^2}
\end{align*}
The mass of the ring shaped element is thus given by:
\begin{align*}
dm = \sigma 2\pi r dr
\end{align*}
where $2\pi r dr$ is the area of the mass element. You can imagine unfolding the mass element into a rectangle of height $dr$ and of length $2\pi r$ to obtain its area. Now that we have expressed the mass element in terms of $r$, we can proceed to calculate the moment of inertia of the disk.

The moment of inertia is given by:
\begin{align*}
I &= \int dm r^2 = \int_0^R \sigma 2\pi r dr r^2 =2\pi \sigma \int_0^R  r^3 dr \\
&=2 \pi \sigma \frac{1}{4}R^4 = 2\pi \left( \frac{M}{\pi R^2} \right) \frac{1}{4}R^4\\
&=\frac{1}{2}MR^2
\end{align*}
where we removed the surface mass density by expressing it in term of the total mass and radius of the disk. 
\textbf{Discussion:} The moment of inertia of a disk of mass $M$ and radius $R$ is half of that of a ring of radius $R$ and mass $M$. It is thus easier to rotate the disk than the ring. 
\end{example}

\subsection{The parallel axis theorem}
The moment of inertia of a solid object can be difficult to calculate, especially if the object is not symmetric. The parallel axis theorem allows us to determine the moment of inertia of an object about an axis, if we already know the moment of inertia of the object about an axis that is parallel and goes through the centre of mass of the object.

Consider an object for which we know the moment of inertia, $I_{CM}$, about an axis that goes through the object's centre of mass. We define a coordinate system such that the origin is located at the centre of mass, and the $z$ axis is parallel to the axis about which we know the moment of inertia, as illustrated in Figure \ref{fig:rotationaldynamics:parallel}. 
\capfig{0.3\textwidth}{figures/RotationalDynamics/parallel.png}{\label{fig:rotationaldynamics:parallel} An object with a coordinate system whose origin is at the object's centre of mass. We wish to determine the object's moment of inertia through a second, parallel, axis located a distance $h$ away from the centre of mass.}
We wish to determine the moment of inertia for the object for an axis that is parallel to the $z$ axis, but goes through a point with coordinates $(x_0,y_0)$ located a distance $h$ away from the centre of mass. The moment of inertia about an axis parallel to the $z$ axis and that goes through that point, $I_h$ is given by:
\begin{align*}
I_h = \sum_i m_i r_i^2
\end{align*}
where $m_i$ is a mass element of the object located at a distance $r_i$ from the axis of rotation. If the mass element is located at a position $(x_i,y_i)$ relative to the centre of mass, we can write the distance $r_i$ in terms of the position of the mass element, and of the position of the axis of rotation:
\begin{align*}
r_i^2 = (x_i-x_0)^2+(y_i-y_0)^2 = x_i^2-2x_ix_0+x_0^2+y_i^2-2y_iy_0+y_0^2
\end{align*}
Note that:
\begin{align*}
x_0^2 + y_0^2 = h^2
\end{align*}
The moment of inertia, $I_h$, can thus be written as:
\begin{align*}
I_h &= \sum_i m_i r_i^2 =\sum_i (m_i(x_i^2+ y_i^2)-2x_0m_ix_i-2y_0m_iy_i+m_ih^2)\\
&=\sum_i m_i(x_i^2+ y_i^2) + h^2\sum_i m_i - 2x_0 \sum_im_ix_i- 2y_0 \sum_im_iy_i
\end{align*}
where we broke the sum up into several sums, and factored constant terms ($h$, $x_0$, $y_0$) out of the sums, since these constants do not depend on which mass element we are considering. The first term is the moment of inertia about the centre of mass, since $x_i^2+y_i^2$ is the distance to the centre of mass. The second term is $h^2$ times the total mass of the object, since the sum of all the $m_i$ is just the mass, $M$, of the object. Now consider the term:
\begin{align*}
-2x_0 \sum_im_ix_i
\end{align*}
The sum, $\sum m_i x_i$ is the numerator in the definition of the $x$ coordinate of the centre of mass! The sum is thus zero, because we choose the origin to be located at the centre of mass. The last two terms in the sum are thus identically zero, because they correspond to the $x$ and $y$ coordinates of the centre of mass!

We can thus write the parallel axis theorem:
\begin{align}
\Aboxed{I_h = I_{CM} + Mh^2}
\end{align}
where $I_{CM}$ is the moment of inertia of an object of mass $M$ about an axis that goes through the centre of mass and, $I_h$, is the moment of inertia about a second axis that is parallel to the first and a distance $h$ away.

\begin{example}{In the previous section, we calculated the moment of inertia of a rod of length $L$ and mass $M$ through an axis that is perpendicular to the rod and through one of its ends, and found that it was given by:
\begin{align*}
I=\frac{1}{3}ML^2
\end{align*}
What is the moment of inertia of the rod about an axis that is perpendicular to the rod and goes through its centre of mass?}
In this case, we know the moment of inertia through an axis that does not go through the centre of mass. The centre of mass is located a distance $h=L/2$ away from the point about which we know the moment of inertia, $I_h$. 

Using the parallel axis theorem, we can find the moment of inertia through the centre of mass:
\begin{align*}
I_{CM} &= I_h - Mh^2\\
&=\frac{1}{3}ML^2 - M \left( \frac{L}{2}\right)^2 = \frac{1}{12}ML^2
\end{align*}

\textbf{Discussion: }We find that the moment of inertia about the centre of mass is smaller than when the rod is rotated about its end. This makes sense because when rotating the rod about its end, more of its mass is further away from the axis of rotation, which results in a larger moment of inertia.
\end{example}




\newpage
\section{Summary}

\begin{chapterSummary}
We can describe the kinematics of rotational motion using vectors to indicate both an axis of rotation and the direction of rotation about that axis. If a particle with velocity vector, $\vec v$, is rotating about an axis, then its angular velocity vector, $\vec\omega$, relative to that axis is defined as:
\begin{align*}
\vec\omega = \frac{1}{r^2}\vec r \times \vec v
\end{align*}
where $\vec r$ is a vector from the axis of rotation to the particle. The particle rotates in a circle that lies in the plane defined by $\vec r$ and $\vec v$, perpendicular to the axis of rotation. The direction of the angular velocity vector is co-linear with the axis of rotation and the direction of rotation is given by the right-hand rule for rotational quantities. 

One can define the angular velocity of a particle relative to a point of rotation, even if the particle is not moving in a circle. In that case, the angular velocity corresponds to the angular velocity of the particle as if it were instantaneously moving about a circle. 

If a particle moving around a circle has a tangential acceleration, $\vec a_\perp$, then its angular acceleration vector is defined as:
\begin{align*}
\vec\alpha = \frac{1}{r^2}\vec r \times \vec a_\perp
\end{align*}


The torque from a force, $\vec F$, exerted at a position $\vec r$, relative to an axis (or point) of rotation is defined as:
\begin{align*}
\vec\tau &= \vec r \times \vec F\\
\end{align*}
Torque is analogous to force in that it is used to model the causes of motion. Torques are only ever defined relative to an axis or point of rotation. The torque vector will be co-linear with the axis about which the object on which the force is exerted would rotate as a result of that force.

The magnitude of the torque can be written using either the component of the force, $F_\perp$ perpendicular to the vector $\vec r$, or the lever arm, $r_\perp$, of the force relative to the axis of rotation:
\begin{align*}
\tau &= rF\sin\phi\\
&=rF_\perp\\
&=r_\perp F
\end{align*}
where $\phi$ is the angle between the vectors $\vec r$ and $\vec F$ when these are placed ``tail to tail''.

Using rotational/angular quantities, we can modify Newton's Second Law to describe rotational dynamics about a given axis (or point) of rotation. For a point particle, this gives:
\begin{align*}
\pvec \tau^{net} = mr^2 \vec\alpha
\end{align*}
where $\pvec \tau^{net}$ is the net torque on the particle (the sum of the torques from each force exerted on the particle) about the axis, and $\vec\alpha$ is the resulting angular acceleration about that axis.

For an object (either continuous or made of point particles), the rotational version of Newton's Second Law for rotation about a specific axis is given by:
\begin{align*}
\pvec \tau^{net} = I\vec\alpha
\end{align*} 
where $I$ is the moment of inertia of the object about that axis.

Objects are in equilibrium if they are not rotating when viewed in their centre of mass frame of reference. Thus, for an object to be in equilibrium, the sum of the torques on the object, in the centre of mass reference frame, must be zero.

An object is in static equilibrium if the centre of mass is not accelerating, and thus the sum of the external forces on the object is zero. To model the torques on an object in static equilibrium, one can choose the axis about which to calculate the torques. A good choice is to choose an axis that is perpendicular to the plane in which the forces on the object are exerted (if such a plane exists), and to choose the axis to go through a point where at least one force is exerted (so that torques exerted at that point are identically zero).

An object is in dynamic equilibrium if the centre of mass is accelerating, but the object does not rotate when viewed in the frame of reference of its centre of mass. In dynamic equilibrium, if one models the torques exerted on the object about an axis that does not go through the centre of mass, then one must remember to include an inertial force exerted at the centre of mass. 

The moment of inertia of an object is given by
\begin{align*}
I = \sum_i m_ir_i^2
\end{align*}
if the object is modelled as a system of point particles of mass $m_i$ each a distance $r_i$ from the axis of rotation. For a continuous object, the moment of inertia is given by:
\begin{align*}
I = \int r^2 dm
\end{align*}
where $dm$ is a small mass element a distance $r$ from the axis of rotation and the integral is over the dimension of the object. Generally, one can set up the integral by expressing $dm$ in terms of $r$ using the density of the object, and then integrating $r$ over the dimension of the object.

If the moment of inertia of an object of mass $M$ about an axis that goes through the centre of mass is given by $I_{CM}$, then the moment of inertia, $I_h$, of the object through an axis that is parallel and a distance $h$ from the centre of mass is given by the parallel axis theorem:
\begin{align*}
I_h = I_{CM} + Mh^2 \quad \text{Parallel axis theorem}
\end{align*}



\end{chapterSummary}

\newpage
\begin{importantEquations}
This is an important equation
\begin{align*}
E = mc^2
\end{align*}

\end{importantEquations}


\newpage
\section{Thinking about the material}
\subsection{Reflect and research}

\begin{enumerate}
\item Something to research more.
\end{enumerate}
\subsection{To try at home}

\begin{tQuestion}Take a large textbook and consider the 3 axes that are parallel to the sides of the textbook and go through the centre of mass. By rotating the book along the three axes successively, determine the axis about which the moment of inertia of the textbook is the largest.\end{tQuestion}

\begin{tQuestion}Confirm that the moment of inertia of a rod is smaller if the rod is rotated about its centre of mass than if it is rotated by one of its ends.\end{tQuestion}


\subsection{To try in the lab}
Measure the moment of inertia of a disk, and compare with a prediction (e.g. build a yoyo...)

\newpage
\section{Sample problems and solutions}
\subsection{Problems}


\newpage
\subsection{Solutions}





%Rotational motion, rolling, rotational energy and momentum, 
%Simple harmonic motion
%Waves
%Fluid Mechanics
%Electric charges and fields
%Gauss Law
%Electric potential
%Electric current
%DC circuits
%Magnetic Force and fields
%Source of magnetic fields
%Induction
%Special relativity
%Quantum mechanics


%
%\appendix
%\renewcommand\chaptername{Appendix}
%%Copyright 2017 R.D. Martin
%This book is free software: you can redistribute it and/or modify it under the terms of the GNU General Public License as published by the Free Software Foundation, either version 3 of the License, or (at your option) any later version.
%
%This book is distributed in the hope that it will be useful, but WITHOUT ANY WARRANTY; without even the implied warranty of MERCHANTABILITY or FITNESS FOR A PARTICULAR PURPOSE.  See the GNU General Public License for more details, http://www.gnu.org/licenses/.
\chapter{Calculus}
\label{app:calculus}
This appendix gives a very brief introduction to calculus with a focus on the tools needed in physics. 
 \vspace{1cm}
\begin{learningObjectives}
{
\item Understand how to determine a derivative and that it measures a rate of change.
\item Understand how to determine partial derivatives and gradients.
\item Understand how to determine anti-derivatives and that integrals are sums.
}
\end{learningObjectives}

\section{Functions of real numbers}
In calculus, we work with functions and their properties, rather than with variables as we do in algebra. We are usually concerned with describing functions in terms of their slope, the area (or volumes) that they enclose, their curvature, their roots (when they have a value of zero) and their continuity. The functions that we will examine are a mapping from one or more \textit{independent} real numbers to one real number. By convention, we will use $x,y,z$ to indicate independent variables, and $f()$ and $g()$, to denote functions. For example, if we say:
\begin{align*}
f(x) &= x^2\\
\therefore f(2) &= 4
\end{align*}
we mean that $f(x)$ is a function that can be evaluated for any real number, $x$, and the result of evaluating the function is to square the number $x$. In the second line, we evaluated the function with $x=2$. Similarly, we can have a function, $g(x,y)$ of multiple variables:
\begin{align*}
g(x,y)&=x^2+2y^2\\
\therefore g(2,3)&=22
\end{align*}

We can easily visualize a function of 1 variable, for example by plotting it in python (see Appendix \ref{app:python}):
\begin{python}[caption = Plotting a function of 1 variable]
#import pacakges for creating arrays of values and for plotting
import numpy as np #arrays
import pylab as pl #plotting

#define the function:
def f(x):
    return x*x
    
#create 100 values of x between -5 and +5
xvals = np.linspace(-5,5,100)

#Plot the function evaluated at the values of x against the values of x:
pl.plot(xvals,f(xvals))
pl.xlabel('x')
pl.ylabel('f(x)')
pl.title('f(x)=$x^2$')
pl.grid()
pl.show()
\end{python}
\begin{poutput}
(*\capfig{0.6\textwidth}{figures/Calculus/xsquared.png}{\label{fig:calculus:xsquared}$f(x)=x^2$ plotted between $x=-5$ and $=+5$.}*)
\end{poutput}

Plotting a function of 2 variables is a little trickier, since we need to do it in three dimensions (one axis for $x$, one axis for $y$, and one axis for $g(x,y)$). This can be done in python with a little more work:
\begin{python}[caption = Plotting a function of 2 variables]
#import pacakges for creating arrays of values and for plotting
import numpy as np #arrays
import pylab as pl #plotting
#import package for handling 3D graphs:
from mpl_toolkits.mplot3d import Axes3D

#define the function:
def g(x,y):
    return x*x+2*y*y
    
#create 100 values of x and y between -5 and +5
xvals = np.linspace(-5,5,100)
yvals = np.linspace(-5,5,100)
#create a grid with the values of x and y:
X,Y = np.meshgrid(xvals,yvals)
#evaluate the function everywhere on the grid
gvals = g(X,Y)

#Plot the function as a surface (create a figure, add 3D, plot it):
fig = pl.figure(figsize=(10,10))
ax = fig.add_subplot(111, projection='3d')
ax.plot_surface(X,Y,gvals,cmap="Blues")
#show contours for the surface, projected on xy plane:
ax.contour(X, Y, gvals,offset=-1,cmap="Blues")
#add some labels
ax.set_xlabel('x')
ax.set_ylabel('y')
ax.set_zlabel('g(x,y)')
ax.set_title("$g(x,y)=x^2+2y^2$")
#choose the view point:
ax.view_init(elev=30, azim=-25)
pl.show()
\end{python}
\begin{poutput}
(*\capfig{0.6\textwidth}{figures/Calculus/gxy.png}{\label{fig:calculus:gxy}$g(x,y)=x^2+2y^2$ plotted for $x$ between -5 and +5 and for $y$ between -5 and +5. A function of two variables can be visualized as a surface in three dimensions. One can also visualize the function by look at its ``contours'' (the lines drawn in the $xy$ plane). }*)
\end{poutput}

Unfortunately, it becomes difficult to visualize functions of more than 2 variables, although one can usually look at projections of those functions to try and visualize some of the features (for example, contour maps are 2D projections of 3D surfaces, as shown in the xy plane of Figure \ref{fig:calculus:gxy}). When you encounter a function, it is good practice to try and visualize it if you can. For example, ask yourself the following questions:
\begin{itemize}
\item Does the function have one or more maxima and/or minima?
\item Does the function cross zero?
\item Is the function continuous everywhere?
\item Is the function always defined for any value of the independent variables?
\end{itemize} 

\section{Derivatives}
Consider the function $f(x)=x^2$ that is plotted in Figure \ref{fig:calculus:xsquared}. For any value of $x$, we can define the slope of the function as the ``steepness of the curve''. For values of $x>0$ the function increases as $x$ increases, so we say that the slope is positive. For values of $x<0$, the function decreases as $x$ increases, so we say that the slope is negative. A synonym for the word slope is ``derivative'', which is the word that we prefer to use in calculus. The derivative of a function $f(x)$ is given the symbol $\frac{df}{dx}$ to indicate that we are referring to the slope of $f(x)$ when plotted as a function of $x$. 

We need to specify which variable we are taking the derivative with respect to when the function has more than one variable but only one of them should be considered \textit{independent}. For example, the function $f(x)=ax^2+c$ will have different values if $a$ and $b$ are changed, so we have to be precise in specifying that we are taking the derivative with respect to $x$. The following notations are equivalent ways to say that we are taking the derivative of $f(x)$ with respect to $x$:
\begin{align*}
\frac{df}{dx}=\frac{d}{dx} f(x) = f'(x) = f'
\end{align*}
The notation with the prime ($f'(x),f'$) can be useful to indicate that the derivative itself is \textit{also} a function of $x$. 

The slope (derivative) of a function tells us how rapidly the value of the function is changing when the independent variable is changing. For $f(x)=x^2$, as $x$ gets more and more positive, the function gets steeper and steeper; the derivative is thus increasing with $x$. The sign of the derivative tells us if the function is increasing or decreasing, whereas its absolute value tells how quickly the function is changing (how steep it is).

We can approximate the derivative by evaluating how much $f(x)$ changes when $x$ changes by a small amount, say, $\Delta x$. In the limit of $\Delta x\to 0$, we get the derivative. In fact, this is the formal definition of the derivative: 
\begin{align}
\label{eqn:Calculus:derdef}
\Aboxed{\frac{df}{dx}=\lim_{\Delta x\to 0}\frac{\Delta f}{\Delta x} =\lim_{\Delta x\to 0}\frac{f(x+\Delta x)-f(x)}{\Delta x} }
\end{align}
where $\Delta f$ is the small change in $f(x)$ that corresponds to the small change, $\Delta x$, in $x$. This makes the notation for the derivative more clear, $dx$ is $\Delta x$ in the limit where $\Delta x\to0$, and $df$ is $\Delta f$, in the same limit of $\Delta x\to 0$.

As an example, let us determine the function $f'(x)$ that is the derivative of $f(x)=x^2$. We start by calculating $\Delta f$:
\begin{align*}
\Delta f &= f(x+\Delta x)-f(x)\\
&=(x+\Delta x)^2 - x^2\\
&=x^2+2x\Delta x+\Delta x^2 -x^2\\
&=2x\Delta x+\Delta x^2
\end{align*}
We now calculate $\frac{\Delta f}{\Delta x}$:
\begin{align*}
\frac{\Delta f}{\Delta x}&=\frac{2x\Delta x+\Delta x^2}{\Delta x}\\
&=2x+\Delta x
\end{align*}
and take the limit $\Delta x\to 0$:
\begin{align*}
\frac{df}{dx}&=\lim_{\Delta x\to 0 }\frac{\Delta f}{\Delta x}\\
&=\lim_{\Delta x\to 0 }(2x+\Delta x)\\
&=2x
\end{align*}
We have thus found that the function, $f'(x)=2x$, is the derivative of the function $f(x)=x^2$. This is illustrated in Figure \ref{fig:Calculus:ffprime}. Note that:
\begin{itemize}
\item For $x>0$, $f'(x)$ is positive and increasing with increasing $x$, just as we described earlier (the function $f(x)$ is increasing and getting steeper).
\item For $x<0$, $f'(x)$ is negative and decreasing in magnitude as $x$ increases. Thus $f(x)$ decreases and gets less steep as $x$ increases.
\item At $x=0$, $f'(x)=0$ indicating that, at the origin, the function $f(x)$ is (momentarily) flat.
\end{itemize}   

\capfig{0.9\textwidth}{figures/Calculus/ffprime.png}{\label{fig:Calculus:ffprime}$f(x)=x^2$ and its derivative, $f'(x)=2x$ plotted for x between -5 and +5.}

\begin{checkpoint}
\begin{MCquestion}{When a function has a maximum, its derivative at that point}
\item also has a maximum
\item is zero \correct
\item has a minimum
\item is infinite
\end{MCquestion}
\end{checkpoint}

\subsection{Common derivatives and properties}
It is beyond the scope of this document to derive the functional form of the derivative for any function using equation \ref{eqn:Calculus:derdef}. Table \ref{tab:Calculus:commonders} below gives the derivatives for common functions. In all cases, $x$ is the independent variable, and all other variables should be thought of as constants:

\begin{center}
\begin{tabular}{l l}
\textbf{Function, $f(x)$} & \textbf{Derivative, $f'(x)$}\\
\hline\hline
$f(x)=a$ & $f'(x)=0$ \\
$f(x)=x^n$ & $f'(x)=nx^{n-1}$ \\
$f(x)=\sin(x)$ & $f'(x)=\cos(x)$ \\
$f(x)=\cos(x)$ & $f'(x)=-\sin(x)$ \\
$f(x)=\tan(x)$ & $f'(x)=\frac{1}{\cos^2(x)}$ \\
$f(x)=e^x$ & $f'(x)=e^x$ \\
$f(x)=\ln(x)$ & $f'(x)=\frac{1}{x}$ \\
\hline
\end{tabular}
\captionof{table}{\label{tab:Calculus:commonders}Common derivatives of functions.}
\end{center}
If two functions of 1 variable, $f(x)$ and $g(x)$, are combined into a third function, $h(x)$, then there are simple rules for finding the derivative, $h'(x)$, based on the derivatives $f'(x)$ and $g'(x)$. These are summarized in Table \ref{tab:Calculus:combders} below.
\begin{center}
\begin{tabular}{l l}
\textbf{Function, $h(x)$} & \textbf{Derivative, $h'(x)$}\\
\hline\hline
$h(x)=f(x)+g(x)$ & $h'(x)=f'(x)+g'(x)$ \\
$h(x)=f(x)-g(x)$ & $h'(x)=f'(x)-g'(x)$ \\
$h(x)=f(x)g(x)$ & $h'(x)=f'(x)g(x)+f(x)g'(x)$ (The product rule) \\
$h(x)=\frac{f(x)}{g(x)}$ & $h'(x)=\frac{f'(x)g(x)-f(x)g'(x)}{g^2(x)}$ (The quotient rule)\\
$h(x)=f(g(x))$ & $h'(x)=f'(g(x))g'(x)$ (The Chain Rule) \\
\hline
\end{tabular}
\captionof{table}{\label{tab:Calculus:combders}Derivatives of combined functions.}
\end{center}
\begin{example}{Use the properties from Table \ref{tab:Calculus:combders} to show that the derivative of $\tan(x)$ is $\frac{1}{\cos^2(x)}$}
Since $\tan(x)=\frac{\sin(x)}{\cos(x)}$, we can write:
\begin{align*}
h(x) &= \frac{f(x)}{g(x)} \\
f(x) &= \sin(x)\\
g(x) &= \cos(x)
\end{align*}
Using the fourth row in Table \ref{tab:Calculus:combders}, and the common derivatives from Table \ref{tab:Calculus:commonders}, we have:
\begin{align*}
f'(x) &= \cos(x) \\
g'(x) &= -\sin(x) \\
g^2(x) &= \cos^2(x) \\
h'(x) &=\frac{f'(x)g(x)-f(x)g'(x)}{g^2(x)}\\ 
&= \frac{\cos(x)\cos(x) - \sin(x) (-\sin(x))}{\cos^2}\\
&=\frac{\cos^2(x)+\sin^2(x)}{\cos^2}\\
&=\frac{1}{\cos^2(x)}
\end{align*}
as required.
\end{example}

\begin{example}{Use the properties from Table \ref{tab:Calculus:combders} to calculate the derivative of $h(x)=\sin^2(x)$}
To calculate the derivative of $h(x)$, we need to use the Chain Rule. $h(x)$ is found by first taking $\sin(x)$ and then taking that result squared. We can thus identify:
\begin{align*}
h(x) &= \sin^2(x) = f(g(x))\\
f(x) &= x^2 \\
g(x) &= \sin(x)
\end{align*}
Using the common derivatives from Table \ref{tab:Calculus:commonders}, we have:
\begin{align*}
f'(x) &= 2x \\
g'(x) &= \cos(x)
\end{align*}
Applying the Chain Rule, we have:
\begin{align*}
h'(x) &= f'(g(x))g'(x)\\
&= 2\sin(x)g'(x)\\
&= 2\sin(x)\cos(x)
\end{align*}
where $f'(g(x))$ means apply the derivative of $f(x)$ to the function $g(x)$. Since the derivative of $f(x)$ is $f'(x)=2x$, when we apply it to $g(x)$ instead of $2x$, we get $2g(x)=2\cos(x)$.
\end{example}

\subsection{Partial derivatives and gradients}
So far, we have only looked at the derivative of a function of a single independent variable and used it to quantify how much the function changes when the independent variable changes. We can proceed analogously for a function of multiple variables, $f(x,y)$, by quantifying how much the function changes along the direction associated with a particular variable. This is illustrated in Figure \ref{fig:Calculus:fxy} for the function $f(x,y)=x^2-2y^2$, which looks somewhat like a saddle. 

\capfig{0.7\textwidth}{figures/Calculus/fxy.png}{\label{fig:Calculus:fxy}$f(x,y)=x^2-2y^2$ plotted for $x$ between -5 and +5 and for $y$ between -5 and +5. The point P labelled on the figure shows the value of the function at $f(-2,-2)$. The two lines show the function evaluated when one of $x$ or $y$ is held constant.}

Suppose that we wish to determine the derivative of the function $f(x)$ at $x=-2$ and $y=-2$. In this case, it does not make sense to simply determine the ``derivative'', but rather, we must specify \textit{in which direction} we want the derivative. That is, we need to specify in which direction we are interested in quantifying the rate of change of the function.

One possibility is to quantify the rate of change in the $x$ direction. The solid line in Figure \ref{fig:Calculus:fxy} shows the part of the function surface where $y$ is fixed at -2, that is, the function evaluated as $f(x,y=-2)$. The point $P$ on the figure shows the value of the function when $x=-2$ and $y=-2$. By looking at the solid line at point $P$, we can see that as $x$ increases, the value of the function is gently decreasing. The derivative of $f(x,y)$ with respect to $x$ when $y$ is held constant and evaluated at $x=-2$ and $y=-2$ is thus negative. Rather than saying ``The derivative of $f(x,y)$ with respect to $x$ when $y$ is held constant'' we say ``The \textbf{partial derivative} of $f(x,y)$ with respect to $x$''.

 Since the partial derivative is different than the ordinary derivative (as it implies that we are holding independent variables fixed), we give it a different symbol, namely, we use $\partial$ instead of $d$:
\begin{align*}
\die{f}{x}=\die{}{x}f(x,y)\text{ (Partial derivative of f with respect to x)}
\end{align*}
Calculating the partial derivative is very easy, as we just treat all variables as constants except for the variable with respect to which we are differentiating\footnote{To take the derivative is to ``differentiate''!}. For the function $f(x,y)=x^2-2y^2$, we have:
\begin{align*}
\die{f}{x}&=\die{}{x}(x^2-2y^2) = 2x\\
\die{f}{y}&=\die{}{y}(x^2-2y^2) = -4y
\end{align*}
At $x=-2$, the partial derivative of $f(x,y)$ is indeed negative, consistent with our observation that, along the solid line, at point $P$, the function is decreasing.

A function will have as many partial derivatives as it has independent variables. Also note that, just like a normal derivative, a partial derivative is still a function. The partial derivative with respect to a variable tells us how steep the function is in the direction in which that variable increases and whether it is increasing or decreasing.

\begin{example}{Determine the partial derivatives of $f(x,y,z)=ax^2+byz-\sin(z)$.}
In this case, we have three partial derivatives to evaluate. Note that $a$ are $b$ constants and can be thought of as numbers that we do not know.
\label{ex:Calculus:partials}
\begin{align*}
\die{f}{x}&=\die{}{x}(ax^2+byz-\sin(z)) = 2ax\\
\die{f}{y}&=\die{}{y}(ax^2+byz-\sin(z)) = bz \\
\die{f}{z}&=\die{}{y}(ax^2+byz-\sin(z)) = by-\cos(z) 
\end{align*} 
\end{example}

Since the partial derivatives tell us how the function changes in a particular direction, we can use them to find the direction in which the function changes \textit{the most rapidly}. For example, suppose that the surface from Figure \ref{fig:Calculus:fxy} corresponds to a real physical surface and that we place a ball at point $P$. We wish to know in which direction the ball will roll. The direction that it will roll in is the opposite of the direction where $f(x,y)$ increases the most rapidly (i.e. it will roll in the direction where $f(x,y)$ decreases the most rapidly). The direction in which the function increases the most rapidly is called the ``gradient'' and denoted by $\nabla f(x,y)$.

Since the gradient is a direction, it cannot be represented by a single number. Rather, we use a ``vector'' to indicate this direction. Since $f(x,y)$ has two independent variables, the gradient will be a vector with two components. The components of the gradient are given by the partial derivatives:
\begin{align*}
\nabla f(x,y) = \die{f}{x}\hat x+\die{f}{y} \hat y
\end{align*}
where $\hat x$ and $\hat y$ are the unit vectors in the $x$ and $y$ directions, respectively (sometimes, the unit vectors are denoted $\hat i$ and $\hat j$). The direction of the gradient tells us in which direction the function increases the fastest, and the magnitude of the gradient tells us how much the function increases in that direction.

\begin{example}{Determine the gradient of the function $f(x,y)=x^2-2y^2$ at the point $x=-2$ and $y=-2$.}
We have already found the partial derivatives that we need to evaluate at $x=-2$ and $y=-2$:
\begin{align*}
\die{f}{x}&= 2x\\
\die{f}{y}&= -4y \\
\therefore \nabla f(x,y) &= \die{f}{x}\hat x+\die{f}{y} \hat y \\
&=2x\hat x-4y\hat y
\end{align*}
Evaluating the gradient at $x=-2$ and $y=-2$:
\begin{align*}
\nabla f(x,y) &= 2x\hat x-4y\hat y\\
&=-4 \hat x + 8 \hat y\\
&=4 (-\hat x+2\hat y)\\
\end{align*}
The gradient vector points in the direction $(-1,2)$. That is, the function increases the most in the direction where you would take 1 pace in the negative $x$ direction and 2 paces in the positive $y$ direction. You can confirm this by looking at point $P$ in Figure \ref{fig:Calculus:fxy} and imagining in which direction you would have to go to climb the surface to get the steepest climb.
\end{example}

The gradient is itself a function, but it is not a real function (in the sense of a real number), since it evaluates to a vector. It is a mapping from real numbers $x,y$ to a vector. As you take more advanced calculus courses, you will eventually encounter ``vector calculus'', which is just the calculus for functions of multiple variables to which you were just introduced. The key point to remember here is that the gradient can be used to find the vector that points in the direction of maximal increase of the corresponding multi-variate function. This is precisely the quantity that we need in physics to determine in which direction a ball will roll when placed on a surface (it will roll in the direction opposite to the gradient vector).

\begin{checkpoint}
\begin{MCquestion}{The gradient of a function of one variable, $f(x)$, is}
\item undefined
\item zero
\item equal to its derivative  \correct
\item infinite
\end{MCquestion}
\end{checkpoint}

\subsection{Common uses of derivatives in physics}
The simplest case of using a derivative is to describe the speed of an object. If an object covers a distance $\Delta x$ in a period of time $\Delta t$, it's ``average speed'', $v_{avg}$, is defined as the distance covered by the object divided by the amount of time it took to cover that distance:
\begin{align*}
v_{avg} = \frac{\Delta x}{\Delta t}
\end{align*}
If the object changes speed (for example it is slowing down) over the distance $\Delta x$, we can still define its ``instantaneous speed'', $v$, by measuring the amount of time, $\Delta t$, that it takes the object to cover a \textit{very small distance}, $\Delta x$. The instantaneous speed is defined in the limit where $\Delta x \to 0$:
\begin{align*}
v = \lim_{\Delta x\to 0}\frac{\Delta x}{\Delta t}=\frac{dx}{dt}
\end{align*} 
which is precisely the derivative of $x(t)$ with respect to $t$. $x(t)$ is a function that gives the position, $x$, of the object along some $x$ axis as a function of time. The speed of the object is thus the rate of change of its position.

Similarly, if the speed is changing with time, then we can define the ``acceleration'', $a$, of an object as the rate of change of its speed:
\begin{align*}
a = \frac{dv}{dt}
\end{align*}


\section{Anti-derivatives and integrals}\label{sec:calculus:integrals}
In the previous section, we were concerned with determining the derivative of a function $f(x)$. The derivative is useful because it tells us how the function $f(x)$ varies as a function of $x$. In physics, we often know how a function varies, but we do not know the actual function. In other words, we often have the opposite problem: we are given the derivative of a function, and wish to determine the actual function. For this case, we will limit our discussion to functions of a single independent variable.

Suppose that we are given a function $f(x)$ and we know that this is the derivative of some other function, $F(x)$, which we do not know. We call $F(x)$ the \textbf{anti-derivative} of $f(x)$. The anti-derivative of a function $f(x)$, written $F(x)$, thus satisfies the property:
\begin{align*}
\frac{dF}{dx}=f(x)
\end{align*}
Since we have a symbol for indicating that we take the derivative with respect to $x$ ($\frac{d}{dx}$), we also have a symbol, $\int dx$, for indicating that we take the anti-derivative with respect to $x$:
\begin{align*}
\int f(x) dx &= F(x) \\
\therefore \frac{d}{dx}\left(\int f(x) dx\right) &= \frac{dF}{dx}=f(x)
\end{align*}
Earlier, we justified the symbol for the derivative by pointing out that it is like $\frac{\Delta f}{\Delta x}$ but for the case when $\Delta x\to 0$. Similarly, we will justify the anti-derivative sign, $\int f(x) dx$, by showing that it is related to a sum of $f(x)\Delta x$, in the limit $\Delta x\to 0$. The $\int$ sign looks like an ``S'' for sum.

While it is possible to exactly determine the derivative of a function $f(x)$, the anti-derivative can only be determined up to a constant. Consider for example a different function, $\tilde F(x)=F(x)+C$, where $C$ is a constant. The derivative of $\tilde F(x)$ with respect to $x$ is given by:
\begin{align*}
\frac{d\tilde{F}}{dx}&=\frac{d}{dx}\left(F(x)+C\right)\\
&=\frac{dF}{dx}+\frac{dC}{dx}\\
&=\frac{dF}{dx}+0\\
&=f(x)
\end{align*}
Hence, the function $\tilde F(x)=F(x)+C$ is also an anti-derivative of $f(x)$. The constant $C$ can often be determined using additional information (sometimes called ``initial conditions''). Recall the function, $f(x)=x^2$, shown in Figure \ref{fig:Calculus:ffprime} (left panel). If you imagine shifting the whole function up or down, the derivative would not change. In other words, if the origin of the axes were not drawn on the left panel, you would still be able to determine the derivative of the function (how steep it is). Adding a constant, $C$, to a function is exactly the same as shifting the function up or down, which does not change its derivative. Thus, when you know the derivative, you cannot know the value of $C$, unless you are also told that the function must go through a specific point (a so-called initial condition).

In order to determine the derivative of a function, we used equation \ref{eqn:Calculus:derdef}. We now need to derive an equivalent prescription for determining the anti-derivative. Suppose that we have the two pieces of information required to determine $F(x)$ completely, namely:
\begin{enumerate}
\item the function $f(x)=\frac{dF}{dx}$ (its derivative).
\item the condition that $F(x)$ must pass through a specific point, $F(x_0)=F_0$.
\end{enumerate}
\capfig{0.6\textwidth}{figures/Calculus/fint.png}{\label{fig:Calculus:fint}Determining the anti-derivative, $F(x)$, given the function $f(x)=2x$ and the initial condition that $F(x)$ passes through the point $(x_0,F_0)=(1,3)$.}

The procedure for determining the anti-derivative $F(x)$ is illustrated above in Figure \ref{fig:Calculus:fint}. We start by drawing the point that we know the function $F(x)$ must go through, $(x_0,F_0)$. We then choose a value of $\Delta x$ and use the derivative, $f(x)$, to calculate $\Delta F_0$, the amount by which $F(x)$ changes when $x$ changes by $\Delta x$. Using the derivative $f(x)$ evaluated at $x_0$, we have:
\begin{align*}
\frac{\Delta F_0}{\Delta x} &\approx f(x_0)\;\;\;\; (\text{in the limit} \Delta x\to 0 )\\
\therefore \Delta F_0 &= f(x_0) \Delta x
\end{align*}
We can then estimate the value of the function $F_1=F(x_1)$ at the next point, $x_1=x_0+\Delta x$, as illustrated by the black arrow in Figure \ref{fig:Calculus:fint} 
\begin{align*}
F_1&=F(x_1)\\
&=F(x+\Delta x) \\
&\approx F_0 + \Delta F_0\\
&\approx F_0+f(x_0)\Delta x
\end{align*}
Now that we have determined the value of the function $F(x)$ at $x=x_1$, we can repeat the procedure to determine the value of the function $F(x)$ at the next point, $x_2=x_1+\Delta x$. Again, we use the derivative evaluated at $x_1$, $f(x_1)$, to determine $\Delta F_1$, and add that to $F_1$ to get $F_2=F(x_2)$, as illustrated by the grey arrow in Figure \ref{fig:Calculus:fint}:
\begin{align*}
F_2&=F(x_1+\Delta x) \\
&\approx F_1+\Delta F_1\\
&\approx F_1+f(x_1)\Delta x\\
&\approx F_0+f(x_0)\Delta x+f(x_1)\Delta x
\end{align*}
Using the summation notation, we can generalize the result and write the function $F(x)$ evaluated at any point, $x_N=x_0+N\Delta x$:
\begin{align*}
F(x_N) \approx F_0+\sum_{i=1}^{i=N} f(x_{i-1}) \Delta x
\end{align*}
The result above will become exactly correct in the limit $\Delta x\to 0$:
\begin{align}
\label{eqn:Calculus:intsum}
F(x_N) = F(x_0)+\lim_{\Delta x\to 0}\sum_{i=1}^{i=N} f(x_{i-1}) \Delta x
\end{align}
Let us take a closer look at the sum. Each term in the sum is of the form $f(x_{i-1})\Delta x$, and is illustrated in Figure \ref{fig:Calculus:fintarea} for the same case as in Figure \ref{fig:Calculus:fint} (that is, Figure \ref{fig:Calculus:fintarea} shows $f(x)$ that we know, and Figure \ref{fig:Calculus:fint} shows $F(x)$ that we are trying to find).
\capfig{0.6\textwidth}{figures/Calculus/fintarea.png}{\label{fig:Calculus:fintarea} The function $f(x)=2x$ and illustration of the terms $f(x_0)\Delta x$ and $f(x_1)\Delta x$ as the area between the curve $f(x)$ and the $x$ axis when $\Delta x\to 0$.}

As you can see, each term in the sum corresponds to the area of a rectangle between the function $f(x)$ and the $x$ axis (with a piece missing). In the limit where $\Delta x\to 0$, the missing pieces (shown by the hashed areas in Figure \ref{fig:Calculus:fintarea}) will vanish and $f(x_i)\Delta x$ will become exactly the area between $f(x)$ and the $x$ axis over a length $\Delta x$. The sum of the rectangular areas will thus approach the area between $f(x)$ and the  $x$ axis between $x_0$ and $x_N$:
\begin{align*}
\lim_{\Delta x\to 0}\sum_{i=1}^{i=N} f(x_{i-1}) \Delta x=\text{Area between f(x) and x axis from $x_0$ to $x_N$}
\end{align*}

Re-arranging equation \ref{eqn:Calculus:intsum} gives us a prescription for determining the anti-derivative:
\begin{align*}
F(x_N) - F(x_0)&=\lim_{\Delta x\to 0}\sum_{i=1}^{i=N} f(x_{i-1}) \Delta x
\end{align*}
We see that if we determine the area between $f(x)$ and the $x$ axis from $x_0$ to $x_N$, we can obtain the difference between the anti-derivative at two points, $F(x_N)-F(x_0)$


The difference between the anti-derivative, $F(x)$, evaluated at two different values of $x$ is called the \textbf{integral} of $f(x)$ and has the following notation:
\begin{align}
\label{eqn:Calculus:intdef}
\Aboxed{\int_{x_0}^{x_N}f(x) dx=F(x_N) - F(x_0)=\lim_{\Delta x\to 0}\sum_{i=1}^{i=N} f(x_{i-1}) \Delta x}
\end{align}
As you can see, the integral has labels that specify the range over which we calculate the area between $f(x)$ and the $x$ axis. A common notation to express the difference $F(x_N) - F(x_0)$ is to use brackets:
\begin{align*}
\int_{x_0}^{x_N}f(x) dx=F(x_N) - F(x_0) =\big [ F(x) \big]_{x_0}^{x_N}
\end{align*}


Recall that we wrote the anti-derivative with the same $\int$ symbol earlier:
\begin{align*}
\int f(x) dx = F(x)
\end{align*}
The symbol $\int f(x) dx$ without the limits is called the \textbf{indefinite integral}. You can also see that when you take the (definite) integral (i.e. the  difference between $F(x)$ evaluated at two points), any constant that is added to $F(x)$ will cancel. Physical quantities are always based on definite integrals, so when we write the constant $C$ it is primarily for completeness and to emphasize that we have an indefinite integral.

As an example, let us determine the integral of $f(x)=2x$ between $x=1$ and $x=4$, as well as the indefinite integral of $f(x)$, which is the case that we illustrated in Figures \ref{fig:Calculus:fint} and \ref{fig:Calculus:fintarea}. Using equation \ref{eqn:Calculus:intdef}, we have:
\begin{align*}
\int_{x_0}^{x_N}f(x) dx&=\lim_{\Delta x\to 0}\sum_{i=1}^{i=N} f(x_{i-1}) \Delta x \\
&=\lim_{\Delta x\to 0}\sum_{i=1}^{i=N} 2x_{i-1} \Delta x 
\end{align*}
where we have:
\begin{align*}
x_0 &=1 \\
x_N &=4 \\
\Delta x &= \frac{x_N-x_0}{N}
\end{align*}
Note that $N$ is the number of times we have $\Delta x$ in the interval between $x_0$ and $x_N$. Thus, taking the limit of $\Delta x\to 0$ is the same as taking the limit $N\to\infty$. Let us illustrate the sum for the case where $N=3$, and thus when $\Delta x=1$, corresponding to the illustration in Figure \ref{fig:Calculus:fintarea}:
\begin{align*}
\sum_{i=1}^{i=N=3} 2x_{i-1} \Delta x &=2x_0\Delta x+2x_1\Delta x+2x_2\Delta x\\
&=2\Delta x (x_0+x_1+x_2) \\
&=2 \frac{x_3-x_0}{N}(x_0+x_1+x_2) \\
&=2 \frac{(4)-(1)}{(3)}(1+2+3) \\
&=12
\end{align*}
where in the second line, we noticed that we could factor out the $2\Delta x$ because it appears in each term. Since we only used 4 points, this is a pretty coarse approximation of the integral, and we expect it to be an underestimate (as the missing area represented by the hashed lines in Figure \ref{fig:Calculus:fintarea} is quite large).

If we repeat this for a larger value of N, $N=6$ ($\Delta x = 0.5$), we should obtain a more accurate answer:
\begin{align*}
\sum_{i=1}^{i=6} 2x_{i-1} \Delta x &=2 \frac{x_6-x_0}{N}(x_0+x_1+x_2+x_3+x_4+x_5)\\
&=2\frac{4-1}{6} (1+1.5+2+2.5+3+3.5)\\
&=13.5
\end{align*}

Writing this out again for the general case so that we can take the limit $N\to\infty$, and factoring out the $2\Delta x$:
\begin{align*}
\sum_{i=1}^{i=N} 2x_{i-1} \Delta x &=2 \Delta x\sum_{i=1}^{i=N}x_{i-1}\\
&=2 \frac{x_N-x_0}{N}\sum_{i=1}^{i=N}x_{i-1}
\end{align*}
Now, consider the combination:
\begin{align*}
\frac{1}{N}\sum_{i=1}^{i=N}x_{i-1}
\end{align*}
that appears above. This corresponds to the arithmetic average of the values from $x_0$ to $x_{N-1}$ (sum the values and divide by the number of values). In the limit where $N\to \infty$, then the value $x_{N-1}\approx x_N$. The average value of $x$ in the interval between $x_0$ and $x_N$ is simply given by the value of $x$ at the midpoint of the interval:
\begin{align*}
\lim_{N\to\infty}\frac{1}{N}\sum_{i=1}^{i=N}x_{i-1}=\frac{1}{2}(x_N+x_0)
\end{align*}
Putting everything together:
\begin{align*}
\lim_{N\to\infty}\sum_{i=1}^{i=N} 2x_{i-1} \Delta x &=2 (x_N+x_0)\lim_{N\to\infty}\frac{1}{N}\sum_{i=1}^{i=N}x_{i-1}\\
&=2 (x_N-x_0)\frac{1}{2}(x_N+x_0)\\
&=x_N^2 - x_0^2\\
&=(4)^2 - (1)^2 = 15
\end{align*}
where in the last line, we substituted in the values of $x_0=1$ and $x_N=4$. Writing this as the integral:
\begin{align*}
\int_{x_0}^{x_N}2x dx=F(x_N) - F(x_0)=x_N^2 - x_0^2
\end{align*}
we can immediately identify the anti-derivative and the indefinite integral:
\begin{align*}
F(x) &= x^2 +C \\
\int 2xdx&=x^2 +C
\end{align*}
This is of course the result that we expected, and we can check our answer by taking the derivative of $F(x)$:
\begin{align*}
\frac{dF}{dx}=\frac{d}{dx}(x^2+C) = 2x
\end{align*}
We have thus confirmed that $F(x)=x^2+C$ is the anti-derivative of $f(x)=2x$.

\begin{checkpoint}
\begin{MCquestion}
{The quantity $\int_{a}^{b}f(t)dt$ is equal to}
\item the area between the function $f(t)$ and the $f$ axis between $t=a$ and $t=b$
\item the sum of $f(t)\Delta t$ in the limit $\Delta t\to 0$ between $t=a$ and $t=b$ \correct
\item the difference $f(b) - f(a)$.
\end{MCquestion}
\end{checkpoint}

\subsection{Common anti-derivative and properties}
Table \ref{tab:Calculus:commonints} below gives the anti-derivatives (indefinite integrals) for common functions. In all cases, $x,$ is the independent variable, and all other variables should be thought of as constants:
\begin{center}
\begin{tabular}{l l}
\textbf{Function, $f(x)$} & \textbf{Anti-derivative, $F(x)$}\\
\hline\hline
$f(x)=a$ & $F(x)=ax+C$ \\
$f(x)=x^n$ & $F(x)=\frac{1}{n+1}x^{n+1}+C$ \\
$f(x)=\frac{1}{x}$ & $F(x)=\ln(x)+C$ \\
$f(x)=\sin(x)$ & $F(x)=-\cos(x)+C$ \\
$f(x)=\cos(x)$ & $F(x)=\sin(x)+C$ \\
$f(x)=\tan(x)$ & $F(x)=-ln(|\cos(x)|)+C$ \\
$f(x)=e^x$ & $F(x)=e^x+C$ \\
$f(x)=\ln(x)$ & $F(x)=x\ln(x)-x+C$ \\
\hline
\end{tabular}
\captionof{table}{\label{tab:Calculus:commonints}Common indefinite integrals of functions.}
\end{center}

Note that, in general, it is much more difficult to obtain the anti-derivative of a function than it is to take its derivative. A few common properties to help evaluate indefinite integrals are shown in Table \ref{tab:Calculus:intprops} below.
\begin{center}
\begin{tabular}{l l}
\textbf{Anti-derivative} & \textbf{Equivalent anti-derivative}\\
\hline\hline
$\int (f(x)+g(x)) dx$ &$\int f(x)dx+\int g(x) dx$ (sum)\\
$\int (f(x)-g(x)) dx$ &$\int f(x)dx-\int g(x) dx$ (subtraction)\\
$\int af(x) dx$ & $a\int f(x)dx$ (multiplication by constant)\\
$\int f'(x)g(x) dx$ & $f(x)g(x)-\int f(x)g'(x) dx$ (integration by parts)\\
\hline
\end{tabular}
\captionof{table}{\label{tab:Calculus:intprops}Some properties of indefinite integrals.}
\end{center}


\subsection{Common uses of integrals in Physics - from a sum to an integral}
Integrals are extremely useful in physics because they are related to sums. If we assume that our mathematician friends (or computers) can determine anti-derivatives for us, using integrals is not that complicated. 

The key idea in physics is that \textbf{integrals are a tool to easily performing sums}. As we saw above, integrals correspond to the area underneath a curve, which is found by \textit{summing} the (different) areas of an infinite number of infinitely small rectangles. In physics, it is often the case that we need to take the sum of an infinite number of small things that keep varying, just as the areas of the rectangles. 

Consider, for example, a rod of length, $L$, and total mass $M$, as shown in Figure \ref{fig:Calculus:rod}. If the rod is uniform in density, then if we cut it into, say, two equal pieces, those two pieces will weigh the same. We can define a ``linear mass density'', $\mu$, for the rod, as the mass per unit length of the rod:
\begin{align*}
\mu = \frac{M}{L}
\end{align*} 
The linear mass density has dimensions of mass over length and can be used to find the mass of any length of rod. For example, if the rod has a mass of $M=\SI{5}{kg}$ and a length of $L=\SI{2}{m}$, then the mass density is:
\begin{align*}
\mu=\frac{M}{L}=\frac{(\SI{5}{kg})}{(\SI{2}{m})}=\SI{2.5}{kg/m}
\end{align*}
Knowing the mass density, we can now easily find the mass, $m$, of a piece of rod that has a length of, say, $l=\SI{10}{cm}$. Using the mass density, the mass of the \SI{10}{cm} rod is given by:
\begin{align*}
m=\mu l=(\SI{2.5}{kg/m})(\SI{0.1}{m})=\SI{0.25}{kg}
\end{align*}
Now suppose that we have a rod of length $L$ that is not uniform, as in Figure \ref{fig:Calculus:rod}, and that does not have a constant linear mass density. Perhaps the rod gets wider and wider, or it has a holes in it that make it not uniform. Imagine that the mass density of the rod is instead given by a function, $\mu(x)$, that depends on the position along the rod, where $x$ is the distance measured from one side of the rod. 

\capfig{0.7\textwidth}{figures/Calculus/rod.png}{\label{fig:Calculus:rod}A rod with a varying linear density. To calculate the mass of the rod, we consider a small mass element $\Delta m_i$ of length $\Delta x$ at position $x_i$. The total mass of the rod is found by summing the mass of the small mass elements.}

Now, we cannot simply determine the mass of the rod by multiplying $\mu(x)$ and $L$, since we do not know which value of $x$ to use. In fact, we have to use all of the values of $x$, between $x=0$ and $x=L$. 

The strategy is to divide the rod up into $N$ pieces of length $\Delta x$. If we label our pieces of rod with an index $i$, we can say that the piece that is at position $x_i$ has a tiny mass, $\Delta m_i$. We assume that $\Delta x$ is small enough so that $\mu(x)$ can be taken as constant over the length of that tiny piece of rod. Then, the tiny piece of rod at $x=x_i$, has a mass, $\Delta m_i$, given by:
\begin{align*}
\Delta m_i = \mu(x_i) \Delta x
\end{align*}
where $\mu(x_i)$ is evaluated at the position, $x_i$, of our tiny piece of rod. The total mass, $M$, of the rod is then the sum of the masses of the tiny rods, in the limit where $\Delta x\to 0$:
\begin{align*}
M &= \lim_{\Delta x\to 0}\sum_{i=1}^{i=N}\Delta m_i \\
  &= \lim_{\Delta x\to 0}\sum_{i=1}^{i=N} \mu(x_i) \Delta x
\end{align*}
But this is precisely the definition of the integral (equation \ref{eqn:Calculus:intsum}), which we can easily evaluate with an anti-derivative:
\begin{align*}
M &=\lim_{\Delta x\to 0}\sum_{i=1}^{i=N} \mu(x_i) \Delta x \\
  &= \int_0^L \mu(x) dx \\
  &= G(L) - G(0)
\end{align*}
where $G(x)$ is the anti-derivative of $\mu(x)$.

Suppose that the mass density is given by the function:
\begin{align*}
\mu(x)=ax^3
\end{align*}
with anti-derivative (Table \ref{tab:Calculus:commonints}):
\begin{align*}
G(x)=a\frac{1}{4}x^4 + C
\end{align*}
Let $a=\SI{5}{kg/m^4}$ and let's say that the length of the rod is $L=\SI{0.5}{m}$. The total mass of the rod is then:
\begin{align*}
M&=\int_0^L \mu(x) dx \\
&=\int_0^L ax^3 dx \\
&= G(L)-G(0)\\
&=\left[ a\frac{1}{4}L^4 \right] - \left[ a\frac{1}{4}0^4 \right]\\
&=\SI{5}{kg/m^4}\frac{1}{4}(\SI{0.5}{m})^4 \\
&=\SI{78}{g}\\
\end{align*}

With a little practice, you can solve this type of problem without writing out the sum explicitly. Picture an \textit{infinitesimal} piece of the rod of length $dx$ at position $x$. It will have an \textit{infinitesimal} mass, $dm$, given by:
\begin{align*}
dm = \mu(x) dx
\end{align*}
The total mass of the rod is the then the sum (i.e. the integral) of the mass \textit{elements}
\begin{align*}
M = \int dm
\end{align*}
and we really can think of the $\int$ sign as a sum, when the things being summed are \textit{infinitesimally} small. In the above equation, we still have not specified the range in $x$ over which we want to take the sum; that is, we need some sort of index for the mass elements to make this a meaningful definite integral. Since we already know how to express $dm$ in terms of $dx$, we can substitute our expression for $dm$ using one with $dx$:
\begin{align*}
M = \int dm = \int_0^L \mu(x) dx
\end{align*}
where we have made the integral definite by specifying the range over which to sum, since we can use $x$ to ``label'' the mass elements.

One should note that coming up with the above integral is physics. Solving it is math. We will worry much more about writing out the integral than evaluating its value. Evaluating the integral can always be done by a mathematician friend or a computer, but determining which integral to write down is the physicist's job!

\newpage
\section{Summary}
\vspace{0.5cm}

\begin{chapterSummary}
The derivative of a function, $f(x)$, with respect to $x$ can be written as:
\begin{align*}
\frac{d}{dx} f(x)=\frac{df}{dx}=f'(x)
\end{align*}
and measures the rate of change of the function with respect to $x$. The derivative of a function is generally itself a function. The derivative is defined as:
\begin{align*}
f'(x) = \lim_{\Delta x \to 0}\frac{f(x+\Delta x)-f(x)}{\Delta x}
\end{align*}
Graphically, the derivative of a function represents the slope of the function, and it is positive if the function is increasing, negative if the function is decreasing and zero if the function is flat.  Derivatives can always be determined analytically for any continuous function.

A partial derivative measures the rate of change of a multi-variate function, $f(x,y)$, with respect to one of its independent variables. The partial derivative with respect to one of the variables is evaluated by taking the derivative of the function with respect to that variable while treating all other independent variables as if they were constant. The partial derivative of a function (with respect to $x$) is written as:
\begin{align*}
\die{f}{x}
\end{align*}
The gradient of a function, $\nabla f(x,y)$, is a vector in the direction in which that function is increasing most rapidly. It is given by:
\begin{align*}
\nabla f(x,y)=\die{f}{x}\hat x + \die{f}{y} \hat y
\end{align*}

Given a function, $f(x)$, its anti-derivative with respect to $x$, $F(x)$, is written:
\begin{align*}
F(x) = \int f(x) dx
\end{align*}
$F(x)$ is such that its derivative with respect to $x$ is $f(x)$:
\begin{align*}
\frac{dF}{dx}=f(x)
\end{align*}
The anti-derivative of a function is only ever defined up to a constant, $C$. We usually write this as:
\begin{align*}
\int f(x) dx = F(x) + C
\end{align*}
since the derivative of $F(x) +C$ will also be equal to $f(x)$. The anti-derivative is also called the ``indefinite integral'' of $f(x)$. 

The definite integral of a function $f(x)$, between $x=a$ and $x=b$, is written:
\begin{align*}
\int_a^b f(x) dx
\end{align*}
and is equal to the difference in the anti-derivative evaluated at $x=a$ and $x=b$:
\begin{align*}
\int_a^b f(x) dx = F(b) - F(a)
\end{align*}
where the constant $C$ no longer matters, since it cancels out. Physical quantities only ever depend on definite integrals, since they must be determined without an arbitrary constant. 

Definite integrals are very useful in physics because they are related to a sum. Given a function $f(x)$, one can relate the sum of terms of the form $f(x_i)\Delta x$ over a range of values from $x=a$ to $x=b$ to the integral of $f(x)$ over that range:
\begin{align*}
\lim_{\Delta x\to 0}\sum_{i=1}^{i=N} f(x_{i-1}) \Delta x = \int_{x_0}^{x_N}f(x) dx=F(x_N) - F(x_0)=
\end{align*}
\end{chapterSummary}

\section{Thinking about the Material}
\begin{chapteractivity}{Reflect and research}
{
\item When was calculus first discovered, and by whom?
\item What is an example of a physical quantity that is given by a derivative (other than speed or acceleration)?
\item What is a case when you would need to perform an integral to evaluate a physical quantity?
}
\end{chapteractivity}

\section{Sample problems and solutions}
\subsection{Problems} 
\begin{problem}{soln:calculus:deriv}{\label{prob:calculus:deriv}You find that the number of customers in your store as a function of time is given by:
\begin{align*}
N(t) = a+bt-ct^2
\end{align*}
where $a$, $b$ and $c$ are constants. At what time does your store have the most customers, and what will the number of customers be? (Give the answer in terms of $a$, $b$ and $c$).}
\end{problem}

\begin{problem}{soln:calculus:int}{\label{prob:calculus:int} You measure the speed, $v(t)$, of an accelerating train as function of time, $t$, to be given by:
\begin{align*}
v(t)=at+bt^2
\end{align*}
where $a$ and $b$ are constants. How far does the train move between $t=t_0$ and $t=t_1$?
}
\end{problem}

\newpage
\subsection{Solutions}

\begin{solution}{prob:calculus:deriv}\label{soln:calculus:deriv}
We need to find the value of $t$ for which the function $N(t)$ is maximal. This will occur when its derivative with respect to $t$ is zero:
\begin{align*}
\frac{dN}{dt} &= b-2ct =0\\
\therefore t &= \frac{b}{2c}
\end{align*}
At that time, the number of customers will be:
\begin{align*}
N\left( t=\frac{b}{2c} \right) &=a+bt-ct^2\\
&=a+\frac{b^2}{2c} - \frac{b^2}{4c} = a+\frac{3b^2}{4c}
\end{align*}
\end{solution}

\begin{solution}{prob:calculus:int}\label{soln:calculus:int} We are given the speed of the train as a function of time, which is the rate of change of its position:
\begin{align*}
v(t)=\frac{dx}{dt}
\end{align*}
We need to find how its position, $x(t)$, changes with time, given the speed. In other words, we need to find the anti-derivative of $v(t)$ to get the function for the position as a function of time, $x(t)$:
\begin{align*}
x(t) &= \int v(t) dt = \int (at+bt^2) dt\\
&=\frac{1}{2}at^2 + \frac{1}{3}bt^3 + C
\end{align*}
where $C$ is an arbitrary constant. The distance covered, $\Delta x$, between time $t_0$ and time $t_1$ is simply the difference in position at those two times:
\begin{align*}
\Delta x &= x(t_1) - x(t_0)\\
&=\frac{1}{2}at_1^2 + \frac{1}{3}bt_1^3 + C - \frac{1}{2}at_0^2 + \frac{1}{3}bt_0^3 - C\\
&=\frac{1}{2}a(t_1^2-t_0^2) + \frac{1}{3}b(t_1^3-t_0^3)
\end{align*}


\end{solution}
%\chapter{Vectors}
\label{app:vectors}
This appendix gives a very brief introduction to coordinate systems and vectors.
 \vspace{1cm}
\begin{learningObjectives}
\item Understand the definition of a coordinate system
\item Understand the definition of a vector and of a scalar
\item Be able to perform algebra with vectors (addition, scalar products, vector products)
\end{learningObjectives}

\section{Coordinate systems}
Coordinate systems are used in order to be able to describe the position of an object in space. A coordinate system is an artificial mathematical tool that we construct in order to describe the position of a real object. 

\subsection{1D Coordinate systems} 
The easiest coordinate system to construct is one where we need to describe the location of objects in one dimensional space. For example, we may wish to describe the location of a train along a straight section of track that runs in the East-West direction. In order to do so, we must first define an ``origin'', which is the reference point of our coordinate system. For example, the origin for our train track may be the Kingston train station. We can describe the position of the train by specifying how far it is from the train station (the origin), using a single real number, say $x_T$. If the train is at position $x_T=0$, then we know that it is at the Kingston station. If the object is not at the origin, then we need to be able to specify on which side (East or West in our train example) of the origin the object is located. We do this by choosing a direction for our one dimensional coordinate $x$. For example, we may choose that the East side of the track corresponds to positive values of $x_T$ and that the West side of the track correspond to the negative values of $x_T$. Thus, in order to fully specify a coordinate system we need to choose:
\begin{itemize}
\item the location of the origin
\item the direction in which the coordinate, $x$, increases
\item the units in which we wish to express $x$
\end{itemize} 

TODO: make figure to illustrate 1D x-axis

In one dimension, it is common to use the variable $x$ to define the position along the ``$x$-axis''. The $x$-axis \textit{is} our coordinate system in one dimension, and we represent it by drawing a line with an arrow in the direction of increasing $x$ and indicate where the origin is located.
 
\subsection{2D Coordinate systems}
\rwcapfig[14]{0.35\textwidth}{figures/Vectors/xyp.png}{\label{fig:Vectors:xyp}Example of Cartesian coordinate system and a point $P$ with coordinates $(x_p,y_p)$.}
To describe the position of an object in two dimensions (e.g. a marble rolling on a table), we need to specify two numbers. The easiest way to do this is to define two axes, $x$ and $y$, whose origin and direction we must define. Figure \ref{fig:Vectors:xyp} shows an example of such a coordinate system. Although it is not necessary to do so, we chose $x$ and $y$ axes that are perpendicular to each other. The origin of the coordinate system is where the two axes intersect. One is free to choose any two directions for the axes (as long as they are not parallel). However, choosing axes that are perpendicular (a ``Cartesian'' coordinate system) is usually the most convenient.

To fully describe the position of an object, we must specify both its position along the $x$ and $y$ axes. For example, point $P$ in Figure \ref{fig:Vectors:xyp} has two \textbf{coordinates}, $x_p$ and $y_p$ that define its position. The $x$ coordinate is found by drawing a line through $P$ that is parallel to the $y$ axis and is given by the intersection of that line with the $x$ axis. The $y$ coordinate is found by drawing a line through point $P$ that is parallel to the $x$ axis and is given by the intersection of that line with the $y$ axis.


\begin{checkpointMC}{Figure \ref{fig:Vectors:xyslant} shows a coordinate system that is not orthogonal (where the $x$ and $y$ axes are not perpendicular). Which value on the figure correctly indicates the $y$ coordinate of point $P$?
\capfig{0.35\textwidth}{figures/Vectors/xyslant.png}{\label{fig:Vectors:xyslant}A non-orthogonal coordinate system (the $x$ and $y$ axes are not perpendicular).}}
\item $y_1$ %correct
\item $y_2$
\item $y_3$
\end{checkpointMC}
\capfig{0.3\textwidth}{figures/Vectors/polarp.png}{\label{fig:Vectors:polarp}Example of a polar coordinate system and a point $P$ with coordinates $(r,\theta)$.}

The most common choice of coordinate system in two dimensions is the Cartesian coordinate system that we just described, where the $x$ and $y$ axes are perpendicular and share a common origin, as shown in Figure \ref{fig:DescribingMotionInND:xyp}. When applicable, by convention, we usually choose the $y$ axis to correspond to the vertical direction.

Another common choice is a ``polar'' coordinate system where the position of an object is specified by a distance to the origin, $r$, and an angle, $\theta$, relative to a specified direction, as shown in Figure \ref{fig:DescribingMotionInND:polarp}. Often, a polar coordinate system is defined alongside a Cartesian system, so that $r$ is the distance to the origin of the Cartesian system and $\theta$ is the angle with respect to the $x$ axis.

One can easily convert between the two Cartesian coordinates, $x,y$, and the two corresponding polar coordinates, $r,\theta$:
\begin{align*}
x&=r\cos(\theta)\\
y&=r\sin(\theta)\\
r&=\sqrt(x^2+y^2)\\
\tan(\theta) &= \frac{y}{x}
\end{align*}
Polar coordinates are often used to describe the motion of an object moving around a circle, as this means that only one of the coordinates ($\theta$) changes with time (if the origin of the coordinate system is chosen to coincide with the centre of the circle).

\subsection{3D Coordinate systems}
In three dimensions, we need to specify three numbers to describe the position of an object (e.g. a bird flying in the air). In a three dimensional Cartesian coordinate system, we simply add a third axis, $z$, that is mutually perpendicular to both $x$ and $y$. The position of an object can then specified using the three coordinates, $x$, $y$, and $z$. 

Two additional coordinate systems are common in three dimensions: ``cylindrical'' and ``spherical coordinates''. All three systems are illustrated in Figure \ref{fig:Vectors:3dcoords} superimposed onto the Cartesian system.
\capfig{0.85\textwidth}{figures/Vectors/3dcoords.png}{\label{fig:Vectors:3dcoords} Cartesian (left), cylindrical (centre) and spherical (right) coordinate systems used in three dimensions. The $y$ and $z$ axes are in the plane of the page, whereas the $x$ axis comes out of the page.}

By convention, we use the $z$ axis to be the vertical direction in three dimensions. In cylindrical coordinates, we keep the same Cartesian coordinate $z$ to indicate the height above the $xy$ plane. However, we use the \textit{azimuthal angle}, $\phi$, and the radius, $\rho$, to describe the position of the projection of a point onto the $xy$ plane. $\phi$ is the angle that the projected point makes with the $x$ axis and $\rho$ is the distance of that projected point to the origin. Thus, cylindrical coordinates are very similar to the polar coordinate system introduced in two dimensions, except with the addition of the $z$ coordinate. Cylindrical coordinates are useful for describing situations with azimuthal symmetry, such as motion along the surface of a cylinder. The cylindrical coordinates are related to the Cartesian coordinates by:
\begin{align*}
\rho &= \sqrt{x^2+y^2}\\
\tan(\phi) &= \frac{y}{x}\\
z&=z
\end{align*}
In spherical coordinates, a point $P$ is described by the radius, $r$, the \textit{polar angle} $\theta$, and the \textit{azimuthal angle}, $\phi$. The radius is the distance between the point and the origin. The polar angle is the angle with the $z$ axis that is made by the line from the origin to the point. The azimuthal angle is defined in the same way as in polar coordinates. Spherical coordinates are useful for describing situations that have spherical symmetry, such as a person walking on the surface of the Earth. The spherical coordinates are related to the Cartesian coordinates by:
\begin{align*}
r &= \sqrt{x^2+y^2+z^2}\\
\tan(\theta) &= \frac{z}{r}=\frac{z}{\sqrt{x^2+y^2+z^2}}\\
\tan(\phi) &= \frac{y}{x}\\
\end{align*}

\section{Vectors}
So far, we have seen how to use a coordinate system to describe the position of a single point in space relative to an origin. In this section, we introduce the notion of a ``vector'', which allows us to describe quantities that have a \textbf{magnitude} and a \textbf{direction}. For example, you can use a vector to describe the fact that you walked \SI{5}{km} in the North direction. A vector can be visualized by an arrow. The length of the arrow is the magnitude that we wish to describe, and the direction of the arrow corresponds to the direction that we would like to describe. 

Unlike a point in space, vectors \textbf{have no location}. That is, vectors are simply an arrow, and you can choose to draw that arrow anywhere you like. In two dimensional space, one requires two numbers to completely define a vector. In three dimensional space, one requires three numbers to completely define a vector. Figure \ref{fig:Vectors:dvec} shows a two dimensional vector, $\vec d$, twice. Because both arrows in the figure have the same magnitude and direction, they represent the \textit{same} vector. When we refer to quantities that are vectors, we usually draw an arrow on top of the quantity ($\vec d$) to indicate that they are vectors. We use the word ``scalar'' to refer to numbers that are not vectors (a regular number is thus also called a scalar to distinguish it from a quantity that is a vector).

\capfig{0.35\textwidth}{figures/Vectors/dvec.png}{\label{fig:Vectors:dvec}A vector $\vec d$ shown twice, once with its Cartesian components ($d_x$, $d_y$) and once with its magnitude and direction ($d$, $\phi$).}

In analogy with coordinate systems, we have multiple ways to choose the numbers that we use to define the vector. The most convenient choice is usually to use the ``Cartesian components'' of the vector which correspond to the length of the vector when projected onto a Cartesian coordinate system. For example, in Figure \ref{fig:Vectors:dvec}, the Cartesian components of the vector $\vec d$ are labelled as ($d_x$, $d_y$) indicating that the vector has a length of $d_x$ in the $x$ direction and $d_y$ in the $y$ direction. Furthermore, the number $d_x$ is negative, since the vector points in the negative $x$ direction. Another common choice is to use the length of the vector, which we label $d$ (the name of the vector without the arrow on top), and the angle, $\phi$ that the vector makes with the $x$-axis, as illustrated in Figure \ref{fig:Vectors:dvec}. In terms of the Cartesian components, the magnitude of the vector is given by:
\begin{align*}
d&= ||\vec d||= \sqrt{d_x^2+d_y^2}
\end{align*}
where we also introduced the notation that placing two vertical bars around a vector ($||\vec d||$) is used to indicated its magnitude.


\subsection{Unit vectors} 
A special category of vectors is ``unit vectors'', which are simply vectors that have a length (magnitude) of 1 (in whichever units the coordinate system is defined). Unit vectors are particularly useful for indicating direction. For example, in Figure \ref{fig:Vectors:dvec}, we may be interested in indicating the direction of the vector $\vec d$. Unit vectors are denoted by using a ``hat'' instead of an arrow. Thus, the vector $\hat d$, is the vector of length 1 that points in the same direction as $\vec d$. The (Cartesian) components of $\hat d$ are easily found by dividing the corresponding components of $\vec d$ by $d$ (the magnitude):
\begin{align*}
(\hat d)_x &= \frac{d_x}{d}=\frac{d_x}{\sqrt{d_x^2+d_y^2}}\\
(\hat d)_y &= \frac{d_y}{d}=\frac{d_y}{\sqrt{d_x^2+d_y^2}}\\
\therefore ||\hat d||&=\sqrt{(\hat d)_x^2+(\hat d)_y^2}=\sqrt{\frac{d_x^2}{d_x^2+d_y^2}+\frac{d_y^2}{d_x^2+d_y^2}}=1
\end{align*}

A specific type of unit vectors are those units vectors that are parallel to the axes of the coordinate system. Those vectors are denoted $\hat x$, $\hat y$, $\hat z$ (and sometimes $\hat i$, $\hat j$, $\hat k$ or $\hat e_x$, $\hat e_y$, $\hat e_z$) for the $x$, $y$, and $z$ axes, respectively. 

\subsection{Notations and representation of vectors}
There are multiple notations for describing a vector using its components. The following, are all equivalent ways to write down the vector $\vec d$ in terms of its components $d_x$ and $d_y$:
\begin{align*}
\vec d &= (d_x,d_y)\quad&\text{row vector}\\
       &=\begin{pmatrix}
           d_x \\
           d_y \\
         \end{pmatrix}\quad&\text{column vector}\\
         &= d_x\hat x +d_y \hat y\quad&\text{using }\hat x,\;\hat y\\
         &=d_x\hat i +d_y \hat j \quad&\text{using }\hat i,\;\hat j
\end{align*}
For example, the unit vector $\hat y$ can be written down as (0,1,0) in three dimensions. 

\begin{checkpointMC}{What is the magnitude (the length) of the vector $5\hat x-2\hat y$?}
\item 3.0
\item 5.4% correct
\item 7.0
\item 10.0
\end{checkpointMC}

Illustrating a vector graphically in two dimensions is straightforward, but difficult in three dimensions. To help remedy this, a notation is introduced in order to draw vectors that point in or out of the page (perpendicular to the plane of the page). The notation comes from imagining that the vector is an arrow. If the vector is coming out of the page (at you!), then you would see the head of the arrow, which we represent as a circle with a dot (the dot is the point of the arrow, the circle is the base of the conically shaped arrowhead). If instead, the vector points into the page, then you would see the back of the arrow, which we represent as a cross (the cross being the feathers in the tail of the arrow). This is illustrated in Figure \ref{fig:Vectors:vector3d}.

\capfig{0.25\textwidth}{figures/Vectors/vector3d.png}{\label{fig:Vectors:vector3d}Geometric representation of three vectors. The vector $\vec a$ lies in the plane of the page, the vector $\vec b$ is pointing out of the page, and the vector $\vec c$ is pointing into the page.}


\section{Vector algebra}
In this section, we describe the various algebraic operations that can be performed using vectors. 
\subsection{Multiplication/division of a vector by a scalar}
One can multiply (or divide) a vector by a scalar (a number). Suppose that we are given a vector $\vec v=(v_c, v_y, v_z)$ and a scalar $a$. The multiplication $a\vec v$ is defined to be a new vector, say $\vec w$, whose components are the components of $\vec v$ multiplied by $a$:
\begin{align*}
\vec w = a\vec v = (av_x, a v_y)
\end{align*}
Similarly, the division of a vector by a scalar is defined analogously:
\begin{align*}
\vec w = \frac{\vec v}{a} = \left(\frac{v_x}{a}, \frac{v_y}{a}\right)
\end{align*}
\begin{checkpointMC}{What happens to the length of a vector if the vector is multiplied by 2?}
\item The length doubles% correct
\item The length is halved
\item The length is quadrupled
\item It depends on the direction of the vector
\end{checkpointMC}

In particular, this makes it easy to determine the unit vector, $\hat v$, that points in the same direction as $\vec v$:
\begin{align*}
\hat v = \frac{\vec v}{v}
\end{align*}
where $v$ is the magnitude of $\vec v$. 

\subsection{Addition/subtraction of two vectors}
The addition (subtraction) of two vectors, $\vec a$ and $\vec b$, is found by adding (subtracting) the components of the two vectors. For example, if $\vec c=\vec a+\vec b$, the components of $\vec c$ are given by:
\begin{align*}
\vec c &= \vec a + \vec b = \begin{pmatrix}
           a_x \\
           a_y \\
         \end{pmatrix} + \begin{pmatrix}
           b_x \\
           b_y \\
         \end{pmatrix}\\
         &=\begin{pmatrix}
           a_x+b_x \\
           a_y+b_y \\
         \end{pmatrix}
\end{align*}
where we chose to use the ``column vector'' notation. The column vector notation highlights the fact that the algebra (addition, subtraction) is performed independently on the $x$ and $y$ components. 
\begin{example}{Given two vectors, $\vec a=2\hat x+3\hat y$, and $\vec b=5\hat x-2\hat y$, calculate the vector $\vec c= 2\vec a- 3\vec b$.}
This can easily be solved algebraically:
\begin{align*}
\vec c &= 2\vec a- 3\vec b\\
&=2 (2\hat x+3\hat y) - 3 (5\hat x-2\hat y) \\
&=(4\hat x+6\hat y)-(15\hat x-6\hat y) \\
&=(4-15)\hat x + (6+6) \hat y\\
&= -11 \hat x + 12 \hat y
\end{align*}
We can think of these operations as being performed independently on the components:
\begin{align*}
c_x&=2a_x-3b_x=-11\\
c_y&=2a_y-3b_y=12
\end{align*} 
\end{example}

Geometrically, one can easily visualize the addition and subtraction of vectors. This is illustrated in Figure \ref{fig:Vectors:aplusbvec} for the case of adding vectors $\vec a$ and $\vec b$ to get the vector $\vec c$. Geometrically, the sum of the vectors $\vec a$ and $\vec b$ (sometimes also called the ``resultant'') can be found by:
\begin{enumerate}
\item Placing the ``tail'' of vector $\vec b$ at the ``head'' of $\vec a$ (think of an arrow, the pointy part is the head and the feathery part is the tail)
\item Drawing the vector goes from the tail of vector $\vec a$ to the head of vector $\vec b$.
\end{enumerate}

\capfig{0.55\textwidth}{figures/Vectors/aplusbvec.png}{\label{fig:Vectors:aplusbvec}Geometric addition of the vectors $\vec a$ and $\vec b$ by placing them ``head to tail''.}

Subtracting two vectors geometrically is done in the same way as addition. For example, the vector $vec c$, given by $\vec c=\vec a -\vec b$ can also be expressed as $\vec c = \vec a + (-1) \vec b$. That is, first multiply the vector $\vec b$ by minus 1 (which just reverses its direction), then add that vector, ``head to tail'', to the vector $\vec a$. 

Now that we know how to add vectors, we can better understand the notation $\vec a = a_x \hat x+ a_y\hat y$. This is not simply a notation, but is in fact algebraically correct. It means: ``multiply the vector $\hat x$ by $a_x$ (thus giving it a length of $a_x$) and then add $a_y$ times the vector $\hat y$''. This is illustrated in Figure \ref{fig:Vectors:acomponents}.

\capfig{0.35\textwidth}{figures/Vectors/acomponents.png}{\label{fig:Vectors:acomponents}Illustration that the notation $\vec a = a_x \hat x+ a_y\hat y$ is in fact the vector addition of $a_x \hat x$ and $a_y \hat y$.}


\subsection{The scalar product}
There are two ways to ``multiply'' vectors: the ``scalar product'' and the ``vector product''. The scalar product (or ``dot product'') takes two vectors and results in a scalar (a number). The vector product (or ``cros product'') takes two vectors and results in a third vector. 

The scalar product, $\vec a \cdot \vec b$, of two vectors $\vec a$ and $\vec b$, is defined as the following:
\begin{align*}
\vec a \cdot \vec b=a_xb_x +a_yb_y
\end{align*}
That is, one multiplies the individual components of the two vectors and then adds those products for each component. This is easily extended to the three dimensional case by adding a term $a_zb_z$ to the sum. One can easily show that the scalar product is also related to the angle between the two vectors when these are placed ``tail to tail'', as in Figure \ref{fig:Vectors:scalarproduct}
\begin{align*}
\vec a \cdot \vec b= ab\cos\theta
\end{align*}

\capfig{0.3\textwidth}{figures/Vectors/scalarproduct.png}{\label{fig:Vectors:scalarproduct}Illustration of the angle between vectors $\vec a$ and $\vec b$ when these are placed tail to tail.}

The scalar product between two vectors of a fixed length will be maximal when the two vectors are parallel ($\cos\theta=1$) and zero when the vectors are perpendicular ($\cos\theta =0$). The scalar product is thus useful when we want to calculate quantities that are maximal when two vectors are parallel. 


\subsection{The vector product}
The vector (or cross) product takes two vectors to produce a third vector that is \textbf{mutually perpendicular} to both vectors. The vector product only has meaning in three dimensions. Two vectors that are not co-linear can always be used to define a plane in three dimensions. The cross product of those two vectors will give a third vector that is thus perpendicular to the plane (thus making it perpendicular to both vectors). 

Algebraically, the three components of the vector product, $\vec a\times \vec b$, of vectors $\vec a$ and $\vec b$ are found as follows:
\begin{align}
\label{eqn:Vectors:crossproduct}
\vec a \times \vec b =\begin{pmatrix}
           a_yb_z - a_z b_y\\
           a_zb_x - a_x b_z\\
           a_xb_y - a_y b_x\\
         \end{pmatrix}
\end{align}

One important property to note is that $\vec a \times \vec b = -\vec b \times \vec a$; that is, the cross product is not commutative (the order matters). The magnitude of the vector obtained by a cross product is given by:
\begin{align}
\label{eqn:Vectors:crossproductmag}
||\vec a \times \vec b ||=ab\sin\theta
\end{align}
where $\theta$ is the angle between the vectors $\vec a$ and $\vec b$ when these are placed tail to tail (Figure \ref{fig:Vectors:scalarproduct}). The vector resulting from a cross product will be null (have a zero length) if the vectors $\vec a$ and $\vec b$ are parallel, and will have a maximal length when these are perpendicular. The cross product is thus useful to determine quantities that are maximal when two vectors are perpendicular (the opposite use case from the scalar product). 

Geometrically, one can determine the direction of the cross product of two vectors by using the ``right hand rule''. This is done by using your right hand, aligning your thumb with the first vector, your index with the second vector, and the cross product will point in the direction of your middle finger (when you hold your middle finger perpendicular to the other two fingers). This is illustrated in Figure TODO. Thus, you can often avoid using equation \ref{eqn:Vectors:crossproduct} and instead use the right hand rule and equation \ref{eqn:Vectors:crossproductmag} to find the vector resulting from a cross product.

TODO: make a figure for the right hand rule

The unit vectors that define a coordinate system have the following properties relative to the cross product:
\begin{align*}
\vec x \times \vec y &= \vec z\\
\vec y \times \vec z &= \vec x\\
\vec z \times \vec x &= \vec y\\
\end{align*}
For these properties to be correct, it should be noted that the direction of the $z$ axis in three dimensions is specified by the choice of $x$ and $y$ axes. That is, one can freely choose the direction of the $x$ and $y$ axes, which then define a plane to which the $z$ axis will be perpendicular. The direction of the $z$ axis must be chosen so that $\vec x \times \vec y = \vec z$ (this guarantees that the coordinate system is ``right handed''). 

\section{Example uses of vectors in physics}
This section gives a quick overview of some applications of vectors in physics.
\subsection{Kinematics and vector equations}
Kinematics is the description of the position and motion of an object (TODO: Chapter reference). The laws of physics are the principles that ultimately allow us to determine how the position of an object changes with time. For example, Newton's Laws are a mathematical framework that introduce the concepts of force and mass in order to model and determine how an object will move through space.

We often use a \textbf{position vector}, $\vec r(t)$, to describe the position of an object as a function of time. Because the object can move, that vector is a function of time. A position vector is a special vector in the sense that it should be considered to be fixed in space; the position vector for an object points from the origin of a coordinate system to the location of the object. 

The three components of the position vector in Cartesian coordinates, are the $x$, $y$, and $z$ coordinates of the object:
\begin{align*}
\vec r(t) = \begin{pmatrix}
           x(t) \\
           y(t) \\
           z(t) \\
         \end{pmatrix}
\end{align*}  
where the three coordinates of the object are functions of time in general if the object is moving relative to the origin of the coordinate system. Suppose that the object was initially at position $\vec r_1=(x_1, y_1, z_1)$ at some time $t=t_1$, and that later, at time $t=t_2$, the object was at as second position, $\vec r_2=(x_1, y_1, z_1)$. We can define the \textbf{displacement vector}, $\vec  d$:
\begin{align*}
 \vec d = \vec r_2 - \vec r_1 =\begin{pmatrix}
           x_2-x_1 \\
           y_2-y_1 \\
           z_2-z_1 \\
         \end{pmatrix} = \begin{pmatrix}
           \Delta x \\
           \Delta y \\
           \Delta z \\
         \end{pmatrix}
\end{align*}
where the components of the displacement vector, $\Delta x$, $\Delta y$, and $\Delta z$ correspond to the displacements along the $x$, $y$, and $z$ axes, respectively. This is illustrated for the two dimensional case in Figure \ref{fig:Vectors:xydvec}.

\capfig{0.3\textwidth}{figures/Vectors/xydvec.png}{\label{fig:Vectors:xydvec}Illustration of a displacement vector, $\vec d = \vec r_2 -\vec r_1$, for an object that was located at position $\vec r_1$ at time $t_1$ and at position $\vec r_2$ at time $t_2$.}


 The velocity vector of the object, $\vec v=(v_x, v_y, v_z)$, is defined to be the displacement vector, $\vec d$, divided by the amount of time that elapsed, $\Delta t=t_2-t_1$:
\begin{align*}
\vec v = \frac{\vec d}{\Delta t}=\begin{pmatrix}
           \frac{\Delta x}{\Delta t} \\
           \frac{\Delta y}{\Delta t} \\
           \frac{\Delta z}{\Delta t} \\
         \end{pmatrix}
\end{align*}
where we used the property that dividing a vector by a scalar ($\Delta t$) is defined as dividing each component by the scalar. If we write the components of the velocity vector out explicitly, we have:
\begin{align*}
\begin{pmatrix}
           v_x \\
           v_y \\
           v_z \\
         \end{pmatrix} = \begin{pmatrix}
           \frac{\Delta x}{\Delta t} \\
           \frac{\Delta y}{\Delta t} \\
           \frac{\Delta z}{\Delta t}
         \end{pmatrix}
\end{align*}
That is, we can think of each row in this ``vector equation'' as an independent equation. That is, when we write the vector equation:
\begin{align*}
\vec v = \frac{\vec d}{\Delta t}
\end{align*}
we are really just using a shorthand notation for writing the three independent equations:
\begin{align*}
v_x &= \frac{\Delta x}{\Delta t} \\
v_y &= \frac{\Delta y}{\Delta t} \\
v_z &= \frac{\Delta z}{\Delta t} \\
\end{align*}
Whenever we write an equation using vectors, we are really writing out multiple equations all at once, one for each component. Newton's Second Law:
\begin{align*}
\vec F = m \vec a
\end{align*}
thus corresponds to the three (scalar) equations:
\begin{align*}
F_x &= ma_x\\
F_y &= ma_y\\
F_z &= ma_z\\
\end{align*}
\subsection{Work and scalar products}
As we will see, work is a scalar quantity that allows us to determine the change in the speed (squared) of an object that results from a force exerted over a particular displacement. Both force and the displacement are vector quantities (a force has a magnitude and is exerted in a particular direction). The work, $W$, done by a force, $\vec F$, over a displacements, $\vec d$, is defined as:
\begin{align*}
W = \vec F \cdot \vec d
\end{align*}
The work energy theorem (TODO: chapter reference) tells us that this work is related to the change in speed squared of the object as it moves along the displacement vector $d$. If the work is zero, the object has the same speed at the beginning and end of the displacement. If the work is positive, the object is moving faster at the end of the displacement (and slower if the work is negative). A one dimensional example is shown in Figure \ref{fig:Vectors:work_scalarprod}, which shows a force $\vec F$ being applied to a block as it slides along the ground over a distance $d$ (represented by the displacement vector $\vec d$).  

\capfig{0.3\textwidth}{figures/Vectors/work_scalarprod.png}{\label{fig:Vectors:work_scalarprod}Example of a force $\vec F$ being applied on an object as it moves along the displacement vector $\vec d$.}

Intuitively, it makes sense that only the horizontal component of the force would contribute to changing the speed of the object as it moves along the horizontal trajectory defined by the vector $\vec d$. The vertical component of the force does not contribute to changing the speed of the object. The scalar product is given by:
\begin{align*}
\vec F \cdot \vec d = Fd\cos\theta = F_{\parallel}d
\end{align*}
where we introduced $F_{\parallel} = F\cos\theta$ as the component of $\vec F$ that is parallel to $\vec d$ (see Figure \ref{fig:Vectors:work_scalarprod}). The scalar product thus ``picks out'' the component of $\vec F$ that is parallel to $\vec d$, which is exactly what we need to in order to calculate work. 

\subsection{Using vectors to describe rotational motion}
Often, we need to describe rotational motion in physics. If an object is rotating, one must specify:
\begin{enumerate}
\item The axis about which the object is rotating
\item The direction around that axis in which the object is rotating (e.g. clockwise or counter-clockwise)
\item How fast the object is rotating
\end{enumerate}
We can also use a vector to describe this type of rotational motion. We choose the direction of the vector to be co-linear with the axis of rotation and the magnitude of the vector to represent the speed with which the object is rotating. We are thus left with two choices for the direction of the vector (it is co-linear with the axis of rotation, but the specific choice of direction has not been made). We choose the direction of the vector by using our right hand in such a way that the vector points in the direction of your thumb when curling your fingers corresponds to the rotational direction, as illustrated in Figure TODO: Figure for rotational right hand rule.


\subsection{Torque and vector products}
We will introduce the concept of a torque in order to describe how a force can cause an object to rotate. Consider the disk illustrated in Figure \ref{fig:Vectors:torque_vectorprod} that is free to rotate about an axis that goes through its centre and that is perpendicular to the plane of the page. A force $\vec F$ is applied at the edge of the disk, as a position that is displaced from the axis of rotation by the vector $\vec r$. The torque, $\vec \tau$, of the force about the centre of the disk is defined to be:
\begin{align*}
\vec\tau=\vec r\times \vec F
\end{align*}
and represents how much the force $\vec F$ will contribute to making the disk rotate about its axis. If the force vector were parallel to the vector $\vec r$, the disk would not rotate; if you pull outwards on a disk, it will not rotate about its centre. However, if the force is perpendicular to the vector $\vec r$ (i.e. tangent to the circumference of the disk), then it will maximally cause the disk to rotate. The magnitude of the torque (cross-product) is given by:
\begin{align*}
\tau =rF\sin\theta=F_{\perp}r=Fr_\perp
\end{align*}
where $\theta$ is the angle between vectors when placed tail to tail, as in the right side of Figure \ref{fig:Vectors:torque_vectorprod}. In the last two equalities, we have defined $F_\perp=F\sin\theta$ or $r_\perp=r\sin\theta$ to refer to the part of the vector $\vec F$ that is perpendicular to the vector $\vec r$ or the part of the vector $\vec r$ that is perpendicular to the vector $\vec F$. That is, the vector product ``picks out'' the part of a vector that is perpendicular to the other, which is exactly the property that we need for the physical quantity of torque.

\capfig{0.3\textwidth}{figures/Vectors/torque_vectorprod.png}{\label{fig:Vectors:torque_vectorprod}A force, $\vec F$, is exerted in the plane of a disk at a position given by the vector $\vec r$ relative to the centre of the disk.}

\begin{checkpointMC}{Referring to Figure \ref{fig:Vectors:torque_vectorprod}, in which direction does the torque vector point?}
\item to the right
\item to the left
\item out of the page %correct
\item into the page
\end{checkpointMC}
%\chapter{Guidelines for lab related activities}
\label{chapter:labs}
This chapter introduces the skills that are necessary for thinking about how to design an experiment and to report on its results.

\begin{learningObjectives}{
 \item Develop skills in general scientific writing.
 \item Learn to write scientific proposals and experimental reports.
 \item Learn to review others' scientific proposals and experimental reports.
 }
\end{learningObjectives}

\section{The process of science and the need for scientific writing}
Conducting experiments that test a scientific theory is integral to the advancement of science and to the refining of scientific theories. In practice, scientists do not have a lab full of equipment ready to go and to be used for testing whichever theory suits their fancy. Instead, they need to write a ``proposal'' for conducting a particular experiment to a funding source (e.g. a funding agency). That funding source will then select a panel of experts in the field to review whether the proposal is feasible and useful in advancing science, to decide whether it should be funded. If the scientist is awarded with funds, they are then expected to carry out their experiment and report on the results in a peer-reviewed scientific journal. Again, before the results are published, the scientific journal will ask a panel of experts to review the results to ensure that they are scientifically valid and interesting.

In order for a proposal to be funded, it must thus propose an experiment that is well-thought out and feasible. For example, the reviewers will want to make sure that the proposed experiment is designed in the best possible way to test a theory. Often, this means that thought has been put into designing an experiment that minimizes the uncertainty on the result, so that the test of the theory is as stringent as possible.

A proposal needs to be well-written and precise. We generally call this type of writing ``scientific writing'', and it is a style of writing that takes some practice. Similarly, when reporting on the results of an experiment, the report will need to be clear and precise as well. For example, in scientific writing, one avoids giving opinions or using sentences that do not add necessary information or that are not factual.

This chapter provides some guidelines for scientific writing, writing proposals, and writing reports. In addition to this, guidelines for reviewing others' proposals and reports are also presented. Not only is it important to develop the ability to critically evaluate others' work, but it is also helpful in learning to reflect and improve on one's own work.

 \vspace{0.25cm}
\section{Scientific writing}
Scientific writing is important in communicating with other scientists. Think of scientific writing as a style of writing where \textbf{every word counts}. It makes for rather ``dry'' reading, but it is important for clearly and precisely communicating factual information. The main guidelines for scientific writing are \textbf{be concise, precise, factual, and clear}. Below are some tips to help with scientific writing:
\begin{itemize}
\item Avoid subjective/imprecise terms: avoid using subjective and imprecise terms, stick to factual statements and avoid opinions.  Instead of saying ``our calculated value of g was much greater than the expected value'', say ``our calculated value of g was greater than the expected value''. Your opinion that it was ''much greater'' does not communicate anything and is imprecise (much greater in relation to what?).
\item Definitive statements: avoid attributing definitive causes to your experimental outcomes. You can never prove a theory to be correct, so at most, your results will be consistent with a theory. For example, instead of saying ``as the data exhibit, we have detected the Purple Particle'', you should state that ``the data are consistent with the detection of the Purple Particle''. 
\item Data is the plural of datum. ``This data shows'' is incorrect, rather, ``these data show'', or ``this set of data show''.
\item Active vs. passive voice: when writing scientific papers, it is recommended to use the third person, passive voice. For example, this would mean saying ``the drop time for balls at various heights was measured'' rather than ``we measured the drop time for balls at various heights''. However, both passive and active voices are acceptable in scientific writing, as long as it is consistent throughout the text.
\item Tense. Generally, for a proposal, you would use the future tense, and the past tense for reporting on your results.
\end{itemize}
\newpage
\begin{studentOpinion}{Emma}
\textbf{Writing and editing - how can I be more concise?}
We've all felt that our writing was lacking at some point or another. Here are some general tips to avoid overall ``wordiness'' and to increase ease of reading when writing scientifically: 
\begin{itemize}
\item What would you want to read? Let's say that you wanted to know the strength of Earth's magnetic field, and how it was found, so you decide to do a literature search. Would you choose a brief, succinct article, or a wordy Magnetic Field Manifesto?
\item The kindergarten test: If you had to explain your concept to a six year old cousin, how would you break it down in a way that they could understand it? If you can't break it down enough to explain to a six year old, perhaps you need to revisit your own understanding of the concept before writing about it scientifically.
\item Avoid unnecessary adjectives: while this might be ok in a creative writing class, in scientific writing, the goal is to get your point across as succinctly as possible. Using ``big'' words might be ok (as long as they properly describe what you are trying to say), but it is important to communicate your message in the simplest manner. 
\item Think about it: every time you use a comma, dash or even an ``and'', you should reconsider the brevity of your statement. In scientific writing, commas are carefully placed, and semicolons are rare. 
\item Cut it in half: For every word you read, think of another that you can cut. For every sentence that you read, think of three sentences that communicate the same idea. Pick the sentence that is the shortest and most concise. 
\item Proofread - the more, the better.
\end{itemize}
\end{studentOpinion}

The following sections provide basic outlines for writing a proposal and a lab report, as well as rubrics for evaluating/reviewing proposals and reports. Additionally, samples of a proposal, proposal review, report, and report review for the experiment ``Measuring g using a pendulum'' are provided. In the sample proposal and lab report, errors are purposefully included and addressed in the reviews. It is important to entirely read the rest of this section to capture the common proposal/lab mistakes and their corresponding corrections. That is, do not take the sample proposal as a ``perfect proposal'', but rather, consider it in the light of the corresponding review.
\newpage


\section{Guide for writing a proposal}
 \vspace{0.25cm}
\textbf{Summary and Goal}

Write a few short sentences briefly summarizing the aim of your experiment, how it will be conducted, and how precise of a result you expect to obtain.

\textbf{Method and equipment}

Clearly describe, in as much detail as required, the method/procedure that you will use to carry out your experiment, and how you will analyse the results. Justify the choices that you made (no need to say you chose to use a ruler because you will need to measure a distance, but perhaps say why you need to measure a given distance, or that you chose to measure something in a particular way as it would reduce the corresponding uncertainty). Provide a list of the equipment that you will need. Also, propose a method of assessing whether or not your project was successful. 

Consider the following questions:
\begin{itemize}
\item What theory are you testing and through what model?
\item How precisely to you estimate you will be able to make your measurement? Estimate the uncertainty that you will obtain with the proposed experiment. Use this in guiding the design of your experiment.
\item What materials, equipment and/or tools are necessary in making your measurements?
\item What are the cost of these materials? Can they be easily obtained?
\item Where should this experiment be conducted? 
\item Are there any safety concerns?
\item How will you make your measurements? How many times will you make them?
\item How will you record your measurements?
\item How will you maximize the precision of your experiments?
\item How will you determine uncertainties? 
\item How will you analyse the data?
\item What issues could arise in your experiment? How do you plan to resolve these issues?
\end{itemize}

\textbf{Timeline and Team}

Provide the names of team members, and assign relevant duties to each member. Give a rough outline of the timeline to conduct the experiment, to analyse the data, and to report on the results.

\newpage
\section{Guide for reviewing a proposal}
 \vspace{0.25cm}
\textbf{Summary}

Summarize your overall evaluation of the proposal in 2-3 sentences. Focus on the experiment's methods and goals. For example, ``The authors wish to drop balls from different heights to determine the value of g''. You don't need to go into the specific details, just give a high level summary of the proposal and your opinion on whether this is a strong proposal. If the proposal is unclear, specify this.

\textbf{Review}

This is where you give your detailed review of the proposal. Consider the following questions:
\begin{itemize}
\item Is the proposed experiment well thought-out and feasible?
\item Is the experimental procedure clear and concise? Could you could carry out the experiment without asking the authors for additional information? Do the authors specify what instruments to use to measure different quantities and how to determine the associated uncertainties?
\item Does the experimental design minimize uncertainties?
\item Is it possible to complete the experiment in a reasonable period of time?
\item Is it possible to obtain the equipment/materials to conduct the experiment?
\item Do the authors describe how to analyse the data (correctly)?
\item Does the plan incorporate a mechanism to assess success?
\item Is a troubleshooting plan in place, in case of unexpected difficulties?
\end{itemize}

\textbf{Overall Rating of the Experiment}

Give the proposal an overall score, based on the criteria described above. Use one of the following to rate the proposal and include a sentence to justify your choice.
\begin{itemize}
\item Excellent
\item Good
\item Satisfactory
\item Needs work
\item Incomplete
\end{itemize}

\newpage
\section{Guide for writing a lab report}
 \vspace{0.25cm}
\textbf{Abstract}

Write a few short sentences briefly summarizing what you did, how you did it, what you found and whether anything went wrong in your experiment.

\textbf{Procedure}

Describe relevant theories that relate to your experiment here, and the steps to carry out your procedure. 

Consider the following questions:
\begin{itemize}
\item What are the relevant theories/principles that you used? 
\item What equations did you use? Show how you modelled your experiment.
\item What materials, equipment and/or tools were necessary in making your measurements?
\item Where was this experiment conducted?
\item How did you make your measurements? How many times did you make them?
\item How did you record your measurements?
\item How did you determine and minimize the uncertainties in your measurements? Why did you choose to measure a specific quantity in a certain way?
\end{itemize}

\textbf{Prediction}
It can be useful to predict the value (and uncertainty) that you expect to measure before conducting the measurement. You should report on this initial prediction in order to help you better understand the data from your experiment.

Consider the following questions:
\begin{itemize}
\item Predict your measured values and uncertainties. How precise do you expect your measurements to be?
\item What assumptions did you have to make to predict your results?
\item Have these predictions influenced how you should approach your procedure? Make relevant adjustments to the procedure based on your predictions.
\end{itemize}

\textbf{Data and Analysis}

Present your data. Include relevant tables/graphs. Describe in detail how you analysed the data, including how you propagated uncertainties. If the data do not agree with your model prediction (or the prediction from your proposal), examine whether you can improve your model. 

Consider the following questions:
\begin{itemize}
\item How did you obtain the ``final'' measurement/value from your collected data? 
\item How did you propagate uncertainties? Why did you do it that way?
\item What is the relative uncertainty on your value(s)?
\end{itemize}

\textbf{Discussion and Conclusion}

Summarize your findings, and address whether or not your model described the data. Discuss possible reasons why your measured value is not consisted with your model expectation (is it the model? is it the data?).

Consider the following questions:
\begin{itemize}
\item Were there any systematic errors that you didn't consider?
\item Did you learn anything that you didn't previously know? (eg. about the subject of your experiment, about the scientific method in general)
\item If you could redo this experiment, what would you change (if anything)?
\end{itemize}

\newpage
\subsection{Guide for reviewing a lab report}
 \vspace{0.25cm}
\textbf{Summary}

Summarize your overall evaluation of the report in 2-3 sentences. Focus on the experiment's method and its result. For example, ``The authors dropped balls from different heights to determine the value of g''. You don't need to go into the specific details, just give a high level summary of the report. If the report is unclear, specify this.

\textbf{Review}

Consider the following questions:
\begin{itemize}
\item Is the the procedure well thought-out, clearly and concisely described? 
\item Do you have sufficient information that you could repeat this experiment?
\item Does the report clearly describe how different quantities were measured and how the uncertainties were determined?
\item Does the report motivate why the specific procedure was chosen? (e.g. to minimize uncertainties).
\item Does the experiment clearly state how uncertainties were propagated and how the data were analysed?
\item Do you believe their result to be scientifically valid?
\end{itemize}


\textbf{Overall Rating of the Experiment}

Give the report an overall score, based on the criteria described above. Use one of the following to rate the proposal and include a sentence to justify your choice.
\begin{itemize}
\item Excellent
\item Good
\item Satisfactory
\item Needs work
\item Incomplete
\end{itemize}

\newpage
\section{Sample proposal (Measuring g using a pendulum)}
 \vspace{0.25cm}
\textbf{Summary and Goal}

One can measure the gravitational constant, $g$, by measuring the period of a pendulum of a known length, requiring only a string, mass, ruler and timer. Because the experimental design can be easily adjusted and the experiment is simple, the experiment has a high chance of success.

\textbf{Method and equipment}

The period of a pendulum of length $L$ is easily shown to be given by:

\begin{align*}
T=2\pi \sqrt {\frac{L}{g}}
\end{align*}

Thus, by measuring the period, $T$, of a pendulum as well as its length, one can determine the value of $g$:

\begin{align*}
g=\frac{4\pi^{2}L}{T^{2}}
\end{align*}

One can carry out the experiment using the following materials:
\begin{itemize}
\item a mass
\item inextensible string
\item a metre stick
\item stand to attach string
\item cell-phone with timer and slow-motion camera
\end{itemize}

The materials listed above are all inexpensive and can be easily obtained.  It is recommended that the experiment be completed indoors at room temperature, in order to minimize any environmental effects. 

One should tie the string to the mass at one end and the stand at the other, and measure the length, $L$, of the string from the point on the stand to the centre of mass of the mass.

The period of the pendulum is measured by timing how long it takes the pendulum to complete 20 oscillations and dividing that time by 20. This will be more precise than trying to time the period of a single oscillation.

The pendulum should be released from $\SI{90}{\degree}$. When releasing the pendulum, the string should be pulled taught, and the team member's eye that is measuring the angle should be situated parallel to the measuring device. 

A slow-motion video will be taken of the pendulum to track the time of the oscillation in order to minimize error due to reaction time. The team member in charge of taking the video will start the video shortly before the pendulum is released. After releasing the pendulum, the team should record 20 oscillations before stopping the pendulum and the video. Data from the video should be entered into a Jupyter Notebook. It is recommended that this measurement be repeated at least 5 times.


The uncertainty in the time should be taken as half of the smallest division of the cell-phone timer, and the uncertainty in the length of the pendulum as half the smallest division of the metre stick used to measure the length of the pendulum.

 Foreseeable issues in this experiment may arise when trying to find a string that is optimally inextensible, as any extensibility will cause error in the results. Additionally, being able to measure exactly \SI{90}{\degree} as the drop-angle for the pendulum could be difficult. In order to correct for this, the team member who is dropping the pendulum must stand directly parallel to the measuring device, minimizing parallax error. 

The measure of success will be determined by the uncertainty and precision of the measured value of $g$. If the measured value of $g$ has a relative uncertainty that is less than 10 \%, and is consistent with the accepted value, then one can consider the experiment to have been carried out successfully. 

\textbf{Team and timeline}

One should be able to complete the experiment and analysis in approximately 1 hour and 30 minutes with the data being collected in the first 30 minutes. The remainder of the time should be spent processing the data and writing the experimental report.  
Following the strengths of the members of the team, the following people should be responsible for leading the following tasks, while everyone participates:

\begin{itemize}
\item Alice: building the pendulum
\item Brice: taking the measurements
\item Chlo\"e: analysing the data
\item Dennis: writing and formatting
\end{itemize}


\newpage
\section{Sample proposal review (Measuring g using a pendulum)}
 \vspace{0.25cm}
\textbf{Summary and Goal}

The authors propose to measure the value of $g$ to within 10\% by measuring the period of a simple pendulum, using the SHM equations and theory. The proposal is reasonably clear, but lacks some details in how to measure the initial angle of the pendulum. The authors propose to use a an amplitude of $\SI{90}{\degree}$ for the pendulum, but at such a large angle, the motion is not expected to be SHM, since it is only so at small angles. By using a smaller angle, the experiment has a good chance of being successful in the proposed timeline.

\textbf{Review}

The experimental methods are described clearly and succinctly, with most information clearly stated. For the materials list, it is stated that ``a mass'' must be used. Here, it should be stated that a small, solid, non-deformable mass should be used to minimize drag and to act as a point mass. The authors refer to a ``measuring device'' when determining the amplitude of the pendulum, but this is not described. Anyhow, the amplitude of the oscillations in irrelevant for a pendulum in SHM, as long as the amplitude is small.

Most equations are described in the theory section, but it is incorrectly assumed that the period of a pendulum is independent of the drop angle for all angles. The small angle approximation is not expected to apply with an oscillation amplitude of \SI{90}{\degree}.

No justification is provided for the use of 20 oscillations prior to measuring the period - it may be necessary to iterate on the reason why 20 oscillations was chosen. 

The equipment can be easily obtained and is fairly inexpensive. Adequate resources are available to the group to perform this experiment. A clear troubleshooting plan is described and a method for evaluating success is included. 

\textbf{Timeline and team}

This experiment is fairly simple and the equipment/setup is not difficult to handle. The proposed team should be qualified to perform this experiment in the proposed amount of time, although I worry a little bit about Dennis, as he seems to be a bit of a menace.

\textbf{Overall Rating of the Proposal}

Good - this proposal was clearly explained and is scientifically sound, apart from the use of a large angle for the oscillations. It was succinctly written, and most components of the experiment were clearly described. A little more detail in the justification for using 20 oscillations is necessary. 


\newpage
\section{Sample lab report (Measuring g using a pendulum)}
 \vspace{0.25cm}
\textbf{Abstract}

In this experiment, we measured $g$ by measuring the period of a pendulum of a known length. We measured $g = \SI{7.65\pm 0.378}{m/s^2}$. This correspond to a relative difference of 22\% with the accepted value ($\SI{9.8}{m/s^2}$), and our result is not consistent with the accepted value.

\textbf{Theory}

A pendulum exhibits simple harmonic motion (SHM), which allowed us to measure the gravitational constant by measuring the period of the pendulum. The period, $T$, of a pendulum of length $L$ undergoing simple harmonic motion is given by:
\begin{align*}
T=2\pi \sqrt {\frac{L}{g}}
\end{align*}

Thus, by measuring the period of a pendulum as well as its length, we can determine the value of $g$:
\begin{align*}
g=\frac{4\pi^{2}L}{T^{2}}
\end{align*}
We assumed that the frequency and period of the pendulum depend on the length of the pendulum string, rather than the angle from which it was dropped. 

\textbf{Predictions}

We built the pendulum with a length $L=\SI{1.0000\pm 0.0005}{m}$ that was measured with a ruler with $\SI{1}{mm}$ graduations (thus a negligible uncertainty in $L$). We plan to measure the period of one oscillation by measuring the time to it takes the pendulum to go through 20 oscillations and dividing that by 20. The period for one oscillation, based on our value of $L$ and the accepted value for $g$, is expected to be $T=\SI{2.0}{s}$. We expect that we can measure the time for 20 oscillations with an uncertainty of $\SI{0.5}{s}$. We thus expect to measure one oscillation with an uncertainty of $\SI{0.025}{s}$ (about 1\% relative uncertainty on the period). We thus expect that we should be able to measure $g$ with a relative uncertainty of the order of 1\%


\textbf{Procedure}

The experiment was conducted in a laboratory indoors.

1. Construction of the pendulum

We constructed the pendulum by attaching a inextensible string to a stand on one end and to a mass on the other end. The mass, string and stand were attached together with knots. We adjusted the knots so that the length of the pendulum was $\SI{1.0000\pm0.0005}{m}$. The uncertainty is given by half of the smallest division of the ruler that we used.


2. Measurement of the period

The pendulum was released from \SI{90}{\degree} and its period was measured by filming the pendulum with a cell-phone camera and using the phone's built-in time. In order to minimize the uncertainty in the period, we measured the time for the pendulum to make 20 oscillations, and divided that time by 20. We repeated this measurement five times. We transcribed the measurements from the cell-phone into a Jupyter Notebook.


\textbf{Data and Analysis}

Using a $\SI{100}{g}$ mass and $\SI{1.0}{m}$ ruler stick, the period of 20 oscillations was measured over 5 trials. The corresponding value of $g$ for each of these trials was calculated. The following data for each trial and corresponding value of $g$ are shown in the table below. 

\begin{table}[H]
\begin{tabular}{|l|l|l|l|l}
\cline{1-4}
Trial & Angle (Degrees) & Measured Period (s) & Value of g ($m/s^2$) &  \\ \cline{1-4}
1     & 90              & 2.24                & 7.87                         &  \\ \cline{1-4}
2     & 90              & 2.37                & 7.03                         &  \\ \cline{1-4}
3     & 90              & 2.28                & 7.59                         &  \\ \cline{1-4}
4     & 90              & 2.26                & 7.73                         &  \\ \cline{1-4}
5     & 90              & 2.22                & 8.01                         &  \\ \cline{1-4}
\end{tabular}
\end{table}

Our final measured value of $g$ is $\SI{7.65\pm 0.378}{m/s^2}$. This was calculated using the mean of the values of g from the last column and the corresponding standard deviation. The relative uncertainty on our measured value of $g$ is 4.9\% and the relative difference with the accepted value of $\SI{9.8}{m/s^2}$ is 22\%, well above our relative uncertainty.

\textbf{Discussion and Conclusion}

In this experiment, we measured $g=\SI{7.65\pm 0.378}{m/s^2}$. This has a relative difference of 22\% with the accepted value and our measured value is not consistent with the accepted value. All of our measured values were systematically lower than expected, as our measured periods were all systematically higher than the $\S{2.0}{s}$ that we expected from our prediction. We also found that our measurement of $g$ had a much larger uncertainty (as determined from the spread in values that we obtained), compared to the 1\% relative uncertainty that we predicted.

We suspect that by using 20 oscillations, the pendulum slowed down due to friction, and this resulted in a deviation from simple harmonic motion. This is consistent with the fact that our measured periods are systematically higher. We also worry that we were not able to accurately measure the angle from which the pendulum was released, as we did not use a protractor.

If this experiment could be redone, measuring 10 oscillations of the pendulum, rather than 20 oscillations, could provide a more precise value of $g$. Additionally, a protractor could be taped to the top of the pendulum stand, with the ruler taped to the protractor. This way, the pendulum could be dropped from a near-perfect $\SI{90}{\degree}$ rather than a rough estimate. 

\newpage
\section{Sample lab report review (Measuring g using a pendulum)}
 \vspace{0.25cm}
\textbf{Summary}

The authors measured the period of a pendulum to determine $g$. They measured $g$ to be $\SI{7.65\pm0.378}{m/s^2}$ which is inconsistent with the accepted value. The authors were incorrect in assuming that the pendulum would undergo simple harmonic motion in the conditions that they used.

\textbf{Review}

The experimental procedure was clearly written and one could mostly reproduce this experiment with the given description.

The authors thought about minimizing uncertainties by measuring the period over several oscillations, although it appears that 20 was perhaps too large, as friction was likely to have an effect. The authors should have taken more care in determining the number of oscillations to use so that the uncertainty in the time is minimized while also keeping the effects of friction negligible. Ultimately, the authors did not specify the uncertainty in the time that they measured.

The authors also claim to have measured the length of the pendulum with a precision of $\SI{0.5}{mm}$, but did not specify the length of the ruler that they used. I would not expect the measurement to be that precise unless they used a very precise ruler that is longer than $\SI{1}{m}$. However, the authors made the length of the pendulum as long as possible so as to minimize the uncertainty in the length.

The authors did not describe the mass that was attached at the end of the pendulum, and whether its size would be expected to cause significant air drag.

The authors made a mistake in assuming that a pendulum would undergo simple harmonic motion with an amplitude of $\SI{90}{\degree}$, as the small angle approximation used to determine the period does not apply in this case.

The experimental procedure was scientifically sound, other than the choices for the number of oscillations and their amplitude.

\textbf{Overall rating of the Experiment}

Satisfactory - The experiment was well described, but the authors should have paid more attention to their choice of 20 oscillations, and they made a mistake in assuming that their pendulum would exhibit simple harmonic oscillation with a large amplitude.


%\chapter{The Python programming language}
\label{python}
This appendix gives a very brief introduction to programming in python and is primarily aimed at introducing tools that are useful for the experimental side of physics. 
 \vspace{1cm}
\begin{learningObjectives}
\item Be able to perform simple algebra using python
\item Be able to plot a function in python
\item Be able to propagate uncertainties in python
\item Be able to plot and fit data to a straight line
\item Understand how to use Python to numerically calculate any integral
\end{learningObjectives}

In this textbook, we will encourage you to use computers to facilitate making calculations and displaying data. We will make use of a popular programming language called Python, as well as several ``modules'' from Python that facilitate working with numbers and data. Do not worry if you do not have any programming experience; we assume that you have none and hope that by the end of this book, you will have some capability to decrease your workload by using computer programming.

The only way to become proficient at programming is through practice. If you want to effectively learn from this chapter, it is important that you take the time to actually type the commands into a Python environment rather than simply reading through the chapter. Reading through the chapter will at least give you a sense of what is possible and some terminology, but it will not teach you programming!

\section{A quick intro to programming}
In Python, as in other programming language, the equal sign is called the \textbf{assignment operator}. Its role is to \textit{assign} the value on its right to the variable on its left. The following code does the following:
\begin{itemize}
\item \textit{assigns} the value of \code{2} to the variable \code{a}
\item \textit{assigns} the values of \code{2*a} to the variable \code{b}
\item prints out the value of the variable \code{b}
\end{itemize}

\begin{python}[caption=Declaring variables in Python] 
#This is a comment, and is ignored by Python
a = 2 
b = 2*a
print(b)
\end{python}
\begin{poutput}
4
\end{poutput}
Note that any text that follows a pound sign (\#) is intended as a comment and will be ignored by Python. Inserting comments in your code is very important for being able to understand your computer program in the future or if you are sharing your code with someone who would like to understand it. In the above example, we called the \code{print()} \textbf{function} and passed to it the variable \code{b} as an \textbf{argument}; this allowed us to print (display) the value of the variable \code{b} and verify that it was indeed equal to the number 4.


In Python, if you want to have access to ``functions'', which are more complex series of operations, then you typically need to load the \textit{module} that defines those operations. 

A large number of functions are provided in Python. Most of these functions need to be ``imported'' from ``modules''. For example, if you want to be able to take the square root of a number, then you need to load (import) the ``math module'' which contains the square root function, as in the following example:
\begin{python}[caption=Using functions from modules] 
#First, we load (import) the math module
import math as m
a = 9
b = m.sqrt(a)
print(b)
\end{python}
\begin{poutput}
3
\end{poutput}
In the above code, we loaded the math module (and renamed it \code{m}); this then allows us to use the functions that are part of that module, including the square root function (\code{m.sqrt()}).

\section{Arrays}
It is often the case that we need to represent a series of numbers. For example, imagine that you have measured the position of an object as a function of time. \textbf{Arrays} are a convenient way to hold a series of numbers that are all alike, for example, all of the values of the position and corresponding time values for the trajectory of the object. In Python, we can define variables that hold arrays instead of a single value (arrays are called ``lists'' in Python):
\begin{python}[caption=Arrays in python]
#define an array of values for the position of the object
position = [0,1,4,9,16,25]
#define an array of values for the corresponding times
time = [0,1,2,3,4,5]
\end{python}

\section{Plotting}
Several modules are available in python for plotting. We will show here how to use the \code{pylab} module (which is equivalent to the \code{matplotlib} module). For example, we can easily plot the data in the two arrays from the previous section in order to plot the position versus time for the object:
\begin{python}[caption=Plotting two arrays]
#import the pylab module
import pylab as pl

#define an array of values for the position of the object
position = [0,1,4,9,16,25]
#define an array of values for the corresponding times
time = [0,1,2,3,4,5]

#make the plot showing points and the line (.-)
pl.plot(time, position)
#add some labels:
pl.xlabel("time") #label for x-axis
pl.ylabel("position") #label for y-axis
#show the plot
pl.show()

\end{python}
\begin{poutput}
(*  \capfig{0.6\textwidth}{figures/Python/positiontime.png}{Plotting two arrays.} *)
\end{poutput}

\begin{checkpointSA}{How would you modify the Python code above to show only the points, and not the line?}
\end{checkpointSA}

We can use Python to plot any mathematical function that we like. It is important to realize that computers do not have a representation of a continuous function. Thus, if we would like to plot a continuous function, we first need to evaluate that function at many points, and then plot those points. The \code{numpy} module provides many useful features for working with arrays of numbers and applying functions directly to those arrays. 

Suppose that we would like to plot the function $f(x) = cos(x^2)$ between $x=-3$ and $x=5$. In order to do this in Python, we will first generate an array of many values of $x$ between $-10$ and $25$ using the \code{numpy} package and the function \code{linspace(min,max,N)} which generates $N$ linearly spaced points between $min$ and $max$. We will then evaluate the function at all of those points to create a second array. Finally, we will plot the two arrays against each other:
\begin{python}[caption=Plotting a function of 1 variable]
#import the pylab and numpy modules
import pylab as pl
import numpy as np

#Use numpy to generate 1000 values of x between -3 and 5:
xvals =np.linspace(-3,5,1000)

#Now, evaluate the function for all of those values of x.
#We use the numpy version of cos, since it allows us to take the cos 
#of all values in the array
fvals = np.cos(xvals**2)

#make the plot showing only a line, and color it
pl.plot(xvals, fvals, color='red')
#add some labels:
pl.xlabel("time") #label for x-axis
pl.ylabel("position") #label for y-axis
#show the plot
pl.show()

\end{python}
\begin{poutput}
(*  \capfig{0.6\textwidth}{figures/Python/functionplot.png}{Plotting a function using arrays.} *)
\end{poutput}

\section{The QExpy python package for experimental physics}
QExpy is a Python module that was developed with students from Queen's University to handle all aspects of undergraduate physics laboratories. In this section, we look at how to use QExpy to propagate uncertainties and to plot experimental data.

\subsection{Propagating uncertainties}
In Chapter \ref{chap:ModelAndExperiment}, we saw how to use the ``derivative method'' to propagate the uncertainty from measurements into the uncertainty in a value that depended on those measurements. In Example \ref{ex:ModelAndExperiment:derivprop}, we propagated the uncertainties $x=\SI{3.00 \pm 0.01}{m}$ and $t=\SI{0.76\pm0.15}{s}$ to the quantity $k=\frac{t}{\sqrt x}$. We show below how easily this can be done with QExpy:

\begin{python}[caption=QExpy to propagate uncertainties] 
#First, we load the QExpy module
import qexpy as q
#Now define our measurements with uncertainties:
t = q.Measurement(0.76, 0.15) # 0.76 +/- 0.15
x = q.Measurement(3,0.1) # 3 +/- 0.1
#Now define k, which depends on t and x:
k = t/q.sqrt(x) # use the QExpy version of sqrt() since x is of type Measurement
#Print the result:
print(k)
\end{python}
\begin{poutput}
0.44 +/- 0.09
\end{poutput}
which is the result that we obtained when manually applying the derivative method. Note that we used the square root function from the QExpy module, as it ``knows'' how to take the square root of a value with uncertainty (a ``Measurement'' in the language of QExpy). 

We also saw that when we had repeated measurements of the same quantity (Section \ref{sec:c2:determiningerrors}), one could define a central value and uncertainty for that quantity by using the mean and standard deviations of the measurements. QExpy can easily take a set of measurements (an array of values) and convert them into a single quantity (a ``Measurement'') with a central value and uncertainty that correspond to the mean and standard deviation of the set of measurements:

\begin{python}[caption=QExpy to calculate mean and standard deviation] 
#First, we load the QExpy module
import qexpy as q
#We define $t$ as an array of values (note the square brackets):
t = q.Measurement([1.01,  0.76,  0.64,  0.73,  0.66])
#Choose the number of significant figures to print:
q.set_sigfigs(2)
#Print the result:
print("t = ",t)
\end{python}
\begin{poutput}
t = 0.76 +/- 0.15
\end{poutput}
By using QExpy, we do not need to tediously calculate the mean and standard deviation, as we had in Example \ref{ex:ModelAndExperiment:stdcalc}.


\subsection{Plotting experimental data with uncertainties}
In Chapter \ref{chap:ModelAndExperiment} we had presented the data in Table \ref{tab:Python:kmes} which corresponded to our measurements of how long it took ($t$) for an object to drop a certain distance, $x$. We had also introduced  Chlo\"e's Theory of gravity that predicted that the data should be described by the following model:
\begin{align*}
t = k \sqrt{x}
\end{align*}
where $k$ was an undetermined constant of proportionality.

\begin{table}[!h]
\centering
\begin{tabular}{cccc} 
\textbf{x} [m]&\textbf{t} [s]&\textbf{$\sqrt x$}  [\si{m^{\frac{1}{2}}}]&\textbf{k}  [\si{s.m^{-\frac{1}{2}}}]\\
\hline
\hline
1.00 &0.33 &1.00 &0.33 \\ \hline
2.00 &0.74 &1.41 &0.52 \\ \hline
3.00 &0.67 &1.73 &0.39 \\ \hline
4.00 &1.07 &2.00 &0.54 \\ \hline
5.00 &1.10 &2.24 &0.49 \\ \hline
\end{tabular}
\caption{\label{tab:Python:kmes} Measurements of the drop times, $t$, for a bowling ball to fall different distances, $x$. We have also computed $\sqrt x$ and the corresponding value of $k$. }
\end{table}

The easiest way to visualize and analyse those data is to plot them. In particular, if we plot (graph) $t$ versus $\sqrt{x}$, we  expect that the points will fall on a straight line that goes through zero, with a slope of $k$ (if the data are described by Chlo\"e's Theory). We can use QExpy to graph the data as well as determine (``fit'') for the slope of the line that best describes the data, since we expect that the slope will correspond to the value of $k$. When plotting data and fitting them to a line (or other function), it is important to make sure that the values have at least an uncertainty in the quantity that is being plotted on the $y$ axis. In this case, we have assumed that all of the measurements of time have an uncertainty of $\SI{0.15}{s}$ and that the measurements of the distance have no (or negligible) uncertainties:

\begin{python}[caption=Using QExPy to plot and fit linear data]
#First, we load the QExpy module:
import qexpy as q

#Use matplotlib as the plot engine (try using 'bokeh' instead of 'mpl')
q.plot_engine = 'mpl'

#Then we enter the data in arrays for the x and y axes.
#The values for the square root of height (x axis):
sqx = [1. , 1.41, 1.73, 2., 2.24]
#and then, the corresponding times (y-axis):
t = [ 0.33,  0.74,  0.67,  1.07,  1.1 ]

#Let us attribute an uncertainty of 0.15 to each measured values of t:
terr = 0.15

#We now make the plot. First, we create the plot object with the data.
#Note that x and y refer to the x and y axes
fig = q.MakePlot( xdata = sqx, xname = "sqrt(distance) [m^0.5]",
                  ydata = t, yerr = terr, yname = "time [s]",
                  data_name = "My data")
                  
#Ask QExpy to also determine the line of best fit                  
fig.fit("linear")
                  
#Then, we show it:
fig.show()          
\end{python}
\begin{poutput}
-----------------Fit results-------------------
Fit of  My data  to  linear
Fit parameters:
My data_linear_fit0_fitpars_intercept = -0.2 +/- 0.2,
My data_linear_fit0_fitpars_slope = 0.6 +/- 0.1

Correlation matrix: 
[[ 1.    -0.968]
 [-0.968  1.   ]]

chi2/ndof = 2.04/2
---------------End fit results----------------
(* \capfig{0.75\textwidth}{figures/Python/tvssqx.png}{\label{fig:Python:tvssqx} QExpy plot of $t$ versus $\sqrt{x}$ and line of best fit.} *)
\end{poutput}
The plot in Figure \ref{fig:Python:tvssqx} shows that the data points are consistent with falling on a straight line, when their error bars are taken into account. We've also asked QExpy to show us the line of best fit to the data, represented by the line with the shaded area. When we asked for the line of best fit, QExpy not only drew the line, but also gave us the values and uncertainties for the slope and the intercept of the line. The shaded area around the line corresponds to other possible lines that one would obtain using different values of the slope and intercept within their corresponding uncertainties. The output also provides a line that tells us that \code{chi2/ndof = 2.04/2}; although you do not need to understand the details, this is a measure of how well the data are described by the line of best fit. Generally, the fit is assumed to be ``good'' if this ratio is close to 1 (the ratio is called ``the reduced chi-squared'').  The ``correlation matrix'' tells us how the best fit value of the slope is linked to the best fit value of the intercept, which you do not need to worry about here.


Since we expect the slope of the data to be $k$, this provides us a method to determine $k$ from the data as \SI{0.61\pm 0.13}{s.m^{-\frac{1}{2}}}. \textbf{Performing a linear fit of the data is the best way to determine a constant of proportionality between the measurements}. Finally, we expect the intercept to be equal to zero according to our model. The best fit line from QExpy has an intercept of \SI{-0.24\pm 0.22}{s}, which is slightly below, but consistent, with zero. From these data, we would conclude that our measurements are consistent with Chlo\"e's Theory. Again, remember that we can never confirm a theory, we can only exclude it; in this case, we cannot exclude Chlo\"e's Theory.

\section{Advanced topics}
This section introduces a few more advanced topics that allow you to use computer programming to simplifying many tasks. In this section, we will show you how you can write your own program to numerically estimate the value of an integral of any function.
\subsection{Defining your own functions}
Although Python provides many modules and functions, it is often useful to be able to define your own functions. For example, suppose that you would like to define a function that calculates $\frac{1}{3}x^2+\frac{1}{4}x^3+\cos(2x)$, for a given value of $x$. This is done easily using the \code{def} keyword in Python:

\begin{python}[caption=Defining a function] 
#import the math module in order to use cos
import math as m

#define our function and call it myfunction:
def myfunction(x):
  return x**2 / 3 + x**3 / 4 + m.cos(2*x)
  
#Test our function by printing out the result of evaluating it at x = 3
print( myfunction(3) )  
\end{python}
\begin{poutput}
10.710170286650365
\end{poutput}
A few things to note about the code above:
\begin{itemize}
\item Functions are defined using the \code{def} keyword followed by the name that we choose for the function (in our case, \code{myfunction})
\item If functions take arguments, those are specified in parenthesis after the name of the function (in our case, we have one argument that we chose to call \code{x})
\item After the name of the function and the arguments, we place a colon
\item The code that belongs to the function, after the colon, must be indented (this allows Python to know where the code for the function ends)
\item The function can ``return'' a value; this is done by using the \code{return} keyword. 
\item We used the ``operator'' \code{**} to take the power of a number (\code{x**2}), and the operator \code{*}, to multiply numbers. In particular, Python would not understand \code{2x} which needs to explicitly have the multiplication operator, \code{2*x} (inside of the cosine function).
\end{itemize}
In the example above, we wrote a Python function to represent a mathematical function. However, one can write a function to execute any set of tasks, not just to apply a mathematical function. Python functions are very useful in order to avoid having to repeatedly type the same code. 

Recall that the numpy module allows us to apply functions to arrays of numbers, instead of a single number. We can modify the code above slight so that, if the argument to the function, \code{x}, is an array, the function will gracefully return an array of numbers to which the function has been applied. This is done by simply replacing the call to the \code{math} version of the \code{cos} function by using the \code{numpy} version:
\begin{python}[caption=Defining a function that works on an array] 
#import the numpy module in order to use cos to an array
import numpy as np

#define our function and call it myfunction:
def myfunction(x):
  return x**2 / 3 + x**3 / 4 + np.cos(2*x)
  
#Test our function by printing out the result of evaluating it at x = 3 (same as before)
print( myfunction(3) )  

#Test it with an array
xvals = np.array([1,2,3])
print ( myfunction(xvals) )  

\end{python}
\begin{poutput}
10.710170286650365
[ 0.1671865   2.67968971 10.71017029]
\end{poutput}
where we created the array \code{xvals} using the \code{numpy} module.

\subsection{Using a loop to calculate an integral}
The ability to define our own functions in Python allows us to easily simplify complex tasks. Using ``loops'' is another way that computer programming can greatly simplify calculations that would otherwise be very tedious. In a loop, one is able to repeat the same task many times. The example below simply prints out a statement five times:
\begin{python}[caption=A simple loop] 
#A loop to print out a statement 5 times:

for i in range(5):
  print("The value of i is ",i)
\end{python}
\begin{poutput}
The value of i is  0
The value of i is  1
The value of i is  2
The value of i is  3
The value of i is  4
\end{poutput}
A few notes on the code above:
\begin{itemize}
\item The loop is defined by using the keywords \code{for ... in}
\item The value after the keyword \code{for} is the ``iterator'' variable and will have a different value each time that the code inside of the loop is run (in our case, we called the variable \code{i})
\item The value after the keyword \code{in} is an array of values that the iterator will take
\item The \code{range(N)} function returns an array of \code{N} integer value between 0 and \code{N-1} (in our case, this returns the five values 0,1,2,3,4)
\item The code to be executed at each ``iteration'' of the loop is preceded by a colon and indented (in the same way as the code for a function also follows a colon and is indented)
\end{itemize}
We now have all of the tools to evaluate an integral numerically. Recall that the integral of the function $f(x)$ between $x_a$ and $x_b$ is simply a sum:
\begin{align*}
\int_{x_a}^{x_b} f(x) dx&=\lim_{\Delta x \to 0} \sum_{i=0}^{i=N-1} f(x_{i})\Delta x\\
\Delta x &= \frac{x_b-x_a}{N}\\
x_i&=x_a+i\Delta x\\
\end{align*}
The limit of $\Delta x \to 0$ is thus equivalent to the limit $N \to \infty$. Our strategy for evaluating the integral is thus:
\begin{enumerate}
\item Define a Python function for $f(x)$
\item Create an array, \code{xvals}, of $N$ values of $x$ between $x_a$ and $x_b$
\item Evaluate the function for all those values and store those into an array, \code{fvals}
\item Loop over all of the values in the array \code{fvals}, multiply them by $\Delta x$, and sum them together.
\end{enumerate}
Let us thus use Python to evaluate the integral of the function $f(x)=4x^3+3x^2+5$ between $x=1$ and $x=5$:
\begin{python}[caption=Numerical integration of a function] 
#import numpy to work with arrays:
import numpy as np

#define our function
def f(x):
  return 4*x**3 + 3*x**2 + 5
  
#Make N and the range of integration variables:
N = 1000
xmin = 1
xmax = 5

#create the array of values of x between xmin and xmax
xvals = np.linspace(xmin, xmax, N)

#evaluate the function at all those values of x
fvals = f(xvals)

#calculate delta x
deltax = (xmax - xmin) / N

#initialize the sum to be zero:
sum = 0

#loop over the values fvals and add them to the sum
for fi in fvals:
  sum = sum + fi*deltax

#print the result:
print("The integral between {} and {} using {} steps is {:.2f} ".format(xmin, xmax, N, sum))

\end{python}
\begin{poutput}
The integral between 1 and 5 using 1000 steps is 768.42 
\end{poutput}
One can easily integrate the above function analytically and obtain the exact result of $\num{768}$. The numerical answer will approach the exact answer as we make $N$ bigger. Of course, the power of numerical integration is to use it when the function cannot be integrated analytically.

\begin{checkpointSA}{What value of $N$ should you use above in order to get within $\num{0.01}$ of the exact analytic answer?}
\end{checkpointSA}
\end{document}
