
\chapter{Potential Energy and Conservation of Energy}
\label{chapter:potentialecons}
In this chapter, we continue to develop the concept of energy in order to introduce a different formulation for Classical Physics that does not use forces. Although we will see that we can describe many phenomena using energy instead of forces, this method is completely equivalent to using Newton's Three Laws, and as such, can be derived from Newton's formulation as we will see. Because energy is a scalar quantity, for many problems, it leads to models that are much easier to develop mathematically than if one had used forces. The chapter will conclude with a presentation of the more modern approach, using ``Lagrangian Mechanics", that is currently preferred in physics and forms the basis for extending our description of physics to the microscopic world (e.g. quantum mechanics). 

\begin{learningObjectives}{
 \item something to learn
 }
\end{learningObjectives}

\begin{opening}
\begin{MCquestion}{A question}
\item a choice
\item another choice %correct
\end{MCquestion}
\end{opening}

\section{Conservative forces}
In Chapter \ref{chap:workenergy}, we introduced the concept of work, $W$, done by a force, $\vec F(\vec r)$, acting on a object as it moves along a path from position $A$ to position $B$:
\begin{align}
\label{eq:potentialecons:workdef}
W = \int_A^B \vec F(\vec r) \cdot d\vec l
\end{align}
where $\vec F(\vec r)$ is a force vector that, in general, is different in different positions in space ($\vec r$). We can also say that $\vec F$ depends on position by writing $\vec F(\vec r)=\vec F(x,y,z)$, since the position vector $\vec r$, is simply the vector $\vec r = x\hat x + y \hat y+ z\hat z$. 

The above integral is in general difficult to evaluate, as it depends on the specific path over which the object moved. In Example \ref{ex:workenergy:workfriction} of Chapter \ref{chap:workenergy}, we calculated the work done by friction on a crate that was slid across the floor along two different paths and indeed found that the work depended on the path that was taken. In Example \ref{ex:workenergy:workgravity} of the same chapter, we saw that the work done by the force of gravity when moving a box along two different paths did not depend on the path chosen\footnote{At least for those two paths that we tried in the example.}.

We call ``conservative forces'' those force for which the work done only depends on the initial and final positions and not on the path taken. ``Non-conservative'' forces are those for which the work done does depend on the path taken. The force of gravity is an example of a conservative force, whereas friction is an example of a non-conservative force.

This means that the work done by a conservative force on a ``closed path'' is zero; that is, the work done by a force on a object that moves along a path that brings the object back to its starting position is zero. Indeed, since the work done by a conservative force only depends on the location of the initial and final positions, and not the path taken between them, the work has to be zero if the object ends back where it started (as a possible path is for the object to not move at all).

Consider the work done by gravity in raising and lowering an object back to its starting position along a vertical path, as depicted in Figure \ref{fig:potentialecons:gravityvertical}.
\capfig{0.2\textwidth}{figures/PotentialECons/gravityvertical.png}{\label{fig:potentialecons:gravityvertical} An object that has moved up and back down.}
The total work done by gravity on this particular closed path is easily shown to be zero, as the work can be broken up into the negative work done as the object moves up (displacement vector $\vec d_1$) and the positive work done on the downwards path (displacement vector $\vec d_2$):
\begin{align*}
W^{tot} = \vec F_g \cdot \vec d_1 + \vec F_g \cdot \vec d_2 = -mgd + mgd = 0 
\end{align*}
In order to write the path integral of the force over a closed path, we introduce a new notation to indicate that the starting and ending position are the same (to avoid writing the integral sign with both limits being the same):
\begin{align*}
\int_A^A \vec F(\vec r) \cdot d\vec l = \oint \vec F(\vec r) \cdot d\vec l
\end{align*}
The condition for a force to be conservative is thus:
\begin{align}
\Aboxed{\oint \vec F(\vec r) \cdot d\vec l = 0}
\end{align}
since this means that the work done over a closed path is zero. The condition on the particular form for this integral to be zero can be found by Stokes' Theorem:
\begin{align*}
\oint \vec F(\vec r) \cdot d\vec l = \int_S \left[\left(\die{F_z}{y}-\die{F_y}{z}\right)\hat x+ \left(\die{F_x}{z}-\die{F_z}{x}\right)\hat y + \left(\die{F_y}{x}-\die{F_x}{y}\right)\hat z \right]\cdot d\vec A
\end{align*}
where the integral on the right is called a ``surface integral'' over the surface, $S$, enclosed by the closed path over which the work is being calculated. Do not worry, it is way beyond the scope of this text to understand this integral or Stokes' Theorem in detail! It is however useful in that it gives us the following conditions on the components of a force for that force to be conservative (by requiring the terms in parentheses to be zero):
\begin{align}
\label{eq:potentialecons:conservative}
\die{F_z}{y}-\die{F_y}{z} &= 0 \nonumber\\
\die{F_x}{z}-\die{F_z}{x} &= 0\nonumber\\
\die{F_y}{x}-\die{F_x}{y} &= 0
\end{align}
In general:
\begin{enumerate}
\item A force can be conservative if it only depends on position in space, and not speed, time, or any other quantity.
\item A force is conservative if it is constant in magnitude and direction.
\end{enumerate}

TODO: Checkpoint: Is the force exerted by a person when pushing a crate conservative?

\begin{example}{\label{ex:potentialecons:gravity}Is the force of gravity on an object of mass $m$, near the surface of the Earth, given by:
\begin{align*}
\vec F(x,y,z) =0\hat x + 0\hat y -mg \hat z
\end{align*}
conservative? Note that we have defined the $z$ axis to be vertical and positive upwards.}
The force is expected to be conservative since it is constant in magnitude and direction. We can verify this explicitly using the conditions in Equation \ref{eq:potentialecons:conservative}:
\begin{align*}
\die{F_z}{y}-\die{F_y}{z} &= \die{}{y}(-mg) - 0 &= 0\\
\die{F_x}{z}-\die{F_z}{x} &= 0 - \die{}{x}(-mg) &= 0\\
\die{F_y}{x}-\die{F_x}{y} &= 0 - 0 &=0
\end{align*}
and the force is indeed conservative since all three conditions are zero.
\end{example}


\begin{example}{Is the force given by:
\begin{align*}
\vec F(x,y,z) = \frac{-k}{r^3}\vec r = \frac{-kx}{(x^2+y^2+z^2)^\frac{3}{2}}\hat x + \frac{-ky}{(x^2+y^2+z^2)^\frac{3}{2}}\hat y + \frac{-kz}{(x^2+y^2+z^2)^\frac{3}{2}}\hat z
\end{align*}
conservative?}
Since the force only depends on position, it could be conservative, so we must check using the conditions from Equation \ref{eq:potentialecons:conservative}:
\begin{align*}
\die{F_z}{y}-\die{F_y}{z} &= \die{}{y}\left(\frac{-kz}{(x^2+y^2+z^2)^\frac{3}{2}}\right)-\die{}{z}\left( \frac{-ky}{(x^2+y^2+z^2)^\frac{3}{2}}\right)\\
&=\frac{3kz(2y)}{2(x^2+y^2+z^2)^\frac{5}{2}}-\frac{3ky(2z)}{2(x^2+y^2+z^2)^\frac{5}{2}} = 0\\
\die{F_x}{z}-\die{F_z}{x} &= \die{}{z}\left(\frac{-kx}{(x^2+y^2+z^2)^\frac{3}{2}}\right)-\die{}{x}\left( \frac{-kz}{(x^2+y^2+z^2)^\frac{3}{2}}\right)\\
&=\frac{3kx(2z)}{2(x^2+y^2+z^2)^\frac{5}{2}}-\frac{3kz(2x)}{2(x^2+y^2+z^2)^\frac{5}{2}} = 0\\
\die{F_y}{x}-\die{F_x}{y} &= \die{}{x}\left(\frac{-ky}{(x^2+y^2+z^2)^\frac{3}{2}}\right)-\die{}{y}\left( \frac{-kx}{(x^2+y^2+z^2)^\frac{3}{2}}\right)\\
&=\frac{3ky(2x)}{2(x^2+y^2+z^2)^\frac{5}{2}}-\frac{3kx(2y)}{2(x^2+y^2+z^2)^\frac{5}{2}} = 0
\end{align*}
where we used the Chain Rule to take the derivatives. Since all of the conditions are zero, the force is conservative. As we will see, the force represented here is similar mathematically to both the force that Newton introduced in his Universal Theory of Gravity, and the force introduced by Coulomb as the electric force, which are both conservative.
\end{example}

TODO: Make taking one of the above partial derivatives a problem in the Math section of the question library!

\section{Potential energy}
In this section, we introduce the concept of ``potential energy''. Potential energy is a scalar function of position that can be defined for any conservative force in a way to make it easy to calculate the work done by that force over any path. Since the work done by a conservative force in going from position $A$ to position $B$ does not depend on the particular path taken, but only on the end points, we can write the work done by a conservative force in terms of a ``potential energy function'', $U(\vec r)$, that can be evaluated at the end points:
\begin{align}
\Aboxed{-W = - \int_A^B \vec F(\vec r) \cdot d\vec l = U(\vec r_B) - U(\vec r_A) = \Delta U}
\end{align}
where we have have chosen to define the function $U(\vec r)$ so that it relates to the \textbf{negative} of the work done for reasons that will be apparent in the next section. Figure \ref{fig:potentialecons:potentialpath} shows an example of an arbitrary path between two points $A$ and $B$ in two dimensions for which one could calculate the work done by a conservative force.
\capfig{0.4\textwidth}{figures/PotentialECons/potentialpath.png}{\label{fig:potentialecons:potentialpath} Illustration of calculating the work of a conservative function along an arbitrary path.}
Once we know the function for the potential energy, $U(\vec r)$, we can calculate the work done by the associated force along any path. In order to determine the function, $U(\vec r)$, we can calculate the work that is done along a path over which the integral for work is easy (usually, a straight line). 

For example, near the surface of the Earth, the force of gravity on an object of mass, $m$, is given by:
\begin{align*}
\vec F_g = -mg \hat z
\end{align*}
where we have defined the $z$ axis to be vertical and positive upwards. We already showed in Example \ref{ex:potentialecons:gravity} that this force is conservative and that we can thus calculate a potential energy function. To do so, we can calculate the work done by the force of gravity over a straight vertical path, from position $A$ to position $B$, as shown in Figure \ref{fig:potentialecons:gravitydl}.
\capfig{0.2\textwidth}{figures/PotentialECons/gravitydl.png}{\label{fig:potentialecons:gravitydl} A vertical path for calculating the work done by gravity.}
The work done by gravity from position $A$ to position $B$ is:
\begin{align*}
W &= \int_A^B \vec F(\vec r) \cdot d\vec l\\
&= \int_{z_A}^{z_B} ( -mg \hat z) \cdot (dz \hat z) \\
&= -mg \int_{z_A}^{z_B} dz\\
&= -mg(z_B-z_A) 
\end{align*} 
By inspection, we can now identify the functional form for the potential energy function, $U(\vec r)$. We require that:
\begin{align*}
-W &= U(\vec r_B) - U(\vec r_A) = U(z_B) - U(z_A)
\end{align*}
where we replaced the position vector $\vec r$, by the $z$ coordinate, since this is a one dimensional situation. Therefore:
\begin{align*}
-W=mg(z_B-z_A)&= U(z_B) - U(z_A)\\
\therefore U(z) &= mgz + C
\end{align*} 
and we have found that, for the force of gravity near the surface of the Earth, one can define a potential energy function, $U(z) mgz +C$.

It is important to note that, since it is only the \textbf{difference} in potential energy that matters when calculating the work done, the potential energy function can have an arbitrary constant, $C$, added to it. Thus, \textbf{the value of the potential energy function is meaningless, and only differences in potential energy are meaningful and related to the work done on an object}. In other words, it does not matter where the potential energy is equal to zero, and by choosing $C$, we can therefore choose a convenient location where the potential function is zero.

\begin{example}{\capfig{0.5\textwidth}{figures/PotentialECons/table.png}{\label{fig:potentialecons:table} A box moved from the ground up onto a table.}
Calculate the work done \textbf{by gravity} when a box of mass, $m$, is moved from the ground up onto a table that is a distance $L$ away horizontally and $H$ vertically, as illustrated in Figure \ref{fig:potentialecons:table}. How much work must be done by a person moving the box?}
Since gravity is a conservative force, we can use the potential energy function $U(z)=mgz+C$ to calculate the work done by gravity when the box is moved. The work done by gravity will only depend on the change in height, $H$, as the potential energy function only depends on the $z$ coordinate of an object.  We can choose the origin of our coordinate system to be the ground and choose the constant $C=-0$, so that the potential energy function at the starting position of the box is:
\begin{align*}
U(z_A=0) = mgz+C= 0
\end{align*}
The potential energy function when the box is on the table, with $z=H$, is given by:
\begin{align*}
U(z_B=H) = mgz + C = mgH
\end{align*}
The change in potential energy, $\Delta U = U(z_B) - U(z_A)$ is equal to the negative of the work done by gravity. The work done by gravity, $W_g$, is thus:
\begin{align*}
-W_g &=  U(z_B) - U(z_A) = mgH - 0\\
\therefore W &= -mgH
\end{align*}
which is the same that we found in Example \ref{ex:workenergy:workgravity} of Chapter \ref{chap:workenergy}. The work done by gravity is negative, as we found previously, which makes sense since gravity has a component in the opposite direction of motion. 

The work done by a person, $W_p$, to move the box can easily be found by considering the net work done on the box. While the box is moving, only the person and gravity are exerting forces on the box, so those are the only two forces performing work. Since the box starts and ends at rest, the net work done on the box must be zero (no change in kinetic energy):
\begin{align*}
W^{net} = 0 &= W_g + W_p\\
\therefore W_p &= -W_g = mgH
\end{align*}
\textbf{Discussion:} We find that the person had to do positive work, which makes sense, since they had to exert a force with a component in the direction of motion (upwards). It is also interesting to note that it does not matter if the person exerted a constant force or whether they varied the force that they exerted on the box as they moved it: the amount of work done by the person is fixed to be the negative of the work done by gravity.
\end{example}

\begin{example}{\label{ex:potentialecons:springpotential}The force exerted by a spring that is extended or compressed by a distance, $x$, is given by Hookes' Law:
\begin{align*}
\vec F(x) = -k x\hat x
\end{align*}
where the $x$ axis is defined to be colinear with the spring with and the origin is located at the rest position of the spring. Show that: (a) the force exerted by the spring onto an object is conservative, and (b), determine the corresponding potential energy function. }
Since the force depends on position, it could be conservative, which we can check with the conditions from Equation \ref{eq:potentialecons:conservative}:
\begin{align*}
\die{F_z}{y}-\die{F_y}{z} &= 0 - 0 &= 0\\
\die{F_x}{z}-\die{F_z}{x} &= \die{}{z}(-kx)) - 0&= 0\\
\die{F_y}{x}-\die{F_x}{y} &= 0 - \die{}{y}(-kx)) &=0
\end{align*}
and the force is indeed conservative. To determine the potential energy function, let us calculate the work done by the spring from position $x_A$ to position $x_B$:
\begin{align*}
W &=\int_A^B \vec F(\vec r) \cdot d\vec l\\
&=\int_A^B (-kx\hat x) \cdot dx \hat x\\
&=\int_{x_A}^{x_B} (-kx)dx=\left[-\frac{1}{2}kx^2  \right]_{x_A}^{x_B}\\
&=-\left( \frac{1}{2}kx_B^2-\frac{1}{2}kx_A^2 \right)
\end{align*}
Again, comparing with:
\begin{align*}
-W &= U(\vec r_B) - U(\vec r_A) = U(x_B) - U(x_A)
\end{align*}
We can identify the potential energy for a spring:
\begin{align*}
U(x) = \frac{1}{2}kx^2 + C
\end{align*}
where, in general, the constant $C$ can take any value. If we choose $C=0$, then the potential energy is zero when the spring is at rest, although it is not important what choice is made.
\end{example}

TODO: Question library problem to find potential energy function for a non linear spring ($F=-k_1x-k_3x^3$) after showing that the force is conservative.

\subsection{Recovering the force from potential energy}
Given a (scalar) potential energy function, $U(\vec r)$, it is possible to determine the (vector) force that is associated with it. Take for example the potential energy from a spring (Example \ref{ex:potentialecons:springpotential}):
\begin{align*}
U(x) = \frac{1}{2}kx^2 + C
\end{align*} 
In one dimension, this is simply the negative of the anti-derivative of the $x$ component of the spring force:
\begin{align*}
F(x) &= -kx\\
U(x) &= -\int F(x) dx = \int (kx) dx = \frac{1}{2}kx^2+C\\
\therefore F(x) &= -\frac{d}{dx}U(x)
\end{align*}
Thus, the force can be obtained by taking the derivative with respect to position of the negative of potential energy function. 

In three dimensions, the situation is similar, although the potential energy function (and the components of the force vector) will generally depend on all three position coordinates, $x$, $y$, and $z$. In three dimensions, the force is given by the gradient of the negative of potential energy function\footnote{As you may recall from Appendix \ref{app:calculus}, the gradient is a vector that points towards the direction of maximal increase in a multi-variate function.}:
\begin{align}
\vec F(\vec r) &= -\vec\nabla U(\vec r)=-\vec\nabla U(x,y,z)\nonumber\\
\therefore F_x(x,y,z) &= -\die{}{x}U(x,y,z)\nonumber\\
\therefore F_y(x,y,z) &= -\die{}{y}U(x,y,z)\nonumber\\
\therefore F_z(x,y,z) &= -\die{}{z}U(x,y,z)
\end{align}

\section{Mechanical energy and conservation of energy}
Recall the Work-Energy Theorem, relating the net work done on an object to its change in kinetic energy along a path:
\begin{align*}
W^{net}=\Delta K = K_B - K_A
\end{align*}
where $K_B$ ($K_A$) is the final (initial) kinetic energy of the object along some path from $A$ to $B$ over which the net work was done. Generally, the net work done is the sum of the work done by conservative forces, $W^C$, and the work done by none conservative forces, $W^{NC}$:
\begin{align*}
W^{net}=W^C+W^{NC}
\end{align*}
The work done by conservative forces can be expressed in terms of changes in potential energy functions. For example, suppose that two conservative forces, $\vec F_1$ and $\vec F_2$, are exerted on the object. The work done by those two forces in terms of the change in potential energy is:
\begin{align*}
W_1 &= -\Delta U_1\\
W_2 &= -\Delta U_2
\end{align*}
where $U_1$ and $U_2$ are the change in potential energy associated with forces $\vec F_1$ and $\vec F_2$, respectively. We can re-arrange the Work-Energy Theorem as follows\footnote{This is why we defined potential energy as negative of the work; it becomes a positive term when we move it to the same side of the equation as the kinetic energy!}:
\begin{align*}
W^{net}=W^C+W^{NC}=-\Delta U_1 - \Delta U_2 +W^{NC} &= \Delta K\\
\therefore W^{NC} = \Delta U_1 + \Delta U_2 + \Delta K
\end{align*}
That is, the work done by non-conservative forces is equal to the sum of the change in potential and kinetic energies. In general, if we write $\Delta U$ as the total change in potential energy of the object (the sum of the changes in potential energies associated with each conservative force), we can write this in a more general form:
\begin{align}
\Aboxed{W^{NC}=\Delta U + \Delta K}
\end{align}
This fundamental result is what we call the ``conservation of mechanical energy''. In particular, note that if there are no non-conservative forces doing work on the object:
\begin{align}
\Aboxed{\Delta K + \Delta U &= 0}\quad\text{if no non-conservative forces}\\
-\Delta U &= \Delta K \nonumber
\end{align}
That is, the sum of the change in potential and kinetic energies of the object is always zero. If the potential energy of the object increases, then the kinetic energy of the object must decrease by the same amount, if there are no non-conservative forces doing work on the object.

We can introduce the ``mechanical energy'', $E$, of an object as the sum of potential and kinetic energies of the object:
\begin{align}
\Aboxed{E = U+K}
\end{align}
If the object started at position $A$, with potential energy $U_A$ and kinetic energy $K_A$, and ended up at position $B$ with potential energy $U_B$ and kinetic energy $K_B$, then we can write the mechanical energy at both positions, and its change as:
\begin{align*}
E_A &= U_A + K_A\\
E_B &= U_B + K_B\\
\therefore \Delta E &= E_B - E_A \\
&= U_B + K_B - U_A - K_A\\
&= \Delta U + \Delta K
\end{align*}
Thus, the change in mechanical energy of the object is equal to the work done by none-conservative forces:
\begin{align*}
W^{NC} = \Delta U + \Delta K = \Delta E
\end{align*}
and if there is no work done by non-conservative forces on the object, then the mechanical energy of the object does not change:
\begin{align*}
\Delta E &= 0\quad\text{if no non-conservative forces}\\
\therefore E &= \text{constant}
\end{align*}
The introduction of mechanical energy gives us a completely different way to think about mechanics. We can now think of an object as having ``energy'' (potential and/or kinetic), and we can think of forces as changing the energy of the object. 

\begin{example}{\label{ex:potentialecons:blockspring}\capfig{0.4\textwidth}{figures/PotentialEcons/blockspring.png}{\label{fig:potentialecons:blockspring} A block is launched along a frictionless surface by compressing a spring by a distance $D$. The top panel shows the spring when at rest, and the bottom panel shows the spring compressed by a distance $D$ just before releasing the block.}
A block of mass $m$ can slide freely along a frictionless surface. A horizontal spring, with spring constant, $k$, is attached to a wall on one end, while the other end can move freely, as shown in Figure \ref{fig:potentialecons:blockspring}. A coordinate system is defined such that the $x$ axis is horizontal and the free end of the spring is at $x=0$ when the spring is at rest. The block is pushed against the spring so that the spring is compressed by a distance $D$. The block is then released. What speed will the block have when it leaves the spring?}
This is again the same example that we saw in Chapters \ref{chap:applyingnewtonslaws} and \ref{chap:workenergy}. We will show here that it is solved very easily with conservation of energy. The forces acting on the block are:
\begin{enumerate}
\item Weight, which does no work since it is perpendicular to the block's displacement.
\item The normal force, which does no work since it is perpendicular to the block's displacement.
\item The force from the spring, which is conservative and can be modelled with a potential energy $U(x)=\frac{1}{2}kx^2$, where $x$ is the position of the end of the spring.
\end{enumerate}

The block starts at rest at position $A$ ($x=-D$), where the spring is compressed by a distance $D$, and leaves the spring at position $B$ ($x=0$), where the spring is at its rest position. 

At position $A$ the kinetic energy of the block is $K_A=0$, and the potential energy from the spring force of the block is $U_A=\frac{1}{2}kD^2$. The mechanical energy of the block at position $A$ is thus:
\begin{align*}
K_A&=0\\
U_A&=\frac{1}{2}kD^2\\
\therefore E_A &= U_A + K_A = \frac{1}{2}kD^2
\end{align*}
At position $B$, the spring potential energy of the block is zero (since the spring is at rest), and all of the energy is kinetic:
\begin{align*}
K_B&=\frac{1}{2}mv_B^2\\
U_B&=0\\
\therefore E_B &= U_B+K_B=\frac{1}{2}mv_B^2
\end{align*}
Since there are no non-conservative forces doing work on the block, the mechanical energies at $A$ and $B$ are the same:
\begin{align*}
W^{NC}&=\Delta E=E_B-E_A= 0\\
\therefore E_B&=E_A\\
\frac{1}{2}mv_B^2&= \frac{1}{2}kD^2\\
 v_B &= \sqrt{\frac{kD^2}{m}}
\end{align*}
as we found previously.
\end{example}

\begin{example}{\label{ex:potentialecons:blockI}
\capfig{0.5\textwidth}{figures/PotentialEcons/blockI.png}{\label{fig:potentialecons:blockI} A block slides down an incline before sliding on a flat surface and stopping. }
A block of mass $m$ is placed at rest on an incline that makes an angle $\theta$ with respect to the horizontal, as shown in Figure \ref{fig:potentialecons:blockI}. The block is nudged slightly so that the force of static friction is overcome and the block starts to accelerate down the incline. At the bottom of the incline, the block slides on a horizontal surface. The coefficient of kinetic friction between the block and the incline is $\mu_{k1}$, and the coefficient of kinetic friction between the block and horizontal surface is $\mu_{k2}$. If one assumes that the block started at rest a distance $L$ from the horizontal surface, how far along the horizontal surface will the block slide before stopping?}
This is the same problem that we solved in Chapter {ref:applyingnewtonslaws} Example \ref{ex:applyingnewtonslaws:blockI}. In that case, we solved for the acceleration of the block using Newton's Second Law and then used kinematics to find how far the block goes. We can solve this problem in a much easier way using conservation of energy. 

It is still a good idea to think about what forces are applied on the object in order to determine if there are non-conservative forces doing work. In this case, the forces on the block are:
\begin{enumerate}
\item The normal force, which does no work, as it is always perpendicular to the motion.
\item Weight, which does work when the height of the object changes, which we can model with a potential energy function.
\item Friction, which is a non-conservative force, whose work we must determine. 
\end{enumerate}
Let us divide the motion into two segments: (1) a segment along the incline (positions $A$ to $B$ in Figure \ref{fig:potentialecons:blockI}, where gravitational potential energy changes, and (2), the horizontal segment from positions $B$ to position $C$ on the figure. We can then apply conservation of energy for each segment. 

Starting with the first segment, we can choose the gravitational potential energy to be zero when the block is at the bottom of the incline. The block starts at a height $h=L\sin\theta$ above the  bottom of the incline. The gravitational potential energy for the beginning and end of the first segment are thus:
\begin{align*}
U_A &= mgL\sin\theta\\
U_B &= 0\\
\Delta U_1 &= U_B-U_A = -mgH
\end{align*}
Since the block starts at rest, its kinetic energy is zero at position $A$, and if the speed of the box is $v_B$ at position $B$, we can write its kinetic energy at both positions as:
\begin{align*}
K_A &=0\\
K_B &= \frac{1}{2}mv_B^2\\
\Delta K_1 &= \frac{1}{2}mv_B^2
\end{align*}
The mechanical energy of the object at positions $A$ and $B$ is thus:
\begin{align*}
E_A &= U_A+K_A = mgL\sin\theta\\
E_B &= U_B+K_B = \frac{1}{2}mv_B^2\\
\Delta E &= E_B - E_A = \frac{1}{2}mv_B^2 - mgL\sin\theta
\end{align*}
Finally, since we have a non-conservative force, the force of kinetic friction acting on the first segment, we need to calculate the work done by that force. We found in Example \ref{ex:applyingnewtonslaws:blockI} that the force of friction had magnitude $f_k=\mu_{k1}N=\mu_{k1}mg\cos\theta$. Since the force of friction is anti-parallel to the displacement vector of length $L$ down the incline, the work done by friction is:
\begin{align*}
W^{NC}=W_f = -f_kL=-\mu_{k1}mg\cos\theta L
\end{align*}
Applying conservation of energy along the first segment, we have:
\begin{align*}
W^{NC} &= \Delta E\\
-\mu_{k1}mg\cos\theta L &= \frac{1}{2}mv_B^2 - mgL\sin\theta\\
\therefore \frac{1}{2}mv_B^2 &= mgL\sin\theta-\mu_{k1}mg\cos\theta L 
\end{align*}
Note that the above equation, in words, could be read as ``the change in kinetic energy is equal to the change in potential energy minus the work done by friction''. In other words, the block had potential energy, which was converted into kinetic energy and heat (the work done by friction can be thought of as thermal energy). 

We now proceed in an analogous way for the second segment, from position $B$ to position $C$. The only force that can do work along this segment, of length $x$, is the force of kinetic friction. Both the weight and normal force are perpendicular to the displacement. The initial kinetic energy is $K_B$, and the final kinetic energy is zero. The change in mechanical energy is thus:
\begin{align*}
\Delta E &= E_C - E_B = K_B - K_C = K_B\\
&=\frac{1}{2}mv_B^2\\
&= mgL\sin\theta-\mu_{k1}mg\cos\theta L 
\end{align*}
where, in the last line, we used the result from the first segment. The work done by the force of friction is:
\begin{align*}
W^{NC}=W_f = -f_kx = -\mu_{k2} N x=-\mu_{k2} mg x
\end{align*}
Finally, applying conservation of energy, we can find $x$:
\begin{align*}
W^{NC} &= \Delta E\\
-\mu_{k2} mg x &= mgL\sin\theta-\mu_{k1}mg\cos\theta L \\
\therefore x&= L\frac{1}{\mu_{k2}}\left(\sin\theta - \mu_{k1}\cos\theta\right)
\end{align*}
which is the same result that we obtained in Example \ref{ex:applyingnewtonslaws:blockI}.

\textbf{Discussion:} By using conservation of energy, we were able to model the motion of the block down the incline in a way that was much easier than what was done in Example \ref{ex:applyingnewtonslaws:blockI}. Furthermore, although we modelled friction as a non-conservative force doing work, we gained some insight into the idea that this could be thought of as an energy loss. In terms of energy, we would say that the block initially had gravitational potential energy, which was then converted into kinetic energy as well as thermal energy (in the heat generated by friction). 
\end{example}

\section{Energy diagrams and equilibria}
We can write the mechanical energy of an object as:
\begin{align*}
E = K + U
\end{align*}
which will be a constant if there are no non-conservative forces doing work on the object. This means that if the potential energy of the object increases, then its kinetic energy must decrease by the same amount, and vice-versa. 

Consider a block that can slide on a frictionless horizontal surface and that is attached to a spring, as in the top panel of Figure \ref{fig:potentialecons:springE}, where $x=0$ is chosen as the position corresponding to the rest length of the spring. If one pushes on the block so as to compress the spring by a distance $D$ and then releases it, the block will initially accelerate because of the spring force in the positive $x$ direction until the block reaches the rest position of the spring ($x=0$ on the diagram). At that point, the spring will exert a force in the opposite direction and the block will continue in the same direciton and decelerate until it stops and turns around. It will then accelerate again towards the spring rest position, and then decelerate once the spring starts being compressed again, until the block stops and the motion repeats. We can say that the block oscillates back and forth about the rest position of the spring.

We can also use energy to describe the motion of the block in terms of its total mechanical energy, $E$, and its potential energy given by:
\begin{align*}
U(x)=\frac{1}{2}kx^2
\end{align*}
The bottom panel of Figure \ref{fig:potentialecons:springE} shows an ``Energy Diagram'' for the block, which allows us to examine how the total energy, $E$, of the block is divided between kinetic and potential energy depending on the position of the block. The vertical axis corresponds to energy and the horizontal axis corresponds to the position of the block. 

 The total mechanical energy, $E=\SI{25}{J}$, is shown by the horizontal red line. Also illustrated are the potential energy function ($U(x)$ in blue), and the kinetic energy, ($K=E-U(x)$, in dotted black).
\capfig{0.7\textwidth}{figures/PotentialEcons/springE.png}{\label{fig:potentialecons:springE} Energy diagram for a spring with spring constant $k=\SI{1}{N/m}$. }
The energy diagram allows us to describe the motion of the object attached to the spring in terms of energy. A few things to note:
\begin{enumerate}
\item At $x=\pm D$, the potential energy is equal to $E$, so the kinetic energy is zero. The block is thus instantaneously at rest at those positions.
\item At $x=0$, the potential energy is zero, and the kinetic energy is maximal. This corresponds to where the block has the highest speed. 
\item The kinetic energy of the block can never be negative, thus, the block cannot be located outside the range $[-D,+D]$, and we would say that the motion of the block is ``bound''. The points between which the motion is bound are called ``turning points''.
\end{enumerate}
An analysis of the energy diagram tells us that the block is bound between the two turning points, which themselves are equidistant from the origin. When we initially compress the spring, we are ``giving'' the block spring potential energy. As the block moves, the potential energy of the block is converted to kinetic energy as it accelerates and then back into potential energy as it decelerates.

TODO: Checkpoint question, for $k=1$, total energy $E=\SI{25}{J}$, what are the positions of the turning points for a spring?

By looking at only the potential energy function, without knowing that it is related to a spring, we can come to the same conclusions; namely that the motion is bound as long as the total mechanical energy is not infinite. We call the point $x=0$ a ``stable equilibrium'', because it is a local minimum of the potential energy function. If the object is displaced from the equilibrium point, it will want to move back towards that point. This can also be understood in terms of the force associated with the potential energy function:
\begin{align*}
F = -\frac{d}{dx}U(x)
\end{align*}
The \textbf{equilibrium point is given by the condition that the force associated with the potential is zero} ($x=0$ in the case of the potential energy from a spring). The equilibrium is a stable equilibrium because the force associated with the potential energy function ($F(x)=-kx$ for the spring) points towards the equilibrium point.

The potential energy function for an object with total mechanical energy, $E$, can be thought of as a little ``roller coaster'', on which you place a marble and watch it ``roll down'' the potential energy function. You can think of placing a marble where $U(x)=E$ and releasing it. The marble would then roll down the potential energy function, just as an actual marble would roll down a real slope, mimicking the motion of the object along the $x$ axis. This is illustrated in Figure \ref{fig:potentialecons:potential} which shows an arbitrary potential energy function and a marble being placed at a location where the potential energy is equal to $E$.
\capfig{0.4\textwidth}{figures/PotentialEcons/potential.png}{\label{fig:potentialecons:potential} Arbitrary potential energy function and illustration of visualizing a marble rolling down the function.}
The motion of the marble will be bound between the two points where the potential energy function is equal to $E$. When the marble is placed as shown, it will roll towards the left, just as if it were a real marble on a surface. Since the potential energy is increasing as a function of $x$ at the point where we placed the marble, the force is in the negative $x$ direction (remember, the force is the negative of the derivative of the potential energy function). With the given energy, the marble would never be able to make it to point $D$, as it does not have enough energy to ``climb up the hill''. It would roll down, through point $C$, up to point $B$, down to point $A$, and then turn around where $U(x)=E$ and return to where it started. 

Locations $A$ and $C$ on the diagram are stable equilibria, because if a marble is placed in those location and nudged slightly, it will come back to the equilibrium point. Points $B$ and $D$ are ``unstable equilibria'', because if the marble is placed there and nudged, it will not come back to those points. Note that if the marble were placed at point $D$ and nudged towards the right, the motion of the marble would be unbond on the right, and it would keep going in that direction. 

\section{Advanced Topic: The Lagrangian formulation of classical physics}
So far, we have seen that based on Newton's Laws, one can formulate a description of motion that is based solely on the concept of energy. A lot of research was done in the 1800s (TODO: check date!) to reformulate a theory of mechanics that would be equivalent to Newton's Theory but whose starting point is the concept of energy instead of the concept of force. This ``modern'' approach to classical mechanics is primarily based on the research by Lagrange and Hamilton. 

Although it is beyond the scope of this text to go into detail into this formulation, it is worth taking a quick look in order to get a better sense of how physicists seek to generalize theories. It is also worth noting that the Lagrangian formulation is the method by which theories are developed for quantum mechanics and modern physics. 

The Lagrangian description of a ``system'' is based on a quantity, $L$, called the ``Lagrangian'', which is defined as:
\begin{align*}
L = K - U
\end{align*}
where $K$ is the kinetic energy of the system, and $U$ is its potential energy. A ``system'' can be a rather complex collection of objects, although we will illustrate how the Lagrangian formulation is implemented for a single object of mass $m$ moving in one dimension under the influence of gravity. Let $x$ be the direction of motion (which is vertical) such that the potential and kinetic energies of the object are given by:
\begin{align*}
U(x) &= mgx\\
K(v_x) &= \frac{1}{2}mv_x^2\\
\therefore L(x,v_x) &= \frac{1}{2}mv_x^2 - mgx
\end{align*}
where we chose the potential energy to be zero at $x=0$, and $v_x$ is the $x$ component of the speed of the object.

In the modern formulation of classical mechanics, the motion of the system will be such that the following integral is minimized:
\begin{align*}
S = \int Ldt
\end{align*}
where $L$ can depend on time explicitly or implicitly (through the fact that position and velocity, on which the Lagrangian depends, are themselves time-dependent). The requirement that the above integral be minimized is called the ``Principle of Least Action''\footnote{The integral, $S$, is called the ``action of the system.}, and is thought to be the fundamental principle that describes all of the laws of physics. The condition of the action to be minimized is given by the Euler-Lagrange equation:
\begin{align*}
\frac{d}{dt}\left(\die{L}{v_x}\right)-\die{L}{x} = 0
\end{align*}
Thus, in the Lagrangian formulation, one first writes down the Lagrangian for the system, and then uses the Euler-Lagrange equation to obtain the ``equations of motion'' for the system (i.e. the kinematic quantities, such as acceleration, for the system). 

Given the Lagrangian that we found above for a particle moving in one dimension under the influence of gravity, we can determine each term in the Euler-Lagrange equation:
\begin{align*}
\die{L}{v_x} &= \die{}{v_x}\left(\frac{1}{2}mv_x^2 - mgx \right)=mv_x\\
\therefore\frac{d}{dt}\left(\die{L}{v_x}\right) &= \frac{d}{dt} (mv_x) = ma_x\\
\therefore\die{L}{x}&= \die{}{x}\frac{1}{2}mv_x^2 - mgx = -mg\\
\end{align*}
Putting these into the Euler-Lagrange equation:
\begin{align*}
\frac{d}{dt}\left(\die{L}{v_x}\right)-\die{L}{x} &= 0\\
(ma_x) - (-mg) &=0\\
\therefore a_x = -g
\end{align*}
which is exactly equivalent to Newton's Second Law. In the Lagrangian formulation, we do not need the concept of force. Instead, we describe possible ``interactions'' by a potential energy function. That is why you may sometimes hear of physicists talking about the ``Weak interaction'' instead of the ``Weak force'' when they are talking about one of the four fundamental interactions (forces) of Nature. This is because, in the modern formulation of physics, one does not use the concept of force, and instead thinks of potential energy functions that can be added into the Lagrangian. 

Emmy Noether, a mathematician in the late 1800s (TODO:check date!!!), derived a theorem that makes the Lagrangian formulation particularly aesthetic. Noether's theorem states that for any symmetry in the Lagrangian, there exist a quantity that is conserved. For example, if the Lagrangian does not depend explicitly on time, then a quantity, which we call energy, is conserved. The Lagrangian that we had above does not depend on time explicitly, and thus energy is conserved (gravitational potential energy is converted into kinetic energy). If Lagrangian did not depend on position, then ``momentum''\footnote{See chapter \ref{chap:momentum}} would be conserved (the Lagrangian that we used depends on $x$ through the potential energy, thus momentum in the $x$ direction is not conserved). 


\newpage
\section{Summary}

\begin{chapterSummary}
A force is conservative if the work done by that force on a closed path is zero:
\begin{align*}
\oint \vec F(\vec r) \cdot d\vec l = 0
\end{align*}
Equivalently, the force is conservative if the work done by the force along a path from position $A$ to position $B$ does not depend on the path. The conditions for a force to be conservative are given by:
\begin{align*}
\die{F_z}{y}-\die{F_y}{z} &= 0 \nonumber\\
\die{F_x}{z}-\die{F_z}{x} &= 0\nonumber\\
\die{F_y}{x}-\die{F_x}{y} &= 0
\end{align*}
In particular, a force that is constant in magnitude and direction will be conservative. A force that depends on quantities other than position (e.g. speed, time) will not be conservative.

For any conservative force, $\vec F(\vec r)$, we can define a potential energy function $U(\vec r)$ that can be used to calculate the work done by the force along any path between position $A$ and position $B$:
\begin{align*}
-W = - \int_A^B \vec F(\vec r) \cdot d\vec l = U(\vec r_B) - U(\vec r_A) = \Delta U
\end{align*}
where the change in potential energy function in going from $A$ to $B$ is equal to the negative of the work done in going from point $A$ to point $B$.

The Work-Energy theorem states that the net work done on an object is equal to its change in kinetic energy. We can break up the net work done on an object as the sum of the work done by conservative and non-conservative forces:
\begin{align*}
W^{net}=W^{NC}+W^{C}
\end{align*}

\end{chapterSummary}

\newpage
\begin{importantEquations}
This is an important equation
\begin{align*}
E = mc^2
\end{align*}

\end{importantEquations}


\newpage
\section{Thinking about the material}
\subsection{Reflect and research}

\begin{enumerate}
\item Something to research more.
\end{enumerate}
\subsection{To try at home}

\begin{tQuestion}Try doing this \end{tQuestion}

\subsection{To try in the lab}

\newpage
\section{Sample problems and solutions}
\subsection{Problems}
\begin{problemParts}{A question\label{Q:chaptertitle:q1}}
\item How close can he get to the hurdle before he has to jump?
\item What maximum height does he reach?
\end{problemParts}

\newpage
\subsection{Solutions}
\begin{solution}{\ref{Q:chaptertitle:q1}}
{
the solution
}
\end{solution}

